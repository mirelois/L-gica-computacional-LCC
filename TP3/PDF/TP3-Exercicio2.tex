\documentclass[11pt]{article}

    \usepackage[breakable]{tcolorbox}
    \usepackage{parskip} % Stop auto-indenting (to mimic markdown behaviour)
    

    % Basic figure setup, for now with no caption control since it's done
    % automatically by Pandoc (which extracts ![](path) syntax from Markdown).
    \usepackage{graphicx}
    % Maintain compatibility with old templates. Remove in nbconvert 6.0
    \let\Oldincludegraphics\includegraphics
    % Ensure that by default, figures have no caption (until we provide a
    % proper Figure object with a Caption API and a way to capture that
    % in the conversion process - todo).
    \usepackage{caption}
    \DeclareCaptionFormat{nocaption}{}
    \captionsetup{format=nocaption,aboveskip=0pt,belowskip=0pt}

    \usepackage{float}
    \floatplacement{figure}{H} % forces figures to be placed at the correct location
    \usepackage{xcolor} % Allow colors to be defined
    \usepackage{enumerate} % Needed for markdown enumerations to work
    \usepackage{geometry} % Used to adjust the document margins
    \usepackage{amsmath} % Equations
    \usepackage{amssymb} % Equations
    \usepackage{textcomp} % defines textquotesingle
    % Hack from http://tex.stackexchange.com/a/47451/13684:
    \AtBeginDocument{%
        \def\PYZsq{\textquotesingle}% Upright quotes in Pygmentized code
    }
    \usepackage{upquote} % Upright quotes for verbatim code
    \usepackage{eurosym} % defines \euro

    \usepackage{iftex}
    \ifPDFTeX
        \usepackage[T1]{fontenc}
        \IfFileExists{alphabeta.sty}{
              \usepackage{alphabeta}
          }{
              \usepackage[mathletters]{ucs}
              \usepackage[utf8x]{inputenc}
          }
    \else
        \usepackage{fontspec}
        \usepackage{unicode-math}
    \fi

    \usepackage{fancyvrb} % verbatim replacement that allows latex
    \usepackage{grffile} % extends the file name processing of package graphics 
                         % to support a larger range
    \makeatletter % fix for old versions of grffile with XeLaTeX
    \@ifpackagelater{grffile}{2019/11/01}
    {
      % Do nothing on new versions
    }
    {
      \def\Gread@@xetex#1{%
        \IfFileExists{"\Gin@base".bb}%
        {\Gread@eps{\Gin@base.bb}}%
        {\Gread@@xetex@aux#1}%
      }
    }
    \makeatother
    \usepackage[Export]{adjustbox} % Used to constrain images to a maximum size
    \adjustboxset{max size={0.9\linewidth}{0.9\paperheight}}

    % The hyperref package gives us a pdf with properly built
    % internal navigation ('pdf bookmarks' for the table of contents,
    % internal cross-reference links, web links for URLs, etc.)
    \usepackage{hyperref}
    % The default LaTeX title has an obnoxious amount of whitespace. By default,
    % titling removes some of it. It also provides customization options.
    \usepackage{titling}
    \usepackage{longtable} % longtable support required by pandoc >1.10
    \usepackage{booktabs}  % table support for pandoc > 1.12.2
    \usepackage{array}     % table support for pandoc >= 2.11.3
    \usepackage{calc}      % table minipage width calculation for pandoc >= 2.11.1
    \usepackage[inline]{enumitem} % IRkernel/repr support (it uses the enumerate* environment)
    \usepackage[normalem]{ulem} % ulem is needed to support strikethroughs (\sout)
                                % normalem makes italics be italics, not underlines
    \usepackage{mathrsfs}
    

    
    % Colors for the hyperref package
    \definecolor{urlcolor}{rgb}{0,.145,.698}
    \definecolor{linkcolor}{rgb}{.71,0.21,0.01}
    \definecolor{citecolor}{rgb}{.12,.54,.11}

    % ANSI colors
    \definecolor{ansi-black}{HTML}{3E424D}
    \definecolor{ansi-black-intense}{HTML}{282C36}
    \definecolor{ansi-red}{HTML}{E75C58}
    \definecolor{ansi-red-intense}{HTML}{B22B31}
    \definecolor{ansi-green}{HTML}{00A250}
    \definecolor{ansi-green-intense}{HTML}{007427}
    \definecolor{ansi-yellow}{HTML}{DDB62B}
    \definecolor{ansi-yellow-intense}{HTML}{B27D12}
    \definecolor{ansi-blue}{HTML}{208FFB}
    \definecolor{ansi-blue-intense}{HTML}{0065CA}
    \definecolor{ansi-magenta}{HTML}{D160C4}
    \definecolor{ansi-magenta-intense}{HTML}{A03196}
    \definecolor{ansi-cyan}{HTML}{60C6C8}
    \definecolor{ansi-cyan-intense}{HTML}{258F8F}
    \definecolor{ansi-white}{HTML}{C5C1B4}
    \definecolor{ansi-white-intense}{HTML}{A1A6B2}
    \definecolor{ansi-default-inverse-fg}{HTML}{FFFFFF}
    \definecolor{ansi-default-inverse-bg}{HTML}{000000}

    % common color for the border for error outputs.
    \definecolor{outerrorbackground}{HTML}{FFDFDF}

    % commands and environments needed by pandoc snippets
    % extracted from the output of `pandoc -s`
    \providecommand{\tightlist}{%
      \setlength{\itemsep}{0pt}\setlength{\parskip}{0pt}}
    \DefineVerbatimEnvironment{Highlighting}{Verbatim}{commandchars=\\\{\}}
    % Add ',fontsize=\small' for more characters per line
    \newenvironment{Shaded}{}{}
    \newcommand{\KeywordTok}[1]{\textcolor[rgb]{0.00,0.44,0.13}{\textbf{{#1}}}}
    \newcommand{\DataTypeTok}[1]{\textcolor[rgb]{0.56,0.13,0.00}{{#1}}}
    \newcommand{\DecValTok}[1]{\textcolor[rgb]{0.25,0.63,0.44}{{#1}}}
    \newcommand{\BaseNTok}[1]{\textcolor[rgb]{0.25,0.63,0.44}{{#1}}}
    \newcommand{\FloatTok}[1]{\textcolor[rgb]{0.25,0.63,0.44}{{#1}}}
    \newcommand{\CharTok}[1]{\textcolor[rgb]{0.25,0.44,0.63}{{#1}}}
    \newcommand{\StringTok}[1]{\textcolor[rgb]{0.25,0.44,0.63}{{#1}}}
    \newcommand{\CommentTok}[1]{\textcolor[rgb]{0.38,0.63,0.69}{\textit{{#1}}}}
    \newcommand{\OtherTok}[1]{\textcolor[rgb]{0.00,0.44,0.13}{{#1}}}
    \newcommand{\AlertTok}[1]{\textcolor[rgb]{1.00,0.00,0.00}{\textbf{{#1}}}}
    \newcommand{\FunctionTok}[1]{\textcolor[rgb]{0.02,0.16,0.49}{{#1}}}
    \newcommand{\RegionMarkerTok}[1]{{#1}}
    \newcommand{\ErrorTok}[1]{\textcolor[rgb]{1.00,0.00,0.00}{\textbf{{#1}}}}
    \newcommand{\NormalTok}[1]{{#1}}
    
    % Additional commands for more recent versions of Pandoc
    \newcommand{\ConstantTok}[1]{\textcolor[rgb]{0.53,0.00,0.00}{{#1}}}
    \newcommand{\SpecialCharTok}[1]{\textcolor[rgb]{0.25,0.44,0.63}{{#1}}}
    \newcommand{\VerbatimStringTok}[1]{\textcolor[rgb]{0.25,0.44,0.63}{{#1}}}
    \newcommand{\SpecialStringTok}[1]{\textcolor[rgb]{0.73,0.40,0.53}{{#1}}}
    \newcommand{\ImportTok}[1]{{#1}}
    \newcommand{\DocumentationTok}[1]{\textcolor[rgb]{0.73,0.13,0.13}{\textit{{#1}}}}
    \newcommand{\AnnotationTok}[1]{\textcolor[rgb]{0.38,0.63,0.69}{\textbf{\textit{{#1}}}}}
    \newcommand{\CommentVarTok}[1]{\textcolor[rgb]{0.38,0.63,0.69}{\textbf{\textit{{#1}}}}}
    \newcommand{\VariableTok}[1]{\textcolor[rgb]{0.10,0.09,0.49}{{#1}}}
    \newcommand{\ControlFlowTok}[1]{\textcolor[rgb]{0.00,0.44,0.13}{\textbf{{#1}}}}
    \newcommand{\OperatorTok}[1]{\textcolor[rgb]{0.40,0.40,0.40}{{#1}}}
    \newcommand{\BuiltInTok}[1]{{#1}}
    \newcommand{\ExtensionTok}[1]{{#1}}
    \newcommand{\PreprocessorTok}[1]{\textcolor[rgb]{0.74,0.48,0.00}{{#1}}}
    \newcommand{\AttributeTok}[1]{\textcolor[rgb]{0.49,0.56,0.16}{{#1}}}
    \newcommand{\InformationTok}[1]{\textcolor[rgb]{0.38,0.63,0.69}{\textbf{\textit{{#1}}}}}
    \newcommand{\WarningTok}[1]{\textcolor[rgb]{0.38,0.63,0.69}{\textbf{\textit{{#1}}}}}
    
    
    % Define a nice break command that doesn't care if a line doesn't already
    % exist.
    \def\br{\hspace*{\fill} \\* }
    % Math Jax compatibility definitions
    \def\gt{>}
    \def\lt{<}
    \let\Oldtex\TeX
    \let\Oldlatex\LaTeX
    \renewcommand{\TeX}{\textrm{\Oldtex}}
    \renewcommand{\LaTeX}{\textrm{\Oldlatex}}
    % Document parameters
    % Document title
    \title{TP3-Exercicio2}
    
    
    
    
    
% Pygments definitions
\makeatletter
\def\PY@reset{\let\PY@it=\relax \let\PY@bf=\relax%
    \let\PY@ul=\relax \let\PY@tc=\relax%
    \let\PY@bc=\relax \let\PY@ff=\relax}
\def\PY@tok#1{\csname PY@tok@#1\endcsname}
\def\PY@toks#1+{\ifx\relax#1\empty\else%
    \PY@tok{#1}\expandafter\PY@toks\fi}
\def\PY@do#1{\PY@bc{\PY@tc{\PY@ul{%
    \PY@it{\PY@bf{\PY@ff{#1}}}}}}}
\def\PY#1#2{\PY@reset\PY@toks#1+\relax+\PY@do{#2}}

\@namedef{PY@tok@w}{\def\PY@tc##1{\textcolor[rgb]{0.73,0.73,0.73}{##1}}}
\@namedef{PY@tok@c}{\let\PY@it=\textit\def\PY@tc##1{\textcolor[rgb]{0.24,0.48,0.48}{##1}}}
\@namedef{PY@tok@cp}{\def\PY@tc##1{\textcolor[rgb]{0.61,0.40,0.00}{##1}}}
\@namedef{PY@tok@k}{\let\PY@bf=\textbf\def\PY@tc##1{\textcolor[rgb]{0.00,0.50,0.00}{##1}}}
\@namedef{PY@tok@kp}{\def\PY@tc##1{\textcolor[rgb]{0.00,0.50,0.00}{##1}}}
\@namedef{PY@tok@kt}{\def\PY@tc##1{\textcolor[rgb]{0.69,0.00,0.25}{##1}}}
\@namedef{PY@tok@o}{\def\PY@tc##1{\textcolor[rgb]{0.40,0.40,0.40}{##1}}}
\@namedef{PY@tok@ow}{\let\PY@bf=\textbf\def\PY@tc##1{\textcolor[rgb]{0.67,0.13,1.00}{##1}}}
\@namedef{PY@tok@nb}{\def\PY@tc##1{\textcolor[rgb]{0.00,0.50,0.00}{##1}}}
\@namedef{PY@tok@nf}{\def\PY@tc##1{\textcolor[rgb]{0.00,0.00,1.00}{##1}}}
\@namedef{PY@tok@nc}{\let\PY@bf=\textbf\def\PY@tc##1{\textcolor[rgb]{0.00,0.00,1.00}{##1}}}
\@namedef{PY@tok@nn}{\let\PY@bf=\textbf\def\PY@tc##1{\textcolor[rgb]{0.00,0.00,1.00}{##1}}}
\@namedef{PY@tok@ne}{\let\PY@bf=\textbf\def\PY@tc##1{\textcolor[rgb]{0.80,0.25,0.22}{##1}}}
\@namedef{PY@tok@nv}{\def\PY@tc##1{\textcolor[rgb]{0.10,0.09,0.49}{##1}}}
\@namedef{PY@tok@no}{\def\PY@tc##1{\textcolor[rgb]{0.53,0.00,0.00}{##1}}}
\@namedef{PY@tok@nl}{\def\PY@tc##1{\textcolor[rgb]{0.46,0.46,0.00}{##1}}}
\@namedef{PY@tok@ni}{\let\PY@bf=\textbf\def\PY@tc##1{\textcolor[rgb]{0.44,0.44,0.44}{##1}}}
\@namedef{PY@tok@na}{\def\PY@tc##1{\textcolor[rgb]{0.41,0.47,0.13}{##1}}}
\@namedef{PY@tok@nt}{\let\PY@bf=\textbf\def\PY@tc##1{\textcolor[rgb]{0.00,0.50,0.00}{##1}}}
\@namedef{PY@tok@nd}{\def\PY@tc##1{\textcolor[rgb]{0.67,0.13,1.00}{##1}}}
\@namedef{PY@tok@s}{\def\PY@tc##1{\textcolor[rgb]{0.73,0.13,0.13}{##1}}}
\@namedef{PY@tok@sd}{\let\PY@it=\textit\def\PY@tc##1{\textcolor[rgb]{0.73,0.13,0.13}{##1}}}
\@namedef{PY@tok@si}{\let\PY@bf=\textbf\def\PY@tc##1{\textcolor[rgb]{0.64,0.35,0.47}{##1}}}
\@namedef{PY@tok@se}{\let\PY@bf=\textbf\def\PY@tc##1{\textcolor[rgb]{0.67,0.36,0.12}{##1}}}
\@namedef{PY@tok@sr}{\def\PY@tc##1{\textcolor[rgb]{0.64,0.35,0.47}{##1}}}
\@namedef{PY@tok@ss}{\def\PY@tc##1{\textcolor[rgb]{0.10,0.09,0.49}{##1}}}
\@namedef{PY@tok@sx}{\def\PY@tc##1{\textcolor[rgb]{0.00,0.50,0.00}{##1}}}
\@namedef{PY@tok@m}{\def\PY@tc##1{\textcolor[rgb]{0.40,0.40,0.40}{##1}}}
\@namedef{PY@tok@gh}{\let\PY@bf=\textbf\def\PY@tc##1{\textcolor[rgb]{0.00,0.00,0.50}{##1}}}
\@namedef{PY@tok@gu}{\let\PY@bf=\textbf\def\PY@tc##1{\textcolor[rgb]{0.50,0.00,0.50}{##1}}}
\@namedef{PY@tok@gd}{\def\PY@tc##1{\textcolor[rgb]{0.63,0.00,0.00}{##1}}}
\@namedef{PY@tok@gi}{\def\PY@tc##1{\textcolor[rgb]{0.00,0.52,0.00}{##1}}}
\@namedef{PY@tok@gr}{\def\PY@tc##1{\textcolor[rgb]{0.89,0.00,0.00}{##1}}}
\@namedef{PY@tok@ge}{\let\PY@it=\textit}
\@namedef{PY@tok@gs}{\let\PY@bf=\textbf}
\@namedef{PY@tok@gp}{\let\PY@bf=\textbf\def\PY@tc##1{\textcolor[rgb]{0.00,0.00,0.50}{##1}}}
\@namedef{PY@tok@go}{\def\PY@tc##1{\textcolor[rgb]{0.44,0.44,0.44}{##1}}}
\@namedef{PY@tok@gt}{\def\PY@tc##1{\textcolor[rgb]{0.00,0.27,0.87}{##1}}}
\@namedef{PY@tok@err}{\def\PY@bc##1{{\setlength{\fboxsep}{\string -\fboxrule}\fcolorbox[rgb]{1.00,0.00,0.00}{1,1,1}{\strut ##1}}}}
\@namedef{PY@tok@kc}{\let\PY@bf=\textbf\def\PY@tc##1{\textcolor[rgb]{0.00,0.50,0.00}{##1}}}
\@namedef{PY@tok@kd}{\let\PY@bf=\textbf\def\PY@tc##1{\textcolor[rgb]{0.00,0.50,0.00}{##1}}}
\@namedef{PY@tok@kn}{\let\PY@bf=\textbf\def\PY@tc##1{\textcolor[rgb]{0.00,0.50,0.00}{##1}}}
\@namedef{PY@tok@kr}{\let\PY@bf=\textbf\def\PY@tc##1{\textcolor[rgb]{0.00,0.50,0.00}{##1}}}
\@namedef{PY@tok@bp}{\def\PY@tc##1{\textcolor[rgb]{0.00,0.50,0.00}{##1}}}
\@namedef{PY@tok@fm}{\def\PY@tc##1{\textcolor[rgb]{0.00,0.00,1.00}{##1}}}
\@namedef{PY@tok@vc}{\def\PY@tc##1{\textcolor[rgb]{0.10,0.09,0.49}{##1}}}
\@namedef{PY@tok@vg}{\def\PY@tc##1{\textcolor[rgb]{0.10,0.09,0.49}{##1}}}
\@namedef{PY@tok@vi}{\def\PY@tc##1{\textcolor[rgb]{0.10,0.09,0.49}{##1}}}
\@namedef{PY@tok@vm}{\def\PY@tc##1{\textcolor[rgb]{0.10,0.09,0.49}{##1}}}
\@namedef{PY@tok@sa}{\def\PY@tc##1{\textcolor[rgb]{0.73,0.13,0.13}{##1}}}
\@namedef{PY@tok@sb}{\def\PY@tc##1{\textcolor[rgb]{0.73,0.13,0.13}{##1}}}
\@namedef{PY@tok@sc}{\def\PY@tc##1{\textcolor[rgb]{0.73,0.13,0.13}{##1}}}
\@namedef{PY@tok@dl}{\def\PY@tc##1{\textcolor[rgb]{0.73,0.13,0.13}{##1}}}
\@namedef{PY@tok@s2}{\def\PY@tc##1{\textcolor[rgb]{0.73,0.13,0.13}{##1}}}
\@namedef{PY@tok@sh}{\def\PY@tc##1{\textcolor[rgb]{0.73,0.13,0.13}{##1}}}
\@namedef{PY@tok@s1}{\def\PY@tc##1{\textcolor[rgb]{0.73,0.13,0.13}{##1}}}
\@namedef{PY@tok@mb}{\def\PY@tc##1{\textcolor[rgb]{0.40,0.40,0.40}{##1}}}
\@namedef{PY@tok@mf}{\def\PY@tc##1{\textcolor[rgb]{0.40,0.40,0.40}{##1}}}
\@namedef{PY@tok@mh}{\def\PY@tc##1{\textcolor[rgb]{0.40,0.40,0.40}{##1}}}
\@namedef{PY@tok@mi}{\def\PY@tc##1{\textcolor[rgb]{0.40,0.40,0.40}{##1}}}
\@namedef{PY@tok@il}{\def\PY@tc##1{\textcolor[rgb]{0.40,0.40,0.40}{##1}}}
\@namedef{PY@tok@mo}{\def\PY@tc##1{\textcolor[rgb]{0.40,0.40,0.40}{##1}}}
\@namedef{PY@tok@ch}{\let\PY@it=\textit\def\PY@tc##1{\textcolor[rgb]{0.24,0.48,0.48}{##1}}}
\@namedef{PY@tok@cm}{\let\PY@it=\textit\def\PY@tc##1{\textcolor[rgb]{0.24,0.48,0.48}{##1}}}
\@namedef{PY@tok@cpf}{\let\PY@it=\textit\def\PY@tc##1{\textcolor[rgb]{0.24,0.48,0.48}{##1}}}
\@namedef{PY@tok@c1}{\let\PY@it=\textit\def\PY@tc##1{\textcolor[rgb]{0.24,0.48,0.48}{##1}}}
\@namedef{PY@tok@cs}{\let\PY@it=\textit\def\PY@tc##1{\textcolor[rgb]{0.24,0.48,0.48}{##1}}}

\def\PYZbs{\char`\\}
\def\PYZus{\char`\_}
\def\PYZob{\char`\{}
\def\PYZcb{\char`\}}
\def\PYZca{\char`\^}
\def\PYZam{\char`\&}
\def\PYZlt{\char`\<}
\def\PYZgt{\char`\>}
\def\PYZsh{\char`\#}
\def\PYZpc{\char`\%}
\def\PYZdl{\char`\$}
\def\PYZhy{\char`\-}
\def\PYZsq{\char`\'}
\def\PYZdq{\char`\"}
\def\PYZti{\char`\~}
% for compatibility with earlier versions
\def\PYZat{@}
\def\PYZlb{[}
\def\PYZrb{]}
\makeatother


    % For linebreaks inside Verbatim environment from package fancyvrb. 
    \makeatletter
        \newbox\Wrappedcontinuationbox 
        \newbox\Wrappedvisiblespacebox 
        \newcommand*\Wrappedvisiblespace {\textcolor{red}{\textvisiblespace}} 
        \newcommand*\Wrappedcontinuationsymbol {\textcolor{red}{\llap{\tiny$\m@th\hookrightarrow$}}} 
        \newcommand*\Wrappedcontinuationindent {3ex } 
        \newcommand*\Wrappedafterbreak {\kern\Wrappedcontinuationindent\copy\Wrappedcontinuationbox} 
        % Take advantage of the already applied Pygments mark-up to insert 
        % potential linebreaks for TeX processing. 
        %        {, <, #, %, $, ' and ": go to next line. 
        %        _, }, ^, &, >, - and ~: stay at end of broken line. 
        % Use of \textquotesingle for straight quote. 
        \newcommand*\Wrappedbreaksatspecials {% 
            \def\PYGZus{\discretionary{\char`\_}{\Wrappedafterbreak}{\char`\_}}% 
            \def\PYGZob{\discretionary{}{\Wrappedafterbreak\char`\{}{\char`\{}}% 
            \def\PYGZcb{\discretionary{\char`\}}{\Wrappedafterbreak}{\char`\}}}% 
            \def\PYGZca{\discretionary{\char`\^}{\Wrappedafterbreak}{\char`\^}}% 
            \def\PYGZam{\discretionary{\char`\&}{\Wrappedafterbreak}{\char`\&}}% 
            \def\PYGZlt{\discretionary{}{\Wrappedafterbreak\char`\<}{\char`\<}}% 
            \def\PYGZgt{\discretionary{\char`\>}{\Wrappedafterbreak}{\char`\>}}% 
            \def\PYGZsh{\discretionary{}{\Wrappedafterbreak\char`\#}{\char`\#}}% 
            \def\PYGZpc{\discretionary{}{\Wrappedafterbreak\char`\%}{\char`\%}}% 
            \def\PYGZdl{\discretionary{}{\Wrappedafterbreak\char`\$}{\char`\$}}% 
            \def\PYGZhy{\discretionary{\char`\-}{\Wrappedafterbreak}{\char`\-}}% 
            \def\PYGZsq{\discretionary{}{\Wrappedafterbreak\textquotesingle}{\textquotesingle}}% 
            \def\PYGZdq{\discretionary{}{\Wrappedafterbreak\char`\"}{\char`\"}}% 
            \def\PYGZti{\discretionary{\char`\~}{\Wrappedafterbreak}{\char`\~}}% 
        } 
        % Some characters . , ; ? ! / are not pygmentized. 
        % This macro makes them "active" and they will insert potential linebreaks 
        \newcommand*\Wrappedbreaksatpunct {% 
            \lccode`\~`\.\lowercase{\def~}{\discretionary{\hbox{\char`\.}}{\Wrappedafterbreak}{\hbox{\char`\.}}}% 
            \lccode`\~`\,\lowercase{\def~}{\discretionary{\hbox{\char`\,}}{\Wrappedafterbreak}{\hbox{\char`\,}}}% 
            \lccode`\~`\;\lowercase{\def~}{\discretionary{\hbox{\char`\;}}{\Wrappedafterbreak}{\hbox{\char`\;}}}% 
            \lccode`\~`\:\lowercase{\def~}{\discretionary{\hbox{\char`\:}}{\Wrappedafterbreak}{\hbox{\char`\:}}}% 
            \lccode`\~`\?\lowercase{\def~}{\discretionary{\hbox{\char`\?}}{\Wrappedafterbreak}{\hbox{\char`\?}}}% 
            \lccode`\~`\!\lowercase{\def~}{\discretionary{\hbox{\char`\!}}{\Wrappedafterbreak}{\hbox{\char`\!}}}% 
            \lccode`\~`\/\lowercase{\def~}{\discretionary{\hbox{\char`\/}}{\Wrappedafterbreak}{\hbox{\char`\/}}}% 
            \catcode`\.\active
            \catcode`\,\active 
            \catcode`\;\active
            \catcode`\:\active
            \catcode`\?\active
            \catcode`\!\active
            \catcode`\/\active 
            \lccode`\~`\~ 	
        }
    \makeatother

    \let\OriginalVerbatim=\Verbatim
    \makeatletter
    \renewcommand{\Verbatim}[1][1]{%
        %\parskip\z@skip
        \sbox\Wrappedcontinuationbox {\Wrappedcontinuationsymbol}%
        \sbox\Wrappedvisiblespacebox {\FV@SetupFont\Wrappedvisiblespace}%
        \def\FancyVerbFormatLine ##1{\hsize\linewidth
            \vtop{\raggedright\hyphenpenalty\z@\exhyphenpenalty\z@
                \doublehyphendemerits\z@\finalhyphendemerits\z@
                \strut ##1\strut}%
        }%
        % If the linebreak is at a space, the latter will be displayed as visible
        % space at end of first line, and a continuation symbol starts next line.
        % Stretch/shrink are however usually zero for typewriter font.
        \def\FV@Space {%
            \nobreak\hskip\z@ plus\fontdimen3\font minus\fontdimen4\font
            \discretionary{\copy\Wrappedvisiblespacebox}{\Wrappedafterbreak}
            {\kern\fontdimen2\font}%
        }%
        
        % Allow breaks at special characters using \PYG... macros.
        \Wrappedbreaksatspecials
        % Breaks at punctuation characters . , ; ? ! and / need catcode=\active 	
        \OriginalVerbatim[#1,codes*=\Wrappedbreaksatpunct]%
    }
    \makeatother

    % Exact colors from NB
    \definecolor{incolor}{HTML}{303F9F}
    \definecolor{outcolor}{HTML}{D84315}
    \definecolor{cellborder}{HTML}{CFCFCF}
    \definecolor{cellbackground}{HTML}{F7F7F7}
    
    % prompt
    \makeatletter
    \newcommand{\boxspacing}{\kern\kvtcb@left@rule\kern\kvtcb@boxsep}
    \makeatother
    \newcommand{\prompt}[4]{
        {\ttfamily\llap{{\color{#2}[#3]:\hspace{3pt}#4}}\vspace{-\baselineskip}}
    }
    

    
    % Prevent overflowing lines due to hard-to-break entities
    \sloppy 
    % Setup hyperref package
    \hypersetup{
      breaklinks=true,  % so long urls are correctly broken across lines
      colorlinks=true,
      urlcolor=urlcolor,
      linkcolor=linkcolor,
      citecolor=citecolor,
      }
    % Slightly bigger margins than the latex defaults
    
    \geometry{verbose,tmargin=1in,bmargin=1in,lmargin=1in,rmargin=1in}
    
    

\begin{document}
    
    \maketitle
    
    

    
    \hypertarget{tp3}{%
\section{TP3}\label{tp3}}

\hypertarget{grupo-15}{%
\subsection{Grupo 15}\label{grupo-15}}

Carlos Eduardo Da Silva Machado A96936

Gonçalo Manuel Maia de Sousa A97485

    \hypertarget{problema-2}{%
\subsection{Problema 2}\label{problema-2}}

    \hypertarget{descriuxe7uxe3o-do-problema}{%
\subsubsection{Descrição do
Problema}\label{descriuxe7uxe3o-do-problema}}

Temos um sistema dinâmico com 4 inversores (\(\,A, B, C, D\,\)) que lêm
um bit num canal input e escrevem num canal output uma transformação
desse bit.
\includegraphics{1.png}

Além disso, temos de respeitar as seguintes regras:

\begin{enumerate}
\def\labelenumi{\arabic{enumi}.}
\item
  Cada inversor tem um bit \(s\) de estado, inicializado com um valor
  aleatório.
\item
  Cada inversor é regido pelas transformações:

  \[\mathbf{invert}\mathtt(in,out)\]
  \[x \gets \mathsf{read}(\mathtt{in})\]
  \[s \gets \neg x\;\;\|\;\; s\gets s\oplus x  \\            
                      \mathsf{write}(\mathtt{out},s)\]
\item
  O estado do sistema é um tuplo definido pelos 4 bits \(s\), e é
  inicializado com um vetor aleatório em \(\{0,1\}^4\;\).
\item
  Caso o estado do sistema seja \((0,0,0,0)\), o sistema termina em
  ERRO.
\end{enumerate}

    \hypertarget{abordagem-do-problema}{%
\subsubsection{Abordagem do problema}\label{abordagem-do-problema}}

Para resolver este problema, construimos, com o auxilio do \emph{pySMT},
um \emph{Safe First Order Transition System} (SFOTS) definido por:
\(\:\:\:\Sigma\;\equiv\;\langle\,\mathsf{X}\,,\,\mathsf{next}\,,\,\mathsf{I}\,,\,\mathsf{T}\,,\,\mathsf{E}\,\rangle\:\:\:\),
onde \(X\) identifica as variáveis base, next operador que gera os
diversos ``clones'' das variáveis, \(I\) determina os estados iniciais,
\(T\) representa a relação de transição e \(E\) a condição de erro. Ao
contrário dos \emph{First Order Transitions Systems} (FOTS), o SFOTS
possui a propriedade de segurança na definição do sistema.

As escolhas e o vetor aleatório que determina os valores dos bits de
estado de cada um dos inversores serão gerados ao nível do input com a
ajuda da bilioteca \emph{random} do \emph{python}, mais precisamente,
com o comando random.choice.

Teremos 4 funções importantes que modelam o problema, \(genState\),
\(init\), \(trans\) e \(error\) cujos objectivos são indicados
juntamente do seu código \emph{python}.

De modo a gerar um traço com \(n\) estados, vamos criar a função
\(genTrace\) que utiliza a abordagem \emph{bounded model checker} (BMC).
Além disso, vamos testar a segurança do sistema através de três métodos
diferentes (BMC, k-indução e model checking com interpolantes) .

    \hypertarget{cuxf3digo-python}{%
\subsubsection{Código python}\label{cuxf3digo-python}}

    Esta secção de codigo serve para importar as bibliotecas necessárias
para a realização do trabalho.

    \begin{tcolorbox}[breakable, size=fbox, boxrule=1pt, pad at break*=1mm,colback=cellbackground, colframe=cellborder]
\prompt{In}{incolor}{1}{\boxspacing}
\begin{Verbatim}[commandchars=\\\{\}]
\PY{k+kn}{from} \PY{n+nn}{pysmt}\PY{n+nn}{.}\PY{n+nn}{shortcuts} \PY{k+kn}{import} \PY{o}{*}
\PY{k+kn}{from} \PY{n+nn}{pysmt}\PY{n+nn}{.}\PY{n+nn}{typing} \PY{k+kn}{import} \PY{o}{*}
\PY{k+kn}{import} \PY{n+nn}{random} \PY{k}{as} \PY{n+nn}{rn}
\PY{k+kn}{import} \PY{n+nn}{itertools} 
\end{Verbatim}
\end{tcolorbox}

    Função \(genState\) que tem como argumentos uma lista com as variáveis
de estado \(vars\), uma etiqueta (do tipo string) \(s\) e um inteiro
\(i\). Devolve uma cópia das variáveis do estado \(state\) identificadas
pela etiqueta e pelo inteiro.

    \begin{tcolorbox}[breakable, size=fbox, boxrule=1pt, pad at break*=1mm,colback=cellbackground, colframe=cellborder]
\prompt{In}{incolor}{2}{\boxspacing}
\begin{Verbatim}[commandchars=\\\{\}]
\PY{k}{def} \PY{n+nf}{genState}\PY{p}{(}\PY{n+nb}{vars}\PY{p}{,}\PY{n}{s}\PY{p}{,}\PY{n}{i}\PY{p}{)}\PY{p}{:}
    \PY{n}{state} \PY{o}{=} \PY{p}{\PYZob{}}\PY{p}{\PYZcb{}}
    \PY{k}{for} \PY{n}{v} \PY{o+ow}{in} \PY{n+nb}{vars}\PY{p}{:}
        \PY{n}{state}\PY{p}{[}\PY{n}{v}\PY{p}{]} \PY{o}{=} \PY{n}{Symbol}\PY{p}{(}\PY{n}{v}\PY{o}{+}\PY{l+s+s1}{\PYZsq{}}\PY{l+s+s1}{!}\PY{l+s+s1}{\PYZsq{}}\PY{o}{+}\PY{n}{s}\PY{o}{+}\PY{n+nb}{str}\PY{p}{(}\PY{n}{i}\PY{p}{)}\PY{p}{,}\PY{n}{BOOL}\PY{p}{)}
    \PY{k}{return} \PY{n}{state}
\end{Verbatim}
\end{tcolorbox}

    Função \(init\) que dado um estado do programa \(state\) e uma lista de
Bools \(s\) que indica o valor dos 4 \emph{bits} iniciais. Devolve um
predicado do \emph{pySMT} que testa se esse estado é um possível estado
inicial do programa. Logo, para garantir que o estado é efetivamente um
estado possível, temos de garantir que os valores são iguais aos gerados
aleatoriamente.

    \begin{tcolorbox}[breakable, size=fbox, boxrule=1pt, pad at break*=1mm,colback=cellbackground, colframe=cellborder]
\prompt{In}{incolor}{3}{\boxspacing}
\begin{Verbatim}[commandchars=\\\{\}]
\PY{k}{def} \PY{n+nf}{init}\PY{p}{(}\PY{n}{state}\PY{p}{,}\PY{n}{s}\PY{p}{)}\PY{p}{:}
    \PY{k}{return} \PY{n}{And}\PY{p}{(}\PY{n}{Iff}\PY{p}{(}\PY{n}{state}\PY{p}{[}\PY{l+s+s1}{\PYZsq{}}\PY{l+s+s1}{A}\PY{l+s+s1}{\PYZsq{}}\PY{p}{]}\PY{p}{,}\PY{n}{Bool}\PY{p}{(}\PY{n}{s}\PY{p}{[}\PY{l+m+mi}{0}\PY{p}{]}\PY{p}{)}\PY{p}{)}\PY{p}{,}\PY{n}{Iff}\PY{p}{(}\PY{n}{state}\PY{p}{[}\PY{l+s+s1}{\PYZsq{}}\PY{l+s+s1}{B}\PY{l+s+s1}{\PYZsq{}}\PY{p}{]}\PY{p}{,}\PY{n}{Bool}\PY{p}{(}\PY{n}{s}\PY{p}{[}\PY{l+m+mi}{1}\PY{p}{]}\PY{p}{)}\PY{p}{)}\PY{p}{,}\PY{n}{Iff}\PY{p}{(}\PY{n}{state}\PY{p}{[}\PY{l+s+s1}{\PYZsq{}}\PY{l+s+s1}{C}\PY{l+s+s1}{\PYZsq{}}\PY{p}{]}\PY{p}{,}\PY{n}{Bool}\PY{p}{(}\PY{n}{s}\PY{p}{[}\PY{l+m+mi}{2}\PY{p}{]}\PY{p}{)}\PY{p}{)}\PY{p}{,}\PY{n}{Iff}\PY{p}{(}\PY{n}{state}\PY{p}{[}\PY{l+s+s1}{\PYZsq{}}\PY{l+s+s1}{D}\PY{l+s+s1}{\PYZsq{}}\PY{p}{]}\PY{p}{,}\PY{n}{Bool}\PY{p}{(}\PY{n}{s}\PY{p}{[}\PY{l+m+mi}{3}\PY{p}{]}\PY{p}{)}\PY{p}{)}\PY{p}{)}
\end{Verbatim}
\end{tcolorbox}

    Função \(trans\) e função auxiliar \(escolha\).

A função escolha tem como argumentos \(x\) que corresponde à parte
\(x \gets \mathsf{read}(\mathtt{in})\), pois o seu valor é o valor do
\emph{bit} do inversor que fornece o \emph{input} para o inversor que
utliza a função escolha, o \(s\) é o valor do \emph{bit} no inversor em
que usamos a função \(escolha\) e \(first\) é um bool que determina a
escolha daquele inversor na parte de código correspondente a
\(s \gets \neg x\;\;\|\;\; s\gets s\oplus x\). Devolve, um valor que é
usado como \emph{output}, ou seja, é escrito no próximo estado
(\(\mathsf{write}(\mathtt{out},s)\)).

Por fim, a função de transição \(trans\) com argumentos \(curr\),
\(prox\) e \(S\), onde os dois primentos indicam o estado atual e o
estado seguinte, respetivamente, e o último uma lista de 4 booleanos que
determinam a escolha em cada inversor. Devolve um predicado do
\emph{pySMT} que testa se é possível transitar do estado \(curr\) para o
estado \(prox\). Os valores dos 4 \emph{bits} são atualizados
simultaneamente, ou seja, a cada transição, o tuplo terá cada um dos
seus valores atualizados.

    \begin{tcolorbox}[breakable, size=fbox, boxrule=1pt, pad at break*=1mm,colback=cellbackground, colframe=cellborder]
\prompt{In}{incolor}{4}{\boxspacing}
\begin{Verbatim}[commandchars=\\\{\}]
\PY{k}{def} \PY{n+nf}{escolha}\PY{p}{(}\PY{n}{x}\PY{p}{,}\PY{n}{s}\PY{p}{,}\PY{n}{first}\PY{p}{)}\PY{p}{:}
    \PY{k}{return} \PY{n}{Not}\PY{p}{(}\PY{n}{x}\PY{p}{)} \PY{k}{if} \PY{n}{first} \PY{k}{else} \PY{n}{Xor}\PY{p}{(}\PY{n}{s}\PY{p}{,}\PY{n}{x}\PY{p}{)}

\PY{k}{def} \PY{n+nf}{trans}\PY{p}{(}\PY{n}{curr}\PY{p}{,}\PY{n}{prox}\PY{p}{,}\PY{n}{S}\PY{p}{)}\PY{p}{:}
    \PY{n}{tA} \PY{o}{=} \PY{n}{Iff}\PY{p}{(}\PY{n}{prox}\PY{p}{[}\PY{l+s+s1}{\PYZsq{}}\PY{l+s+s1}{A}\PY{l+s+s1}{\PYZsq{}}\PY{p}{]}\PY{p}{,}\PY{n}{escolha}\PY{p}{(}\PY{n}{curr}\PY{p}{[}\PY{l+s+s1}{\PYZsq{}}\PY{l+s+s1}{C}\PY{l+s+s1}{\PYZsq{}}\PY{p}{]}\PY{p}{,}\PY{n}{curr}\PY{p}{[}\PY{l+s+s1}{\PYZsq{}}\PY{l+s+s1}{A}\PY{l+s+s1}{\PYZsq{}}\PY{p}{]}\PY{p}{,}\PY{n}{S}\PY{p}{[}\PY{l+m+mi}{0}\PY{p}{]}\PY{p}{)}\PY{p}{)}
    \PY{n}{tB} \PY{o}{=} \PY{n}{Iff}\PY{p}{(}\PY{n}{prox}\PY{p}{[}\PY{l+s+s1}{\PYZsq{}}\PY{l+s+s1}{B}\PY{l+s+s1}{\PYZsq{}}\PY{p}{]}\PY{p}{,}\PY{n}{escolha}\PY{p}{(}\PY{n}{curr}\PY{p}{[}\PY{l+s+s1}{\PYZsq{}}\PY{l+s+s1}{A}\PY{l+s+s1}{\PYZsq{}}\PY{p}{]}\PY{p}{,}\PY{n}{curr}\PY{p}{[}\PY{l+s+s1}{\PYZsq{}}\PY{l+s+s1}{B}\PY{l+s+s1}{\PYZsq{}}\PY{p}{]}\PY{p}{,}\PY{n}{S}\PY{p}{[}\PY{l+m+mi}{1}\PY{p}{]}\PY{p}{)}\PY{p}{)}
    \PY{n}{tC} \PY{o}{=} \PY{n}{Iff}\PY{p}{(}\PY{n}{prox}\PY{p}{[}\PY{l+s+s1}{\PYZsq{}}\PY{l+s+s1}{C}\PY{l+s+s1}{\PYZsq{}}\PY{p}{]}\PY{p}{,}\PY{n}{escolha}\PY{p}{(}\PY{n}{curr}\PY{p}{[}\PY{l+s+s1}{\PYZsq{}}\PY{l+s+s1}{D}\PY{l+s+s1}{\PYZsq{}}\PY{p}{]}\PY{p}{,}\PY{n}{curr}\PY{p}{[}\PY{l+s+s1}{\PYZsq{}}\PY{l+s+s1}{C}\PY{l+s+s1}{\PYZsq{}}\PY{p}{]}\PY{p}{,}\PY{n}{S}\PY{p}{[}\PY{l+m+mi}{2}\PY{p}{]}\PY{p}{)}\PY{p}{)}
    \PY{n}{tD} \PY{o}{=} \PY{n}{Iff}\PY{p}{(}\PY{n}{prox}\PY{p}{[}\PY{l+s+s1}{\PYZsq{}}\PY{l+s+s1}{D}\PY{l+s+s1}{\PYZsq{}}\PY{p}{]}\PY{p}{,}\PY{n}{escolha}\PY{p}{(}\PY{n}{curr}\PY{p}{[}\PY{l+s+s1}{\PYZsq{}}\PY{l+s+s1}{B}\PY{l+s+s1}{\PYZsq{}}\PY{p}{]}\PY{p}{,}\PY{n}{curr}\PY{p}{[}\PY{l+s+s1}{\PYZsq{}}\PY{l+s+s1}{D}\PY{l+s+s1}{\PYZsq{}}\PY{p}{]}\PY{p}{,}\PY{n}{S}\PY{p}{[}\PY{l+m+mi}{3}\PY{p}{]}\PY{p}{)}\PY{p}{)}
    
    \PY{k}{return} \PY{n}{And}\PY{p}{(}\PY{n}{tA}\PY{p}{,}\PY{n}{tB}\PY{p}{,}\PY{n}{tC}\PY{p}{,}\PY{n}{tD}\PY{p}{)}
\end{Verbatim}
\end{tcolorbox}

    Função \(error\) com um único argumento \(state\) que indica o estado em
que estamos a testar a condição de erro. Devolve um predicado do
\emph{pySMT} que testa se o estado \(state\) é um possível estado de
erro do programa. No caso deste problema, a condição de erro será
(0,0,0,0), isto é, o \emph{bit} de cada um dos inversores ser 0 naquele
estado.

    \begin{tcolorbox}[breakable, size=fbox, boxrule=1pt, pad at break*=1mm,colback=cellbackground, colframe=cellborder]
\prompt{In}{incolor}{5}{\boxspacing}
\begin{Verbatim}[commandchars=\\\{\}]
\PY{k}{def} \PY{n+nf}{error}\PY{p}{(}\PY{n}{state}\PY{p}{)}\PY{p}{:}
    \PY{k}{return} \PY{n}{And}\PY{p}{(}\PY{n}{Iff}\PY{p}{(}\PY{n}{state}\PY{p}{[}\PY{l+s+s1}{\PYZsq{}}\PY{l+s+s1}{A}\PY{l+s+s1}{\PYZsq{}}\PY{p}{]}\PY{p}{,}\PY{n}{Bool}\PY{p}{(}\PY{k+kc}{False}\PY{p}{)}\PY{p}{)}\PY{p}{,}\PY{n}{Iff}\PY{p}{(}\PY{n}{state}\PY{p}{[}\PY{l+s+s1}{\PYZsq{}}\PY{l+s+s1}{B}\PY{l+s+s1}{\PYZsq{}}\PY{p}{]}\PY{p}{,}\PY{n}{Bool}\PY{p}{(}\PY{k+kc}{False}\PY{p}{)}\PY{p}{)}\PY{p}{,}\PY{n}{Iff}\PY{p}{(}\PY{n}{state}\PY{p}{[}\PY{l+s+s1}{\PYZsq{}}\PY{l+s+s1}{C}\PY{l+s+s1}{\PYZsq{}}\PY{p}{]}\PY{p}{,}\PY{n}{Bool}\PY{p}{(}\PY{k+kc}{False}\PY{p}{)}\PY{p}{)}\PY{p}{,}\PY{n}{Iff}\PY{p}{(}\PY{n}{state}\PY{p}{[}\PY{l+s+s1}{\PYZsq{}}\PY{l+s+s1}{D}\PY{l+s+s1}{\PYZsq{}}\PY{p}{]}\PY{p}{,}\PY{n}{Bool}\PY{p}{(}\PY{k+kc}{False}\PY{p}{)}\PY{p}{)}\PY{p}{)}
\end{Verbatim}
\end{tcolorbox}

    A Função \(genTrace\) gera um possível traço de execução com um número
finito de transições com argumentos \(vars\) (variáveis de estado),
\(init\) (função init acima criada), \(estadoInicial\), \(trans\)
(função trans, também, criada acima), \(S\), \(error\) (função error) e
\(n\) que indica o número de transições.

    \begin{tcolorbox}[breakable, size=fbox, boxrule=1pt, pad at break*=1mm,colback=cellbackground, colframe=cellborder]
\prompt{In}{incolor}{6}{\boxspacing}
\begin{Verbatim}[commandchars=\\\{\}]
\PY{k}{def} \PY{n+nf}{genTrace}\PY{p}{(}\PY{n+nb}{vars}\PY{p}{,}\PY{n}{init}\PY{p}{,}\PY{n}{estadoInicial}\PY{p}{,}\PY{n}{trans}\PY{p}{,}\PY{n}{S}\PY{p}{,}\PY{n}{error}\PY{p}{,}\PY{n}{n}\PY{p}{)}\PY{p}{:}
    \PY{k}{with} \PY{n}{Solver}\PY{p}{(}\PY{n}{name}\PY{o}{=}\PY{l+s+s2}{\PYZdq{}}\PY{l+s+s2}{z3}\PY{l+s+s2}{\PYZdq{}}\PY{p}{)} \PY{k}{as} \PY{n}{s}\PY{p}{:}
        
        \PY{n}{X} \PY{o}{=} \PY{p}{[}\PY{n}{genState}\PY{p}{(}\PY{n+nb}{vars}\PY{p}{,}\PY{l+s+s1}{\PYZsq{}}\PY{l+s+s1}{X}\PY{l+s+s1}{\PYZsq{}}\PY{p}{,}\PY{n}{i}\PY{p}{)} \PY{k}{for} \PY{n}{i} \PY{o+ow}{in} \PY{n+nb}{range}\PY{p}{(}\PY{n}{n}\PY{o}{+}\PY{l+m+mi}{1}\PY{p}{)}\PY{p}{]}   \PY{c+c1}{\PYZsh{} cria n+1 estados (com etiqueta X)}
        \PY{n}{I} \PY{o}{=} \PY{n}{init}\PY{p}{(}\PY{n}{X}\PY{p}{[}\PY{l+m+mi}{0}\PY{p}{]}\PY{p}{,}\PY{n}{estadoInicial}\PY{p}{)}
        \PY{n}{Tks} \PY{o}{=} \PY{p}{[} \PY{n}{trans}\PY{p}{(}\PY{n}{X}\PY{p}{[}\PY{n}{i}\PY{p}{]}\PY{p}{,}\PY{n}{X}\PY{p}{[}\PY{n}{i}\PY{o}{+}\PY{l+m+mi}{1}\PY{p}{]}\PY{p}{,}\PY{n}{S}\PY{p}{)} \PY{k}{for} \PY{n}{i} \PY{o+ow}{in} \PY{n+nb}{range}\PY{p}{(}\PY{n}{n}\PY{p}{)} \PY{p}{]}
        
        \PY{k}{if} \PY{n}{s}\PY{o}{.}\PY{n}{solve}\PY{p}{(}\PY{p}{[}\PY{n}{I}\PY{p}{,}\PY{n}{And}\PY{p}{(}\PY{n}{Tks}\PY{p}{)}\PY{p}{]}\PY{p}{)}\PY{p}{:}      \PY{c+c1}{\PYZsh{} testa se I /\PYZbs{} T\PYZca{}n  é satisfazível}
            \PY{k}{for} \PY{n}{i} \PY{o+ow}{in} \PY{n+nb}{range}\PY{p}{(}\PY{n}{n}\PY{p}{)}\PY{p}{:}
                \PY{n+nb}{print}\PY{p}{(}\PY{l+s+s2}{\PYZdq{}}\PY{l+s+s2}{Estado:}\PY{l+s+s2}{\PYZdq{}}\PY{p}{,}\PY{n}{i}\PY{p}{)}
                \PY{k}{for} \PY{n}{v} \PY{o+ow}{in} \PY{n}{X}\PY{p}{[}\PY{n}{i}\PY{p}{]}\PY{p}{:}
                    \PY{n+nb}{print}\PY{p}{(}\PY{l+s+s2}{\PYZdq{}}\PY{l+s+s2}{          }\PY{l+s+s2}{\PYZdq{}}\PY{p}{,}\PY{n}{v}\PY{p}{,}\PY{l+s+s1}{\PYZsq{}}\PY{l+s+s1}{=}\PY{l+s+s1}{\PYZsq{}}\PY{p}{,}\PY{n}{s}\PY{o}{.}\PY{n}{get\PYZus{}value}\PY{p}{(}\PY{n}{X}\PY{p}{[}\PY{n}{i}\PY{p}{]}\PY{p}{[}\PY{n}{v}\PY{p}{]}\PY{p}{)}\PY{p}{)}
        \PY{k}{else}\PY{p}{:}
            \PY{n+nb}{print}\PY{p}{(}\PY{l+s+s2}{\PYZdq{}}\PY{l+s+s2}{Não foi possível resolver}\PY{l+s+s2}{\PYZdq{}}\PY{p}{)}
\end{Verbatim}
\end{tcolorbox}

    Invariante de não ocorrer erro, isto é, a propriedade ``não ocorre
erro'' deve ser verdade em todos os estados da execução do programa. No
caso do problema, o caso contrário ao erro é a sua negação, ou seja, o
não do erro, simplificando, em vez de querermos um estado em que todos
os inversores não podem ser falso (\(0\)), basta verificar que um deles
é verdadeiro (\(1\)).

    \begin{tcolorbox}[breakable, size=fbox, boxrule=1pt, pad at break*=1mm,colback=cellbackground, colframe=cellborder]
\prompt{In}{incolor}{7}{\boxspacing}
\begin{Verbatim}[commandchars=\\\{\}]
\PY{k}{def} \PY{n+nf}{Noerror}\PY{p}{(}\PY{n}{state}\PY{p}{)}\PY{p}{:}
    \PY{k}{return} \PY{n}{Or}\PY{p}{(}\PY{n}{Iff}\PY{p}{(}\PY{n}{state}\PY{p}{[}\PY{l+s+s1}{\PYZsq{}}\PY{l+s+s1}{A}\PY{l+s+s1}{\PYZsq{}}\PY{p}{]}\PY{p}{,}\PY{n}{Bool}\PY{p}{(}\PY{k+kc}{True}\PY{p}{)}\PY{p}{)}\PY{p}{,}\PY{n}{Iff}\PY{p}{(}\PY{n}{state}\PY{p}{[}\PY{l+s+s1}{\PYZsq{}}\PY{l+s+s1}{B}\PY{l+s+s1}{\PYZsq{}}\PY{p}{]}\PY{p}{,}\PY{n}{Bool}\PY{p}{(}\PY{k+kc}{True}\PY{p}{)}\PY{p}{)}\PY{p}{,}\PY{n}{Iff}\PY{p}{(}\PY{n}{state}\PY{p}{[}\PY{l+s+s1}{\PYZsq{}}\PY{l+s+s1}{C}\PY{l+s+s1}{\PYZsq{}}\PY{p}{]}\PY{p}{,}\PY{n}{Bool}\PY{p}{(}\PY{k+kc}{True}\PY{p}{)}\PY{p}{)}\PY{p}{,}\PY{n}{Iff}\PY{p}{(}\PY{n}{state}\PY{p}{[}\PY{l+s+s1}{\PYZsq{}}\PY{l+s+s1}{D}\PY{l+s+s1}{\PYZsq{}}\PY{p}{]}\PY{p}{,}\PY{n}{Bool}\PY{p}{(}\PY{k+kc}{True}\PY{p}{)}\PY{p}{)}\PY{p}{)}
\end{Verbatim}
\end{tcolorbox}

    Função de ordem superior criada na ficha 7 que dada uma função que gera
uma cópia das variáveis do estado, um predicado que testa se um estado é
inicial, um predicado que testa se um par de estados é uma transição
válida, um invariante a verificar, uma lista que indica o estado incial
dos inversores, uma lista que indica as escolhas dos inversores e um
número de passos \(K\), testa se o invariante é sempre válido para os
primeiros \(K-1\) passos com o auxilio do \emph{SMT solver}. Caso não
seja verificável, devolve um contra-exemplo.

    \begin{tcolorbox}[breakable, size=fbox, boxrule=1pt, pad at break*=1mm,colback=cellbackground, colframe=cellborder]
\prompt{In}{incolor}{8}{\boxspacing}
\begin{Verbatim}[commandchars=\\\{\}]
\PY{k}{def} \PY{n+nf}{bmc\PYZus{}always}\PY{p}{(}\PY{n}{genState}\PY{p}{,}\PY{n}{init}\PY{p}{,}\PY{n}{trans}\PY{p}{,}\PY{n}{inv}\PY{p}{,}\PY{n}{K}\PY{p}{,}\PY{n}{estadoInicial}\PY{p}{,}\PY{n}{escolhas}\PY{p}{)}\PY{p}{:}
    \PY{k}{for} \PY{n}{k} \PY{o+ow}{in} \PY{n+nb}{range}\PY{p}{(}\PY{l+m+mi}{1}\PY{p}{,}\PY{n}{K}\PY{o}{+}\PY{l+m+mi}{1}\PY{p}{)}\PY{p}{:}
        \PY{k}{with} \PY{n}{Solver}\PY{p}{(}\PY{n}{name}\PY{o}{=}\PY{l+s+s2}{\PYZdq{}}\PY{l+s+s2}{z3}\PY{l+s+s2}{\PYZdq{}}\PY{p}{)} \PY{k}{as} \PY{n}{s}\PY{p}{:}

            \PY{n}{trace} \PY{o}{=} \PY{p}{[}\PY{n}{genState}\PY{p}{(}\PY{p}{[}\PY{l+s+s1}{\PYZsq{}}\PY{l+s+s1}{A}\PY{l+s+s1}{\PYZsq{}}\PY{p}{,}\PY{l+s+s1}{\PYZsq{}}\PY{l+s+s1}{B}\PY{l+s+s1}{\PYZsq{}}\PY{p}{,}\PY{l+s+s1}{\PYZsq{}}\PY{l+s+s1}{C}\PY{l+s+s1}{\PYZsq{}}\PY{p}{,}\PY{l+s+s1}{\PYZsq{}}\PY{l+s+s1}{D}\PY{l+s+s1}{\PYZsq{}}\PY{p}{]}\PY{p}{,}\PY{l+s+s1}{\PYZsq{}}\PY{l+s+s1}{X}\PY{l+s+s1}{\PYZsq{}}\PY{p}{,}\PY{n}{i}\PY{p}{)} \PY{k}{for} \PY{n}{i} \PY{o+ow}{in} \PY{n+nb}{range}\PY{p}{(}\PY{n}{k}\PY{p}{)}\PY{p}{]}
    
            \PY{n}{s}\PY{o}{.}\PY{n}{add\PYZus{}assertion}\PY{p}{(}\PY{n}{init}\PY{p}{(}\PY{n}{trace}\PY{p}{[}\PY{l+m+mi}{0}\PY{p}{]}\PY{p}{,}\PY{n}{estadoInicial}\PY{p}{)}\PY{p}{)}
            \PY{k}{for} \PY{n}{i} \PY{o+ow}{in} \PY{n+nb}{range}\PY{p}{(}\PY{n}{k}\PY{o}{\PYZhy{}}\PY{l+m+mi}{1}\PY{p}{)}\PY{p}{:}
                \PY{n}{s}\PY{o}{.}\PY{n}{add\PYZus{}assertion}\PY{p}{(}\PY{n}{trans}\PY{p}{(}\PY{n}{trace}\PY{p}{[}\PY{n}{i}\PY{p}{]}\PY{p}{,} \PY{n}{trace}\PY{p}{[}\PY{n}{i}\PY{o}{+}\PY{l+m+mi}{1}\PY{p}{]}\PY{p}{,}\PY{n}{escolhas}\PY{p}{)}\PY{p}{)}
            
            \PY{c+c1}{\PYZsh{} }
            \PY{n}{s}\PY{o}{.}\PY{n}{add\PYZus{}assertion}\PY{p}{(}\PY{n}{Not}\PY{p}{(}\PY{n}{inv}\PY{p}{(}\PY{n}{trace}\PY{p}{[}\PY{n}{k}\PY{o}{\PYZhy{}}\PY{l+m+mi}{1}\PY{p}{]}\PY{p}{)}\PY{p}{)}\PY{p}{)}
            \PY{k}{if} \PY{n}{s}\PY{o}{.}\PY{n}{solve}\PY{p}{(}\PY{p}{)}\PY{p}{:}
                \PY{n+nb}{print}\PY{p}{(}\PY{l+s+s1}{\PYZsq{}}\PY{l+s+s1}{A propriedade não é valida}\PY{l+s+s1}{\PYZsq{}}\PY{p}{)}
                \PY{k}{for} \PY{n}{i} \PY{o+ow}{in} \PY{n+nb}{range}\PY{p}{(}\PY{n}{k}\PY{p}{)}\PY{p}{:}
                    \PY{n+nb}{print}\PY{p}{(}\PY{n}{i}\PY{p}{)}
                    \PY{n+nb}{print}\PY{p}{(}\PY{l+s+s2}{\PYZdq{}}\PY{l+s+s2}{A=}\PY{l+s+s2}{\PYZdq{}}\PY{p}{,} \PY{n}{s}\PY{o}{.}\PY{n}{get\PYZus{}value}\PY{p}{(}\PY{n}{trace}\PY{p}{[}\PY{n}{i}\PY{p}{]}\PY{p}{[}\PY{l+s+s1}{\PYZsq{}}\PY{l+s+s1}{A}\PY{l+s+s1}{\PYZsq{}}\PY{p}{]}\PY{p}{)}\PY{p}{)}
                    \PY{n+nb}{print}\PY{p}{(}\PY{l+s+s2}{\PYZdq{}}\PY{l+s+s2}{B=}\PY{l+s+s2}{\PYZdq{}}\PY{p}{,} \PY{n}{s}\PY{o}{.}\PY{n}{get\PYZus{}value}\PY{p}{(}\PY{n}{trace}\PY{p}{[}\PY{n}{i}\PY{p}{]}\PY{p}{[}\PY{l+s+s1}{\PYZsq{}}\PY{l+s+s1}{B}\PY{l+s+s1}{\PYZsq{}}\PY{p}{]}\PY{p}{)}\PY{p}{)}
                    \PY{n+nb}{print}\PY{p}{(}\PY{l+s+s2}{\PYZdq{}}\PY{l+s+s2}{C=}\PY{l+s+s2}{\PYZdq{}}\PY{p}{,} \PY{n}{s}\PY{o}{.}\PY{n}{get\PYZus{}value}\PY{p}{(}\PY{n}{trace}\PY{p}{[}\PY{n}{i}\PY{p}{]}\PY{p}{[}\PY{l+s+s1}{\PYZsq{}}\PY{l+s+s1}{C}\PY{l+s+s1}{\PYZsq{}}\PY{p}{]}\PY{p}{)}\PY{p}{)}
                    \PY{n+nb}{print}\PY{p}{(}\PY{l+s+s2}{\PYZdq{}}\PY{l+s+s2}{D=}\PY{l+s+s2}{\PYZdq{}}\PY{p}{,} \PY{n}{s}\PY{o}{.}\PY{n}{get\PYZus{}value}\PY{p}{(}\PY{n}{trace}\PY{p}{[}\PY{n}{i}\PY{p}{]}\PY{p}{[}\PY{l+s+s1}{\PYZsq{}}\PY{l+s+s1}{D}\PY{l+s+s1}{\PYZsq{}}\PY{p}{]}\PY{p}{)}\PY{p}{)}
                \PY{k}{return}
            
    \PY{n+nb}{print}\PY{p}{(}\PY{l+s+s2}{\PYZdq{}}\PY{l+s+s2}{A propriedade é válida para traços de tamanho até }\PY{l+s+s2}{\PYZdq{}} \PY{o}{+} \PY{n+nb}{str}\PY{p}{(}\PY{n}{k}\PY{p}{)}\PY{p}{)}
\end{Verbatim}
\end{tcolorbox}

    Função kinduction\_always para verificação de invariantes por k-indução,
criada na ficha 8. Onde a ideia é generalizar a indução simples
assumindo no passo indutivo que o invariante é válido nos \(k\) estados
anteriores. Tendo que verificar se a propriedade é válida nos \(k\)
primeiros estados.

    \begin{tcolorbox}[breakable, size=fbox, boxrule=1pt, pad at break*=1mm,colback=cellbackground, colframe=cellborder]
\prompt{In}{incolor}{9}{\boxspacing}
\begin{Verbatim}[commandchars=\\\{\}]
\PY{k}{def} \PY{n+nf}{kinduction\PYZus{}always}\PY{p}{(}\PY{n}{declare}\PY{p}{,}\PY{n}{init}\PY{p}{,}\PY{n}{trans}\PY{p}{,}\PY{n}{inv}\PY{p}{,}\PY{n}{k}\PY{p}{,}\PY{n}{estadoInicial}\PY{p}{,}\PY{n}{escolhas}\PY{p}{)}\PY{p}{:}
    \PY{k}{with} \PY{n}{Solver}\PY{p}{(}\PY{n}{name}\PY{o}{=}\PY{l+s+s2}{\PYZdq{}}\PY{l+s+s2}{z3}\PY{l+s+s2}{\PYZdq{}}\PY{p}{)} \PY{k}{as} \PY{n}{s}\PY{p}{:}
       \PY{c+c1}{\PYZsh{} criar k+1 cópias do estado}
    
        \PY{n}{traco} \PY{o}{=} \PY{p}{[}\PY{n}{genState}\PY{p}{(}\PY{p}{[}\PY{l+s+s1}{\PYZsq{}}\PY{l+s+s1}{A}\PY{l+s+s1}{\PYZsq{}}\PY{p}{,}\PY{l+s+s1}{\PYZsq{}}\PY{l+s+s1}{B}\PY{l+s+s1}{\PYZsq{}}\PY{p}{,}\PY{l+s+s1}{\PYZsq{}}\PY{l+s+s1}{C}\PY{l+s+s1}{\PYZsq{}}\PY{p}{,}\PY{l+s+s1}{\PYZsq{}}\PY{l+s+s1}{D}\PY{l+s+s1}{\PYZsq{}}\PY{p}{]}\PY{p}{,}\PY{l+s+s1}{\PYZsq{}}\PY{l+s+s1}{X}\PY{l+s+s1}{\PYZsq{}}\PY{p}{,}\PY{n}{i}\PY{p}{)} \PY{k}{for} \PY{n}{i} \PY{o+ow}{in} \PY{n+nb}{range}\PY{p}{(}\PY{n}{k}\PY{o}{+}\PY{l+m+mi}{1}\PY{p}{)}\PY{p}{]}
        
       \PY{c+c1}{\PYZsh{}testar os k passos iniciais}

        \PY{n}{s}\PY{o}{.}\PY{n}{push}\PY{p}{(}\PY{p}{)}
        \PY{n}{s}\PY{o}{.}\PY{n}{add\PYZus{}assertion}\PY{p}{(}\PY{n}{init}\PY{p}{(}\PY{n}{traco}\PY{p}{[}\PY{l+m+mi}{0}\PY{p}{]}\PY{p}{,}\PY{n}{estadoInicial}\PY{p}{)}\PY{p}{)}
        
        \PY{k}{for} \PY{n}{i} \PY{o+ow}{in} \PY{n+nb}{range}\PY{p}{(}\PY{n}{k}\PY{o}{\PYZhy{}}\PY{l+m+mi}{1}\PY{p}{)}\PY{p}{:}
            \PY{n}{s}\PY{o}{.}\PY{n}{add\PYZus{}assertion}\PY{p}{(}\PY{n}{trans}\PY{p}{(}\PY{n}{traco}\PY{p}{[}\PY{n}{i}\PY{p}{]}\PY{p}{,}\PY{n}{traco}\PY{p}{[}\PY{n}{i}\PY{o}{+}\PY{l+m+mi}{1}\PY{p}{]}\PY{p}{,}\PY{n}{escolhas}\PY{p}{)}\PY{p}{)}
        
        \PY{n}{s}\PY{o}{.}\PY{n}{add\PYZus{}assertion}\PY{p}{(}\PY{n}{Or}\PY{p}{(}\PY{p}{[}\PY{n}{Not}\PY{p}{(}\PY{n}{inv}\PY{p}{(}\PY{n}{traco}\PY{p}{[}\PY{n}{i}\PY{p}{]}\PY{p}{)}\PY{p}{)} \PY{k}{for} \PY{n}{i} \PY{o+ow}{in} \PY{n+nb}{range}\PY{p}{(}\PY{n}{k}\PY{p}{)} \PY{p}{]}\PY{p}{)}\PY{p}{)}
        
        \PY{k}{if} \PY{n}{s}\PY{o}{.}\PY{n}{solve}\PY{p}{(}\PY{p}{)}\PY{p}{:}
            \PY{n+nb}{print}\PY{p}{(}\PY{l+s+s1}{\PYZsq{}}\PY{l+s+s1}{A propriedade não é válida nos k estados iniciais.}\PY{l+s+s1}{\PYZsq{}}\PY{p}{)}
            \PY{c+c1}{\PYZsh{}for v in traco[0]:}
                \PY{c+c1}{\PYZsh{}print(v,\PYZsq{}=\PYZsq{},s.get\PYZus{}value(traco[0][v]))}
            \PY{k}{for} \PY{n}{i} \PY{o+ow}{in} \PY{n+nb}{range}\PY{p}{(}\PY{n}{k}\PY{p}{)}\PY{p}{:}
                    \PY{n+nb}{print}\PY{p}{(}\PY{n}{i}\PY{p}{)}
                    \PY{n+nb}{print}\PY{p}{(}\PY{l+s+s2}{\PYZdq{}}\PY{l+s+s2}{A=}\PY{l+s+s2}{\PYZdq{}}\PY{p}{,} \PY{n}{s}\PY{o}{.}\PY{n}{get\PYZus{}value}\PY{p}{(}\PY{n}{traco}\PY{p}{[}\PY{n}{i}\PY{p}{]}\PY{p}{[}\PY{l+s+s1}{\PYZsq{}}\PY{l+s+s1}{A}\PY{l+s+s1}{\PYZsq{}}\PY{p}{]}\PY{p}{)}\PY{p}{)}
                    \PY{n+nb}{print}\PY{p}{(}\PY{l+s+s2}{\PYZdq{}}\PY{l+s+s2}{B=}\PY{l+s+s2}{\PYZdq{}}\PY{p}{,} \PY{n}{s}\PY{o}{.}\PY{n}{get\PYZus{}value}\PY{p}{(}\PY{n}{traco}\PY{p}{[}\PY{n}{i}\PY{p}{]}\PY{p}{[}\PY{l+s+s1}{\PYZsq{}}\PY{l+s+s1}{B}\PY{l+s+s1}{\PYZsq{}}\PY{p}{]}\PY{p}{)}\PY{p}{)}
                    \PY{n+nb}{print}\PY{p}{(}\PY{l+s+s2}{\PYZdq{}}\PY{l+s+s2}{C=}\PY{l+s+s2}{\PYZdq{}}\PY{p}{,} \PY{n}{s}\PY{o}{.}\PY{n}{get\PYZus{}value}\PY{p}{(}\PY{n}{traco}\PY{p}{[}\PY{n}{i}\PY{p}{]}\PY{p}{[}\PY{l+s+s1}{\PYZsq{}}\PY{l+s+s1}{C}\PY{l+s+s1}{\PYZsq{}}\PY{p}{]}\PY{p}{)}\PY{p}{)}
                    \PY{n+nb}{print}\PY{p}{(}\PY{l+s+s2}{\PYZdq{}}\PY{l+s+s2}{D=}\PY{l+s+s2}{\PYZdq{}}\PY{p}{,} \PY{n}{s}\PY{o}{.}\PY{n}{get\PYZus{}value}\PY{p}{(}\PY{n}{traco}\PY{p}{[}\PY{n}{i}\PY{p}{]}\PY{p}{[}\PY{l+s+s1}{\PYZsq{}}\PY{l+s+s1}{D}\PY{l+s+s1}{\PYZsq{}}\PY{p}{]}\PY{p}{)}\PY{p}{)}
            \PY{k}{return}
        
        \PY{n}{s}\PY{o}{.}\PY{n}{pop}\PY{p}{(}\PY{p}{)}
        
        \PY{c+c1}{\PYZsh{} testar o passo k\PYZhy{}indutivo}
        \PY{n}{s}\PY{o}{.}\PY{n}{push}\PY{p}{(}\PY{p}{)}
        
        \PY{k}{for} \PY{n}{i} \PY{o+ow}{in} \PY{n+nb}{range}\PY{p}{(}\PY{n}{k}\PY{p}{)}\PY{p}{:}
            \PY{n}{s}\PY{o}{.}\PY{n}{add\PYZus{}assertion}\PY{p}{(}\PY{n}{trans}\PY{p}{(}\PY{n}{traco}\PY{p}{[}\PY{n}{i}\PY{p}{]}\PY{p}{,}\PY{n}{traco}\PY{p}{[}\PY{n}{i}\PY{o}{+}\PY{l+m+mi}{1}\PY{p}{]}\PY{p}{,}\PY{n}{escolhas}\PY{p}{)}\PY{p}{)}
            \PY{n}{s}\PY{o}{.}\PY{n}{add\PYZus{}assertion}\PY{p}{(}\PY{n}{inv}\PY{p}{(}\PY{n}{traco}\PY{p}{[}\PY{n}{i}\PY{p}{]}\PY{p}{)}\PY{p}{)}
        \PY{n}{s}\PY{o}{.}\PY{n}{add\PYZus{}assertion}\PY{p}{(}\PY{n}{Not}\PY{p}{(}\PY{n}{inv}\PY{p}{(}\PY{n}{traco}\PY{p}{[}\PY{n}{k}\PY{p}{]}\PY{p}{)}\PY{p}{)}\PY{p}{)}
        
        \PY{k}{if} \PY{n}{s}\PY{o}{.}\PY{n}{solve}\PY{p}{(}\PY{p}{)}\PY{p}{:}
            \PY{n+nb}{print}\PY{p}{(}\PY{l+s+s1}{\PYZsq{}}\PY{l+s+s1}{O passo }\PY{l+s+si}{\PYZpc{}s}\PY{l+s+s1}{ indutivo nao preserva a propriedade.}\PY{l+s+s1}{\PYZsq{}} \PY{o}{\PYZpc{}} \PY{n}{k}\PY{p}{)}
            \PY{n+nb}{print}\PY{p}{(}\PY{l+s+s2}{\PYZdq{}}\PY{l+s+s2}{O ponto onde falha começa aqui:}\PY{l+s+s2}{\PYZdq{}}\PY{p}{)}
            \PY{k}{for} \PY{n}{v} \PY{o+ow}{in} \PY{n}{traco}\PY{p}{[}\PY{l+m+mi}{0}\PY{p}{]}\PY{p}{:}
                \PY{n+nb}{print}\PY{p}{(}\PY{n}{v}\PY{p}{,}\PY{l+s+s1}{\PYZsq{}}\PY{l+s+s1}{=}\PY{l+s+s1}{\PYZsq{}}\PY{p}{,}\PY{n}{s}\PY{o}{.}\PY{n}{get\PYZus{}value}\PY{p}{(}\PY{n}{traco}\PY{p}{[}\PY{l+m+mi}{0}\PY{p}{]}\PY{p}{[}\PY{n}{v}\PY{p}{]}\PY{p}{)}\PY{p}{)}
            \PY{k}{return}
        
        \PY{n+nb}{print}\PY{p}{(}\PY{l+s+s2}{\PYZdq{}}\PY{l+s+s2}{A propriedade é sempre válida}\PY{l+s+s2}{\PYZdq{}}\PY{p}{)}
        \PY{n}{s}\PY{o}{.}\PY{n}{pop}\PY{p}{(}\PY{p}{)}
\end{Verbatim}
\end{tcolorbox}

    Funções \(baseName\), \(rename\), \(same\) e \(invert\) definidas na
ficha9, a primeira é fornece o nome da variável, a segunda com o aulixio
da primeira renomeia uma fórmula de acordo com um dado estado, a
penúltima verifica se dois estados são iguais e a última é uma função de
ordem superior que recebe a função de transição e devolve a sua inversa.

    \begin{tcolorbox}[breakable, size=fbox, boxrule=1pt, pad at break*=1mm,colback=cellbackground, colframe=cellborder]
\prompt{In}{incolor}{10}{\boxspacing}
\begin{Verbatim}[commandchars=\\\{\}]
\PY{k}{def} \PY{n+nf}{baseName}\PY{p}{(}\PY{n}{s}\PY{p}{)}\PY{p}{:}
    \PY{k}{return} \PY{l+s+s1}{\PYZsq{}}\PY{l+s+s1}{\PYZsq{}}\PY{o}{.}\PY{n}{join}\PY{p}{(}\PY{n+nb}{list}\PY{p}{(}\PY{n}{itertools}\PY{o}{.}\PY{n}{takewhile}\PY{p}{(}\PY{k}{lambda} \PY{n}{x}\PY{p}{:} \PY{n}{x}\PY{o}{!=}\PY{l+s+s1}{\PYZsq{}}\PY{l+s+s1}{!}\PY{l+s+s1}{\PYZsq{}}\PY{p}{,} \PY{n}{s}\PY{p}{)}\PY{p}{)}\PY{p}{)}

\PY{k}{def} \PY{n+nf}{rename}\PY{p}{(}\PY{n}{form}\PY{p}{,}\PY{n}{state}\PY{p}{)}\PY{p}{:}
    \PY{n}{vs} \PY{o}{=} \PY{n}{get\PYZus{}free\PYZus{}variables}\PY{p}{(}\PY{n}{form}\PY{p}{)}
    \PY{n}{pairs} \PY{o}{=} \PY{p}{[} \PY{p}{(}\PY{n}{x}\PY{p}{,}\PY{n}{state}\PY{p}{[}\PY{n}{baseName}\PY{p}{(}\PY{n}{x}\PY{o}{.}\PY{n}{symbol\PYZus{}name}\PY{p}{(}\PY{p}{)}\PY{p}{)}\PY{p}{]}\PY{p}{)} \PY{k}{for} \PY{n}{x} \PY{o+ow}{in} \PY{n}{vs} \PY{p}{]} \PY{c+c1}{\PYZsh{} Descobrir os pares \PYZsh{} symbol\PYZus{}name dá o nome aka string da var}
    \PY{k}{return} \PY{n}{form}\PY{o}{.}\PY{n}{substitute}\PY{p}{(}\PY{n+nb}{dict}\PY{p}{(}\PY{n}{pairs}\PY{p}{)}\PY{p}{)} \PY{c+c1}{\PYZsh{} recebe um dicionário e substitui as chaves do dicionário pelo o que está no value}

\PY{k}{def} \PY{n+nf}{same}\PY{p}{(}\PY{n}{state1}\PY{p}{,}\PY{n}{state2}\PY{p}{)}\PY{p}{:} \PY{c+c1}{\PYZsh{} ver se as duas vars têm o mesmo valor}
    \PY{k}{return} \PY{n}{And}\PY{p}{(}\PY{p}{[}\PY{n}{Iff}\PY{p}{(}\PY{n}{state1}\PY{p}{[}\PY{n}{x}\PY{p}{]}\PY{p}{,}\PY{n}{state2}\PY{p}{[}\PY{n}{x}\PY{p}{]}\PY{p}{)} \PY{k}{for} \PY{n}{x} \PY{o+ow}{in} \PY{n}{state1}\PY{p}{]}\PY{p}{)}

\PY{k}{def} \PY{n+nf}{invert}\PY{p}{(}\PY{n}{trans}\PY{p}{,}\PY{n}{S}\PY{p}{)}\PY{p}{:}
    \PY{k}{return} \PY{p}{(}\PY{k}{lambda} \PY{n}{u}\PY{p}{,} \PY{n}{v}\PY{p}{:} \PY{n}{trans}\PY{p}{(}\PY{n}{v}\PY{p}{,}\PY{n}{u}\PY{p}{,}\PY{n}{S}\PY{p}{)}\PY{p}{)}
\end{Verbatim}
\end{tcolorbox}

    Função \(model-checking\) de ordem superior da ficha 9 que dada a lista
de nomes das variáveis do sistema \(vars\) , um predicado que testa se
um estado é inicial \(init\), um predicado que testa se um par de
estados é uma transição válida \(trans\) , um predicado que testa se um
estado é de erro \(error\), dois números positivos \(N\) e \(M\) que são
os limites máximos para os indices n e m, uma lista de \emph{bools} com
que indica os valores dos \emph{bits} no estado inicial e outra lista
com elementos do mesmo tipo que indica cada a escolha de cada um dos 4
inversores. O seu objetivo é implementar o algoritmo de
``model-checking'' com o auxílio do \emph{SMT} \emph{solver} e o
algoritmo para encontrar o majorante S descrito na ficha.

    \begin{tcolorbox}[breakable, size=fbox, boxrule=1pt, pad at break*=1mm,colback=cellbackground, colframe=cellborder]
\prompt{In}{incolor}{11}{\boxspacing}
\begin{Verbatim}[commandchars=\\\{\}]
\PY{k}{def} \PY{n+nf}{model\PYZus{}checking}\PY{p}{(}\PY{n+nb}{vars}\PY{p}{,}\PY{n}{init}\PY{p}{,}\PY{n}{trans}\PY{p}{,}\PY{n}{error}\PY{p}{,}\PY{n}{N}\PY{p}{,}\PY{n}{M}\PY{p}{,}\PY{n}{estadoInicial}\PY{p}{,}\PY{n}{escolhas}\PY{p}{)}\PY{p}{:}
    \PY{k}{with} \PY{n}{Solver}\PY{p}{(}\PY{n}{name}\PY{o}{=}\PY{l+s+s2}{\PYZdq{}}\PY{l+s+s2}{z3}\PY{l+s+s2}{\PYZdq{}}\PY{p}{)} \PY{k}{as} \PY{n}{s}\PY{p}{:}
        
        \PY{c+c1}{\PYZsh{} Criar todos os estados que poderão vir a ser necessários.}
        \PY{n}{X} \PY{o}{=} \PY{p}{[}\PY{n}{genState}\PY{p}{(}\PY{n+nb}{vars}\PY{p}{,}\PY{l+s+s1}{\PYZsq{}}\PY{l+s+s1}{X}\PY{l+s+s1}{\PYZsq{}}\PY{p}{,}\PY{n}{i}\PY{p}{)} \PY{k}{for} \PY{n}{i} \PY{o+ow}{in} \PY{n+nb}{range}\PY{p}{(}\PY{n}{N}\PY{o}{+}\PY{l+m+mi}{1}\PY{p}{)}\PY{p}{]} \PY{c+c1}{\PYZsh{} Com a função genState, criar todos os estados que possam ser necessário, TODOS. \PYZsh{} X SFOTS original}
        \PY{n}{Y} \PY{o}{=} \PY{p}{[}\PY{n}{genState}\PY{p}{(}\PY{n+nb}{vars}\PY{p}{,}\PY{l+s+s1}{\PYZsq{}}\PY{l+s+s1}{Y}\PY{l+s+s1}{\PYZsq{}}\PY{p}{,}\PY{n}{i}\PY{p}{)} \PY{k}{for} \PY{n}{i} \PY{o+ow}{in} \PY{n+nb}{range}\PY{p}{(}\PY{n}{M}\PY{o}{+}\PY{l+m+mi}{1}\PY{p}{)}\PY{p}{]} \PY{c+c1}{\PYZsh{} Y SFOTS invertido}

        \PY{c+c1}{\PYZsh{} Estabelecer a ordem pela qual os pares (n,m) vão surgir. Por exemplo:}
        \PY{n}{order} \PY{o}{=} \PY{n+nb}{sorted}\PY{p}{(}\PY{p}{[}\PY{p}{(}\PY{n}{a}\PY{p}{,}\PY{n}{b}\PY{p}{)} \PY{k}{for} \PY{n}{a} \PY{o+ow}{in} \PY{n+nb}{range}\PY{p}{(}\PY{l+m+mi}{1}\PY{p}{,}\PY{n}{N}\PY{o}{+}\PY{l+m+mi}{1}\PY{p}{)} \PY{k}{for} \PY{n}{b} \PY{o+ow}{in} \PY{n+nb}{range}\PY{p}{(}\PY{l+m+mi}{1}\PY{p}{,}\PY{n}{M}\PY{o}{+}\PY{l+m+mi}{1}\PY{p}{)}\PY{p}{]}\PY{p}{,}\PY{n}{key}\PY{o}{=}\PY{k}{lambda} \PY{n}{tup}\PY{p}{:}\PY{n}{tup}\PY{p}{[}\PY{l+m+mi}{0}\PY{p}{]}\PY{o}{+}\PY{n}{tup}\PY{p}{[}\PY{l+m+mi}{1}\PY{p}{]}\PY{p}{)} \PY{c+c1}{\PYZsh{} Estabelecer ordem que criamos o n e o m \PYZsh{} ideia da stora: usar o sorted,}
                                                                                                         \PY{c+c1}{\PYZsh{} gerar todos os pares possíveis }
                                                                                                         \PY{c+c1}{\PYZsh{} e ter como critério de ordenação as soma dos elementos dos pares}
        
        \PY{k}{for} \PY{p}{(}\PY{n}{n}\PY{p}{,}\PY{n}{m}\PY{p}{)} \PY{o+ow}{in} \PY{n}{order}\PY{p}{:} \PY{c+c1}{\PYZsh{} Seguir o algoritmo}
            \PY{c+c1}{\PYZsh{} completar}
            \PY{n}{I} \PY{o}{=} \PY{n}{init}\PY{p}{(}\PY{n}{X}\PY{p}{[}\PY{l+m+mi}{0}\PY{p}{]}\PY{p}{,}\PY{n}{estadoInicial}\PY{p}{)} \PY{c+c1}{\PYZsh{} o X é uma lista de estados}
            \PY{n}{Tn} \PY{o}{=} \PY{n}{And}\PY{p}{(}\PY{p}{[}\PY{n}{trans}\PY{p}{(}\PY{n}{X}\PY{p}{[}\PY{n}{i}\PY{p}{]}\PY{p}{,}\PY{n}{X}\PY{p}{[}\PY{n}{i}\PY{o}{+}\PY{l+m+mi}{1}\PY{p}{]}\PY{p}{,}\PY{n}{escolhas}\PY{p}{)} \PY{k}{for} \PY{n}{i} \PY{o+ow}{in} \PY{n+nb}{range}\PY{p}{(}\PY{n}{n}\PY{p}{)}\PY{p}{]}\PY{p}{)}
            \PY{n}{Rn} \PY{o}{=} \PY{n}{And}\PY{p}{(}\PY{n}{I}\PY{p}{,}\PY{n}{Tn}\PY{p}{)} \PY{c+c1}{\PYZsh{} estados acessíveis em n transições}
            
            \PY{n}{E} \PY{o}{=} \PY{n}{error}\PY{p}{(}\PY{n}{Y}\PY{p}{[}\PY{l+m+mi}{0}\PY{p}{]}\PY{p}{)}
            \PY{n}{Bm} \PY{o}{=} \PY{n}{And}\PY{p}{(}\PY{p}{[}\PY{n}{invert}\PY{p}{(}\PY{n}{trans}\PY{p}{,}\PY{n}{escolhas}\PY{p}{)}\PY{p}{(}\PY{n}{Y}\PY{p}{[}\PY{n}{i}\PY{p}{]}\PY{p}{,}\PY{n}{Y}\PY{p}{[}\PY{n}{i}\PY{o}{+}\PY{l+m+mi}{1}\PY{p}{]}\PY{p}{)} \PY{k}{for} \PY{n}{i} \PY{o+ow}{in} \PY{n+nb}{range}\PY{p}{(}\PY{n}{m}\PY{p}{)}\PY{p}{]}\PY{p}{)}
            \PY{n}{Um} \PY{o}{=} \PY{n}{And}\PY{p}{(}\PY{n}{E}\PY{p}{,}\PY{n}{Bm}\PY{p}{)} \PY{c+c1}{\PYZsh{} estados inseguros em m transições}
            
            \PY{n}{Vnm} \PY{o}{=} \PY{n}{And}\PY{p}{(}\PY{n}{Rn}\PY{p}{,}\PY{n}{same}\PY{p}{(}\PY{n}{X}\PY{p}{[}\PY{n}{n}\PY{p}{]}\PY{p}{,}\PY{n}{Y}\PY{p}{[}\PY{n}{m}\PY{p}{]}\PY{p}{)}\PY{p}{,}\PY{n}{Um}\PY{p}{)} \PY{c+c1}{\PYZsh{} temos de testar se dois estados estão iguais e, portanto, usamos a função same dada acima}
            
            \PY{k}{if} \PY{n}{s}\PY{o}{.}\PY{n}{solve}\PY{p}{(}\PY{p}{[}\PY{n}{Vnm}\PY{p}{]}\PY{p}{)}\PY{p}{:}
                \PY{n+nb}{print}\PY{p}{(}\PY{l+s+s2}{\PYZdq{}}\PY{l+s+s2}{unsafe}\PY{l+s+s2}{\PYZdq{}}\PY{p}{)}
                \PY{k}{return} 
           
            \PY{c+c1}{\PYZsh{} Se for insatisfazível, temos de criar o interpolante para n fórmulas}
            \PY{n}{C} \PY{o}{=} \PY{n}{binary\PYZus{}interpolant}\PY{p}{(}\PY{n}{And}\PY{p}{(}\PY{n}{Rn}\PY{p}{,}\PY{n}{same}\PY{p}{(}\PY{n}{X}\PY{p}{[}\PY{n}{n}\PY{p}{]}\PY{p}{,}\PY{n}{Y}\PY{p}{[}\PY{n}{m}\PY{p}{]}\PY{p}{)}\PY{p}{)}\PY{p}{,} \PY{n}{Um}\PY{p}{)}
            \PY{k}{if} \PY{n}{C} \PY{o+ow}{is} \PY{k+kc}{None}\PY{p}{:}
                \PY{n+nb}{print}\PY{p}{(}\PY{l+s+s2}{\PYZdq{}}\PY{l+s+s2}{Interpolante None}\PY{l+s+s2}{\PYZdq{}}\PY{p}{)}
                \PY{k}{continue}
            
            \PY{n}{C0} \PY{o}{=} \PY{n}{rename}\PY{p}{(}\PY{n}{C}\PY{p}{,}\PY{n}{X}\PY{p}{[}\PY{l+m+mi}{0}\PY{p}{]}\PY{p}{)} \PY{c+c1}{\PYZsh{} Rename do C com o estado envolvido, neste caso o X[0] }
            \PY{n}{C1} \PY{o}{=} \PY{n}{rename}\PY{p}{(}\PY{n}{C}\PY{p}{,}\PY{n}{X}\PY{p}{[}\PY{l+m+mi}{1}\PY{p}{]}\PY{p}{)}
            \PY{c+c1}{\PYZsh{} Trabalhamos com X[0] e X[1] porque T pode ser escrito como T = (X0,X1)}
            
            \PY{n}{T} \PY{o}{=} \PY{n}{trans}\PY{p}{(}\PY{n}{X}\PY{p}{[}\PY{l+m+mi}{0}\PY{p}{]}\PY{p}{,}\PY{n}{X}\PY{p}{[}\PY{l+m+mi}{1}\PY{p}{]}\PY{p}{,}\PY{n}{escolhas}\PY{p}{)}
            
            \PY{k}{if} \PY{o+ow}{not} \PY{n}{s}\PY{o}{.}\PY{n}{solve}\PY{p}{(}\PY{p}{[}\PY{n}{C0}\PY{p}{,}\PY{n}{T}\PY{p}{,}\PY{n}{Not}\PY{p}{(}\PY{n}{C1}\PY{p}{)}\PY{p}{]}\PY{p}{)}\PY{p}{:}
                \PY{n+nb}{print}\PY{p}{(}\PY{l+s+s2}{\PYZdq{}}\PY{l+s+s2}{Safe}\PY{l+s+s2}{\PYZdq{}}\PY{p}{)}
                \PY{k}{return}
            \PY{k}{else}\PY{p}{:}
                    \PY{c+c1}{\PYZsh{}\PYZsh{}\PYZsh{}\PYZsh{} gerar o S (Parte que descreve o Majorante S)}
                
                \PY{c+c1}{\PYZsh{}Passo 1:}
                \PY{n}{S} \PY{o}{=} \PY{n}{rename}\PY{p}{(}\PY{n}{C}\PY{p}{,}\PY{n}{X}\PY{p}{[}\PY{n}{n}\PY{p}{]}\PY{p}{)}
                \PY{k}{while} \PY{k+kc}{True}\PY{p}{:}
                    \PY{c+c1}{\PYZsh{}Passo 2:}
                    \PY{n}{A} \PY{o}{=} \PY{n}{And}\PY{p}{(}\PY{n}{S}\PY{p}{,}\PY{n}{trans}\PY{p}{(}\PY{n}{X}\PY{p}{[}\PY{n}{n}\PY{p}{]}\PY{p}{,}\PY{n}{Y}\PY{p}{[}\PY{n}{m}\PY{p}{]}\PY{p}{,}\PY{n}{escolhas}\PY{p}{)}\PY{p}{)}
                    \PY{k}{if} \PY{n}{s}\PY{o}{.}\PY{n}{solve}\PY{p}{(}\PY{p}{[}\PY{n}{A}\PY{p}{,}\PY{n}{Um}\PY{p}{]}\PY{p}{)}\PY{p}{:}
                        \PY{n+nb}{print}\PY{p}{(}\PY{l+s+s2}{\PYZdq{}}\PY{l+s+s2}{Não foi possível encontrar o majorante.}\PY{l+s+s2}{\PYZdq{}}\PY{p}{)}
                        \PY{k}{break}
                    \PY{k}{else}\PY{p}{:}
                        \PY{n}{CNew} \PY{o}{=} \PY{n}{binary\PYZus{}interpolant}\PY{p}{(}\PY{n}{A}\PY{p}{,}\PY{n}{Um}\PY{p}{)} \PY{c+c1}{\PYZsh{} as duas formulas têm de ser inconsistentes para que faça sentido para usar binary\PYZus{}interpolant}
                        \PY{n}{Cn} \PY{o}{=} \PY{n}{rename}\PY{p}{(}\PY{n}{CNew}\PY{p}{,}\PY{n}{X}\PY{p}{[}\PY{n}{n}\PY{p}{]}\PY{p}{)}
                        
                        \PY{k}{if} \PY{n}{s}\PY{o}{.}\PY{n}{solve}\PY{p}{(}\PY{p}{[}\PY{n}{Cn}\PY{p}{,}\PY{n}{Not}\PY{p}{(}\PY{n}{S}\PY{p}{)}\PY{p}{]}\PY{p}{)}\PY{p}{:}
                            \PY{n}{S} \PY{o}{=} \PY{n}{Or}\PY{p}{(}\PY{n}{S}\PY{p}{,}\PY{n}{Cn}\PY{p}{)}
                        \PY{k}{else}\PY{p}{:}
                            \PY{n+nb}{print}\PY{p}{(}\PY{l+s+s2}{\PYZdq{}}\PY{l+s+s2}{Safe}\PY{l+s+s2}{\PYZdq{}}\PY{p}{)}
                            \PY{k}{return}
            
        \PY{n+nb}{print}\PY{p}{(}\PY{l+s+s2}{\PYZdq{}}\PY{l+s+s2}{unknown}\PY{l+s+s2}{\PYZdq{}}\PY{p}{)}  
\end{Verbatim}
\end{tcolorbox}

    \hypertarget{conclusuxf5es-acerca-das-3-funuxe7uxf5es-usadas-para-a-veficauxe7uxe3o-da-seguranuxe7a-do-programa}{%
\paragraph{Conclusões acerca das 3 funções usadas para a veficação da
segurança do
programa}\label{conclusuxf5es-acerca-das-3-funuxe7uxf5es-usadas-para-a-veficauxe7uxe3o-da-seguranuxe7a-do-programa}}

Notamos que existe um número finito de estados possíveis neste problema
uma vez que existem 4 inversores onde os seus valores podem tomar um
valor binário (\(0 \lor 1\)), e por isso, temos no maximo \(2^4 = 16\)
estados possíveis. Assim, para a função \textbf{bmc\_always} basta
verificar para K = 16 temos a garantia de que todos os estados foram
testados. Para a função \textbf{kinduction\_always} para obter um
comportamento semelhante ao do \textbf{bmc\_always}, embora sendo muitas
das vezes um valor maior do que o necessário, se fizermos para K=16
temos sempre a garantia de que a função está a abranger todos os casos.
Apesar disso, nos testes mais abaixo nem sempre iremos fazer para K=16.
A função \textbf{model\_checking} apresentou um comportamento diferente
das outras duas, a função consegue verificar que é unsafe para exemplos
em que o estado de erro acontece \((0,0,0,0)\), porém quando o contrário
ocorre, não é capaz de provar que é safe. Após alguns testes,
descobrimos que a função não consegue detetar loops com periodo maior
que \(1\), visto que, no problema anterior (TP3-Exercício1) notamos que
a função consegue imprimir para o output a string ``safe'' nos casos
corretos e, da maneira que este estava modelado, após atingir o estado
stop, a transição não altera o estado. Para verificar isso com maior
rigor, num ficheiro à parte trocamos a condição de erro para
\((1,1,1,1)\) e testamos exemplos em que há loops de periodo 1, como o
caso de o estado inicial ser \((0,0,0,0)\) e xor como escolhas de todos
os inversores, após correr a função \textbf{model\_checking}, a mesma
rapidamente deu como resultado ``safe''. Portanto, supomos que no
\textbf{model\_checking} também se aplique o facto de termos números
limitados de estados, e portanto, para N = 16 e M = 16 ou mais podemos
suspeitamos que o resultado ``unknown'' possa ser traduzido para
``safe''.

    \hypertarget{exemplos-e-testes-de-aplicauxe7uxe3o}{%
\subsubsection{Exemplos e Testes de
Aplicação}\label{exemplos-e-testes-de-aplicauxe7uxe3o}}

    \hypertarget{exemplo-1}{%
\paragraph{Exemplo 1}\label{exemplo-1}}

    Neste exemplo testamos a k-indução para um k = 4.

    \begin{tcolorbox}[breakable, size=fbox, boxrule=1pt, pad at break*=1mm,colback=cellbackground, colframe=cellborder]
\prompt{In}{incolor}{19}{\boxspacing}
\begin{Verbatim}[commandchars=\\\{\}]
\PY{n}{estadoInicial} \PY{o}{=} \PY{p}{[}\PY{n}{rn}\PY{o}{.}\PY{n}{choice}\PY{p}{(}\PY{p}{[}\PY{k+kc}{True}\PY{p}{,}\PY{k+kc}{False}\PY{p}{]}\PY{p}{)} \PY{k}{for} \PY{n}{i} \PY{o+ow}{in} \PY{n+nb}{range}\PY{p}{(}\PY{l+m+mi}{4}\PY{p}{)}\PY{p}{]}
\PY{n+nb}{print}\PY{p}{(}\PY{l+s+sa}{f}\PY{l+s+s2}{\PYZdq{}}\PY{l+s+s2}{Estado inicial: }\PY{l+s+si}{\PYZob{}}\PY{n+nb}{tuple}\PY{p}{(}\PY{n}{estadoInicial}\PY{p}{)}\PY{l+s+si}{\PYZcb{}}\PY{l+s+s2}{\PYZdq{}}\PY{p}{)}
\PY{n}{escolhas} \PY{o}{=} \PY{p}{[}\PY{n}{rn}\PY{o}{.}\PY{n}{choice}\PY{p}{(}\PY{p}{[}\PY{k+kc}{True}\PY{p}{,}\PY{k+kc}{False}\PY{p}{]}\PY{p}{)} \PY{k}{for} \PY{n}{i} \PY{o+ow}{in} \PY{n+nb}{range}\PY{p}{(}\PY{l+m+mi}{4}\PY{p}{)}\PY{p}{]}
\PY{n+nb}{print}\PY{p}{(}\PY{l+s+sa}{f}\PY{l+s+s2}{\PYZdq{}}\PY{l+s+s2}{Escolhas: }\PY{l+s+si}{\PYZob{}}\PY{n+nb}{tuple}\PY{p}{(}\PY{n}{escolhas}\PY{p}{)}\PY{l+s+si}{\PYZcb{}}\PY{l+s+s2}{\PYZdq{}}\PY{p}{)}
\PY{n+nb}{print}\PY{p}{(}\PY{p}{)}
\PY{n}{n\PYZus{}trasicoes} \PY{o}{=} \PY{l+m+mi}{20}
\PY{n+nb}{print}\PY{p}{(}\PY{l+s+sa}{f}\PY{l+s+s2}{\PYZdq{}}\PY{l+s+s2}{Traço de }\PY{l+s+si}{\PYZob{}}\PY{n}{n\PYZus{}trasicoes}\PY{l+s+si}{\PYZcb{}}\PY{l+s+s2}{ transições:}\PY{l+s+s2}{\PYZdq{}}\PY{p}{)}
\PY{n}{genTrace}\PY{p}{(}\PY{p}{[}\PY{l+s+s1}{\PYZsq{}}\PY{l+s+s1}{A}\PY{l+s+s1}{\PYZsq{}}\PY{p}{,}\PY{l+s+s1}{\PYZsq{}}\PY{l+s+s1}{B}\PY{l+s+s1}{\PYZsq{}}\PY{p}{,}\PY{l+s+s1}{\PYZsq{}}\PY{l+s+s1}{C}\PY{l+s+s1}{\PYZsq{}}\PY{p}{,}\PY{l+s+s1}{\PYZsq{}}\PY{l+s+s1}{D}\PY{l+s+s1}{\PYZsq{}}\PY{p}{]}\PY{p}{,}\PY{n}{init}\PY{p}{,}\PY{n}{estadoInicial}\PY{p}{,}\PY{n}{trans}\PY{p}{,}\PY{n}{escolhas}\PY{p}{,}\PY{n}{error}\PY{p}{,}\PY{n}{n\PYZus{}trasicoes}\PY{p}{)}
\PY{n+nb}{print}\PY{p}{(}\PY{p}{)}
\PY{n+nb}{print}\PY{p}{(}\PY{l+s+s2}{\PYZdq{}}\PY{l+s+s2}{Verificação BMC do sistema:}\PY{l+s+s2}{\PYZdq{}}\PY{p}{)}
\PY{n}{bmc\PYZus{}always}\PY{p}{(}\PY{n}{genState}\PY{p}{,}\PY{n}{init}\PY{p}{,}\PY{n}{trans}\PY{p}{,}\PY{n}{Noerror}\PY{p}{,}\PY{l+m+mi}{16}\PY{p}{,}\PY{n}{estadoInicial}\PY{p}{,}\PY{n}{escolhas}\PY{p}{)}
\PY{n+nb}{print}\PY{p}{(}\PY{p}{)}
\PY{n+nb}{print}\PY{p}{(}\PY{l+s+s2}{\PYZdq{}}\PY{l+s+s2}{Verificação com k\PYZhy{}indução do sistema:}\PY{l+s+s2}{\PYZdq{}}\PY{p}{)}
\PY{n}{kinduction\PYZus{}always}\PY{p}{(}\PY{n}{genState}\PY{p}{,}\PY{n}{init}\PY{p}{,}\PY{n}{trans}\PY{p}{,}\PY{n}{Noerror}\PY{p}{,}\PY{l+m+mi}{4}\PY{p}{,}\PY{n}{estadoInicial}\PY{p}{,}\PY{n}{escolhas}\PY{p}{)}
\PY{n+nb}{print}\PY{p}{(}\PY{p}{)}
\PY{n+nb}{print}\PY{p}{(}\PY{l+s+s2}{\PYZdq{}}\PY{l+s+s2}{Verificação model checking com interpolantes do sistema:}\PY{l+s+s2}{\PYZdq{}}\PY{p}{)}
\PY{n}{model\PYZus{}checking}\PY{p}{(}\PY{p}{[}\PY{l+s+s1}{\PYZsq{}}\PY{l+s+s1}{A}\PY{l+s+s1}{\PYZsq{}}\PY{p}{,}\PY{l+s+s1}{\PYZsq{}}\PY{l+s+s1}{B}\PY{l+s+s1}{\PYZsq{}}\PY{p}{,}\PY{l+s+s1}{\PYZsq{}}\PY{l+s+s1}{C}\PY{l+s+s1}{\PYZsq{}}\PY{p}{,}\PY{l+s+s1}{\PYZsq{}}\PY{l+s+s1}{D}\PY{l+s+s1}{\PYZsq{}}\PY{p}{]}\PY{p}{,} \PY{n}{init}\PY{p}{,}\PY{n}{trans}\PY{p}{,} \PY{n}{error}\PY{p}{,} \PY{l+m+mi}{16}\PY{p}{,} \PY{l+m+mi}{16}\PY{p}{,}\PY{n}{estadoInicial}\PY{p}{,}\PY{n}{escolhas}\PY{p}{)}
\end{Verbatim}
\end{tcolorbox}

    \begin{Verbatim}[commandchars=\\\{\}]
Estado inicial: (False, False, True, False)
Escolhas: (True, True, False, False)

Traço de 20 transições:
Estado: 0
           A = False
           B = False
           C = True
           D = False
Estado: 1
           A = False
           B = True
           C = True
           D = False
Estado: 2
           A = False
           B = True
           C = True
           D = True
Estado: 3
           A = False
           B = True
           C = False
           D = False
Estado: 4
           A = True
           B = True
           C = False
           D = True
Estado: 5
           A = True
           B = False
           C = True
           D = False
Estado: 6
           A = False
           B = False
           C = True
           D = False
Estado: 7
           A = False
           B = True
           C = True
           D = False
Estado: 8
           A = False
           B = True
           C = True
           D = True
Estado: 9
           A = False
           B = True
           C = False
           D = False
Estado: 10
           A = True
           B = True
           C = False
           D = True
Estado: 11
           A = True
           B = False
           C = True
           D = False
Estado: 12
           A = False
           B = False
           C = True
           D = False
Estado: 13
           A = False
           B = True
           C = True
           D = False
Estado: 14
           A = False
           B = True
           C = True
           D = True
Estado: 15
           A = False
           B = True
           C = False
           D = False
Estado: 16
           A = True
           B = True
           C = False
           D = True
Estado: 17
           A = True
           B = False
           C = True
           D = False
Estado: 18
           A = False
           B = False
           C = True
           D = False
Estado: 19
           A = False
           B = True
           C = True
           D = False

Verificação BMC do sistema:
A propriedade é válida para traços de tamanho até 16

Verificação com k-indução do sistema:
O passo 4 indutivo nao preserva a propriedade.
O ponto onde falha começa aqui:
A = True
B = False
C = False
D = True

Verificação model checking com interpolantes do sistema:
Não foi possível encontrar o majorante.
Não foi possível encontrar o majorante.
Não foi possível encontrar o majorante.
Não foi possível encontrar o majorante.
Não foi possível encontrar o majorante.
Não foi possível encontrar o majorante.
Não foi possível encontrar o majorante.
Não foi possível encontrar o majorante.
Não foi possível encontrar o majorante.
Não foi possível encontrar o majorante.
Não foi possível encontrar o majorante.
Não foi possível encontrar o majorante.
Não foi possível encontrar o majorante.
Não foi possível encontrar o majorante.
Não foi possível encontrar o majorante.
Não foi possível encontrar o majorante.
Não foi possível encontrar o majorante.
Não foi possível encontrar o majorante.
Não foi possível encontrar o majorante.
Não foi possível encontrar o majorante.
Não foi possível encontrar o majorante.
Não foi possível encontrar o majorante.
Não foi possível encontrar o majorante.
Não foi possível encontrar o majorante.
Não foi possível encontrar o majorante.
Não foi possível encontrar o majorante.
Não foi possível encontrar o majorante.
Não foi possível encontrar o majorante.
Não foi possível encontrar o majorante.
Não foi possível encontrar o majorante.
Não foi possível encontrar o majorante.
Não foi possível encontrar o majorante.
Não foi possível encontrar o majorante.
Não foi possível encontrar o majorante.
Não foi possível encontrar o majorante.
Não foi possível encontrar o majorante.
Não foi possível encontrar o majorante.
Não foi possível encontrar o majorante.
Não foi possível encontrar o majorante.
Não foi possível encontrar o majorante.
Não foi possível encontrar o majorante.
Não foi possível encontrar o majorante.
Não foi possível encontrar o majorante.
Não foi possível encontrar o majorante.
Não foi possível encontrar o majorante.
Não foi possível encontrar o majorante.
Não foi possível encontrar o majorante.
Não foi possível encontrar o majorante.
Não foi possível encontrar o majorante.
Não foi possível encontrar o majorante.
Não foi possível encontrar o majorante.
Não foi possível encontrar o majorante.
Não foi possível encontrar o majorante.
Não foi possível encontrar o majorante.
Não foi possível encontrar o majorante.
Não foi possível encontrar o majorante.
Não foi possível encontrar o majorante.
Não foi possível encontrar o majorante.
Não foi possível encontrar o majorante.
Não foi possível encontrar o majorante.
Não foi possível encontrar o majorante.
Não foi possível encontrar o majorante.
Não foi possível encontrar o majorante.
Não foi possível encontrar o majorante.
Não foi possível encontrar o majorante.
Não foi possível encontrar o majorante.
Não foi possível encontrar o majorante.
Não foi possível encontrar o majorante.
Não foi possível encontrar o majorante.
Não foi possível encontrar o majorante.
Não foi possível encontrar o majorante.
Não foi possível encontrar o majorante.
Não foi possível encontrar o majorante.
Não foi possível encontrar o majorante.
Não foi possível encontrar o majorante.
Não foi possível encontrar o majorante.
Não foi possível encontrar o majorante.
Não foi possível encontrar o majorante.
Não foi possível encontrar o majorante.
Não foi possível encontrar o majorante.
Não foi possível encontrar o majorante.
Não foi possível encontrar o majorante.
Não foi possível encontrar o majorante.
Não foi possível encontrar o majorante.
Não foi possível encontrar o majorante.
Não foi possível encontrar o majorante.
Não foi possível encontrar o majorante.
Não foi possível encontrar o majorante.
Não foi possível encontrar o majorante.
Não foi possível encontrar o majorante.
Não foi possível encontrar o majorante.
Não foi possível encontrar o majorante.
Não foi possível encontrar o majorante.
Não foi possível encontrar o majorante.
Não foi possível encontrar o majorante.
Não foi possível encontrar o majorante.
Não foi possível encontrar o majorante.
Não foi possível encontrar o majorante.
Não foi possível encontrar o majorante.
Não foi possível encontrar o majorante.
Não foi possível encontrar o majorante.
Não foi possível encontrar o majorante.
Não foi possível encontrar o majorante.
Não foi possível encontrar o majorante.
Não foi possível encontrar o majorante.
Não foi possível encontrar o majorante.
Não foi possível encontrar o majorante.
Não foi possível encontrar o majorante.
Não foi possível encontrar o majorante.
Não foi possível encontrar o majorante.
Não foi possível encontrar o majorante.
Não foi possível encontrar o majorante.
Não foi possível encontrar o majorante.
Não foi possível encontrar o majorante.
Não foi possível encontrar o majorante.
Não foi possível encontrar o majorante.
Não foi possível encontrar o majorante.
Não foi possível encontrar o majorante.
Não foi possível encontrar o majorante.
Não foi possível encontrar o majorante.
Não foi possível encontrar o majorante.
Não foi possível encontrar o majorante.
Não foi possível encontrar o majorante.
Não foi possível encontrar o majorante.
Não foi possível encontrar o majorante.
Não foi possível encontrar o majorante.
Não foi possível encontrar o majorante.
Não foi possível encontrar o majorante.
Não foi possível encontrar o majorante.
Não foi possível encontrar o majorante.
Não foi possível encontrar o majorante.
Não foi possível encontrar o majorante.
Não foi possível encontrar o majorante.
Não foi possível encontrar o majorante.
Não foi possível encontrar o majorante.
Não foi possível encontrar o majorante.
Não foi possível encontrar o majorante.
Não foi possível encontrar o majorante.
Não foi possível encontrar o majorante.
Não foi possível encontrar o majorante.
Não foi possível encontrar o majorante.
Não foi possível encontrar o majorante.
Não foi possível encontrar o majorante.
Não foi possível encontrar o majorante.
Não foi possível encontrar o majorante.
Não foi possível encontrar o majorante.
Não foi possível encontrar o majorante.
Não foi possível encontrar o majorante.
Não foi possível encontrar o majorante.
Não foi possível encontrar o majorante.
Não foi possível encontrar o majorante.
Não foi possível encontrar o majorante.
Não foi possível encontrar o majorante.
Não foi possível encontrar o majorante.
Não foi possível encontrar o majorante.
Não foi possível encontrar o majorante.
Não foi possível encontrar o majorante.
Não foi possível encontrar o majorante.
Não foi possível encontrar o majorante.
Não foi possível encontrar o majorante.
Não foi possível encontrar o majorante.
Não foi possível encontrar o majorante.
Não foi possível encontrar o majorante.
Não foi possível encontrar o majorante.
Não foi possível encontrar o majorante.
Não foi possível encontrar o majorante.
Não foi possível encontrar o majorante.
Não foi possível encontrar o majorante.
Não foi possível encontrar o majorante.
Não foi possível encontrar o majorante.
Não foi possível encontrar o majorante.
Não foi possível encontrar o majorante.
Não foi possível encontrar o majorante.
Não foi possível encontrar o majorante.
Não foi possível encontrar o majorante.
Não foi possível encontrar o majorante.
Não foi possível encontrar o majorante.
Não foi possível encontrar o majorante.
Não foi possível encontrar o majorante.
Não foi possível encontrar o majorante.
Não foi possível encontrar o majorante.
Não foi possível encontrar o majorante.
Não foi possível encontrar o majorante.
Não foi possível encontrar o majorante.
Não foi possível encontrar o majorante.
Não foi possível encontrar o majorante.
Não foi possível encontrar o majorante.
Não foi possível encontrar o majorante.
Não foi possível encontrar o majorante.
Não foi possível encontrar o majorante.
Não foi possível encontrar o majorante.
Não foi possível encontrar o majorante.
Não foi possível encontrar o majorante.
Não foi possível encontrar o majorante.
Não foi possível encontrar o majorante.
Não foi possível encontrar o majorante.
Não foi possível encontrar o majorante.
Não foi possível encontrar o majorante.
Não foi possível encontrar o majorante.
Não foi possível encontrar o majorante.
Não foi possível encontrar o majorante.
Não foi possível encontrar o majorante.
Não foi possível encontrar o majorante.
Não foi possível encontrar o majorante.
Não foi possível encontrar o majorante.
Não foi possível encontrar o majorante.
Não foi possível encontrar o majorante.
Não foi possível encontrar o majorante.
Não foi possível encontrar o majorante.
Não foi possível encontrar o majorante.
Não foi possível encontrar o majorante.
Não foi possível encontrar o majorante.
Não foi possível encontrar o majorante.
Não foi possível encontrar o majorante.
Não foi possível encontrar o majorante.
Não foi possível encontrar o majorante.
Não foi possível encontrar o majorante.
Não foi possível encontrar o majorante.
Não foi possível encontrar o majorante.
Não foi possível encontrar o majorante.
Não foi possível encontrar o majorante.
Não foi possível encontrar o majorante.
Não foi possível encontrar o majorante.
Não foi possível encontrar o majorante.
Não foi possível encontrar o majorante.
Não foi possível encontrar o majorante.
Não foi possível encontrar o majorante.
Não foi possível encontrar o majorante.
Não foi possível encontrar o majorante.
Não foi possível encontrar o majorante.
Não foi possível encontrar o majorante.
Não foi possível encontrar o majorante.
Não foi possível encontrar o majorante.
Não foi possível encontrar o majorante.
Não foi possível encontrar o majorante.
Não foi possível encontrar o majorante.
Não foi possível encontrar o majorante.
Não foi possível encontrar o majorante.
Não foi possível encontrar o majorante.
Não foi possível encontrar o majorante.
Não foi possível encontrar o majorante.
Não foi possível encontrar o majorante.
Não foi possível encontrar o majorante.
Não foi possível encontrar o majorante.
Não foi possível encontrar o majorante.
Não foi possível encontrar o majorante.
Não foi possível encontrar o majorante.
Não foi possível encontrar o majorante.
Não foi possível encontrar o majorante.
Não foi possível encontrar o majorante.
Não foi possível encontrar o majorante.
Não foi possível encontrar o majorante.
Não foi possível encontrar o majorante.
Não foi possível encontrar o majorante.
Não foi possível encontrar o majorante.
Não foi possível encontrar o majorante.
unknown
    \end{Verbatim}

    Para confirmar que de facto a função de k-indução funciona corretamente,
pegando no exemplo em específico, a partir de k = 6, a função consegue
provar o invariante.

    \begin{tcolorbox}[breakable, size=fbox, boxrule=1pt, pad at break*=1mm,colback=cellbackground, colframe=cellborder]
\prompt{In}{incolor}{21}{\boxspacing}
\begin{Verbatim}[commandchars=\\\{\}]
\PY{n}{estadoInicial} \PY{o}{=} \PY{p}{[}\PY{k+kc}{False}\PY{p}{,} \PY{k+kc}{False}\PY{p}{,} \PY{k+kc}{True}\PY{p}{,} \PY{k+kc}{False}\PY{p}{]}
\PY{n+nb}{print}\PY{p}{(}\PY{l+s+sa}{f}\PY{l+s+s2}{\PYZdq{}}\PY{l+s+s2}{Estado inicial: }\PY{l+s+si}{\PYZob{}}\PY{n+nb}{tuple}\PY{p}{(}\PY{n}{estadoInicial}\PY{p}{)}\PY{l+s+si}{\PYZcb{}}\PY{l+s+s2}{\PYZdq{}}\PY{p}{)}
\PY{n}{escolhas} \PY{o}{=} \PY{p}{[}\PY{k+kc}{True}\PY{p}{,} \PY{k+kc}{True}\PY{p}{,} \PY{k+kc}{False}\PY{p}{,} \PY{k+kc}{False}\PY{p}{]}
\PY{n+nb}{print}\PY{p}{(}\PY{l+s+sa}{f}\PY{l+s+s2}{\PYZdq{}}\PY{l+s+s2}{Escolhas: }\PY{l+s+si}{\PYZob{}}\PY{n+nb}{tuple}\PY{p}{(}\PY{n}{escolhas}\PY{p}{)}\PY{l+s+si}{\PYZcb{}}\PY{l+s+s2}{\PYZdq{}}\PY{p}{)}
\PY{n+nb}{print}\PY{p}{(}\PY{p}{)}
\PY{n}{n\PYZus{}trasicoes} \PY{o}{=} \PY{l+m+mi}{20}
\PY{n+nb}{print}\PY{p}{(}\PY{l+s+sa}{f}\PY{l+s+s2}{\PYZdq{}}\PY{l+s+s2}{Traço de }\PY{l+s+si}{\PYZob{}}\PY{n}{n\PYZus{}trasicoes}\PY{l+s+si}{\PYZcb{}}\PY{l+s+s2}{ transições:}\PY{l+s+s2}{\PYZdq{}}\PY{p}{)}
\PY{n}{genTrace}\PY{p}{(}\PY{p}{[}\PY{l+s+s1}{\PYZsq{}}\PY{l+s+s1}{A}\PY{l+s+s1}{\PYZsq{}}\PY{p}{,}\PY{l+s+s1}{\PYZsq{}}\PY{l+s+s1}{B}\PY{l+s+s1}{\PYZsq{}}\PY{p}{,}\PY{l+s+s1}{\PYZsq{}}\PY{l+s+s1}{C}\PY{l+s+s1}{\PYZsq{}}\PY{p}{,}\PY{l+s+s1}{\PYZsq{}}\PY{l+s+s1}{D}\PY{l+s+s1}{\PYZsq{}}\PY{p}{]}\PY{p}{,}\PY{n}{init}\PY{p}{,}\PY{n}{estadoInicial}\PY{p}{,}\PY{n}{trans}\PY{p}{,}\PY{n}{escolhas}\PY{p}{,}\PY{n}{error}\PY{p}{,}\PY{n}{n\PYZus{}trasicoes}\PY{p}{)}
\PY{n+nb}{print}\PY{p}{(}\PY{p}{)}
\PY{n+nb}{print}\PY{p}{(}\PY{l+s+s2}{\PYZdq{}}\PY{l+s+s2}{Verificação BMC do sistema:}\PY{l+s+s2}{\PYZdq{}}\PY{p}{)}
\PY{n}{bmc\PYZus{}always}\PY{p}{(}\PY{n}{genState}\PY{p}{,}\PY{n}{init}\PY{p}{,}\PY{n}{trans}\PY{p}{,}\PY{n}{Noerror}\PY{p}{,}\PY{l+m+mi}{16}\PY{p}{,}\PY{n}{estadoInicial}\PY{p}{,}\PY{n}{escolhas}\PY{p}{)}
\PY{n+nb}{print}\PY{p}{(}\PY{p}{)}
\PY{n+nb}{print}\PY{p}{(}\PY{l+s+s2}{\PYZdq{}}\PY{l+s+s2}{Verificação com k\PYZhy{}indução do sistema:}\PY{l+s+s2}{\PYZdq{}}\PY{p}{)}
\PY{n}{kinduction\PYZus{}always}\PY{p}{(}\PY{n}{genState}\PY{p}{,}\PY{n}{init}\PY{p}{,}\PY{n}{trans}\PY{p}{,}\PY{n}{Noerror}\PY{p}{,}\PY{l+m+mi}{6}\PY{p}{,}\PY{n}{estadoInicial}\PY{p}{,}\PY{n}{escolhas}\PY{p}{)}
\PY{n+nb}{print}\PY{p}{(}\PY{p}{)}
\PY{n+nb}{print}\PY{p}{(}\PY{l+s+s2}{\PYZdq{}}\PY{l+s+s2}{Verificação model checking com interpolantes do sistema:}\PY{l+s+s2}{\PYZdq{}}\PY{p}{)}
\PY{n}{model\PYZus{}checking}\PY{p}{(}\PY{p}{[}\PY{l+s+s1}{\PYZsq{}}\PY{l+s+s1}{A}\PY{l+s+s1}{\PYZsq{}}\PY{p}{,}\PY{l+s+s1}{\PYZsq{}}\PY{l+s+s1}{B}\PY{l+s+s1}{\PYZsq{}}\PY{p}{,}\PY{l+s+s1}{\PYZsq{}}\PY{l+s+s1}{C}\PY{l+s+s1}{\PYZsq{}}\PY{p}{,}\PY{l+s+s1}{\PYZsq{}}\PY{l+s+s1}{D}\PY{l+s+s1}{\PYZsq{}}\PY{p}{]}\PY{p}{,} \PY{n}{init}\PY{p}{,}\PY{n}{trans}\PY{p}{,} \PY{n}{error}\PY{p}{,} \PY{l+m+mi}{16}\PY{p}{,} \PY{l+m+mi}{16}\PY{p}{,}\PY{n}{estadoInicial}\PY{p}{,}\PY{n}{escolhas}\PY{p}{)}
\end{Verbatim}
\end{tcolorbox}

    \begin{Verbatim}[commandchars=\\\{\}]
Estado inicial: (False, False, True, False)
Escolhas: (True, True, False, False)

Traço de 20 transições:
Estado: 0
           A = False
           B = False
           C = True
           D = False
Estado: 1
           A = False
           B = True
           C = True
           D = False
Estado: 2
           A = False
           B = True
           C = True
           D = True
Estado: 3
           A = False
           B = True
           C = False
           D = False
Estado: 4
           A = True
           B = True
           C = False
           D = True
Estado: 5
           A = True
           B = False
           C = True
           D = False
Estado: 6
           A = False
           B = False
           C = True
           D = False
Estado: 7
           A = False
           B = True
           C = True
           D = False
Estado: 8
           A = False
           B = True
           C = True
           D = True
Estado: 9
           A = False
           B = True
           C = False
           D = False
Estado: 10
           A = True
           B = True
           C = False
           D = True
Estado: 11
           A = True
           B = False
           C = True
           D = False
Estado: 12
           A = False
           B = False
           C = True
           D = False
Estado: 13
           A = False
           B = True
           C = True
           D = False
Estado: 14
           A = False
           B = True
           C = True
           D = True
Estado: 15
           A = False
           B = True
           C = False
           D = False
Estado: 16
           A = True
           B = True
           C = False
           D = True
Estado: 17
           A = True
           B = False
           C = True
           D = False
Estado: 18
           A = False
           B = False
           C = True
           D = False
Estado: 19
           A = False
           B = True
           C = True
           D = False

Verificação BMC do sistema:
A propriedade é válida para traços de tamanho até 16

Verificação com k-indução do sistema:
A propriedade é sempre válida

Verificação model checking com interpolantes do sistema:
Não foi possível encontrar o majorante.
Não foi possível encontrar o majorante.
Não foi possível encontrar o majorante.
Não foi possível encontrar o majorante.
Não foi possível encontrar o majorante.
Não foi possível encontrar o majorante.
Não foi possível encontrar o majorante.
Não foi possível encontrar o majorante.
Não foi possível encontrar o majorante.
Não foi possível encontrar o majorante.
Não foi possível encontrar o majorante.
Não foi possível encontrar o majorante.
Não foi possível encontrar o majorante.
Não foi possível encontrar o majorante.
Não foi possível encontrar o majorante.
Não foi possível encontrar o majorante.
Não foi possível encontrar o majorante.
Não foi possível encontrar o majorante.
Não foi possível encontrar o majorante.
Não foi possível encontrar o majorante.
Não foi possível encontrar o majorante.
Não foi possível encontrar o majorante.
Não foi possível encontrar o majorante.
Não foi possível encontrar o majorante.
Não foi possível encontrar o majorante.
Não foi possível encontrar o majorante.
Não foi possível encontrar o majorante.
Não foi possível encontrar o majorante.
Não foi possível encontrar o majorante.
Não foi possível encontrar o majorante.
Não foi possível encontrar o majorante.
Não foi possível encontrar o majorante.
Não foi possível encontrar o majorante.
Não foi possível encontrar o majorante.
Não foi possível encontrar o majorante.
Não foi possível encontrar o majorante.
Não foi possível encontrar o majorante.
Não foi possível encontrar o majorante.
Não foi possível encontrar o majorante.
Não foi possível encontrar o majorante.
Não foi possível encontrar o majorante.
Não foi possível encontrar o majorante.
Não foi possível encontrar o majorante.
Não foi possível encontrar o majorante.
Não foi possível encontrar o majorante.
Não foi possível encontrar o majorante.
Não foi possível encontrar o majorante.
Não foi possível encontrar o majorante.
Não foi possível encontrar o majorante.
Não foi possível encontrar o majorante.
Não foi possível encontrar o majorante.
Não foi possível encontrar o majorante.
Não foi possível encontrar o majorante.
Não foi possível encontrar o majorante.
Não foi possível encontrar o majorante.
Não foi possível encontrar o majorante.
Não foi possível encontrar o majorante.
Não foi possível encontrar o majorante.
Não foi possível encontrar o majorante.
Não foi possível encontrar o majorante.
Não foi possível encontrar o majorante.
Não foi possível encontrar o majorante.
Não foi possível encontrar o majorante.
Não foi possível encontrar o majorante.
Não foi possível encontrar o majorante.
Não foi possível encontrar o majorante.
Não foi possível encontrar o majorante.
Não foi possível encontrar o majorante.
Não foi possível encontrar o majorante.
Não foi possível encontrar o majorante.
Não foi possível encontrar o majorante.
Não foi possível encontrar o majorante.
Não foi possível encontrar o majorante.
Não foi possível encontrar o majorante.
Não foi possível encontrar o majorante.
Não foi possível encontrar o majorante.
Não foi possível encontrar o majorante.
Não foi possível encontrar o majorante.
Não foi possível encontrar o majorante.
Não foi possível encontrar o majorante.
Não foi possível encontrar o majorante.
Não foi possível encontrar o majorante.
Não foi possível encontrar o majorante.
Não foi possível encontrar o majorante.
Não foi possível encontrar o majorante.
Não foi possível encontrar o majorante.
Não foi possível encontrar o majorante.
Não foi possível encontrar o majorante.
Não foi possível encontrar o majorante.
Não foi possível encontrar o majorante.
Não foi possível encontrar o majorante.
Não foi possível encontrar o majorante.
Não foi possível encontrar o majorante.
Não foi possível encontrar o majorante.
Não foi possível encontrar o majorante.
Não foi possível encontrar o majorante.
Não foi possível encontrar o majorante.
Não foi possível encontrar o majorante.
Não foi possível encontrar o majorante.
Não foi possível encontrar o majorante.
Não foi possível encontrar o majorante.
Não foi possível encontrar o majorante.
Não foi possível encontrar o majorante.
Não foi possível encontrar o majorante.
Não foi possível encontrar o majorante.
Não foi possível encontrar o majorante.
Não foi possível encontrar o majorante.
Não foi possível encontrar o majorante.
Não foi possível encontrar o majorante.
Não foi possível encontrar o majorante.
Não foi possível encontrar o majorante.
Não foi possível encontrar o majorante.
Não foi possível encontrar o majorante.
Não foi possível encontrar o majorante.
Não foi possível encontrar o majorante.
Não foi possível encontrar o majorante.
Não foi possível encontrar o majorante.
Não foi possível encontrar o majorante.
Não foi possível encontrar o majorante.
Não foi possível encontrar o majorante.
Não foi possível encontrar o majorante.
Não foi possível encontrar o majorante.
Não foi possível encontrar o majorante.
Não foi possível encontrar o majorante.
Não foi possível encontrar o majorante.
Não foi possível encontrar o majorante.
Não foi possível encontrar o majorante.
Não foi possível encontrar o majorante.
Não foi possível encontrar o majorante.
Não foi possível encontrar o majorante.
Não foi possível encontrar o majorante.
Não foi possível encontrar o majorante.
Não foi possível encontrar o majorante.
Não foi possível encontrar o majorante.
Não foi possível encontrar o majorante.
Não foi possível encontrar o majorante.
Não foi possível encontrar o majorante.
Não foi possível encontrar o majorante.
Não foi possível encontrar o majorante.
Não foi possível encontrar o majorante.
Não foi possível encontrar o majorante.
Não foi possível encontrar o majorante.
Não foi possível encontrar o majorante.
Não foi possível encontrar o majorante.
Não foi possível encontrar o majorante.
Não foi possível encontrar o majorante.
Não foi possível encontrar o majorante.
Não foi possível encontrar o majorante.
Não foi possível encontrar o majorante.
Não foi possível encontrar o majorante.
Não foi possível encontrar o majorante.
Não foi possível encontrar o majorante.
Não foi possível encontrar o majorante.
Não foi possível encontrar o majorante.
Não foi possível encontrar o majorante.
Não foi possível encontrar o majorante.
Não foi possível encontrar o majorante.
Não foi possível encontrar o majorante.
Não foi possível encontrar o majorante.
Não foi possível encontrar o majorante.
Não foi possível encontrar o majorante.
Não foi possível encontrar o majorante.
Não foi possível encontrar o majorante.
Não foi possível encontrar o majorante.
Não foi possível encontrar o majorante.
Não foi possível encontrar o majorante.
Não foi possível encontrar o majorante.
Não foi possível encontrar o majorante.
Não foi possível encontrar o majorante.
Não foi possível encontrar o majorante.
Não foi possível encontrar o majorante.
Não foi possível encontrar o majorante.
Não foi possível encontrar o majorante.
Não foi possível encontrar o majorante.
Não foi possível encontrar o majorante.
Não foi possível encontrar o majorante.
Não foi possível encontrar o majorante.
Não foi possível encontrar o majorante.
Não foi possível encontrar o majorante.
Não foi possível encontrar o majorante.
Não foi possível encontrar o majorante.
Não foi possível encontrar o majorante.
Não foi possível encontrar o majorante.
Não foi possível encontrar o majorante.
Não foi possível encontrar o majorante.
Não foi possível encontrar o majorante.
Não foi possível encontrar o majorante.
Não foi possível encontrar o majorante.
Não foi possível encontrar o majorante.
Não foi possível encontrar o majorante.
Não foi possível encontrar o majorante.
Não foi possível encontrar o majorante.
Não foi possível encontrar o majorante.
Não foi possível encontrar o majorante.
Não foi possível encontrar o majorante.
Não foi possível encontrar o majorante.
Não foi possível encontrar o majorante.
Não foi possível encontrar o majorante.
Não foi possível encontrar o majorante.
Não foi possível encontrar o majorante.
Não foi possível encontrar o majorante.
Não foi possível encontrar o majorante.
Não foi possível encontrar o majorante.
Não foi possível encontrar o majorante.
Não foi possível encontrar o majorante.
Não foi possível encontrar o majorante.
Não foi possível encontrar o majorante.
Não foi possível encontrar o majorante.
Não foi possível encontrar o majorante.
Não foi possível encontrar o majorante.
Não foi possível encontrar o majorante.
Não foi possível encontrar o majorante.
Não foi possível encontrar o majorante.
Não foi possível encontrar o majorante.
Não foi possível encontrar o majorante.
Não foi possível encontrar o majorante.
Não foi possível encontrar o majorante.
Não foi possível encontrar o majorante.
Não foi possível encontrar o majorante.
Não foi possível encontrar o majorante.
Não foi possível encontrar o majorante.
Não foi possível encontrar o majorante.
Não foi possível encontrar o majorante.
Não foi possível encontrar o majorante.
Não foi possível encontrar o majorante.
Não foi possível encontrar o majorante.
Não foi possível encontrar o majorante.
Não foi possível encontrar o majorante.
Não foi possível encontrar o majorante.
Não foi possível encontrar o majorante.
Não foi possível encontrar o majorante.
Não foi possível encontrar o majorante.
Não foi possível encontrar o majorante.
Não foi possível encontrar o majorante.
Não foi possível encontrar o majorante.
Não foi possível encontrar o majorante.
Não foi possível encontrar o majorante.
Não foi possível encontrar o majorante.
Não foi possível encontrar o majorante.
Não foi possível encontrar o majorante.
Não foi possível encontrar o majorante.
Não foi possível encontrar o majorante.
Não foi possível encontrar o majorante.
Não foi possível encontrar o majorante.
Não foi possível encontrar o majorante.
Não foi possível encontrar o majorante.
Não foi possível encontrar o majorante.
Não foi possível encontrar o majorante.
Não foi possível encontrar o majorante.
Não foi possível encontrar o majorante.
Não foi possível encontrar o majorante.
Não foi possível encontrar o majorante.
Não foi possível encontrar o majorante.
Não foi possível encontrar o majorante.
Não foi possível encontrar o majorante.
Não foi possível encontrar o majorante.
unknown
    \end{Verbatim}

    \hypertarget{exemplo-2}{%
\paragraph{Exemplo 2}\label{exemplo-2}}

    Neste exemplo, deu unsafe e foi possível verificar com as três funções.
Além disso, através do genTrace denotamos que o estado de erro é 11º
estado gerado pela função (o índice começa em 0).

    \begin{tcolorbox}[breakable, size=fbox, boxrule=1pt, pad at break*=1mm,colback=cellbackground, colframe=cellborder]
\prompt{In}{incolor}{13}{\boxspacing}
\begin{Verbatim}[commandchars=\\\{\}]
\PY{n}{estadoInicial} \PY{o}{=} \PY{p}{[}\PY{n}{rn}\PY{o}{.}\PY{n}{choice}\PY{p}{(}\PY{p}{[}\PY{k+kc}{True}\PY{p}{,}\PY{k+kc}{False}\PY{p}{]}\PY{p}{)} \PY{k}{for} \PY{n}{i} \PY{o+ow}{in} \PY{n+nb}{range}\PY{p}{(}\PY{l+m+mi}{4}\PY{p}{)}\PY{p}{]}
\PY{n+nb}{print}\PY{p}{(}\PY{l+s+sa}{f}\PY{l+s+s2}{\PYZdq{}}\PY{l+s+s2}{Estado inicial: }\PY{l+s+si}{\PYZob{}}\PY{n+nb}{tuple}\PY{p}{(}\PY{n}{estadoInicial}\PY{p}{)}\PY{l+s+si}{\PYZcb{}}\PY{l+s+s2}{\PYZdq{}}\PY{p}{)}
\PY{n}{escolhas} \PY{o}{=} \PY{p}{[}\PY{n}{rn}\PY{o}{.}\PY{n}{choice}\PY{p}{(}\PY{p}{[}\PY{k+kc}{True}\PY{p}{,}\PY{k+kc}{False}\PY{p}{]}\PY{p}{)} \PY{k}{for} \PY{n}{i} \PY{o+ow}{in} \PY{n+nb}{range}\PY{p}{(}\PY{l+m+mi}{4}\PY{p}{)}\PY{p}{]}
\PY{n+nb}{print}\PY{p}{(}\PY{l+s+sa}{f}\PY{l+s+s2}{\PYZdq{}}\PY{l+s+s2}{Escolhas: }\PY{l+s+si}{\PYZob{}}\PY{n+nb}{tuple}\PY{p}{(}\PY{n}{escolhas}\PY{p}{)}\PY{l+s+si}{\PYZcb{}}\PY{l+s+s2}{\PYZdq{}}\PY{p}{)}
\PY{n+nb}{print}\PY{p}{(}\PY{p}{)}
\PY{n}{n\PYZus{}trasicoes} \PY{o}{=} \PY{l+m+mi}{20}
\PY{n+nb}{print}\PY{p}{(}\PY{l+s+sa}{f}\PY{l+s+s2}{\PYZdq{}}\PY{l+s+s2}{Traço de }\PY{l+s+si}{\PYZob{}}\PY{n}{n\PYZus{}trasicoes}\PY{l+s+si}{\PYZcb{}}\PY{l+s+s2}{ transições:}\PY{l+s+s2}{\PYZdq{}}\PY{p}{)}
\PY{n}{genTrace}\PY{p}{(}\PY{p}{[}\PY{l+s+s1}{\PYZsq{}}\PY{l+s+s1}{A}\PY{l+s+s1}{\PYZsq{}}\PY{p}{,}\PY{l+s+s1}{\PYZsq{}}\PY{l+s+s1}{B}\PY{l+s+s1}{\PYZsq{}}\PY{p}{,}\PY{l+s+s1}{\PYZsq{}}\PY{l+s+s1}{C}\PY{l+s+s1}{\PYZsq{}}\PY{p}{,}\PY{l+s+s1}{\PYZsq{}}\PY{l+s+s1}{D}\PY{l+s+s1}{\PYZsq{}}\PY{p}{]}\PY{p}{,}\PY{n}{init}\PY{p}{,}\PY{n}{estadoInicial}\PY{p}{,}\PY{n}{trans}\PY{p}{,}\PY{n}{escolhas}\PY{p}{,}\PY{n}{error}\PY{p}{,}\PY{n}{n\PYZus{}trasicoes}\PY{p}{)}
\PY{n+nb}{print}\PY{p}{(}\PY{p}{)}
\PY{n+nb}{print}\PY{p}{(}\PY{l+s+s2}{\PYZdq{}}\PY{l+s+s2}{Verificação BMC do sistema:}\PY{l+s+s2}{\PYZdq{}}\PY{p}{)}
\PY{n}{bmc\PYZus{}always}\PY{p}{(}\PY{n}{genState}\PY{p}{,}\PY{n}{init}\PY{p}{,}\PY{n}{trans}\PY{p}{,}\PY{n}{Noerror}\PY{p}{,}\PY{l+m+mi}{20}\PY{p}{,}\PY{n}{estadoInicial}\PY{p}{,}\PY{n}{escolhas}\PY{p}{)}
\PY{n+nb}{print}\PY{p}{(}\PY{p}{)}
\PY{n+nb}{print}\PY{p}{(}\PY{l+s+s2}{\PYZdq{}}\PY{l+s+s2}{Verificação com k\PYZhy{}indução do sistema:}\PY{l+s+s2}{\PYZdq{}}\PY{p}{)}
\PY{n}{kinduction\PYZus{}always}\PY{p}{(}\PY{n}{genState}\PY{p}{,}\PY{n}{init}\PY{p}{,}\PY{n}{trans}\PY{p}{,}\PY{n}{Noerror}\PY{p}{,}\PY{l+m+mi}{16}\PY{p}{,}\PY{n}{estadoInicial}\PY{p}{,}\PY{n}{escolhas}\PY{p}{)}
\PY{n+nb}{print}\PY{p}{(}\PY{p}{)}
\PY{n+nb}{print}\PY{p}{(}\PY{l+s+s2}{\PYZdq{}}\PY{l+s+s2}{Verificação model checking com interpolantes do sistema:}\PY{l+s+s2}{\PYZdq{}}\PY{p}{)}
\PY{n}{model\PYZus{}checking}\PY{p}{(}\PY{p}{[}\PY{l+s+s1}{\PYZsq{}}\PY{l+s+s1}{A}\PY{l+s+s1}{\PYZsq{}}\PY{p}{,}\PY{l+s+s1}{\PYZsq{}}\PY{l+s+s1}{B}\PY{l+s+s1}{\PYZsq{}}\PY{p}{,}\PY{l+s+s1}{\PYZsq{}}\PY{l+s+s1}{C}\PY{l+s+s1}{\PYZsq{}}\PY{p}{,}\PY{l+s+s1}{\PYZsq{}}\PY{l+s+s1}{D}\PY{l+s+s1}{\PYZsq{}}\PY{p}{]}\PY{p}{,} \PY{n}{init}\PY{p}{,}\PY{n}{trans}\PY{p}{,} \PY{n}{error}\PY{p}{,} \PY{l+m+mi}{16}\PY{p}{,} \PY{l+m+mi}{16}\PY{p}{,}\PY{n}{estadoInicial}\PY{p}{,}\PY{n}{escolhas}\PY{p}{)}
\end{Verbatim}
\end{tcolorbox}

    \begin{Verbatim}[commandchars=\\\{\}]
Estado inicial: (False, True, True, True)
Escolhas: (True, False, True, True)

Traço de 20 transições:
Estado: 0
           A = False
           B = True
           C = True
           D = True
Estado: 1
           A = False
           B = True
           C = False
           D = False
Estado: 2
           A = True
           B = True
           C = True
           D = False
Estado: 3
           A = False
           B = False
           C = True
           D = False
Estado: 4
           A = False
           B = False
           C = True
           D = True
Estado: 5
           A = False
           B = False
           C = False
           D = True
Estado: 6
           A = True
           B = False
           C = False
           D = True
Estado: 7
           A = True
           B = True
           C = False
           D = True
Estado: 8
           A = True
           B = False
           C = False
           D = False
Estado: 9
           A = True
           B = True
           C = True
           D = True
Estado: 10
           A = False
           B = False
           C = False
           D = False
Estado: 11
           A = True
           B = False
           C = True
           D = True
Estado: 12
           A = False
           B = True
           C = False
           D = True
Estado: 13
           A = True
           B = True
           C = False
           D = False
Estado: 14
           A = True
           B = False
           C = True
           D = False
Estado: 15
           A = False
           B = True
           C = True
           D = True
Estado: 16
           A = False
           B = True
           C = False
           D = False
Estado: 17
           A = True
           B = True
           C = True
           D = False
Estado: 18
           A = False
           B = False
           C = True
           D = False
Estado: 19
           A = False
           B = False
           C = True
           D = True

Verificação BMC do sistema:
A propriedade não é valida
0
A= False
B= True
C= True
D= True
1
A= False
B= True
C= False
D= False
2
A= True
B= True
C= True
D= False
3
A= False
B= False
C= True
D= False
4
A= False
B= False
C= True
D= True
5
A= False
B= False
C= False
D= True
6
A= True
B= False
C= False
D= True
7
A= True
B= True
C= False
D= True
8
A= True
B= False
C= False
D= False
9
A= True
B= True
C= True
D= True
10
A= False
B= False
C= False
D= False

Verificação com k-indução do sistema:
A propriedade não é válida nos k estados iniciais.
0
A= False
B= True
C= True
D= True
1
A= False
B= True
C= False
D= False
2
A= True
B= True
C= True
D= False
3
A= False
B= False
C= True
D= False
4
A= False
B= False
C= True
D= True
5
A= False
B= False
C= False
D= True
6
A= True
B= False
C= False
D= True
7
A= True
B= True
C= False
D= True
8
A= True
B= False
C= False
D= False
9
A= True
B= True
C= True
D= True
10
A= False
B= False
C= False
D= False
11
A= True
B= False
C= True
D= True
12
A= False
B= True
C= False
D= True
13
A= True
B= True
C= False
D= False
14
A= True
B= False
C= True
D= False
15
A= False
B= True
C= True
D= True

Verificação model checking com interpolantes do sistema:
Não foi possível encontrar o majorante.
Não foi possível encontrar o majorante.
Não foi possível encontrar o majorante.
Não foi possível encontrar o majorante.
Não foi possível encontrar o majorante.
Não foi possível encontrar o majorante.
Não foi possível encontrar o majorante.
Não foi possível encontrar o majorante.
Não foi possível encontrar o majorante.
Não foi possível encontrar o majorante.
Não foi possível encontrar o majorante.
Não foi possível encontrar o majorante.
Não foi possível encontrar o majorante.
Não foi possível encontrar o majorante.
Não foi possível encontrar o majorante.
Não foi possível encontrar o majorante.
Não foi possível encontrar o majorante.
Não foi possível encontrar o majorante.
Não foi possível encontrar o majorante.
Não foi possível encontrar o majorante.
Não foi possível encontrar o majorante.
Não foi possível encontrar o majorante.
Não foi possível encontrar o majorante.
Não foi possível encontrar o majorante.
Não foi possível encontrar o majorante.
Não foi possível encontrar o majorante.
Não foi possível encontrar o majorante.
Não foi possível encontrar o majorante.
Não foi possível encontrar o majorante.
Não foi possível encontrar o majorante.
Não foi possível encontrar o majorante.
Não foi possível encontrar o majorante.
Não foi possível encontrar o majorante.
Não foi possível encontrar o majorante.
Não foi possível encontrar o majorante.
Não foi possível encontrar o majorante.
unsafe
    \end{Verbatim}

    \hypertarget{exemplo-3}{%
\paragraph{Exemplo 3}\label{exemplo-3}}

    Neste exemplo, deu unsafe e foi possível verificar com as três funções.
Além disso, através do genTrace denotamos que o primeiro estado de erro
é 4º estado gerado pela função. (o índice começa em 0)

    \begin{tcolorbox}[breakable, size=fbox, boxrule=1pt, pad at break*=1mm,colback=cellbackground, colframe=cellborder]
\prompt{In}{incolor}{14}{\boxspacing}
\begin{Verbatim}[commandchars=\\\{\}]
\PY{n}{estadoInicial} \PY{o}{=} \PY{p}{[}\PY{n}{rn}\PY{o}{.}\PY{n}{choice}\PY{p}{(}\PY{p}{[}\PY{k+kc}{True}\PY{p}{,}\PY{k+kc}{False}\PY{p}{]}\PY{p}{)} \PY{k}{for} \PY{n}{i} \PY{o+ow}{in} \PY{n+nb}{range}\PY{p}{(}\PY{l+m+mi}{4}\PY{p}{)}\PY{p}{]}
\PY{n+nb}{print}\PY{p}{(}\PY{l+s+sa}{f}\PY{l+s+s2}{\PYZdq{}}\PY{l+s+s2}{Estado inicial: }\PY{l+s+si}{\PYZob{}}\PY{n+nb}{tuple}\PY{p}{(}\PY{n}{estadoInicial}\PY{p}{)}\PY{l+s+si}{\PYZcb{}}\PY{l+s+s2}{\PYZdq{}}\PY{p}{)}
\PY{n}{escolhas} \PY{o}{=} \PY{p}{[}\PY{n}{rn}\PY{o}{.}\PY{n}{choice}\PY{p}{(}\PY{p}{[}\PY{k+kc}{True}\PY{p}{,}\PY{k+kc}{False}\PY{p}{]}\PY{p}{)} \PY{k}{for} \PY{n}{i} \PY{o+ow}{in} \PY{n+nb}{range}\PY{p}{(}\PY{l+m+mi}{4}\PY{p}{)}\PY{p}{]}
\PY{n+nb}{print}\PY{p}{(}\PY{l+s+sa}{f}\PY{l+s+s2}{\PYZdq{}}\PY{l+s+s2}{Escolhas: }\PY{l+s+si}{\PYZob{}}\PY{n+nb}{tuple}\PY{p}{(}\PY{n}{escolhas}\PY{p}{)}\PY{l+s+si}{\PYZcb{}}\PY{l+s+s2}{\PYZdq{}}\PY{p}{)}
\PY{n+nb}{print}\PY{p}{(}\PY{p}{)}
\PY{n}{n\PYZus{}trasicoes} \PY{o}{=} \PY{l+m+mi}{20}
\PY{n+nb}{print}\PY{p}{(}\PY{l+s+sa}{f}\PY{l+s+s2}{\PYZdq{}}\PY{l+s+s2}{Traço de }\PY{l+s+si}{\PYZob{}}\PY{n}{n\PYZus{}trasicoes}\PY{l+s+si}{\PYZcb{}}\PY{l+s+s2}{ transições:}\PY{l+s+s2}{\PYZdq{}}\PY{p}{)}
\PY{n}{genTrace}\PY{p}{(}\PY{p}{[}\PY{l+s+s1}{\PYZsq{}}\PY{l+s+s1}{A}\PY{l+s+s1}{\PYZsq{}}\PY{p}{,}\PY{l+s+s1}{\PYZsq{}}\PY{l+s+s1}{B}\PY{l+s+s1}{\PYZsq{}}\PY{p}{,}\PY{l+s+s1}{\PYZsq{}}\PY{l+s+s1}{C}\PY{l+s+s1}{\PYZsq{}}\PY{p}{,}\PY{l+s+s1}{\PYZsq{}}\PY{l+s+s1}{D}\PY{l+s+s1}{\PYZsq{}}\PY{p}{]}\PY{p}{,}\PY{n}{init}\PY{p}{,}\PY{n}{estadoInicial}\PY{p}{,}\PY{n}{trans}\PY{p}{,}\PY{n}{escolhas}\PY{p}{,}\PY{n}{error}\PY{p}{,}\PY{n}{n\PYZus{}trasicoes}\PY{p}{)}
\PY{n+nb}{print}\PY{p}{(}\PY{p}{)}
\PY{n+nb}{print}\PY{p}{(}\PY{l+s+s2}{\PYZdq{}}\PY{l+s+s2}{Verificação BMC do sistema:}\PY{l+s+s2}{\PYZdq{}}\PY{p}{)}
\PY{n}{bmc\PYZus{}always}\PY{p}{(}\PY{n}{genState}\PY{p}{,}\PY{n}{init}\PY{p}{,}\PY{n}{trans}\PY{p}{,}\PY{n}{Noerror}\PY{p}{,}\PY{l+m+mi}{20}\PY{p}{,}\PY{n}{estadoInicial}\PY{p}{,}\PY{n}{escolhas}\PY{p}{)}
\PY{n+nb}{print}\PY{p}{(}\PY{p}{)}
\PY{n+nb}{print}\PY{p}{(}\PY{l+s+s2}{\PYZdq{}}\PY{l+s+s2}{Verificação com k\PYZhy{}indução do sistema:}\PY{l+s+s2}{\PYZdq{}}\PY{p}{)}
\PY{n}{kinduction\PYZus{}always}\PY{p}{(}\PY{n}{genState}\PY{p}{,}\PY{n}{init}\PY{p}{,}\PY{n}{trans}\PY{p}{,}\PY{n}{Noerror}\PY{p}{,}\PY{l+m+mi}{16}\PY{p}{,}\PY{n}{estadoInicial}\PY{p}{,}\PY{n}{escolhas}\PY{p}{)}
\PY{n+nb}{print}\PY{p}{(}\PY{p}{)}
\PY{n+nb}{print}\PY{p}{(}\PY{l+s+s2}{\PYZdq{}}\PY{l+s+s2}{Verificação model checking com interpolantes do sistema:}\PY{l+s+s2}{\PYZdq{}}\PY{p}{)}
\PY{n}{model\PYZus{}checking}\PY{p}{(}\PY{p}{[}\PY{l+s+s1}{\PYZsq{}}\PY{l+s+s1}{A}\PY{l+s+s1}{\PYZsq{}}\PY{p}{,}\PY{l+s+s1}{\PYZsq{}}\PY{l+s+s1}{B}\PY{l+s+s1}{\PYZsq{}}\PY{p}{,}\PY{l+s+s1}{\PYZsq{}}\PY{l+s+s1}{C}\PY{l+s+s1}{\PYZsq{}}\PY{p}{,}\PY{l+s+s1}{\PYZsq{}}\PY{l+s+s1}{D}\PY{l+s+s1}{\PYZsq{}}\PY{p}{]}\PY{p}{,} \PY{n}{init}\PY{p}{,}\PY{n}{trans}\PY{p}{,} \PY{n}{error}\PY{p}{,} \PY{l+m+mi}{16}\PY{p}{,} \PY{l+m+mi}{16}\PY{p}{,}\PY{n}{estadoInicial}\PY{p}{,}\PY{n}{escolhas}\PY{p}{)}
\end{Verbatim}
\end{tcolorbox}

    \begin{Verbatim}[commandchars=\\\{\}]
Estado inicial: (True, True, False, True)
Escolhas: (False, False, True, True)

Traço de 20 transições:
Estado: 0
           A = True
           B = True
           C = False
           D = True
Estado: 1
           A = True
           B = False
           C = False
           D = False
Estado: 2
           A = True
           B = True
           C = True
           D = True
Estado: 3
           A = False
           B = False
           C = False
           D = False
Estado: 4
           A = False
           B = False
           C = True
           D = True
Estado: 5
           A = True
           B = False
           C = False
           D = True
Estado: 6
           A = True
           B = True
           C = False
           D = True
Estado: 7
           A = True
           B = False
           C = False
           D = False
Estado: 8
           A = True
           B = True
           C = True
           D = True
Estado: 9
           A = False
           B = False
           C = False
           D = False
Estado: 10
           A = False
           B = False
           C = True
           D = True
Estado: 11
           A = True
           B = False
           C = False
           D = True
Estado: 12
           A = True
           B = True
           C = False
           D = True
Estado: 13
           A = True
           B = False
           C = False
           D = False
Estado: 14
           A = True
           B = True
           C = True
           D = True
Estado: 15
           A = False
           B = False
           C = False
           D = False
Estado: 16
           A = False
           B = False
           C = True
           D = True
Estado: 17
           A = True
           B = False
           C = False
           D = True
Estado: 18
           A = True
           B = True
           C = False
           D = True
Estado: 19
           A = True
           B = False
           C = False
           D = False

Verificação BMC do sistema:
A propriedade não é valida
0
A= True
B= True
C= False
D= True
1
A= True
B= False
C= False
D= False
2
A= True
B= True
C= True
D= True
3
A= False
B= False
C= False
D= False

Verificação com k-indução do sistema:
A propriedade não é válida nos k estados iniciais.
0
A= True
B= True
C= False
D= True
1
A= True
B= False
C= False
D= False
2
A= True
B= True
C= True
D= True
3
A= False
B= False
C= False
D= False
4
A= False
B= False
C= True
D= True
5
A= True
B= False
C= False
D= True
6
A= True
B= True
C= False
D= True
7
A= True
B= False
C= False
D= False
8
A= True
B= True
C= True
D= True
9
A= False
B= False
C= False
D= False
10
A= False
B= False
C= True
D= True
11
A= True
B= False
C= False
D= True
12
A= True
B= True
C= False
D= True
13
A= True
B= False
C= False
D= False
14
A= True
B= True
C= True
D= True
15
A= False
B= False
C= False
D= False

Verificação model checking com interpolantes do sistema:
Não foi possível encontrar o majorante.
unsafe
    \end{Verbatim}

    \hypertarget{exemplo-4}{%
\paragraph{Exemplo 4}\label{exemplo-4}}

    Neste caso, conseguimos verificar que é ``safe'', k-indução(k=16) e
seguindo a suposição feita a par das funções, o \textbf{model\_checking}
também poderá signifcar que é ``safe''.

    \begin{tcolorbox}[breakable, size=fbox, boxrule=1pt, pad at break*=1mm,colback=cellbackground, colframe=cellborder]
\prompt{In}{incolor}{15}{\boxspacing}
\begin{Verbatim}[commandchars=\\\{\}]
\PY{n}{estadoInicial} \PY{o}{=} \PY{p}{[}\PY{n}{rn}\PY{o}{.}\PY{n}{choice}\PY{p}{(}\PY{p}{[}\PY{k+kc}{True}\PY{p}{,}\PY{k+kc}{False}\PY{p}{]}\PY{p}{)} \PY{k}{for} \PY{n}{i} \PY{o+ow}{in} \PY{n+nb}{range}\PY{p}{(}\PY{l+m+mi}{4}\PY{p}{)}\PY{p}{]}
\PY{n+nb}{print}\PY{p}{(}\PY{l+s+sa}{f}\PY{l+s+s2}{\PYZdq{}}\PY{l+s+s2}{Estado inicial: }\PY{l+s+si}{\PYZob{}}\PY{n+nb}{tuple}\PY{p}{(}\PY{n}{estadoInicial}\PY{p}{)}\PY{l+s+si}{\PYZcb{}}\PY{l+s+s2}{\PYZdq{}}\PY{p}{)}
\PY{n}{escolhas} \PY{o}{=} \PY{p}{[}\PY{n}{rn}\PY{o}{.}\PY{n}{choice}\PY{p}{(}\PY{p}{[}\PY{k+kc}{True}\PY{p}{,}\PY{k+kc}{False}\PY{p}{]}\PY{p}{)} \PY{k}{for} \PY{n}{i} \PY{o+ow}{in} \PY{n+nb}{range}\PY{p}{(}\PY{l+m+mi}{4}\PY{p}{)}\PY{p}{]}
\PY{n+nb}{print}\PY{p}{(}\PY{l+s+sa}{f}\PY{l+s+s2}{\PYZdq{}}\PY{l+s+s2}{Escolhas: }\PY{l+s+si}{\PYZob{}}\PY{n+nb}{tuple}\PY{p}{(}\PY{n}{escolhas}\PY{p}{)}\PY{l+s+si}{\PYZcb{}}\PY{l+s+s2}{\PYZdq{}}\PY{p}{)}
\PY{n+nb}{print}\PY{p}{(}\PY{p}{)}
\PY{n}{n\PYZus{}trasicoes} \PY{o}{=} \PY{l+m+mi}{20}
\PY{n+nb}{print}\PY{p}{(}\PY{l+s+sa}{f}\PY{l+s+s2}{\PYZdq{}}\PY{l+s+s2}{Traço de }\PY{l+s+si}{\PYZob{}}\PY{n}{n\PYZus{}trasicoes}\PY{l+s+si}{\PYZcb{}}\PY{l+s+s2}{ transições:}\PY{l+s+s2}{\PYZdq{}}\PY{p}{)}
\PY{n}{genTrace}\PY{p}{(}\PY{p}{[}\PY{l+s+s1}{\PYZsq{}}\PY{l+s+s1}{A}\PY{l+s+s1}{\PYZsq{}}\PY{p}{,}\PY{l+s+s1}{\PYZsq{}}\PY{l+s+s1}{B}\PY{l+s+s1}{\PYZsq{}}\PY{p}{,}\PY{l+s+s1}{\PYZsq{}}\PY{l+s+s1}{C}\PY{l+s+s1}{\PYZsq{}}\PY{p}{,}\PY{l+s+s1}{\PYZsq{}}\PY{l+s+s1}{D}\PY{l+s+s1}{\PYZsq{}}\PY{p}{]}\PY{p}{,}\PY{n}{init}\PY{p}{,}\PY{n}{estadoInicial}\PY{p}{,}\PY{n}{trans}\PY{p}{,}\PY{n}{escolhas}\PY{p}{,}\PY{n}{error}\PY{p}{,}\PY{n}{n\PYZus{}trasicoes}\PY{p}{)}
\PY{n+nb}{print}\PY{p}{(}\PY{p}{)}
\PY{n+nb}{print}\PY{p}{(}\PY{l+s+s2}{\PYZdq{}}\PY{l+s+s2}{Verificação BMC do sistema:}\PY{l+s+s2}{\PYZdq{}}\PY{p}{)}
\PY{n}{bmc\PYZus{}always}\PY{p}{(}\PY{n}{genState}\PY{p}{,}\PY{n}{init}\PY{p}{,}\PY{n}{trans}\PY{p}{,}\PY{n}{Noerror}\PY{p}{,}\PY{l+m+mi}{20}\PY{p}{,}\PY{n}{estadoInicial}\PY{p}{,}\PY{n}{escolhas}\PY{p}{)}
\PY{n+nb}{print}\PY{p}{(}\PY{p}{)}
\PY{n+nb}{print}\PY{p}{(}\PY{l+s+s2}{\PYZdq{}}\PY{l+s+s2}{Verificação com k\PYZhy{}indução do sistema:}\PY{l+s+s2}{\PYZdq{}}\PY{p}{)}
\PY{n}{kinduction\PYZus{}always}\PY{p}{(}\PY{n}{genState}\PY{p}{,}\PY{n}{init}\PY{p}{,}\PY{n}{trans}\PY{p}{,}\PY{n}{Noerror}\PY{p}{,}\PY{l+m+mi}{16}\PY{p}{,}\PY{n}{estadoInicial}\PY{p}{,}\PY{n}{escolhas}\PY{p}{)}
\PY{n+nb}{print}\PY{p}{(}\PY{p}{)}
\PY{n+nb}{print}\PY{p}{(}\PY{l+s+s2}{\PYZdq{}}\PY{l+s+s2}{Verificação model checking com interpolantes do sistema:}\PY{l+s+s2}{\PYZdq{}}\PY{p}{)}
\PY{n}{model\PYZus{}checking}\PY{p}{(}\PY{p}{[}\PY{l+s+s1}{\PYZsq{}}\PY{l+s+s1}{A}\PY{l+s+s1}{\PYZsq{}}\PY{p}{,}\PY{l+s+s1}{\PYZsq{}}\PY{l+s+s1}{B}\PY{l+s+s1}{\PYZsq{}}\PY{p}{,}\PY{l+s+s1}{\PYZsq{}}\PY{l+s+s1}{C}\PY{l+s+s1}{\PYZsq{}}\PY{p}{,}\PY{l+s+s1}{\PYZsq{}}\PY{l+s+s1}{D}\PY{l+s+s1}{\PYZsq{}}\PY{p}{]}\PY{p}{,} \PY{n}{init}\PY{p}{,}\PY{n}{trans}\PY{p}{,} \PY{n}{error}\PY{p}{,} \PY{l+m+mi}{16}\PY{p}{,} \PY{l+m+mi}{16}\PY{p}{,}\PY{n}{estadoInicial}\PY{p}{,}\PY{n}{escolhas}\PY{p}{)}
\end{Verbatim}
\end{tcolorbox}

    \begin{Verbatim}[commandchars=\\\{\}]
Estado inicial: (False, True, False, True)
Escolhas: (False, False, False, True)

Traço de 20 transições:
Estado: 0
           A = False
           B = True
           C = False
           D = True
Estado: 1
           A = False
           B = True
           C = True
           D = False
Estado: 2
           A = True
           B = True
           C = True
           D = False
Estado: 3
           A = False
           B = False
           C = True
           D = False
Estado: 4
           A = True
           B = False
           C = True
           D = True
Estado: 5
           A = False
           B = True
           C = False
           D = True
Estado: 6
           A = False
           B = True
           C = True
           D = False
Estado: 7
           A = True
           B = True
           C = True
           D = False
Estado: 8
           A = False
           B = False
           C = True
           D = False
Estado: 9
           A = True
           B = False
           C = True
           D = True
Estado: 10
           A = False
           B = True
           C = False
           D = True
Estado: 11
           A = False
           B = True
           C = True
           D = False
Estado: 12
           A = True
           B = True
           C = True
           D = False
Estado: 13
           A = False
           B = False
           C = True
           D = False
Estado: 14
           A = True
           B = False
           C = True
           D = True
Estado: 15
           A = False
           B = True
           C = False
           D = True
Estado: 16
           A = False
           B = True
           C = True
           D = False
Estado: 17
           A = True
           B = True
           C = True
           D = False
Estado: 18
           A = False
           B = False
           C = True
           D = False
Estado: 19
           A = True
           B = False
           C = True
           D = True

Verificação BMC do sistema:
A propriedade é válida para traços de tamanho até 20

Verificação com k-indução do sistema:
A propriedade é sempre válida

Verificação model checking com interpolantes do sistema:
Não foi possível encontrar o majorante.
Não foi possível encontrar o majorante.
Não foi possível encontrar o majorante.
Não foi possível encontrar o majorante.
Não foi possível encontrar o majorante.
Não foi possível encontrar o majorante.
Não foi possível encontrar o majorante.
Não foi possível encontrar o majorante.
Não foi possível encontrar o majorante.
Não foi possível encontrar o majorante.
Não foi possível encontrar o majorante.
Não foi possível encontrar o majorante.
Não foi possível encontrar o majorante.
Não foi possível encontrar o majorante.
Não foi possível encontrar o majorante.
Não foi possível encontrar o majorante.
Não foi possível encontrar o majorante.
Não foi possível encontrar o majorante.
Não foi possível encontrar o majorante.
Não foi possível encontrar o majorante.
Não foi possível encontrar o majorante.
Não foi possível encontrar o majorante.
Não foi possível encontrar o majorante.
Não foi possível encontrar o majorante.
Não foi possível encontrar o majorante.
Não foi possível encontrar o majorante.
Não foi possível encontrar o majorante.
Não foi possível encontrar o majorante.
Não foi possível encontrar o majorante.
Não foi possível encontrar o majorante.
Não foi possível encontrar o majorante.
Não foi possível encontrar o majorante.
Não foi possível encontrar o majorante.
Não foi possível encontrar o majorante.
Não foi possível encontrar o majorante.
Não foi possível encontrar o majorante.
Não foi possível encontrar o majorante.
Não foi possível encontrar o majorante.
Não foi possível encontrar o majorante.
Não foi possível encontrar o majorante.
Não foi possível encontrar o majorante.
Não foi possível encontrar o majorante.
Não foi possível encontrar o majorante.
Não foi possível encontrar o majorante.
Não foi possível encontrar o majorante.
Não foi possível encontrar o majorante.
Não foi possível encontrar o majorante.
Não foi possível encontrar o majorante.
Não foi possível encontrar o majorante.
Não foi possível encontrar o majorante.
Não foi possível encontrar o majorante.
Não foi possível encontrar o majorante.
Não foi possível encontrar o majorante.
Não foi possível encontrar o majorante.
Não foi possível encontrar o majorante.
Não foi possível encontrar o majorante.
Não foi possível encontrar o majorante.
Não foi possível encontrar o majorante.
Não foi possível encontrar o majorante.
Não foi possível encontrar o majorante.
Não foi possível encontrar o majorante.
Não foi possível encontrar o majorante.
Não foi possível encontrar o majorante.
Não foi possível encontrar o majorante.
Não foi possível encontrar o majorante.
Não foi possível encontrar o majorante.
Não foi possível encontrar o majorante.
Não foi possível encontrar o majorante.
Não foi possível encontrar o majorante.
Não foi possível encontrar o majorante.
Não foi possível encontrar o majorante.
Não foi possível encontrar o majorante.
Não foi possível encontrar o majorante.
Não foi possível encontrar o majorante.
Não foi possível encontrar o majorante.
Não foi possível encontrar o majorante.
Não foi possível encontrar o majorante.
Não foi possível encontrar o majorante.
Não foi possível encontrar o majorante.
Não foi possível encontrar o majorante.
Não foi possível encontrar o majorante.
Não foi possível encontrar o majorante.
Não foi possível encontrar o majorante.
Não foi possível encontrar o majorante.
Não foi possível encontrar o majorante.
Não foi possível encontrar o majorante.
Não foi possível encontrar o majorante.
Não foi possível encontrar o majorante.
Não foi possível encontrar o majorante.
Não foi possível encontrar o majorante.
Não foi possível encontrar o majorante.
Não foi possível encontrar o majorante.
Não foi possível encontrar o majorante.
Não foi possível encontrar o majorante.
Não foi possível encontrar o majorante.
Não foi possível encontrar o majorante.
Não foi possível encontrar o majorante.
Não foi possível encontrar o majorante.
Não foi possível encontrar o majorante.
Não foi possível encontrar o majorante.
Não foi possível encontrar o majorante.
Não foi possível encontrar o majorante.
Não foi possível encontrar o majorante.
Não foi possível encontrar o majorante.
Não foi possível encontrar o majorante.
Não foi possível encontrar o majorante.
Não foi possível encontrar o majorante.
Não foi possível encontrar o majorante.
Não foi possível encontrar o majorante.
Não foi possível encontrar o majorante.
Não foi possível encontrar o majorante.
Não foi possível encontrar o majorante.
Não foi possível encontrar o majorante.
Não foi possível encontrar o majorante.
Não foi possível encontrar o majorante.
Não foi possível encontrar o majorante.
Não foi possível encontrar o majorante.
Não foi possível encontrar o majorante.
Não foi possível encontrar o majorante.
Não foi possível encontrar o majorante.
Não foi possível encontrar o majorante.
Não foi possível encontrar o majorante.
Não foi possível encontrar o majorante.
Não foi possível encontrar o majorante.
Não foi possível encontrar o majorante.
Não foi possível encontrar o majorante.
Não foi possível encontrar o majorante.
Não foi possível encontrar o majorante.
Não foi possível encontrar o majorante.
Não foi possível encontrar o majorante.
Não foi possível encontrar o majorante.
Não foi possível encontrar o majorante.
Não foi possível encontrar o majorante.
Não foi possível encontrar o majorante.
Não foi possível encontrar o majorante.
Não foi possível encontrar o majorante.
Não foi possível encontrar o majorante.
Não foi possível encontrar o majorante.
Não foi possível encontrar o majorante.
Não foi possível encontrar o majorante.
Não foi possível encontrar o majorante.
Não foi possível encontrar o majorante.
Não foi possível encontrar o majorante.
Não foi possível encontrar o majorante.
Não foi possível encontrar o majorante.
Não foi possível encontrar o majorante.
Não foi possível encontrar o majorante.
Não foi possível encontrar o majorante.
Não foi possível encontrar o majorante.
Não foi possível encontrar o majorante.
Não foi possível encontrar o majorante.
Não foi possível encontrar o majorante.
Não foi possível encontrar o majorante.
Não foi possível encontrar o majorante.
Não foi possível encontrar o majorante.
Não foi possível encontrar o majorante.
Não foi possível encontrar o majorante.
Não foi possível encontrar o majorante.
Não foi possível encontrar o majorante.
Não foi possível encontrar o majorante.
Não foi possível encontrar o majorante.
Não foi possível encontrar o majorante.
Não foi possível encontrar o majorante.
Não foi possível encontrar o majorante.
Não foi possível encontrar o majorante.
Não foi possível encontrar o majorante.
Não foi possível encontrar o majorante.
Não foi possível encontrar o majorante.
Não foi possível encontrar o majorante.
Não foi possível encontrar o majorante.
Não foi possível encontrar o majorante.
Não foi possível encontrar o majorante.
Não foi possível encontrar o majorante.
Não foi possível encontrar o majorante.
Não foi possível encontrar o majorante.
Não foi possível encontrar o majorante.
Não foi possível encontrar o majorante.
Não foi possível encontrar o majorante.
Não foi possível encontrar o majorante.
Não foi possível encontrar o majorante.
Não foi possível encontrar o majorante.
Não foi possível encontrar o majorante.
Não foi possível encontrar o majorante.
Não foi possível encontrar o majorante.
Não foi possível encontrar o majorante.
Não foi possível encontrar o majorante.
Não foi possível encontrar o majorante.
Não foi possível encontrar o majorante.
Não foi possível encontrar o majorante.
Não foi possível encontrar o majorante.
Não foi possível encontrar o majorante.
Não foi possível encontrar o majorante.
Não foi possível encontrar o majorante.
Não foi possível encontrar o majorante.
Não foi possível encontrar o majorante.
Não foi possível encontrar o majorante.
Não foi possível encontrar o majorante.
Não foi possível encontrar o majorante.
Não foi possível encontrar o majorante.
Não foi possível encontrar o majorante.
Não foi possível encontrar o majorante.
Não foi possível encontrar o majorante.
Não foi possível encontrar o majorante.
Não foi possível encontrar o majorante.
Não foi possível encontrar o majorante.
Não foi possível encontrar o majorante.
Não foi possível encontrar o majorante.
Não foi possível encontrar o majorante.
Não foi possível encontrar o majorante.
Não foi possível encontrar o majorante.
Não foi possível encontrar o majorante.
Não foi possível encontrar o majorante.
Não foi possível encontrar o majorante.
Não foi possível encontrar o majorante.
Não foi possível encontrar o majorante.
Não foi possível encontrar o majorante.
Não foi possível encontrar o majorante.
Não foi possível encontrar o majorante.
Não foi possível encontrar o majorante.
Não foi possível encontrar o majorante.
Não foi possível encontrar o majorante.
Não foi possível encontrar o majorante.
Não foi possível encontrar o majorante.
Não foi possível encontrar o majorante.
Não foi possível encontrar o majorante.
Não foi possível encontrar o majorante.
Não foi possível encontrar o majorante.
Não foi possível encontrar o majorante.
Não foi possível encontrar o majorante.
Não foi possível encontrar o majorante.
Não foi possível encontrar o majorante.
Não foi possível encontrar o majorante.
Não foi possível encontrar o majorante.
Não foi possível encontrar o majorante.
Não foi possível encontrar o majorante.
Não foi possível encontrar o majorante.
Não foi possível encontrar o majorante.
Não foi possível encontrar o majorante.
Não foi possível encontrar o majorante.
Não foi possível encontrar o majorante.
Não foi possível encontrar o majorante.
Não foi possível encontrar o majorante.
Não foi possível encontrar o majorante.
Não foi possível encontrar o majorante.
Não foi possível encontrar o majorante.
Não foi possível encontrar o majorante.
Não foi possível encontrar o majorante.
Não foi possível encontrar o majorante.
Não foi possível encontrar o majorante.
Não foi possível encontrar o majorante.
Não foi possível encontrar o majorante.
Não foi possível encontrar o majorante.
Não foi possível encontrar o majorante.
Não foi possível encontrar o majorante.
Não foi possível encontrar o majorante.
Não foi possível encontrar o majorante.
unknown
    \end{Verbatim}

    \hypertarget{exemplo-5}{%
\paragraph{Exemplo 5}\label{exemplo-5}}

    Outro exemplo para ``unsafe''

    \begin{tcolorbox}[breakable, size=fbox, boxrule=1pt, pad at break*=1mm,colback=cellbackground, colframe=cellborder]
\prompt{In}{incolor}{16}{\boxspacing}
\begin{Verbatim}[commandchars=\\\{\}]
\PY{n}{estadoInicial} \PY{o}{=} \PY{p}{[}\PY{n}{rn}\PY{o}{.}\PY{n}{choice}\PY{p}{(}\PY{p}{[}\PY{k+kc}{True}\PY{p}{,}\PY{k+kc}{False}\PY{p}{]}\PY{p}{)} \PY{k}{for} \PY{n}{i} \PY{o+ow}{in} \PY{n+nb}{range}\PY{p}{(}\PY{l+m+mi}{4}\PY{p}{)}\PY{p}{]}
\PY{n+nb}{print}\PY{p}{(}\PY{l+s+sa}{f}\PY{l+s+s2}{\PYZdq{}}\PY{l+s+s2}{Estado inicial: }\PY{l+s+si}{\PYZob{}}\PY{n+nb}{tuple}\PY{p}{(}\PY{n}{estadoInicial}\PY{p}{)}\PY{l+s+si}{\PYZcb{}}\PY{l+s+s2}{\PYZdq{}}\PY{p}{)}
\PY{n}{escolhas} \PY{o}{=} \PY{p}{[}\PY{n}{rn}\PY{o}{.}\PY{n}{choice}\PY{p}{(}\PY{p}{[}\PY{k+kc}{True}\PY{p}{,}\PY{k+kc}{False}\PY{p}{]}\PY{p}{)} \PY{k}{for} \PY{n}{i} \PY{o+ow}{in} \PY{n+nb}{range}\PY{p}{(}\PY{l+m+mi}{4}\PY{p}{)}\PY{p}{]}
\PY{n+nb}{print}\PY{p}{(}\PY{l+s+sa}{f}\PY{l+s+s2}{\PYZdq{}}\PY{l+s+s2}{Escolhas: }\PY{l+s+si}{\PYZob{}}\PY{n+nb}{tuple}\PY{p}{(}\PY{n}{escolhas}\PY{p}{)}\PY{l+s+si}{\PYZcb{}}\PY{l+s+s2}{\PYZdq{}}\PY{p}{)}
\PY{n+nb}{print}\PY{p}{(}\PY{p}{)}
\PY{n}{n\PYZus{}trasicoes} \PY{o}{=} \PY{l+m+mi}{20}
\PY{n+nb}{print}\PY{p}{(}\PY{l+s+sa}{f}\PY{l+s+s2}{\PYZdq{}}\PY{l+s+s2}{Traço de }\PY{l+s+si}{\PYZob{}}\PY{n}{n\PYZus{}trasicoes}\PY{l+s+si}{\PYZcb{}}\PY{l+s+s2}{ transições:}\PY{l+s+s2}{\PYZdq{}}\PY{p}{)}
\PY{n}{genTrace}\PY{p}{(}\PY{p}{[}\PY{l+s+s1}{\PYZsq{}}\PY{l+s+s1}{A}\PY{l+s+s1}{\PYZsq{}}\PY{p}{,}\PY{l+s+s1}{\PYZsq{}}\PY{l+s+s1}{B}\PY{l+s+s1}{\PYZsq{}}\PY{p}{,}\PY{l+s+s1}{\PYZsq{}}\PY{l+s+s1}{C}\PY{l+s+s1}{\PYZsq{}}\PY{p}{,}\PY{l+s+s1}{\PYZsq{}}\PY{l+s+s1}{D}\PY{l+s+s1}{\PYZsq{}}\PY{p}{]}\PY{p}{,}\PY{n}{init}\PY{p}{,}\PY{n}{estadoInicial}\PY{p}{,}\PY{n}{trans}\PY{p}{,}\PY{n}{escolhas}\PY{p}{,}\PY{n}{error}\PY{p}{,}\PY{n}{n\PYZus{}trasicoes}\PY{p}{)}
\PY{n+nb}{print}\PY{p}{(}\PY{p}{)}
\PY{n+nb}{print}\PY{p}{(}\PY{l+s+s2}{\PYZdq{}}\PY{l+s+s2}{Verificação BMC do sistema:}\PY{l+s+s2}{\PYZdq{}}\PY{p}{)}
\PY{n}{bmc\PYZus{}always}\PY{p}{(}\PY{n}{genState}\PY{p}{,}\PY{n}{init}\PY{p}{,}\PY{n}{trans}\PY{p}{,}\PY{n}{Noerror}\PY{p}{,}\PY{l+m+mi}{20}\PY{p}{,}\PY{n}{estadoInicial}\PY{p}{,}\PY{n}{escolhas}\PY{p}{)}
\PY{n+nb}{print}\PY{p}{(}\PY{p}{)}
\PY{n+nb}{print}\PY{p}{(}\PY{l+s+s2}{\PYZdq{}}\PY{l+s+s2}{Verificação com k\PYZhy{}indução do sistema:}\PY{l+s+s2}{\PYZdq{}}\PY{p}{)}
\PY{n}{kinduction\PYZus{}always}\PY{p}{(}\PY{n}{genState}\PY{p}{,}\PY{n}{init}\PY{p}{,}\PY{n}{trans}\PY{p}{,}\PY{n}{Noerror}\PY{p}{,}\PY{l+m+mi}{16}\PY{p}{,}\PY{n}{estadoInicial}\PY{p}{,}\PY{n}{escolhas}\PY{p}{)}
\PY{n+nb}{print}\PY{p}{(}\PY{p}{)}
\PY{n+nb}{print}\PY{p}{(}\PY{l+s+s2}{\PYZdq{}}\PY{l+s+s2}{Verificação model checking com interpolantes do sistema:}\PY{l+s+s2}{\PYZdq{}}\PY{p}{)}
\PY{n}{model\PYZus{}checking}\PY{p}{(}\PY{p}{[}\PY{l+s+s1}{\PYZsq{}}\PY{l+s+s1}{A}\PY{l+s+s1}{\PYZsq{}}\PY{p}{,}\PY{l+s+s1}{\PYZsq{}}\PY{l+s+s1}{B}\PY{l+s+s1}{\PYZsq{}}\PY{p}{,}\PY{l+s+s1}{\PYZsq{}}\PY{l+s+s1}{C}\PY{l+s+s1}{\PYZsq{}}\PY{p}{,}\PY{l+s+s1}{\PYZsq{}}\PY{l+s+s1}{D}\PY{l+s+s1}{\PYZsq{}}\PY{p}{]}\PY{p}{,} \PY{n}{init}\PY{p}{,}\PY{n}{trans}\PY{p}{,} \PY{n}{error}\PY{p}{,} \PY{l+m+mi}{16}\PY{p}{,} \PY{l+m+mi}{16}\PY{p}{,}\PY{n}{estadoInicial}\PY{p}{,}\PY{n}{escolhas}\PY{p}{)}
\end{Verbatim}
\end{tcolorbox}

    \begin{Verbatim}[commandchars=\\\{\}]
Estado inicial: (False, True, False, False)
Escolhas: (True, False, True, True)

Traço de 20 transições:
Estado: 0
           A = False
           B = True
           C = False
           D = False
Estado: 1
           A = True
           B = True
           C = True
           D = False
Estado: 2
           A = False
           B = False
           C = True
           D = False
Estado: 3
           A = False
           B = False
           C = True
           D = True
Estado: 4
           A = False
           B = False
           C = False
           D = True
Estado: 5
           A = True
           B = False
           C = False
           D = True
Estado: 6
           A = True
           B = True
           C = False
           D = True
Estado: 7
           A = True
           B = False
           C = False
           D = False
Estado: 8
           A = True
           B = True
           C = True
           D = True
Estado: 9
           A = False
           B = False
           C = False
           D = False
Estado: 10
           A = True
           B = False
           C = True
           D = True
Estado: 11
           A = False
           B = True
           C = False
           D = True
Estado: 12
           A = True
           B = True
           C = False
           D = False
Estado: 13
           A = True
           B = False
           C = True
           D = False
Estado: 14
           A = False
           B = True
           C = True
           D = True
Estado: 15
           A = False
           B = True
           C = False
           D = False
Estado: 16
           A = True
           B = True
           C = True
           D = False
Estado: 17
           A = False
           B = False
           C = True
           D = False
Estado: 18
           A = False
           B = False
           C = True
           D = True
Estado: 19
           A = False
           B = False
           C = False
           D = True

Verificação BMC do sistema:
A propriedade não é valida
0
A= False
B= True
C= False
D= False
1
A= True
B= True
C= True
D= False
2
A= False
B= False
C= True
D= False
3
A= False
B= False
C= True
D= True
4
A= False
B= False
C= False
D= True
5
A= True
B= False
C= False
D= True
6
A= True
B= True
C= False
D= True
7
A= True
B= False
C= False
D= False
8
A= True
B= True
C= True
D= True
9
A= False
B= False
C= False
D= False

Verificação com k-indução do sistema:
A propriedade não é válida nos k estados iniciais.
0
A= False
B= True
C= False
D= False
1
A= True
B= True
C= True
D= False
2
A= False
B= False
C= True
D= False
3
A= False
B= False
C= True
D= True
4
A= False
B= False
C= False
D= True
5
A= True
B= False
C= False
D= True
6
A= True
B= True
C= False
D= True
7
A= True
B= False
C= False
D= False
8
A= True
B= True
C= True
D= True
9
A= False
B= False
C= False
D= False
10
A= True
B= False
C= True
D= True
11
A= False
B= True
C= False
D= True
12
A= True
B= True
C= False
D= False
13
A= True
B= False
C= True
D= False
14
A= False
B= True
C= True
D= True
15
A= False
B= True
C= False
D= False

Verificação model checking com interpolantes do sistema:
Não foi possível encontrar o majorante.
Não foi possível encontrar o majorante.
Não foi possível encontrar o majorante.
Não foi possível encontrar o majorante.
Não foi possível encontrar o majorante.
Não foi possível encontrar o majorante.
Não foi possível encontrar o majorante.
Não foi possível encontrar o majorante.
Não foi possível encontrar o majorante.
Não foi possível encontrar o majorante.
Não foi possível encontrar o majorante.
Não foi possível encontrar o majorante.
Não foi possível encontrar o majorante.
Não foi possível encontrar o majorante.
Não foi possível encontrar o majorante.
Não foi possível encontrar o majorante.
Não foi possível encontrar o majorante.
Não foi possível encontrar o majorante.
Não foi possível encontrar o majorante.
Não foi possível encontrar o majorante.
Não foi possível encontrar o majorante.
Não foi possível encontrar o majorante.
Não foi possível encontrar o majorante.
Não foi possível encontrar o majorante.
Não foi possível encontrar o majorante.
Não foi possível encontrar o majorante.
Não foi possível encontrar o majorante.
Não foi possível encontrar o majorante.
unsafe
    \end{Verbatim}

    \begin{tcolorbox}[breakable, size=fbox, boxrule=1pt, pad at break*=1mm,colback=cellbackground, colframe=cellborder]
\prompt{In}{incolor}{ }{\boxspacing}
\begin{Verbatim}[commandchars=\\\{\}]

\end{Verbatim}
\end{tcolorbox}


    % Add a bibliography block to the postdoc
    
    
    
\end{document}
