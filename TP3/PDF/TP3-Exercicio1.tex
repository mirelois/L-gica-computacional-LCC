\documentclass[11pt]{article}

    \usepackage[breakable]{tcolorbox}
    \usepackage{parskip} % Stop auto-indenting (to mimic markdown behaviour)
    

    % Basic figure setup, for now with no caption control since it's done
    % automatically by Pandoc (which extracts ![](path) syntax from Markdown).
    \usepackage{graphicx}
    % Maintain compatibility with old templates. Remove in nbconvert 6.0
    \let\Oldincludegraphics\includegraphics
    % Ensure that by default, figures have no caption (until we provide a
    % proper Figure object with a Caption API and a way to capture that
    % in the conversion process - todo).
    \usepackage{caption}
    \DeclareCaptionFormat{nocaption}{}
    \captionsetup{format=nocaption,aboveskip=0pt,belowskip=0pt}

    \usepackage{float}
    \floatplacement{figure}{H} % forces figures to be placed at the correct location
    \usepackage{xcolor} % Allow colors to be defined
    \usepackage{enumerate} % Needed for markdown enumerations to work
    \usepackage{geometry} % Used to adjust the document margins
    \usepackage{amsmath} % Equations
    \usepackage{amssymb} % Equations
    \usepackage{textcomp} % defines textquotesingle
    % Hack from http://tex.stackexchange.com/a/47451/13684:
    \AtBeginDocument{%
        \def\PYZsq{\textquotesingle}% Upright quotes in Pygmentized code
    }
    \usepackage{upquote} % Upright quotes for verbatim code
    \usepackage{eurosym} % defines \euro

    \usepackage{iftex}
    \ifPDFTeX
        \usepackage[T1]{fontenc}
        \IfFileExists{alphabeta.sty}{
              \usepackage{alphabeta}
          }{
              \usepackage[mathletters]{ucs}
              \usepackage[utf8x]{inputenc}
          }
    \else
        \usepackage{fontspec}
        \usepackage{unicode-math}
    \fi

    \usepackage{fancyvrb} % verbatim replacement that allows latex
    \usepackage{grffile} % extends the file name processing of package graphics 
                         % to support a larger range
    \makeatletter % fix for old versions of grffile with XeLaTeX
    \@ifpackagelater{grffile}{2019/11/01}
    {
      % Do nothing on new versions
    }
    {
      \def\Gread@@xetex#1{%
        \IfFileExists{"\Gin@base".bb}%
        {\Gread@eps{\Gin@base.bb}}%
        {\Gread@@xetex@aux#1}%
      }
    }
    \makeatother
    \usepackage[Export]{adjustbox} % Used to constrain images to a maximum size
    \adjustboxset{max size={0.9\linewidth}{0.9\paperheight}}

    % The hyperref package gives us a pdf with properly built
    % internal navigation ('pdf bookmarks' for the table of contents,
    % internal cross-reference links, web links for URLs, etc.)
    \usepackage{hyperref}
    % The default LaTeX title has an obnoxious amount of whitespace. By default,
    % titling removes some of it. It also provides customization options.
    \usepackage{titling}
    \usepackage{longtable} % longtable support required by pandoc >1.10
    \usepackage{booktabs}  % table support for pandoc > 1.12.2
    \usepackage{array}     % table support for pandoc >= 2.11.3
    \usepackage{calc}      % table minipage width calculation for pandoc >= 2.11.1
    \usepackage[inline]{enumitem} % IRkernel/repr support (it uses the enumerate* environment)
    \usepackage[normalem]{ulem} % ulem is needed to support strikethroughs (\sout)
                                % normalem makes italics be italics, not underlines
    \usepackage{mathrsfs}
    

    
    % Colors for the hyperref package
    \definecolor{urlcolor}{rgb}{0,.145,.698}
    \definecolor{linkcolor}{rgb}{.71,0.21,0.01}
    \definecolor{citecolor}{rgb}{.12,.54,.11}

    % ANSI colors
    \definecolor{ansi-black}{HTML}{3E424D}
    \definecolor{ansi-black-intense}{HTML}{282C36}
    \definecolor{ansi-red}{HTML}{E75C58}
    \definecolor{ansi-red-intense}{HTML}{B22B31}
    \definecolor{ansi-green}{HTML}{00A250}
    \definecolor{ansi-green-intense}{HTML}{007427}
    \definecolor{ansi-yellow}{HTML}{DDB62B}
    \definecolor{ansi-yellow-intense}{HTML}{B27D12}
    \definecolor{ansi-blue}{HTML}{208FFB}
    \definecolor{ansi-blue-intense}{HTML}{0065CA}
    \definecolor{ansi-magenta}{HTML}{D160C4}
    \definecolor{ansi-magenta-intense}{HTML}{A03196}
    \definecolor{ansi-cyan}{HTML}{60C6C8}
    \definecolor{ansi-cyan-intense}{HTML}{258F8F}
    \definecolor{ansi-white}{HTML}{C5C1B4}
    \definecolor{ansi-white-intense}{HTML}{A1A6B2}
    \definecolor{ansi-default-inverse-fg}{HTML}{FFFFFF}
    \definecolor{ansi-default-inverse-bg}{HTML}{000000}

    % common color for the border for error outputs.
    \definecolor{outerrorbackground}{HTML}{FFDFDF}

    % commands and environments needed by pandoc snippets
    % extracted from the output of `pandoc -s`
    \providecommand{\tightlist}{%
      \setlength{\itemsep}{0pt}\setlength{\parskip}{0pt}}
    \DefineVerbatimEnvironment{Highlighting}{Verbatim}{commandchars=\\\{\}}
    % Add ',fontsize=\small' for more characters per line
    \newenvironment{Shaded}{}{}
    \newcommand{\KeywordTok}[1]{\textcolor[rgb]{0.00,0.44,0.13}{\textbf{{#1}}}}
    \newcommand{\DataTypeTok}[1]{\textcolor[rgb]{0.56,0.13,0.00}{{#1}}}
    \newcommand{\DecValTok}[1]{\textcolor[rgb]{0.25,0.63,0.44}{{#1}}}
    \newcommand{\BaseNTok}[1]{\textcolor[rgb]{0.25,0.63,0.44}{{#1}}}
    \newcommand{\FloatTok}[1]{\textcolor[rgb]{0.25,0.63,0.44}{{#1}}}
    \newcommand{\CharTok}[1]{\textcolor[rgb]{0.25,0.44,0.63}{{#1}}}
    \newcommand{\StringTok}[1]{\textcolor[rgb]{0.25,0.44,0.63}{{#1}}}
    \newcommand{\CommentTok}[1]{\textcolor[rgb]{0.38,0.63,0.69}{\textit{{#1}}}}
    \newcommand{\OtherTok}[1]{\textcolor[rgb]{0.00,0.44,0.13}{{#1}}}
    \newcommand{\AlertTok}[1]{\textcolor[rgb]{1.00,0.00,0.00}{\textbf{{#1}}}}
    \newcommand{\FunctionTok}[1]{\textcolor[rgb]{0.02,0.16,0.49}{{#1}}}
    \newcommand{\RegionMarkerTok}[1]{{#1}}
    \newcommand{\ErrorTok}[1]{\textcolor[rgb]{1.00,0.00,0.00}{\textbf{{#1}}}}
    \newcommand{\NormalTok}[1]{{#1}}
    
    % Additional commands for more recent versions of Pandoc
    \newcommand{\ConstantTok}[1]{\textcolor[rgb]{0.53,0.00,0.00}{{#1}}}
    \newcommand{\SpecialCharTok}[1]{\textcolor[rgb]{0.25,0.44,0.63}{{#1}}}
    \newcommand{\VerbatimStringTok}[1]{\textcolor[rgb]{0.25,0.44,0.63}{{#1}}}
    \newcommand{\SpecialStringTok}[1]{\textcolor[rgb]{0.73,0.40,0.53}{{#1}}}
    \newcommand{\ImportTok}[1]{{#1}}
    \newcommand{\DocumentationTok}[1]{\textcolor[rgb]{0.73,0.13,0.13}{\textit{{#1}}}}
    \newcommand{\AnnotationTok}[1]{\textcolor[rgb]{0.38,0.63,0.69}{\textbf{\textit{{#1}}}}}
    \newcommand{\CommentVarTok}[1]{\textcolor[rgb]{0.38,0.63,0.69}{\textbf{\textit{{#1}}}}}
    \newcommand{\VariableTok}[1]{\textcolor[rgb]{0.10,0.09,0.49}{{#1}}}
    \newcommand{\ControlFlowTok}[1]{\textcolor[rgb]{0.00,0.44,0.13}{\textbf{{#1}}}}
    \newcommand{\OperatorTok}[1]{\textcolor[rgb]{0.40,0.40,0.40}{{#1}}}
    \newcommand{\BuiltInTok}[1]{{#1}}
    \newcommand{\ExtensionTok}[1]{{#1}}
    \newcommand{\PreprocessorTok}[1]{\textcolor[rgb]{0.74,0.48,0.00}{{#1}}}
    \newcommand{\AttributeTok}[1]{\textcolor[rgb]{0.49,0.56,0.16}{{#1}}}
    \newcommand{\InformationTok}[1]{\textcolor[rgb]{0.38,0.63,0.69}{\textbf{\textit{{#1}}}}}
    \newcommand{\WarningTok}[1]{\textcolor[rgb]{0.38,0.63,0.69}{\textbf{\textit{{#1}}}}}
    
    
    % Define a nice break command that doesn't care if a line doesn't already
    % exist.
    \def\br{\hspace*{\fill} \\* }
    % Math Jax compatibility definitions
    \def\gt{>}
    \def\lt{<}
    \let\Oldtex\TeX
    \let\Oldlatex\LaTeX
    \renewcommand{\TeX}{\textrm{\Oldtex}}
    \renewcommand{\LaTeX}{\textrm{\Oldlatex}}
    % Document parameters
    % Document title
    \title{TP3-Exercicio1}
    
    
    
    
    
% Pygments definitions
\makeatletter
\def\PY@reset{\let\PY@it=\relax \let\PY@bf=\relax%
    \let\PY@ul=\relax \let\PY@tc=\relax%
    \let\PY@bc=\relax \let\PY@ff=\relax}
\def\PY@tok#1{\csname PY@tok@#1\endcsname}
\def\PY@toks#1+{\ifx\relax#1\empty\else%
    \PY@tok{#1}\expandafter\PY@toks\fi}
\def\PY@do#1{\PY@bc{\PY@tc{\PY@ul{%
    \PY@it{\PY@bf{\PY@ff{#1}}}}}}}
\def\PY#1#2{\PY@reset\PY@toks#1+\relax+\PY@do{#2}}

\@namedef{PY@tok@w}{\def\PY@tc##1{\textcolor[rgb]{0.73,0.73,0.73}{##1}}}
\@namedef{PY@tok@c}{\let\PY@it=\textit\def\PY@tc##1{\textcolor[rgb]{0.24,0.48,0.48}{##1}}}
\@namedef{PY@tok@cp}{\def\PY@tc##1{\textcolor[rgb]{0.61,0.40,0.00}{##1}}}
\@namedef{PY@tok@k}{\let\PY@bf=\textbf\def\PY@tc##1{\textcolor[rgb]{0.00,0.50,0.00}{##1}}}
\@namedef{PY@tok@kp}{\def\PY@tc##1{\textcolor[rgb]{0.00,0.50,0.00}{##1}}}
\@namedef{PY@tok@kt}{\def\PY@tc##1{\textcolor[rgb]{0.69,0.00,0.25}{##1}}}
\@namedef{PY@tok@o}{\def\PY@tc##1{\textcolor[rgb]{0.40,0.40,0.40}{##1}}}
\@namedef{PY@tok@ow}{\let\PY@bf=\textbf\def\PY@tc##1{\textcolor[rgb]{0.67,0.13,1.00}{##1}}}
\@namedef{PY@tok@nb}{\def\PY@tc##1{\textcolor[rgb]{0.00,0.50,0.00}{##1}}}
\@namedef{PY@tok@nf}{\def\PY@tc##1{\textcolor[rgb]{0.00,0.00,1.00}{##1}}}
\@namedef{PY@tok@nc}{\let\PY@bf=\textbf\def\PY@tc##1{\textcolor[rgb]{0.00,0.00,1.00}{##1}}}
\@namedef{PY@tok@nn}{\let\PY@bf=\textbf\def\PY@tc##1{\textcolor[rgb]{0.00,0.00,1.00}{##1}}}
\@namedef{PY@tok@ne}{\let\PY@bf=\textbf\def\PY@tc##1{\textcolor[rgb]{0.80,0.25,0.22}{##1}}}
\@namedef{PY@tok@nv}{\def\PY@tc##1{\textcolor[rgb]{0.10,0.09,0.49}{##1}}}
\@namedef{PY@tok@no}{\def\PY@tc##1{\textcolor[rgb]{0.53,0.00,0.00}{##1}}}
\@namedef{PY@tok@nl}{\def\PY@tc##1{\textcolor[rgb]{0.46,0.46,0.00}{##1}}}
\@namedef{PY@tok@ni}{\let\PY@bf=\textbf\def\PY@tc##1{\textcolor[rgb]{0.44,0.44,0.44}{##1}}}
\@namedef{PY@tok@na}{\def\PY@tc##1{\textcolor[rgb]{0.41,0.47,0.13}{##1}}}
\@namedef{PY@tok@nt}{\let\PY@bf=\textbf\def\PY@tc##1{\textcolor[rgb]{0.00,0.50,0.00}{##1}}}
\@namedef{PY@tok@nd}{\def\PY@tc##1{\textcolor[rgb]{0.67,0.13,1.00}{##1}}}
\@namedef{PY@tok@s}{\def\PY@tc##1{\textcolor[rgb]{0.73,0.13,0.13}{##1}}}
\@namedef{PY@tok@sd}{\let\PY@it=\textit\def\PY@tc##1{\textcolor[rgb]{0.73,0.13,0.13}{##1}}}
\@namedef{PY@tok@si}{\let\PY@bf=\textbf\def\PY@tc##1{\textcolor[rgb]{0.64,0.35,0.47}{##1}}}
\@namedef{PY@tok@se}{\let\PY@bf=\textbf\def\PY@tc##1{\textcolor[rgb]{0.67,0.36,0.12}{##1}}}
\@namedef{PY@tok@sr}{\def\PY@tc##1{\textcolor[rgb]{0.64,0.35,0.47}{##1}}}
\@namedef{PY@tok@ss}{\def\PY@tc##1{\textcolor[rgb]{0.10,0.09,0.49}{##1}}}
\@namedef{PY@tok@sx}{\def\PY@tc##1{\textcolor[rgb]{0.00,0.50,0.00}{##1}}}
\@namedef{PY@tok@m}{\def\PY@tc##1{\textcolor[rgb]{0.40,0.40,0.40}{##1}}}
\@namedef{PY@tok@gh}{\let\PY@bf=\textbf\def\PY@tc##1{\textcolor[rgb]{0.00,0.00,0.50}{##1}}}
\@namedef{PY@tok@gu}{\let\PY@bf=\textbf\def\PY@tc##1{\textcolor[rgb]{0.50,0.00,0.50}{##1}}}
\@namedef{PY@tok@gd}{\def\PY@tc##1{\textcolor[rgb]{0.63,0.00,0.00}{##1}}}
\@namedef{PY@tok@gi}{\def\PY@tc##1{\textcolor[rgb]{0.00,0.52,0.00}{##1}}}
\@namedef{PY@tok@gr}{\def\PY@tc##1{\textcolor[rgb]{0.89,0.00,0.00}{##1}}}
\@namedef{PY@tok@ge}{\let\PY@it=\textit}
\@namedef{PY@tok@gs}{\let\PY@bf=\textbf}
\@namedef{PY@tok@gp}{\let\PY@bf=\textbf\def\PY@tc##1{\textcolor[rgb]{0.00,0.00,0.50}{##1}}}
\@namedef{PY@tok@go}{\def\PY@tc##1{\textcolor[rgb]{0.44,0.44,0.44}{##1}}}
\@namedef{PY@tok@gt}{\def\PY@tc##1{\textcolor[rgb]{0.00,0.27,0.87}{##1}}}
\@namedef{PY@tok@err}{\def\PY@bc##1{{\setlength{\fboxsep}{\string -\fboxrule}\fcolorbox[rgb]{1.00,0.00,0.00}{1,1,1}{\strut ##1}}}}
\@namedef{PY@tok@kc}{\let\PY@bf=\textbf\def\PY@tc##1{\textcolor[rgb]{0.00,0.50,0.00}{##1}}}
\@namedef{PY@tok@kd}{\let\PY@bf=\textbf\def\PY@tc##1{\textcolor[rgb]{0.00,0.50,0.00}{##1}}}
\@namedef{PY@tok@kn}{\let\PY@bf=\textbf\def\PY@tc##1{\textcolor[rgb]{0.00,0.50,0.00}{##1}}}
\@namedef{PY@tok@kr}{\let\PY@bf=\textbf\def\PY@tc##1{\textcolor[rgb]{0.00,0.50,0.00}{##1}}}
\@namedef{PY@tok@bp}{\def\PY@tc##1{\textcolor[rgb]{0.00,0.50,0.00}{##1}}}
\@namedef{PY@tok@fm}{\def\PY@tc##1{\textcolor[rgb]{0.00,0.00,1.00}{##1}}}
\@namedef{PY@tok@vc}{\def\PY@tc##1{\textcolor[rgb]{0.10,0.09,0.49}{##1}}}
\@namedef{PY@tok@vg}{\def\PY@tc##1{\textcolor[rgb]{0.10,0.09,0.49}{##1}}}
\@namedef{PY@tok@vi}{\def\PY@tc##1{\textcolor[rgb]{0.10,0.09,0.49}{##1}}}
\@namedef{PY@tok@vm}{\def\PY@tc##1{\textcolor[rgb]{0.10,0.09,0.49}{##1}}}
\@namedef{PY@tok@sa}{\def\PY@tc##1{\textcolor[rgb]{0.73,0.13,0.13}{##1}}}
\@namedef{PY@tok@sb}{\def\PY@tc##1{\textcolor[rgb]{0.73,0.13,0.13}{##1}}}
\@namedef{PY@tok@sc}{\def\PY@tc##1{\textcolor[rgb]{0.73,0.13,0.13}{##1}}}
\@namedef{PY@tok@dl}{\def\PY@tc##1{\textcolor[rgb]{0.73,0.13,0.13}{##1}}}
\@namedef{PY@tok@s2}{\def\PY@tc##1{\textcolor[rgb]{0.73,0.13,0.13}{##1}}}
\@namedef{PY@tok@sh}{\def\PY@tc##1{\textcolor[rgb]{0.73,0.13,0.13}{##1}}}
\@namedef{PY@tok@s1}{\def\PY@tc##1{\textcolor[rgb]{0.73,0.13,0.13}{##1}}}
\@namedef{PY@tok@mb}{\def\PY@tc##1{\textcolor[rgb]{0.40,0.40,0.40}{##1}}}
\@namedef{PY@tok@mf}{\def\PY@tc##1{\textcolor[rgb]{0.40,0.40,0.40}{##1}}}
\@namedef{PY@tok@mh}{\def\PY@tc##1{\textcolor[rgb]{0.40,0.40,0.40}{##1}}}
\@namedef{PY@tok@mi}{\def\PY@tc##1{\textcolor[rgb]{0.40,0.40,0.40}{##1}}}
\@namedef{PY@tok@il}{\def\PY@tc##1{\textcolor[rgb]{0.40,0.40,0.40}{##1}}}
\@namedef{PY@tok@mo}{\def\PY@tc##1{\textcolor[rgb]{0.40,0.40,0.40}{##1}}}
\@namedef{PY@tok@ch}{\let\PY@it=\textit\def\PY@tc##1{\textcolor[rgb]{0.24,0.48,0.48}{##1}}}
\@namedef{PY@tok@cm}{\let\PY@it=\textit\def\PY@tc##1{\textcolor[rgb]{0.24,0.48,0.48}{##1}}}
\@namedef{PY@tok@cpf}{\let\PY@it=\textit\def\PY@tc##1{\textcolor[rgb]{0.24,0.48,0.48}{##1}}}
\@namedef{PY@tok@c1}{\let\PY@it=\textit\def\PY@tc##1{\textcolor[rgb]{0.24,0.48,0.48}{##1}}}
\@namedef{PY@tok@cs}{\let\PY@it=\textit\def\PY@tc##1{\textcolor[rgb]{0.24,0.48,0.48}{##1}}}

\def\PYZbs{\char`\\}
\def\PYZus{\char`\_}
\def\PYZob{\char`\{}
\def\PYZcb{\char`\}}
\def\PYZca{\char`\^}
\def\PYZam{\char`\&}
\def\PYZlt{\char`\<}
\def\PYZgt{\char`\>}
\def\PYZsh{\char`\#}
\def\PYZpc{\char`\%}
\def\PYZdl{\char`\$}
\def\PYZhy{\char`\-}
\def\PYZsq{\char`\'}
\def\PYZdq{\char`\"}
\def\PYZti{\char`\~}
% for compatibility with earlier versions
\def\PYZat{@}
\def\PYZlb{[}
\def\PYZrb{]}
\makeatother


    % For linebreaks inside Verbatim environment from package fancyvrb. 
    \makeatletter
        \newbox\Wrappedcontinuationbox 
        \newbox\Wrappedvisiblespacebox 
        \newcommand*\Wrappedvisiblespace {\textcolor{red}{\textvisiblespace}} 
        \newcommand*\Wrappedcontinuationsymbol {\textcolor{red}{\llap{\tiny$\m@th\hookrightarrow$}}} 
        \newcommand*\Wrappedcontinuationindent {3ex } 
        \newcommand*\Wrappedafterbreak {\kern\Wrappedcontinuationindent\copy\Wrappedcontinuationbox} 
        % Take advantage of the already applied Pygments mark-up to insert 
        % potential linebreaks for TeX processing. 
        %        {, <, #, %, $, ' and ": go to next line. 
        %        _, }, ^, &, >, - and ~: stay at end of broken line. 
        % Use of \textquotesingle for straight quote. 
        \newcommand*\Wrappedbreaksatspecials {% 
            \def\PYGZus{\discretionary{\char`\_}{\Wrappedafterbreak}{\char`\_}}% 
            \def\PYGZob{\discretionary{}{\Wrappedafterbreak\char`\{}{\char`\{}}% 
            \def\PYGZcb{\discretionary{\char`\}}{\Wrappedafterbreak}{\char`\}}}% 
            \def\PYGZca{\discretionary{\char`\^}{\Wrappedafterbreak}{\char`\^}}% 
            \def\PYGZam{\discretionary{\char`\&}{\Wrappedafterbreak}{\char`\&}}% 
            \def\PYGZlt{\discretionary{}{\Wrappedafterbreak\char`\<}{\char`\<}}% 
            \def\PYGZgt{\discretionary{\char`\>}{\Wrappedafterbreak}{\char`\>}}% 
            \def\PYGZsh{\discretionary{}{\Wrappedafterbreak\char`\#}{\char`\#}}% 
            \def\PYGZpc{\discretionary{}{\Wrappedafterbreak\char`\%}{\char`\%}}% 
            \def\PYGZdl{\discretionary{}{\Wrappedafterbreak\char`\$}{\char`\$}}% 
            \def\PYGZhy{\discretionary{\char`\-}{\Wrappedafterbreak}{\char`\-}}% 
            \def\PYGZsq{\discretionary{}{\Wrappedafterbreak\textquotesingle}{\textquotesingle}}% 
            \def\PYGZdq{\discretionary{}{\Wrappedafterbreak\char`\"}{\char`\"}}% 
            \def\PYGZti{\discretionary{\char`\~}{\Wrappedafterbreak}{\char`\~}}% 
        } 
        % Some characters . , ; ? ! / are not pygmentized. 
        % This macro makes them "active" and they will insert potential linebreaks 
        \newcommand*\Wrappedbreaksatpunct {% 
            \lccode`\~`\.\lowercase{\def~}{\discretionary{\hbox{\char`\.}}{\Wrappedafterbreak}{\hbox{\char`\.}}}% 
            \lccode`\~`\,\lowercase{\def~}{\discretionary{\hbox{\char`\,}}{\Wrappedafterbreak}{\hbox{\char`\,}}}% 
            \lccode`\~`\;\lowercase{\def~}{\discretionary{\hbox{\char`\;}}{\Wrappedafterbreak}{\hbox{\char`\;}}}% 
            \lccode`\~`\:\lowercase{\def~}{\discretionary{\hbox{\char`\:}}{\Wrappedafterbreak}{\hbox{\char`\:}}}% 
            \lccode`\~`\?\lowercase{\def~}{\discretionary{\hbox{\char`\?}}{\Wrappedafterbreak}{\hbox{\char`\?}}}% 
            \lccode`\~`\!\lowercase{\def~}{\discretionary{\hbox{\char`\!}}{\Wrappedafterbreak}{\hbox{\char`\!}}}% 
            \lccode`\~`\/\lowercase{\def~}{\discretionary{\hbox{\char`\/}}{\Wrappedafterbreak}{\hbox{\char`\/}}}% 
            \catcode`\.\active
            \catcode`\,\active 
            \catcode`\;\active
            \catcode`\:\active
            \catcode`\?\active
            \catcode`\!\active
            \catcode`\/\active 
            \lccode`\~`\~ 	
        }
    \makeatother

    \let\OriginalVerbatim=\Verbatim
    \makeatletter
    \renewcommand{\Verbatim}[1][1]{%
        %\parskip\z@skip
        \sbox\Wrappedcontinuationbox {\Wrappedcontinuationsymbol}%
        \sbox\Wrappedvisiblespacebox {\FV@SetupFont\Wrappedvisiblespace}%
        \def\FancyVerbFormatLine ##1{\hsize\linewidth
            \vtop{\raggedright\hyphenpenalty\z@\exhyphenpenalty\z@
                \doublehyphendemerits\z@\finalhyphendemerits\z@
                \strut ##1\strut}%
        }%
        % If the linebreak is at a space, the latter will be displayed as visible
        % space at end of first line, and a continuation symbol starts next line.
        % Stretch/shrink are however usually zero for typewriter font.
        \def\FV@Space {%
            \nobreak\hskip\z@ plus\fontdimen3\font minus\fontdimen4\font
            \discretionary{\copy\Wrappedvisiblespacebox}{\Wrappedafterbreak}
            {\kern\fontdimen2\font}%
        }%
        
        % Allow breaks at special characters using \PYG... macros.
        \Wrappedbreaksatspecials
        % Breaks at punctuation characters . , ; ? ! and / need catcode=\active 	
        \OriginalVerbatim[#1,codes*=\Wrappedbreaksatpunct]%
    }
    \makeatother

    % Exact colors from NB
    \definecolor{incolor}{HTML}{303F9F}
    \definecolor{outcolor}{HTML}{D84315}
    \definecolor{cellborder}{HTML}{CFCFCF}
    \definecolor{cellbackground}{HTML}{F7F7F7}
    
    % prompt
    \makeatletter
    \newcommand{\boxspacing}{\kern\kvtcb@left@rule\kern\kvtcb@boxsep}
    \makeatother
    \newcommand{\prompt}[4]{
        {\ttfamily\llap{{\color{#2}[#3]:\hspace{3pt}#4}}\vspace{-\baselineskip}}
    }
    

    
    % Prevent overflowing lines due to hard-to-break entities
    \sloppy 
    % Setup hyperref package
    \hypersetup{
      breaklinks=true,  % so long urls are correctly broken across lines
      colorlinks=true,
      urlcolor=urlcolor,
      linkcolor=linkcolor,
      citecolor=citecolor,
      }
    % Slightly bigger margins than the latex defaults
    
    \geometry{verbose,tmargin=1in,bmargin=1in,lmargin=1in,rmargin=1in}
    
    

\begin{document}
    
    \maketitle
    
    

    
    \hypertarget{tp3}{%
\section{TP3}\label{tp3}}

\hypertarget{grupo-15}{%
\subsection{Grupo 15}\label{grupo-15}}

Carlos Eduardo Da Silva Machado A96936

Gonçalo Manuel Maia de Sousa A97485

    \hypertarget{problema-1}{%
\subsection{Problema 1}\label{problema-1}}

    \hypertarget{descriuxe7uxe3o-do-problema}{%
\subsubsection{Descrição do
Problema}\label{descriuxe7uxe3o-do-problema}}

    \begin{enumerate}
\def\labelenumi{\arabic{enumi}.}
\tightlist
\item
  Pretende-se construir uma implementação simplificada do algoritmo
  ``model checking'' orientado aos interpolantes. Para tal seguimos a
  estrutura apresentada nos apontamentos onde, no passo \((n,m)\,\) na
  impossibilidade de encontrar um interpolante invariante se dá ao
  utilizador a possibilidade de incrementar um dos índices \(n\) e \(m\)
  à sua escolha. Pretende-se aplicar este algoritmo ao problema da da
  multiplicação de inteiros positivos em \texttt{BitVec} (apresentado no
  TP2).
\end{enumerate}

    \hypertarget{abordagem-do-problema}{%
\subsubsection{Abordagem do Problema}\label{abordagem-do-problema}}

    Numa abordagem inicial decidimos aplicar o algoritmo com a solução
apresentado no TP2 sem qualquer alteração. No entanto, rapidamente
descobrimos que, sendo a variavel \(pc\) do tipo de dados \texttt{Int} o
\(solver\) não é capaz de encontrar um interpolante. Como modo de
resolver este problema decidimos alterar ligeiramente a solução original
de forma a que \(pc\) seja agora de tipo \texttt{BitVec} mantendo a
mesma lógica da solução original. Além disso removemos todas as
transições para estados de erro, pois estas não são necessárias.

    \hypertarget{cuxf3digo-python}{%
\subsection{Código Python}\label{cuxf3digo-python}}

    Esta secção de codigo serve para importar as bibliotecas necessárias
para a realização do trabalho.

    \begin{tcolorbox}[breakable, size=fbox, boxrule=1pt, pad at break*=1mm,colback=cellbackground, colframe=cellborder]
\prompt{In}{incolor}{3}{\boxspacing}
\begin{Verbatim}[commandchars=\\\{\}]
\PY{k+kn}{from} \PY{n+nn}{pysmt}\PY{n+nn}{.}\PY{n+nn}{shortcuts} \PY{k+kn}{import} \PY{o}{*}
\PY{k+kn}{import} \PY{n+nn}{itertools}
\PY{k+kn}{from} \PY{n+nn}{pysmt}\PY{n+nn}{.}\PY{n+nn}{typing} \PY{k+kn}{import} \PY{n}{INT}
\PY{k+kn}{import} \PY{n+nn}{random} \PY{k}{as} \PY{n+nn}{rn}
\end{Verbatim}
\end{tcolorbox}

    Aqui estão apresentadas as funções \texttt{genState} e
\texttt{init\_fixed}, em tudo análogas às funções \texttt{declare} e
\texttt{init} do TP2. Como forma de poupar espaço \(pc\) é declarado com
tamanho \(2\) visto que é suficiente para representar todos os valores
que \(pc\) possa tomar na execução do algoritmo. Além disso,
implementamos duas versões adicionais da função \texttt{init}: Uma
delas, a função \texttt{init\_unbounded} não limita o valor inicial de
\(x\) e \(y\) de modo a que o solver teste se é possivel alcançar um
estado de erro a partir de \(x\) e \(y\) arbitrários. (O resultado será
sempre unsafe pois para qualquer tamanho é possivel encontrar dois
valoras cujo produto resulte em ``overflow'' ) A outra, a função
\texttt{init\_bounded} limita o valor de \(x\) e \(y\) superiormente de
modo a que o solver teste se é possivel alcançar um estado de erro a
partir \(x\), \(y\) menores ou iguais a um certo limite.

    \begin{tcolorbox}[breakable, size=fbox, boxrule=1pt, pad at break*=1mm,colback=cellbackground, colframe=cellborder]
\prompt{In}{incolor}{4}{\boxspacing}
\begin{Verbatim}[commandchars=\\\{\}]
\PY{k}{def} \PY{n+nf}{genState}\PY{p}{(}\PY{n+nb}{vars}\PY{p}{,}\PY{n}{s}\PY{p}{,}\PY{n}{i}\PY{p}{,}\PY{n}{n}\PY{p}{)}\PY{p}{:}
    \PY{n}{state} \PY{o}{=} \PY{p}{\PYZob{}}\PY{p}{\PYZcb{}}
    \PY{k}{for} \PY{n}{v} \PY{o+ow}{in} \PY{n+nb}{vars}\PY{p}{:}
        \PY{k}{if} \PY{n}{v} \PY{o}{==} \PY{l+s+s1}{\PYZsq{}}\PY{l+s+s1}{pc}\PY{l+s+s1}{\PYZsq{}}\PY{p}{:}
            \PY{n}{state}\PY{p}{[}\PY{n}{v}\PY{p}{]} \PY{o}{=} \PY{n}{Symbol}\PY{p}{(}\PY{n}{v}\PY{o}{+}\PY{l+s+s1}{\PYZsq{}}\PY{l+s+s1}{!}\PY{l+s+s1}{\PYZsq{}}\PY{o}{+}\PY{n}{s}\PY{o}{+}\PY{n+nb}{str}\PY{p}{(}\PY{n}{i}\PY{p}{)}\PY{p}{,}\PY{n}{types}\PY{o}{.}\PY{n}{BVType}\PY{p}{(}\PY{l+m+mi}{2}\PY{p}{)}\PY{p}{)}
        \PY{k}{else}\PY{p}{:}
            \PY{n}{state}\PY{p}{[}\PY{n}{v}\PY{p}{]} \PY{o}{=} \PY{n}{Symbol}\PY{p}{(}\PY{n}{v}\PY{o}{+}\PY{l+s+s1}{\PYZsq{}}\PY{l+s+s1}{!}\PY{l+s+s1}{\PYZsq{}}\PY{o}{+}\PY{n}{s}\PY{o}{+}\PY{n+nb}{str}\PY{p}{(}\PY{n}{i}\PY{p}{)}\PY{p}{,}\PY{n}{types}\PY{o}{.}\PY{n}{BVType}\PY{p}{(}\PY{n}{n}\PY{o}{+}\PY{l+m+mi}{1}\PY{p}{)}\PY{p}{)}
    \PY{k}{return} \PY{n}{state}

\PY{k}{def} \PY{n+nf}{init\PYZus{}fixed}\PY{p}{(}\PY{n}{state}\PY{p}{,}\PY{n}{a}\PY{p}{,}\PY{n}{b}\PY{p}{,}\PY{n}{n}\PY{p}{)}\PY{p}{:}
    \PY{k}{if} \PY{n}{b} \PY{o}{\PYZgt{}} \PY{n}{a}\PY{p}{:}
        \PY{n}{a}\PY{p}{,}\PY{n}{b} \PY{o}{=} \PY{n}{b}\PY{p}{,}\PY{n}{a}
        
    \PY{n}{tPc} \PY{o}{=} \PY{n}{Equals}\PY{p}{(}\PY{n}{state}\PY{p}{[}\PY{l+s+s1}{\PYZsq{}}\PY{l+s+s1}{pc}\PY{l+s+s1}{\PYZsq{}}\PY{p}{]}\PY{p}{,}\PY{n}{BVZero}\PY{p}{(}\PY{l+m+mi}{2}\PY{p}{)}\PY{p}{)} \PY{c+c1}{\PYZsh{} Program counter a zero}
    \PY{n}{tZ}  \PY{o}{=} \PY{n}{Equals}\PY{p}{(}\PY{n}{state}\PY{p}{[}\PY{l+s+s1}{\PYZsq{}}\PY{l+s+s1}{z}\PY{l+s+s1}{\PYZsq{}}\PY{p}{]}\PY{p}{,}\PY{n}{BVZero}\PY{p}{(}\PY{n}{n}\PY{o}{+}\PY{l+m+mi}{1}\PY{p}{)}\PY{p}{)} \PY{c+c1}{\PYZsh{} Z a zero}
    \PY{n}{tX}  \PY{o}{=} \PY{n}{Equals}\PY{p}{(}\PY{n}{state}\PY{p}{[}\PY{l+s+s1}{\PYZsq{}}\PY{l+s+s1}{x}\PY{l+s+s1}{\PYZsq{}}\PY{p}{]}\PY{p}{,} \PY{n}{BV}\PY{p}{(}\PY{n}{a}\PY{p}{,}\PY{n}{n}\PY{o}{+}\PY{l+m+mi}{1}\PY{p}{)}\PY{p}{)} \PY{c+c1}{\PYZsh{} x inicilizado com valor de a}
    \PY{n}{tY}  \PY{o}{=} \PY{n}{Equals}\PY{p}{(}\PY{n}{state}\PY{p}{[}\PY{l+s+s1}{\PYZsq{}}\PY{l+s+s1}{y}\PY{l+s+s1}{\PYZsq{}}\PY{p}{]}\PY{p}{,} \PY{n}{BV}\PY{p}{(}\PY{n}{b}\PY{p}{,}\PY{n}{n}\PY{o}{+}\PY{l+m+mi}{1}\PY{p}{)}\PY{p}{)} \PY{c+c1}{\PYZsh{} y inicilizado com valor de b}
    \PY{k}{return} \PY{n}{And}\PY{p}{(}\PY{n}{tPc}\PY{p}{,}\PY{n}{tX}\PY{p}{,}\PY{n}{tY}\PY{p}{,}\PY{n}{tZ}\PY{p}{)}
    
\PY{k}{def} \PY{n+nf}{init\PYZus{}unbounded}\PY{p}{(}\PY{n}{state}\PY{p}{,}\PY{n}{a}\PY{p}{,}\PY{n}{b}\PY{p}{,}\PY{n}{n}\PY{p}{)}\PY{p}{:}
    \PY{n}{tPc} \PY{o}{=} \PY{n}{Equals}\PY{p}{(}\PY{n}{state}\PY{p}{[}\PY{l+s+s1}{\PYZsq{}}\PY{l+s+s1}{pc}\PY{l+s+s1}{\PYZsq{}}\PY{p}{]}\PY{p}{,}\PY{n}{BVZero}\PY{p}{(}\PY{l+m+mi}{2}\PY{p}{)}\PY{p}{)} \PY{c+c1}{\PYZsh{} Program counter a zero}
    \PY{n}{tZ}  \PY{o}{=} \PY{n}{Equals}\PY{p}{(}\PY{n}{state}\PY{p}{[}\PY{l+s+s1}{\PYZsq{}}\PY{l+s+s1}{z}\PY{l+s+s1}{\PYZsq{}}\PY{p}{]}\PY{p}{,}\PY{n}{BVZero}\PY{p}{(}\PY{n}{n}\PY{o}{+}\PY{l+m+mi}{1}\PY{p}{)}\PY{p}{)} \PY{c+c1}{\PYZsh{} Z a zero}
    \PY{k}{return} \PY{n}{And}\PY{p}{(}\PY{n}{tPc}\PY{p}{,}\PY{n}{tZ}\PY{p}{)}

\PY{k}{def} \PY{n+nf}{init\PYZus{}bounded}\PY{p}{(}\PY{n}{state}\PY{p}{,}\PY{n}{upper\PYZus{}a}\PY{p}{,}\PY{n}{upper\PYZus{}b}\PY{p}{,}\PY{n}{n}\PY{p}{)}\PY{p}{:}
    
    \PY{n}{tPc} \PY{o}{=} \PY{n}{Equals}\PY{p}{(}\PY{n}{state}\PY{p}{[}\PY{l+s+s1}{\PYZsq{}}\PY{l+s+s1}{pc}\PY{l+s+s1}{\PYZsq{}}\PY{p}{]}\PY{p}{,}\PY{n}{BVZero}\PY{p}{(}\PY{l+m+mi}{2}\PY{p}{)}\PY{p}{)} \PY{c+c1}{\PYZsh{} Program counter a zero}
    \PY{n}{tZ}  \PY{o}{=} \PY{n}{Equals}\PY{p}{(}\PY{n}{state}\PY{p}{[}\PY{l+s+s1}{\PYZsq{}}\PY{l+s+s1}{z}\PY{l+s+s1}{\PYZsq{}}\PY{p}{]}\PY{p}{,}\PY{n}{BVZero}\PY{p}{(}\PY{n}{n}\PY{o}{+}\PY{l+m+mi}{1}\PY{p}{)}\PY{p}{)} \PY{c+c1}{\PYZsh{} Z a zero}
    \PY{n}{tX}  \PY{o}{=} \PY{n}{BVULE}\PY{p}{(}\PY{n}{state}\PY{p}{[}\PY{l+s+s1}{\PYZsq{}}\PY{l+s+s1}{x}\PY{l+s+s1}{\PYZsq{}}\PY{p}{]}\PY{p}{,} \PY{n}{BV}\PY{p}{(}\PY{n}{upper\PYZus{}a}\PY{p}{,}\PY{n}{n}\PY{o}{+}\PY{l+m+mi}{1}\PY{p}{)}\PY{p}{)} \PY{c+c1}{\PYZsh{} x inicilizado com valor de a}
    \PY{n}{tY}  \PY{o}{=} \PY{n}{BVULE}\PY{p}{(}\PY{n}{state}\PY{p}{[}\PY{l+s+s1}{\PYZsq{}}\PY{l+s+s1}{y}\PY{l+s+s1}{\PYZsq{}}\PY{p}{]}\PY{p}{,} \PY{n}{BV}\PY{p}{(}\PY{n}{upper\PYZus{}b}\PY{p}{,}\PY{n}{n}\PY{o}{+}\PY{l+m+mi}{1}\PY{p}{)}\PY{p}{)} \PY{c+c1}{\PYZsh{} y inicilizado com valor de b}
    \PY{k}{return} \PY{n}{And}\PY{p}{(}\PY{n}{tPc}\PY{p}{,}\PY{n}{tZ}\PY{p}{,}\PY{n}{tX}\PY{p}{,}\PY{n}{tY}\PY{p}{)}
\end{Verbatim}
\end{tcolorbox}

    Aqui é apresentada a função de transição(\texttt{trans}), análoga à
apresentada no TP2 a menos da remoção de todas as transições para
estados de erro e da mudança do tipo de dados do \(pc\) para
\texttt{BitVec}. Além disso apresentamos uma função de transição extra,
\texttt{trans\_more\_states} que procura seguir de forma mais rigida o
modelo dado no enunciado do TP2. Nota: após alguns testes notamos que
esta nova versão é mais lenta do que a versão anterior.

    Função \texttt{trans} apresentada no TP2.

    \begin{tcolorbox}[breakable, size=fbox, boxrule=1pt, pad at break*=1mm,colback=cellbackground, colframe=cellborder]
\prompt{In}{incolor}{5}{\boxspacing}
\begin{Verbatim}[commandchars=\\\{\}]
\PY{k}{def} \PY{n+nf}{BVFirst}\PY{p}{(}\PY{n}{n}\PY{p}{)}\PY{p}{:}
    \PY{k}{return} \PY{n}{BV}\PY{p}{(}\PY{l+m+mi}{2}\PY{o}{*}\PY{o}{*}\PY{p}{(}\PY{n}{n}\PY{o}{\PYZhy{}}\PY{l+m+mi}{1}\PY{p}{)}\PY{p}{,}\PY{n}{n}\PY{p}{)}

\PY{k}{def} \PY{n+nf}{tEven}\PY{p}{(}\PY{n}{curr}\PY{p}{,}\PY{n}{prox}\PY{p}{,}\PY{n}{n}\PY{p}{)}\PY{p}{:}
    \PY{n}{tPcZero} \PY{o}{=} \PY{n}{Equals}\PY{p}{(}\PY{n}{curr}\PY{p}{[}\PY{l+s+s1}{\PYZsq{}}\PY{l+s+s1}{pc}\PY{l+s+s1}{\PYZsq{}}\PY{p}{]}\PY{p}{,}\PY{n}{BVZero}\PY{p}{(}\PY{l+m+mi}{2}\PY{p}{)}\PY{p}{)}
    \PY{n}{tYLast}  \PY{o}{=} \PY{n}{Equals}\PY{p}{(}\PY{n}{BVAnd}\PY{p}{(}\PY{n}{curr}\PY{p}{[}\PY{l+s+s1}{\PYZsq{}}\PY{l+s+s1}{y}\PY{l+s+s1}{\PYZsq{}}\PY{p}{]}\PY{p}{,}\PY{n}{BVOne}\PY{p}{(}\PY{n}{n}\PY{o}{+}\PY{l+m+mi}{1}\PY{p}{)}\PY{p}{)}\PY{p}{,}\PY{n}{BVZero}\PY{p}{(}\PY{n}{n}\PY{o}{+}\PY{l+m+mi}{1}\PY{p}{)}\PY{p}{)}\PY{c+c1}{\PYZsh{}ultimo bit = 0}
    \PY{n}{tYGt}    \PY{o}{=} \PY{n}{BVUGT}\PY{p}{(}\PY{n}{curr}\PY{p}{[}\PY{l+s+s1}{\PYZsq{}}\PY{l+s+s1}{y}\PY{l+s+s1}{\PYZsq{}}\PY{p}{]}\PY{p}{,}\PY{n}{BVZero}\PY{p}{(}\PY{n}{n}\PY{o}{+}\PY{l+m+mi}{1}\PY{p}{)}\PY{p}{)}\PY{c+c1}{\PYZsh{}y \PYZgt{} 0}
    \PY{n}{tX}      \PY{o}{=} \PY{n}{Equals}\PY{p}{(}\PY{n}{prox}\PY{p}{[}\PY{l+s+s1}{\PYZsq{}}\PY{l+s+s1}{x}\PY{l+s+s1}{\PYZsq{}}\PY{p}{]}\PY{p}{,} \PY{n}{BVLShl}\PY{p}{(}\PY{n}{curr}\PY{p}{[}\PY{l+s+s1}{\PYZsq{}}\PY{l+s+s1}{x}\PY{l+s+s1}{\PYZsq{}}\PY{p}{]}\PY{p}{,}\PY{n}{BVOne}\PY{p}{(}\PY{n}{n}\PY{o}{+}\PY{l+m+mi}{1}\PY{p}{)}\PY{p}{)}\PY{p}{)}\PY{c+c1}{\PYZsh{}2*x}
    \PY{n}{tY}      \PY{o}{=} \PY{n}{Equals}\PY{p}{(}\PY{n}{prox}\PY{p}{[}\PY{l+s+s1}{\PYZsq{}}\PY{l+s+s1}{y}\PY{l+s+s1}{\PYZsq{}}\PY{p}{]}\PY{p}{,} \PY{n}{BVLShr}\PY{p}{(}\PY{n}{curr}\PY{p}{[}\PY{l+s+s1}{\PYZsq{}}\PY{l+s+s1}{y}\PY{l+s+s1}{\PYZsq{}}\PY{p}{]}\PY{p}{,}\PY{n}{BVOne}\PY{p}{(}\PY{n}{n}\PY{o}{+}\PY{l+m+mi}{1}\PY{p}{)}\PY{p}{)}\PY{p}{)}\PY{c+c1}{\PYZsh{}y/2}
    \PY{n}{tZ}      \PY{o}{=} \PY{n}{Equals}\PY{p}{(}\PY{n}{prox}\PY{p}{[}\PY{l+s+s1}{\PYZsq{}}\PY{l+s+s1}{z}\PY{l+s+s1}{\PYZsq{}}\PY{p}{]}\PY{p}{,}\PY{n}{curr}\PY{p}{[}\PY{l+s+s1}{\PYZsq{}}\PY{l+s+s1}{z}\PY{l+s+s1}{\PYZsq{}}\PY{p}{]}\PY{p}{)}\PY{c+c1}{\PYZsh{}z}
    \PY{n}{tPc}     \PY{o}{=} \PY{n}{Equals}\PY{p}{(}\PY{n}{prox}\PY{p}{[}\PY{l+s+s1}{\PYZsq{}}\PY{l+s+s1}{pc}\PY{l+s+s1}{\PYZsq{}}\PY{p}{]}\PY{p}{,}\PY{n}{BVZero}\PY{p}{(}\PY{l+m+mi}{2}\PY{p}{)}\PY{p}{)}
    \PY{k}{return} \PY{n}{And}\PY{p}{(}\PY{n}{tPcZero}\PY{p}{,}\PY{n}{tYLast}\PY{p}{,}\PY{n}{tYGt}\PY{p}{,}\PY{n}{tX}\PY{p}{,}\PY{n}{tY}\PY{p}{,}\PY{n}{tZ}\PY{p}{,}\PY{n}{tPc}\PY{p}{)}

\PY{k}{def} \PY{n+nf}{tOdd}\PY{p}{(}\PY{n}{curr}\PY{p}{,}\PY{n}{prox}\PY{p}{,}\PY{n}{n}\PY{p}{)}\PY{p}{:}
    \PY{n}{tPcZero} \PY{o}{=} \PY{n}{Equals}\PY{p}{(}\PY{n}{curr}\PY{p}{[}\PY{l+s+s1}{\PYZsq{}}\PY{l+s+s1}{pc}\PY{l+s+s1}{\PYZsq{}}\PY{p}{]}\PY{p}{,}\PY{n}{BVZero}\PY{p}{(}\PY{l+m+mi}{2}\PY{p}{)}\PY{p}{)}
    \PY{n}{tYLast}  \PY{o}{=} \PY{n}{Equals}\PY{p}{(}\PY{n}{BVAnd}\PY{p}{(}\PY{n}{curr}\PY{p}{[}\PY{l+s+s1}{\PYZsq{}}\PY{l+s+s1}{y}\PY{l+s+s1}{\PYZsq{}}\PY{p}{]}\PY{p}{,}\PY{n}{BVOne}\PY{p}{(}\PY{n}{n}\PY{o}{+}\PY{l+m+mi}{1}\PY{p}{)}\PY{p}{)}\PY{p}{,}\PY{n}{BVOne}\PY{p}{(}\PY{n}{n}\PY{o}{+}\PY{l+m+mi}{1}\PY{p}{)}\PY{p}{)}
    \PY{n}{tX}      \PY{o}{=} \PY{n}{Equals}\PY{p}{(}\PY{n}{prox}\PY{p}{[}\PY{l+s+s1}{\PYZsq{}}\PY{l+s+s1}{x}\PY{l+s+s1}{\PYZsq{}}\PY{p}{]}\PY{p}{,} \PY{n}{curr}\PY{p}{[}\PY{l+s+s1}{\PYZsq{}}\PY{l+s+s1}{x}\PY{l+s+s1}{\PYZsq{}}\PY{p}{]}\PY{p}{)}
    \PY{n}{tY}      \PY{o}{=} \PY{n}{Equals}\PY{p}{(}\PY{n}{prox}\PY{p}{[}\PY{l+s+s1}{\PYZsq{}}\PY{l+s+s1}{y}\PY{l+s+s1}{\PYZsq{}}\PY{p}{]}\PY{p}{,}\PY{n}{BVSub}\PY{p}{(}\PY{n}{curr}\PY{p}{[}\PY{l+s+s1}{\PYZsq{}}\PY{l+s+s1}{y}\PY{l+s+s1}{\PYZsq{}}\PY{p}{]}\PY{p}{,}\PY{n}{BVOne}\PY{p}{(}\PY{n}{n}\PY{o}{+}\PY{l+m+mi}{1}\PY{p}{)}\PY{p}{)}\PY{p}{)}
    \PY{n}{tZ}      \PY{o}{=} \PY{n}{Equals}\PY{p}{(}\PY{n}{prox}\PY{p}{[}\PY{l+s+s1}{\PYZsq{}}\PY{l+s+s1}{z}\PY{l+s+s1}{\PYZsq{}}\PY{p}{]}\PY{p}{,}\PY{n}{BVAdd}\PY{p}{(}\PY{n}{curr}\PY{p}{[}\PY{l+s+s1}{\PYZsq{}}\PY{l+s+s1}{x}\PY{l+s+s1}{\PYZsq{}}\PY{p}{]}\PY{p}{,}\PY{n}{curr}\PY{p}{[}\PY{l+s+s1}{\PYZsq{}}\PY{l+s+s1}{z}\PY{l+s+s1}{\PYZsq{}}\PY{p}{]}\PY{p}{)}\PY{p}{)} 
    \PY{n}{tPc}     \PY{o}{=} \PY{n}{Equals}\PY{p}{(}\PY{n}{prox}\PY{p}{[}\PY{l+s+s1}{\PYZsq{}}\PY{l+s+s1}{pc}\PY{l+s+s1}{\PYZsq{}}\PY{p}{]}\PY{p}{,}\PY{n}{BVZero}\PY{p}{(}\PY{l+m+mi}{2}\PY{p}{)}\PY{p}{)}   
    \PY{k}{return} \PY{n}{And}\PY{p}{(}\PY{n}{tPcZero}\PY{p}{,}\PY{n}{tYLast}\PY{p}{,}\PY{n}{tX}\PY{p}{,}\PY{n}{tY}\PY{p}{,}\PY{n}{tZ}\PY{p}{,}\PY{n}{tPc}\PY{p}{)}

\PY{k}{def} \PY{n+nf}{tStop}\PY{p}{(}\PY{n}{curr}\PY{p}{,}\PY{n}{prox}\PY{p}{,}\PY{n}{n}\PY{p}{)}\PY{p}{:}
    \PY{n}{tPcZero} \PY{o}{=} \PY{n}{Equals}\PY{p}{(}\PY{n}{curr}\PY{p}{[}\PY{l+s+s1}{\PYZsq{}}\PY{l+s+s1}{pc}\PY{l+s+s1}{\PYZsq{}}\PY{p}{]}\PY{p}{,}\PY{n}{BVZero}\PY{p}{(}\PY{l+m+mi}{2}\PY{p}{)}\PY{p}{)}
    \PY{n}{tYZero}  \PY{o}{=} \PY{n}{Equals}\PY{p}{(}\PY{n}{curr}\PY{p}{[}\PY{l+s+s1}{\PYZsq{}}\PY{l+s+s1}{y}\PY{l+s+s1}{\PYZsq{}}\PY{p}{]}\PY{p}{,}\PY{n}{BVZero}\PY{p}{(}\PY{n}{n}\PY{o}{+}\PY{l+m+mi}{1}\PY{p}{)}\PY{p}{)}\PY{c+c1}{\PYZsh{}y=0}
    \PY{n}{tX}      \PY{o}{=} \PY{n}{Equals}\PY{p}{(}\PY{n}{prox}\PY{p}{[}\PY{l+s+s1}{\PYZsq{}}\PY{l+s+s1}{x}\PY{l+s+s1}{\PYZsq{}}\PY{p}{]}\PY{p}{,}\PY{n}{curr}\PY{p}{[}\PY{l+s+s1}{\PYZsq{}}\PY{l+s+s1}{x}\PY{l+s+s1}{\PYZsq{}}\PY{p}{]}\PY{p}{)}
    \PY{n}{tY}      \PY{o}{=} \PY{n}{Equals}\PY{p}{(}\PY{n}{prox}\PY{p}{[}\PY{l+s+s1}{\PYZsq{}}\PY{l+s+s1}{y}\PY{l+s+s1}{\PYZsq{}}\PY{p}{]}\PY{p}{,}\PY{n}{curr}\PY{p}{[}\PY{l+s+s1}{\PYZsq{}}\PY{l+s+s1}{y}\PY{l+s+s1}{\PYZsq{}}\PY{p}{]}\PY{p}{)}
    \PY{n}{tZ}      \PY{o}{=} \PY{n}{Equals}\PY{p}{(}\PY{n}{prox}\PY{p}{[}\PY{l+s+s1}{\PYZsq{}}\PY{l+s+s1}{z}\PY{l+s+s1}{\PYZsq{}}\PY{p}{]}\PY{p}{,}\PY{n}{curr}\PY{p}{[}\PY{l+s+s1}{\PYZsq{}}\PY{l+s+s1}{z}\PY{l+s+s1}{\PYZsq{}}\PY{p}{]}\PY{p}{)}
    \PY{n}{tPc}     \PY{o}{=} \PY{n}{Equals}\PY{p}{(}\PY{n}{prox}\PY{p}{[}\PY{l+s+s1}{\PYZsq{}}\PY{l+s+s1}{pc}\PY{l+s+s1}{\PYZsq{}}\PY{p}{]}\PY{p}{,}\PY{n}{BVOne}\PY{p}{(}\PY{l+m+mi}{2}\PY{p}{)}\PY{p}{)}
    \PY{k}{return} \PY{n}{And}\PY{p}{(}\PY{n}{tYZero}\PY{p}{,}\PY{n}{tPcZero}\PY{p}{,}\PY{n}{tX}\PY{p}{,}\PY{n}{tY}\PY{p}{,}\PY{n}{tZ}\PY{p}{,}\PY{n}{tPc}\PY{p}{)}

\PY{k}{def} \PY{n+nf}{tEnd}\PY{p}{(}\PY{n}{curr}\PY{p}{,}\PY{n}{prox}\PY{p}{,}\PY{n}{n}\PY{p}{)}\PY{p}{:}
    \PY{n}{tPcOne} \PY{o}{=} \PY{n}{Equals}\PY{p}{(}\PY{n}{curr}\PY{p}{[}\PY{l+s+s1}{\PYZsq{}}\PY{l+s+s1}{pc}\PY{l+s+s1}{\PYZsq{}}\PY{p}{]}\PY{p}{,}\PY{n}{BVOne}\PY{p}{(}\PY{l+m+mi}{2}\PY{p}{)}\PY{p}{)}
    \PY{n}{tX}     \PY{o}{=} \PY{n}{Equals}\PY{p}{(}\PY{n}{prox}\PY{p}{[}\PY{l+s+s1}{\PYZsq{}}\PY{l+s+s1}{x}\PY{l+s+s1}{\PYZsq{}}\PY{p}{]}\PY{p}{,}\PY{n}{curr}\PY{p}{[}\PY{l+s+s1}{\PYZsq{}}\PY{l+s+s1}{x}\PY{l+s+s1}{\PYZsq{}}\PY{p}{]}\PY{p}{)}
    \PY{n}{tY}     \PY{o}{=} \PY{n}{Equals}\PY{p}{(}\PY{n}{prox}\PY{p}{[}\PY{l+s+s1}{\PYZsq{}}\PY{l+s+s1}{y}\PY{l+s+s1}{\PYZsq{}}\PY{p}{]}\PY{p}{,}\PY{n}{curr}\PY{p}{[}\PY{l+s+s1}{\PYZsq{}}\PY{l+s+s1}{y}\PY{l+s+s1}{\PYZsq{}}\PY{p}{]}\PY{p}{)}
    \PY{n}{tZ}     \PY{o}{=} \PY{n}{Equals}\PY{p}{(}\PY{n}{prox}\PY{p}{[}\PY{l+s+s1}{\PYZsq{}}\PY{l+s+s1}{z}\PY{l+s+s1}{\PYZsq{}}\PY{p}{]}\PY{p}{,}\PY{n}{curr}\PY{p}{[}\PY{l+s+s1}{\PYZsq{}}\PY{l+s+s1}{z}\PY{l+s+s1}{\PYZsq{}}\PY{p}{]}\PY{p}{)}
    \PY{n}{tPc}    \PY{o}{=} \PY{n}{Equals}\PY{p}{(}\PY{n}{prox}\PY{p}{[}\PY{l+s+s1}{\PYZsq{}}\PY{l+s+s1}{pc}\PY{l+s+s1}{\PYZsq{}}\PY{p}{]}\PY{p}{,}\PY{n}{BVOne}\PY{p}{(}\PY{l+m+mi}{2}\PY{p}{)}\PY{p}{)}
    \PY{k}{return} \PY{n}{And}\PY{p}{(}\PY{n}{tPcOne}\PY{p}{,}\PY{n}{tX}\PY{p}{,}\PY{n}{tY}\PY{p}{,}\PY{n}{tZ}\PY{p}{,}\PY{n}{tPc}\PY{p}{)}


\PY{k}{def} \PY{n+nf}{trans}\PY{p}{(}\PY{n}{curr}\PY{p}{,}\PY{n}{prox}\PY{p}{,}\PY{n}{n}\PY{p}{)}\PY{p}{:}
    \PY{n}{tToStop} \PY{o}{=} \PY{n}{tStop}\PY{p}{(}\PY{n}{curr}\PY{p}{,}\PY{n}{prox}\PY{p}{,}\PY{n}{n}\PY{p}{)}
    \PY{n}{tToEven} \PY{o}{=} \PY{n}{tEven}\PY{p}{(}\PY{n}{curr}\PY{p}{,}\PY{n}{prox}\PY{p}{,}\PY{n}{n}\PY{p}{)}
    \PY{n}{tToOdd}  \PY{o}{=} \PY{n}{tOdd}\PY{p}{(}\PY{n}{curr}\PY{p}{,}\PY{n}{prox}\PY{p}{,}\PY{n}{n}\PY{p}{)}
    \PY{n}{tToEnd}  \PY{o}{=} \PY{n}{tEnd}\PY{p}{(}\PY{n}{curr}\PY{p}{,}\PY{n}{prox}\PY{p}{,}\PY{n}{n}\PY{p}{)}
    \PY{k}{return} \PY{n}{Or}\PY{p}{(}\PY{n}{tToStop}\PY{p}{,}\PY{n}{tToEven}\PY{p}{,}\PY{n}{tToOdd}\PY{p}{,}\PY{n}{tToEnd}\PY{p}{)}
\end{Verbatim}
\end{tcolorbox}

    Versão nova da função.

    \begin{tcolorbox}[breakable, size=fbox, boxrule=1pt, pad at break*=1mm,colback=cellbackground, colframe=cellborder]
\prompt{In}{incolor}{6}{\boxspacing}
\begin{Verbatim}[commandchars=\\\{\}]
\PY{k}{def} \PY{n+nf}{initToEnven}\PY{p}{(}\PY{n}{curr}\PY{p}{,}\PY{n}{prox}\PY{p}{,}\PY{n}{n}\PY{p}{)}\PY{p}{:}
    \PY{n}{tPcZero} \PY{o}{=} \PY{n}{Equals}\PY{p}{(}\PY{n}{curr}\PY{p}{[}\PY{l+s+s1}{\PYZsq{}}\PY{l+s+s1}{pc}\PY{l+s+s1}{\PYZsq{}}\PY{p}{]}\PY{p}{,}\PY{n}{BVZero}\PY{p}{(}\PY{l+m+mi}{2}\PY{p}{)}\PY{p}{)}
    \PY{n}{tYLast} \PY{o}{=} \PY{n}{Equals}\PY{p}{(}\PY{n}{BVAnd}\PY{p}{(}\PY{n}{curr}\PY{p}{[}\PY{l+s+s1}{\PYZsq{}}\PY{l+s+s1}{y}\PY{l+s+s1}{\PYZsq{}}\PY{p}{]}\PY{p}{,}\PY{n}{BVOne}\PY{p}{(}\PY{n}{n}\PY{o}{+}\PY{l+m+mi}{1}\PY{p}{)}\PY{p}{)}\PY{p}{,}\PY{n}{BVZero}\PY{p}{(}\PY{n}{n}\PY{o}{+}\PY{l+m+mi}{1}\PY{p}{)}\PY{p}{)}\PY{c+c1}{\PYZsh{}ultimo bit = 0}
    \PY{n}{tYGt} \PY{o}{=} \PY{n}{BVUGT}\PY{p}{(}\PY{n}{curr}\PY{p}{[}\PY{l+s+s1}{\PYZsq{}}\PY{l+s+s1}{y}\PY{l+s+s1}{\PYZsq{}}\PY{p}{]}\PY{p}{,}\PY{n}{BVZero}\PY{p}{(}\PY{n}{n}\PY{o}{+}\PY{l+m+mi}{1}\PY{p}{)}\PY{p}{)}\PY{c+c1}{\PYZsh{}y \PYZgt{} 0}
    \PY{n}{tX} \PY{o}{=} \PY{n}{Equals}\PY{p}{(}\PY{n}{prox}\PY{p}{[}\PY{l+s+s1}{\PYZsq{}}\PY{l+s+s1}{x}\PY{l+s+s1}{\PYZsq{}}\PY{p}{]}\PY{p}{,} \PY{n}{curr}\PY{p}{[}\PY{l+s+s1}{\PYZsq{}}\PY{l+s+s1}{x}\PY{l+s+s1}{\PYZsq{}}\PY{p}{]}\PY{p}{)}
    \PY{n}{tY} \PY{o}{=} \PY{n}{Equals}\PY{p}{(}\PY{n}{prox}\PY{p}{[}\PY{l+s+s1}{\PYZsq{}}\PY{l+s+s1}{y}\PY{l+s+s1}{\PYZsq{}}\PY{p}{]}\PY{p}{,} \PY{n}{curr}\PY{p}{[}\PY{l+s+s1}{\PYZsq{}}\PY{l+s+s1}{y}\PY{l+s+s1}{\PYZsq{}}\PY{p}{]}\PY{p}{)}\PY{c+c1}{\PYZsh{}y/2}
    \PY{n}{tZ} \PY{o}{=} \PY{n}{Equals}\PY{p}{(}\PY{n}{prox}\PY{p}{[}\PY{l+s+s1}{\PYZsq{}}\PY{l+s+s1}{z}\PY{l+s+s1}{\PYZsq{}}\PY{p}{]}\PY{p}{,}\PY{n}{curr}\PY{p}{[}\PY{l+s+s1}{\PYZsq{}}\PY{l+s+s1}{z}\PY{l+s+s1}{\PYZsq{}}\PY{p}{]}\PY{p}{)}\PY{c+c1}{\PYZsh{}z}
    \PY{n}{tPc} \PY{o}{=} \PY{n}{Equals}\PY{p}{(}\PY{n}{prox}\PY{p}{[}\PY{l+s+s1}{\PYZsq{}}\PY{l+s+s1}{pc}\PY{l+s+s1}{\PYZsq{}}\PY{p}{]}\PY{p}{,}\PY{n}{BV}\PY{p}{(}\PY{l+m+mi}{2}\PY{p}{,}\PY{l+m+mi}{2}\PY{p}{)}\PY{p}{)}
    \PY{k}{return} \PY{n}{And}\PY{p}{(}\PY{n}{tPcZero}\PY{p}{,}\PY{n}{tYLast}\PY{p}{,}\PY{n}{tYGt}\PY{p}{,}\PY{n}{tX}\PY{p}{,}\PY{n}{tY}\PY{p}{,}\PY{n}{tZ}\PY{p}{,}\PY{n}{tPc}\PY{p}{)} 
    

\PY{k}{def} \PY{n+nf}{initToOdd}\PY{p}{(}\PY{n}{curr}\PY{p}{,}\PY{n}{prox}\PY{p}{,}\PY{n}{n}\PY{p}{)}\PY{p}{:}
    \PY{n}{tPcZero} \PY{o}{=} \PY{n}{Equals}\PY{p}{(}\PY{n}{curr}\PY{p}{[}\PY{l+s+s1}{\PYZsq{}}\PY{l+s+s1}{pc}\PY{l+s+s1}{\PYZsq{}}\PY{p}{]}\PY{p}{,}\PY{n}{BVZero}\PY{p}{(}\PY{l+m+mi}{2}\PY{p}{)}\PY{p}{)}
    \PY{n}{tYLast} \PY{o}{=} \PY{n}{Equals}\PY{p}{(}\PY{n}{BVAnd}\PY{p}{(}\PY{n}{curr}\PY{p}{[}\PY{l+s+s1}{\PYZsq{}}\PY{l+s+s1}{y}\PY{l+s+s1}{\PYZsq{}}\PY{p}{]}\PY{p}{,}\PY{n}{BVOne}\PY{p}{(}\PY{n}{n}\PY{o}{+}\PY{l+m+mi}{1}\PY{p}{)}\PY{p}{)}\PY{p}{,}\PY{n}{BVOne}\PY{p}{(}\PY{n}{n}\PY{o}{+}\PY{l+m+mi}{1}\PY{p}{)}\PY{p}{)}\PY{c+c1}{\PYZsh{}ultimo bit = 1}
    \PY{n}{tX} \PY{o}{=} \PY{n}{Equals}\PY{p}{(}\PY{n}{prox}\PY{p}{[}\PY{l+s+s1}{\PYZsq{}}\PY{l+s+s1}{x}\PY{l+s+s1}{\PYZsq{}}\PY{p}{]}\PY{p}{,} \PY{n}{curr}\PY{p}{[}\PY{l+s+s1}{\PYZsq{}}\PY{l+s+s1}{x}\PY{l+s+s1}{\PYZsq{}}\PY{p}{]}\PY{p}{)}
    \PY{n}{tY} \PY{o}{=} \PY{n}{Equals}\PY{p}{(}\PY{n}{prox}\PY{p}{[}\PY{l+s+s1}{\PYZsq{}}\PY{l+s+s1}{y}\PY{l+s+s1}{\PYZsq{}}\PY{p}{]}\PY{p}{,}\PY{n}{curr}\PY{p}{[}\PY{l+s+s1}{\PYZsq{}}\PY{l+s+s1}{y}\PY{l+s+s1}{\PYZsq{}}\PY{p}{]}\PY{p}{)}
    \PY{n}{tZ} \PY{o}{=} \PY{n}{Equals}\PY{p}{(}\PY{n}{prox}\PY{p}{[}\PY{l+s+s1}{\PYZsq{}}\PY{l+s+s1}{z}\PY{l+s+s1}{\PYZsq{}}\PY{p}{]}\PY{p}{,}\PY{n}{curr}\PY{p}{[}\PY{l+s+s1}{\PYZsq{}}\PY{l+s+s1}{z}\PY{l+s+s1}{\PYZsq{}}\PY{p}{]}\PY{p}{)} 
    \PY{n}{tPc} \PY{o}{=} \PY{n}{Equals}\PY{p}{(}\PY{n}{prox}\PY{p}{[}\PY{l+s+s1}{\PYZsq{}}\PY{l+s+s1}{pc}\PY{l+s+s1}{\PYZsq{}}\PY{p}{]}\PY{p}{,}\PY{n}{BVOne}\PY{p}{(}\PY{l+m+mi}{2}\PY{p}{)}\PY{p}{)}   
    \PY{k}{return} \PY{n}{And}\PY{p}{(}\PY{n}{tPcZero}\PY{p}{,}\PY{n}{tYLast}\PY{p}{,}\PY{n}{tX}\PY{p}{,}\PY{n}{tY}\PY{p}{,}\PY{n}{tZ}\PY{p}{,}\PY{n}{tPc}\PY{p}{)}
    
\PY{k}{def} \PY{n+nf}{OddToInit}\PY{p}{(}\PY{n}{curr}\PY{p}{,}\PY{n}{prox}\PY{p}{,}\PY{n}{n}\PY{p}{)}\PY{p}{:}
    \PY{n}{cpc} \PY{o}{=} \PY{n}{Equals}\PY{p}{(}\PY{n}{curr}\PY{p}{[}\PY{l+s+s1}{\PYZsq{}}\PY{l+s+s1}{pc}\PY{l+s+s1}{\PYZsq{}}\PY{p}{]}\PY{p}{,} \PY{n}{BVOne}\PY{p}{(}\PY{l+m+mi}{2}\PY{p}{)}\PY{p}{)}
    \PY{n}{cxp} \PY{o}{=} \PY{n}{Equals}\PY{p}{(}\PY{n}{curr}\PY{p}{[}\PY{l+s+s1}{\PYZsq{}}\PY{l+s+s1}{x}\PY{l+s+s1}{\PYZsq{}}\PY{p}{]}\PY{p}{,}\PY{n}{prox}\PY{p}{[}\PY{l+s+s1}{\PYZsq{}}\PY{l+s+s1}{x}\PY{l+s+s1}{\PYZsq{}}\PY{p}{]}\PY{p}{)}
    \PY{n}{cyp} \PY{o}{=} \PY{n}{Equals}\PY{p}{(}\PY{n}{prox}\PY{p}{[}\PY{l+s+s1}{\PYZsq{}}\PY{l+s+s1}{y}\PY{l+s+s1}{\PYZsq{}}\PY{p}{]}\PY{p}{,}\PY{n}{BVSub}\PY{p}{(}\PY{n}{curr}\PY{p}{[}\PY{l+s+s1}{\PYZsq{}}\PY{l+s+s1}{y}\PY{l+s+s1}{\PYZsq{}}\PY{p}{]}\PY{p}{,}\PY{n}{BVOne}\PY{p}{(}\PY{n}{n}\PY{o}{+}\PY{l+m+mi}{1}\PY{p}{)}\PY{p}{)}\PY{p}{)}    
    \PY{n}{czp} \PY{o}{=} \PY{n}{Equals}\PY{p}{(}\PY{n}{prox}\PY{p}{[}\PY{l+s+s1}{\PYZsq{}}\PY{l+s+s1}{z}\PY{l+s+s1}{\PYZsq{}}\PY{p}{]}\PY{p}{,}\PY{n}{BVAdd}\PY{p}{(}\PY{n}{curr}\PY{p}{[}\PY{l+s+s1}{\PYZsq{}}\PY{l+s+s1}{x}\PY{l+s+s1}{\PYZsq{}}\PY{p}{]}\PY{p}{,}\PY{n}{curr}\PY{p}{[}\PY{l+s+s1}{\PYZsq{}}\PY{l+s+s1}{z}\PY{l+s+s1}{\PYZsq{}}\PY{p}{]}\PY{p}{)}\PY{p}{)}    
    \PY{n}{ppc} \PY{o}{=} \PY{n}{Equals}\PY{p}{(}\PY{n}{prox}\PY{p}{[}\PY{l+s+s1}{\PYZsq{}}\PY{l+s+s1}{pc}\PY{l+s+s1}{\PYZsq{}}\PY{p}{]}\PY{p}{,} \PY{n}{BVZero}\PY{p}{(}\PY{l+m+mi}{2}\PY{p}{)}\PY{p}{)}
    \PY{k}{return} \PY{n}{And}\PY{p}{(}\PY{n}{cpc}\PY{p}{,}\PY{n}{cxp}\PY{p}{,}\PY{n}{cyp}\PY{p}{,}\PY{n}{czp}\PY{p}{,}\PY{n}{ppc}\PY{p}{)}
    
\PY{k}{def} \PY{n+nf}{EvenToInit}\PY{p}{(}\PY{n}{curr}\PY{p}{,}\PY{n}{prox}\PY{p}{,}\PY{n}{n}\PY{p}{)}\PY{p}{:}
    \PY{n}{cpc} \PY{o}{=} \PY{n}{Equals}\PY{p}{(}\PY{n}{curr}\PY{p}{[}\PY{l+s+s1}{\PYZsq{}}\PY{l+s+s1}{pc}\PY{l+s+s1}{\PYZsq{}}\PY{p}{]}\PY{p}{,} \PY{n}{BV}\PY{p}{(}\PY{l+m+mi}{2}\PY{p}{,}\PY{l+m+mi}{2}\PY{p}{)}\PY{p}{)}
    \PY{n}{cxp} \PY{o}{=} \PY{n}{Equals}\PY{p}{(}\PY{n}{prox}\PY{p}{[}\PY{l+s+s1}{\PYZsq{}}\PY{l+s+s1}{x}\PY{l+s+s1}{\PYZsq{}}\PY{p}{]}\PY{p}{,}\PY{n}{BVLShl}\PY{p}{(}\PY{n}{curr}\PY{p}{[}\PY{l+s+s1}{\PYZsq{}}\PY{l+s+s1}{x}\PY{l+s+s1}{\PYZsq{}}\PY{p}{]}\PY{p}{,}\PY{n}{BVOne}\PY{p}{(}\PY{n}{n}\PY{o}{+}\PY{l+m+mi}{1}\PY{p}{)}\PY{p}{)}\PY{p}{)}
    \PY{n}{cyp} \PY{o}{=} \PY{n}{Equals}\PY{p}{(}\PY{n}{prox}\PY{p}{[}\PY{l+s+s1}{\PYZsq{}}\PY{l+s+s1}{y}\PY{l+s+s1}{\PYZsq{}}\PY{p}{]}\PY{p}{,}\PY{n}{BVLShr}\PY{p}{(}\PY{n}{curr}\PY{p}{[}\PY{l+s+s1}{\PYZsq{}}\PY{l+s+s1}{y}\PY{l+s+s1}{\PYZsq{}}\PY{p}{]}\PY{p}{,}\PY{n}{BVOne}\PY{p}{(}\PY{n}{n}\PY{o}{+}\PY{l+m+mi}{1}\PY{p}{)}\PY{p}{)}\PY{p}{)}    
    \PY{n}{czp} \PY{o}{=} \PY{n}{Equals}\PY{p}{(}\PY{n}{curr}\PY{p}{[}\PY{l+s+s1}{\PYZsq{}}\PY{l+s+s1}{z}\PY{l+s+s1}{\PYZsq{}}\PY{p}{]}\PY{p}{,}\PY{n}{prox}\PY{p}{[}\PY{l+s+s1}{\PYZsq{}}\PY{l+s+s1}{z}\PY{l+s+s1}{\PYZsq{}}\PY{p}{]}\PY{p}{)}    
    \PY{n}{ppc} \PY{o}{=} \PY{n}{Equals}\PY{p}{(}\PY{n}{prox}\PY{p}{[}\PY{l+s+s1}{\PYZsq{}}\PY{l+s+s1}{pc}\PY{l+s+s1}{\PYZsq{}}\PY{p}{]}\PY{p}{,} \PY{n}{BVZero}\PY{p}{(}\PY{l+m+mi}{2}\PY{p}{)}\PY{p}{)}
    \PY{k}{return} \PY{n}{And}\PY{p}{(}\PY{n}{cpc}\PY{p}{,}\PY{n}{cxp}\PY{p}{,}\PY{n}{cyp}\PY{p}{,}\PY{n}{czp}\PY{p}{,}\PY{n}{ppc}\PY{p}{)}
    
\PY{k}{def} \PY{n+nf}{initToStop}\PY{p}{(}\PY{n}{curr}\PY{p}{,}\PY{n}{prox}\PY{p}{,}\PY{n}{n}\PY{p}{)}\PY{p}{:}
    \PY{n}{tPcZero} \PY{o}{=} \PY{n}{Equals}\PY{p}{(}\PY{n}{curr}\PY{p}{[}\PY{l+s+s1}{\PYZsq{}}\PY{l+s+s1}{pc}\PY{l+s+s1}{\PYZsq{}}\PY{p}{]}\PY{p}{,}\PY{n}{BVZero}\PY{p}{(}\PY{l+m+mi}{2}\PY{p}{)}\PY{p}{)}
    \PY{n}{tYZero} \PY{o}{=} \PY{n}{Equals}\PY{p}{(}\PY{n}{curr}\PY{p}{[}\PY{l+s+s1}{\PYZsq{}}\PY{l+s+s1}{y}\PY{l+s+s1}{\PYZsq{}}\PY{p}{]}\PY{p}{,}\PY{n}{BVZero}\PY{p}{(}\PY{n}{n}\PY{o}{+}\PY{l+m+mi}{1}\PY{p}{)}\PY{p}{)}\PY{c+c1}{\PYZsh{}y=0}
    \PY{n}{tZFirst} \PY{o}{=} \PY{n}{Equals}\PY{p}{(}\PY{n}{BVAnd}\PY{p}{(}\PY{n}{curr}\PY{p}{[}\PY{l+s+s1}{\PYZsq{}}\PY{l+s+s1}{z}\PY{l+s+s1}{\PYZsq{}}\PY{p}{]}\PY{p}{,}\PY{n}{BVFirst}\PY{p}{(}\PY{n}{n}\PY{o}{+}\PY{l+m+mi}{1}\PY{p}{)}\PY{p}{)}\PY{p}{,}\PY{n}{BVZero}\PY{p}{(}\PY{n}{n}\PY{o}{+}\PY{l+m+mi}{1}\PY{p}{)}\PY{p}{)}\PY{c+c1}{\PYZsh{}primriro bit de z = 0}
    \PY{n}{tX} \PY{o}{=} \PY{n}{Equals}\PY{p}{(}\PY{n}{prox}\PY{p}{[}\PY{l+s+s1}{\PYZsq{}}\PY{l+s+s1}{x}\PY{l+s+s1}{\PYZsq{}}\PY{p}{]}\PY{p}{,}\PY{n}{curr}\PY{p}{[}\PY{l+s+s1}{\PYZsq{}}\PY{l+s+s1}{x}\PY{l+s+s1}{\PYZsq{}}\PY{p}{]}\PY{p}{)}
    \PY{n}{tY} \PY{o}{=} \PY{n}{Equals}\PY{p}{(}\PY{n}{prox}\PY{p}{[}\PY{l+s+s1}{\PYZsq{}}\PY{l+s+s1}{y}\PY{l+s+s1}{\PYZsq{}}\PY{p}{]}\PY{p}{,}\PY{n}{curr}\PY{p}{[}\PY{l+s+s1}{\PYZsq{}}\PY{l+s+s1}{y}\PY{l+s+s1}{\PYZsq{}}\PY{p}{]}\PY{p}{)}
    \PY{n}{tZ} \PY{o}{=} \PY{n}{Equals}\PY{p}{(}\PY{n}{prox}\PY{p}{[}\PY{l+s+s1}{\PYZsq{}}\PY{l+s+s1}{z}\PY{l+s+s1}{\PYZsq{}}\PY{p}{]}\PY{p}{,}\PY{n}{curr}\PY{p}{[}\PY{l+s+s1}{\PYZsq{}}\PY{l+s+s1}{z}\PY{l+s+s1}{\PYZsq{}}\PY{p}{]}\PY{p}{)}
    \PY{n}{tPc} \PY{o}{=} \PY{n}{Equals}\PY{p}{(}\PY{n}{prox}\PY{p}{[}\PY{l+s+s1}{\PYZsq{}}\PY{l+s+s1}{pc}\PY{l+s+s1}{\PYZsq{}}\PY{p}{]}\PY{p}{,}\PY{n}{BV}\PY{p}{(}\PY{l+m+mi}{3}\PY{p}{,}\PY{l+m+mi}{2}\PY{p}{)}\PY{p}{)}
    \PY{k}{return} \PY{n}{And}\PY{p}{(}\PY{n}{tYZero}\PY{p}{,}\PY{n}{tZFirst}\PY{p}{,}\PY{n}{tPcZero}\PY{p}{,}\PY{n}{tX}\PY{p}{,}\PY{n}{tY}\PY{p}{,}\PY{n}{tZ}\PY{p}{,}\PY{n}{tPc}\PY{p}{)}

\PY{k}{def} \PY{n+nf}{stopToStop}\PY{p}{(}\PY{n}{curr}\PY{p}{,}\PY{n}{prox}\PY{p}{,}\PY{n}{n}\PY{p}{)}\PY{p}{:}
    \PY{n}{tPcOne} \PY{o}{=} \PY{n}{Equals}\PY{p}{(}\PY{n}{curr}\PY{p}{[}\PY{l+s+s1}{\PYZsq{}}\PY{l+s+s1}{pc}\PY{l+s+s1}{\PYZsq{}}\PY{p}{]}\PY{p}{,}\PY{n}{BV}\PY{p}{(}\PY{l+m+mi}{3}\PY{p}{,}\PY{l+m+mi}{2}\PY{p}{)}\PY{p}{)}
    \PY{n}{tX} \PY{o}{=} \PY{n}{Equals}\PY{p}{(}\PY{n}{prox}\PY{p}{[}\PY{l+s+s1}{\PYZsq{}}\PY{l+s+s1}{x}\PY{l+s+s1}{\PYZsq{}}\PY{p}{]}\PY{p}{,}\PY{n}{curr}\PY{p}{[}\PY{l+s+s1}{\PYZsq{}}\PY{l+s+s1}{x}\PY{l+s+s1}{\PYZsq{}}\PY{p}{]}\PY{p}{)}
    \PY{n}{tY} \PY{o}{=} \PY{n}{Equals}\PY{p}{(}\PY{n}{prox}\PY{p}{[}\PY{l+s+s1}{\PYZsq{}}\PY{l+s+s1}{y}\PY{l+s+s1}{\PYZsq{}}\PY{p}{]}\PY{p}{,}\PY{n}{curr}\PY{p}{[}\PY{l+s+s1}{\PYZsq{}}\PY{l+s+s1}{y}\PY{l+s+s1}{\PYZsq{}}\PY{p}{]}\PY{p}{)}
    \PY{n}{tZ} \PY{o}{=} \PY{n}{Equals}\PY{p}{(}\PY{n}{prox}\PY{p}{[}\PY{l+s+s1}{\PYZsq{}}\PY{l+s+s1}{z}\PY{l+s+s1}{\PYZsq{}}\PY{p}{]}\PY{p}{,}\PY{n}{curr}\PY{p}{[}\PY{l+s+s1}{\PYZsq{}}\PY{l+s+s1}{z}\PY{l+s+s1}{\PYZsq{}}\PY{p}{]}\PY{p}{)}
    \PY{n}{tPc} \PY{o}{=} \PY{n}{Equals}\PY{p}{(}\PY{n}{prox}\PY{p}{[}\PY{l+s+s1}{\PYZsq{}}\PY{l+s+s1}{pc}\PY{l+s+s1}{\PYZsq{}}\PY{p}{]}\PY{p}{,}\PY{n}{BV}\PY{p}{(}\PY{l+m+mi}{3}\PY{p}{,}\PY{l+m+mi}{2}\PY{p}{)}\PY{p}{)}
    \PY{k}{return} \PY{n}{And}\PY{p}{(}\PY{n}{tPcOne}\PY{p}{,}\PY{n}{tX}\PY{p}{,}\PY{n}{tY}\PY{p}{,}\PY{n}{tZ}\PY{p}{,}\PY{n}{tPc}\PY{p}{)}

\PY{k}{def} \PY{n+nf}{trans\PYZus{}more\PYZus{}states}\PY{p}{(}\PY{n}{curr}\PY{p}{,}\PY{n}{prox}\PY{p}{,}\PY{n}{n}\PY{p}{)}\PY{p}{:}
    \PY{n}{ite} \PY{o}{=} \PY{n}{initToEnven}\PY{p}{(}\PY{n}{curr}\PY{p}{,}\PY{n}{prox}\PY{p}{,}\PY{n}{n}\PY{p}{)}
    \PY{n}{ito} \PY{o}{=} \PY{n}{initToOdd}\PY{p}{(}\PY{n}{curr}\PY{p}{,}\PY{n}{prox}\PY{p}{,}\PY{n}{n}\PY{p}{)}
    \PY{n}{oti} \PY{o}{=} \PY{n}{OddToInit}\PY{p}{(}\PY{n}{curr}\PY{p}{,}\PY{n}{prox}\PY{p}{,}\PY{n}{n}\PY{p}{)}
    \PY{n}{eti} \PY{o}{=} \PY{n}{EvenToInit}\PY{p}{(}\PY{n}{curr}\PY{p}{,}\PY{n}{prox}\PY{p}{,}\PY{n}{n}\PY{p}{)}
    \PY{n}{its} \PY{o}{=} \PY{n}{initToStop}\PY{p}{(}\PY{n}{curr}\PY{p}{,}\PY{n}{prox}\PY{p}{,}\PY{n}{n}\PY{p}{)}
    \PY{n}{sts} \PY{o}{=} \PY{n}{stopToStop}\PY{p}{(}\PY{n}{curr}\PY{p}{,}\PY{n}{prox}\PY{p}{,}\PY{n}{n}\PY{p}{)}
    \PY{k}{return} \PY{n}{Or}\PY{p}{(}\PY{n}{ite}\PY{p}{,}\PY{n}{ito}\PY{p}{,}\PY{n}{oti}\PY{p}{,}\PY{n}{eti}\PY{p}{,}\PY{n}{its}\PY{p}{,}\PY{n}{sts}\PY{p}{)}
\end{Verbatim}
\end{tcolorbox}

    Aqui apresentamos a condição de erro na forma de uma função
\texttt{error()} que, dado um estado e o tamanho máximo de \(bits\) para
uma variável, devolve uma condição de erro para esse estado. Para tal
apenas é necessário verificar se o bit mais á esquerda do
\texttt{BitVec} declarado é \(1\).

    \begin{tcolorbox}[breakable, size=fbox, boxrule=1pt, pad at break*=1mm,colback=cellbackground, colframe=cellborder]
\prompt{In}{incolor}{7}{\boxspacing}
\begin{Verbatim}[commandchars=\\\{\}]
\PY{k}{def} \PY{n+nf}{error}\PY{p}{(}\PY{n}{state}\PY{p}{,}\PY{n}{n}\PY{p}{)}\PY{p}{:}
    \PY{n}{tYFirst} \PY{o}{=} \PY{n}{Equals}\PY{p}{(}\PY{n}{BVAnd}\PY{p}{(}\PY{n}{state}\PY{p}{[}\PY{l+s+s1}{\PYZsq{}}\PY{l+s+s1}{y}\PY{l+s+s1}{\PYZsq{}}\PY{p}{]}\PY{p}{,}\PY{n}{BVFirst}\PY{p}{(}\PY{n}{n}\PY{o}{+}\PY{l+m+mi}{1}\PY{p}{)}\PY{p}{)}\PY{p}{,}\PY{n}{BVFirst}\PY{p}{(}\PY{n}{n}\PY{o}{+}\PY{l+m+mi}{1}\PY{p}{)}\PY{p}{)}
    \PY{n}{tXFirst} \PY{o}{=} \PY{n}{Equals}\PY{p}{(}\PY{n}{BVAnd}\PY{p}{(}\PY{n}{state}\PY{p}{[}\PY{l+s+s1}{\PYZsq{}}\PY{l+s+s1}{x}\PY{l+s+s1}{\PYZsq{}}\PY{p}{]}\PY{p}{,}\PY{n}{BVFirst}\PY{p}{(}\PY{n}{n}\PY{o}{+}\PY{l+m+mi}{1}\PY{p}{)}\PY{p}{)}\PY{p}{,}\PY{n}{BVFirst}\PY{p}{(}\PY{n}{n}\PY{o}{+}\PY{l+m+mi}{1}\PY{p}{)}\PY{p}{)}
    \PY{n}{tZFirst} \PY{o}{=} \PY{n}{Equals}\PY{p}{(}\PY{n}{BVAnd}\PY{p}{(}\PY{n}{state}\PY{p}{[}\PY{l+s+s1}{\PYZsq{}}\PY{l+s+s1}{z}\PY{l+s+s1}{\PYZsq{}}\PY{p}{]}\PY{p}{,}\PY{n}{BVFirst}\PY{p}{(}\PY{n}{n}\PY{o}{+}\PY{l+m+mi}{1}\PY{p}{)}\PY{p}{)}\PY{p}{,}\PY{n}{BVFirst}\PY{p}{(}\PY{n}{n}\PY{o}{+}\PY{l+m+mi}{1}\PY{p}{)}\PY{p}{)}
    \PY{k}{return} \PY{n}{Or}\PY{p}{(}\PY{n}{tYFirst}\PY{p}{,}\PY{n}{tXFirst}\PY{p}{,}\PY{n}{tZFirst}\PY{p}{)}
\end{Verbatim}
\end{tcolorbox}

    Função que gera um traço de tamanho n. Esta função serve apenas para
fins ilustrativos.

    \begin{tcolorbox}[breakable, size=fbox, boxrule=1pt, pad at break*=1mm,colback=cellbackground, colframe=cellborder]
\prompt{In}{incolor}{8}{\boxspacing}
\begin{Verbatim}[commandchars=\\\{\}]
\PY{k}{def} \PY{n+nf}{genTrace\PYZus{}fixed}\PY{p}{(}\PY{n+nb}{vars}\PY{p}{,}\PY{n}{init}\PY{p}{,}\PY{n}{trans}\PY{p}{,}\PY{n}{error}\PY{p}{,}\PY{n}{n}\PY{p}{,}\PY{n}{tam}\PY{p}{,}\PY{n}{a}\PY{p}{,}\PY{n}{b}\PY{p}{)}\PY{p}{:}
    \PY{k}{with} \PY{n}{Solver}\PY{p}{(}\PY{n}{name}\PY{o}{=}\PY{l+s+s2}{\PYZdq{}}\PY{l+s+s2}{z3}\PY{l+s+s2}{\PYZdq{}}\PY{p}{)} \PY{k}{as} \PY{n}{s}\PY{p}{:}
        
        \PY{n}{X} \PY{o}{=} \PY{p}{[}\PY{n}{genState}\PY{p}{(}\PY{n+nb}{vars}\PY{p}{,}\PY{l+s+s1}{\PYZsq{}}\PY{l+s+s1}{X}\PY{l+s+s1}{\PYZsq{}}\PY{p}{,}\PY{n}{i}\PY{p}{,}\PY{n}{tam}\PY{p}{)} \PY{k}{for} \PY{n}{i} \PY{o+ow}{in} \PY{n+nb}{range}\PY{p}{(}\PY{n}{n}\PY{o}{+}\PY{l+m+mi}{1}\PY{p}{)}\PY{p}{]}   \PY{c+c1}{\PYZsh{} cria n+1 estados (com etiqueta X)}
        \PY{n}{I} \PY{o}{=} \PY{n}{init}\PY{p}{(}\PY{n}{X}\PY{p}{[}\PY{l+m+mi}{0}\PY{p}{]}\PY{p}{,}\PY{n}{a}\PY{p}{,}\PY{n}{b}\PY{p}{,}\PY{n}{tam}\PY{p}{)}
        \PY{n}{Tks} \PY{o}{=} \PY{p}{[} \PY{n}{trans}\PY{p}{(}\PY{n}{X}\PY{p}{[}\PY{n}{i}\PY{p}{]}\PY{p}{,}\PY{n}{X}\PY{p}{[}\PY{n}{i}\PY{o}{+}\PY{l+m+mi}{1}\PY{p}{]}\PY{p}{,}\PY{n}{tam}\PY{p}{)} \PY{k}{for} \PY{n}{i} \PY{o+ow}{in} \PY{n+nb}{range}\PY{p}{(}\PY{n}{n}\PY{p}{)} \PY{p}{]}
        
        \PY{k}{if} \PY{n}{s}\PY{o}{.}\PY{n}{solve}\PY{p}{(}\PY{p}{[}\PY{n}{I}\PY{p}{,}\PY{n}{And}\PY{p}{(}\PY{n}{Tks}\PY{p}{)}\PY{p}{]}\PY{p}{)}\PY{p}{:}      \PY{c+c1}{\PYZsh{} testa se I /\PYZbs{} T\PYZca{}n  é satisfazível}
            \PY{k}{for} \PY{n}{i} \PY{o+ow}{in} \PY{n+nb}{range}\PY{p}{(}\PY{n}{n}\PY{p}{)}\PY{p}{:}
                \PY{n+nb}{print}\PY{p}{(}\PY{l+s+s2}{\PYZdq{}}\PY{l+s+s2}{Estado:}\PY{l+s+s2}{\PYZdq{}}\PY{p}{,}\PY{n}{i}\PY{p}{)}
                \PY{k}{for} \PY{n}{v} \PY{o+ow}{in} \PY{n}{X}\PY{p}{[}\PY{n}{i}\PY{p}{]}\PY{p}{:}
                    \PY{n+nb}{print}\PY{p}{(}\PY{l+s+s2}{\PYZdq{}}\PY{l+s+s2}{          }\PY{l+s+s2}{\PYZdq{}}\PY{p}{,}\PY{n}{v}\PY{p}{,}\PY{l+s+s1}{\PYZsq{}}\PY{l+s+s1}{=}\PY{l+s+s1}{\PYZsq{}}\PY{p}{,}\PY{n}{s}\PY{o}{.}\PY{n}{get\PYZus{}value}\PY{p}{(}\PY{n}{X}\PY{p}{[}\PY{n}{i}\PY{p}{]}\PY{p}{[}\PY{n}{v}\PY{p}{]}\PY{p}{)}\PY{o}{.}\PY{n}{constant\PYZus{}value}\PY{p}{(}\PY{p}{)}\PY{p}{)}
\end{Verbatim}
\end{tcolorbox}

    Aqui apresentamos algumas funções auxiliares utilizadas na função de
``model checking''.

    \begin{tcolorbox}[breakable, size=fbox, boxrule=1pt, pad at break*=1mm,colback=cellbackground, colframe=cellborder]
\prompt{In}{incolor}{9}{\boxspacing}
\begin{Verbatim}[commandchars=\\\{\}]
\PY{k}{def} \PY{n+nf}{baseName}\PY{p}{(}\PY{n}{s}\PY{p}{)}\PY{p}{:}
    \PY{k}{return} \PY{l+s+s1}{\PYZsq{}}\PY{l+s+s1}{\PYZsq{}}\PY{o}{.}\PY{n}{join}\PY{p}{(}\PY{n+nb}{list}\PY{p}{(}\PY{n}{itertools}\PY{o}{.}\PY{n}{takewhile}\PY{p}{(}\PY{k}{lambda} \PY{n}{x}\PY{p}{:} \PY{n}{x}\PY{o}{!=}\PY{l+s+s1}{\PYZsq{}}\PY{l+s+s1}{!}\PY{l+s+s1}{\PYZsq{}}\PY{p}{,} \PY{n}{s}\PY{p}{)}\PY{p}{)}\PY{p}{)}

\PY{k}{def} \PY{n+nf}{rename}\PY{p}{(}\PY{n}{form}\PY{p}{,}\PY{n}{state}\PY{p}{)}\PY{p}{:}
    \PY{n}{vs} \PY{o}{=} \PY{n}{get\PYZus{}free\PYZus{}variables}\PY{p}{(}\PY{n}{form}\PY{p}{)}
    \PY{n}{pairs} \PY{o}{=} \PY{p}{[} \PY{p}{(}\PY{n}{x}\PY{p}{,}\PY{n}{state}\PY{p}{[}\PY{n}{baseName}\PY{p}{(}\PY{n}{x}\PY{o}{.}\PY{n}{symbol\PYZus{}name}\PY{p}{(}\PY{p}{)}\PY{p}{)}\PY{p}{]}\PY{p}{)} \PY{k}{for} \PY{n}{x} \PY{o+ow}{in} \PY{n}{vs} \PY{p}{]} \PY{c+c1}{\PYZsh{} Descobrir os pares \PYZsh{} symbol\PYZus{}name dá o nome aka string da var}
    \PY{k}{return} \PY{n}{form}\PY{o}{.}\PY{n}{substitute}\PY{p}{(}\PY{n+nb}{dict}\PY{p}{(}\PY{n}{pairs}\PY{p}{)}\PY{p}{)} \PY{c+c1}{\PYZsh{} recebe um dicionário e substitui as chaves do dicionário pelo o que está no value}

\PY{k}{def} \PY{n+nf}{same}\PY{p}{(}\PY{n}{state1}\PY{p}{,}\PY{n}{state2}\PY{p}{)}\PY{p}{:} \PY{c+c1}{\PYZsh{} ver se as duas vars têm o mesmo valor}
    \PY{k}{return} \PY{n}{And}\PY{p}{(}\PY{p}{[}\PY{n}{Equals}\PY{p}{(}\PY{n}{state1}\PY{p}{[}\PY{n}{x}\PY{p}{]}\PY{p}{,}\PY{n}{state2}\PY{p}{[}\PY{n}{x}\PY{p}{]}\PY{p}{)} \PY{k}{for} \PY{n}{x} \PY{o+ow}{in} \PY{n}{state1}\PY{p}{]}\PY{p}{)}

\PY{k}{def} \PY{n+nf}{invert}\PY{p}{(}\PY{n}{trans}\PY{p}{,}\PY{n}{n}\PY{p}{)}\PY{p}{:}
    \PY{k}{return} \PY{p}{(}\PY{k}{lambda} \PY{n}{u}\PY{p}{,} \PY{n}{v}\PY{p}{:} \PY{n}{trans}\PY{p}{(}\PY{n}{v}\PY{p}{,}\PY{n}{u}\PY{p}{,}\PY{n}{n}\PY{p}{)}\PY{p}{)}
\end{Verbatim}
\end{tcolorbox}

    \hypertarget{funuxe7uxe3o-model-checking.}{%
\subsubsection{Função ``Model
checking''.}\label{funuxe7uxe3o-model-checking.}}

Nesta secção do relatório apresentamos duas implementações semelhantes
de ``model checking''.

Uma delas percorre todos os pares \((n,m)\) possíveis à procura de um
interpolante invariante não necessita, por isso input do utilizador.

Outra testa inicialmente o par \((1,1)\) e, caso não seja capaz de
encontrar um interpolante invariante pede ao utilizador para incrementar
uma das variáveis \(n\) ou \(m\) e um valor pelo qual incrementar.

    Versão sem input.

    \begin{tcolorbox}[breakable, size=fbox, boxrule=1pt, pad at break*=1mm,colback=cellbackground, colframe=cellborder]
\prompt{In}{incolor}{10}{\boxspacing}
\begin{Verbatim}[commandchars=\\\{\}]
\PY{k}{def} \PY{n+nf}{model\PYZus{}checking}\PY{p}{(}\PY{n+nb}{vars}\PY{p}{,}\PY{n}{init}\PY{p}{,}\PY{n}{trans}\PY{p}{,}\PY{n}{error}\PY{p}{,}\PY{n}{N}\PY{p}{,}\PY{n}{M}\PY{p}{,}\PY{n}{a}\PY{p}{,}\PY{n}{b}\PY{p}{,}\PY{n}{tamanho}\PY{p}{)}\PY{p}{:}
    \PY{k}{with} \PY{n}{Solver}\PY{p}{(}\PY{n}{name}\PY{o}{=}\PY{l+s+s2}{\PYZdq{}}\PY{l+s+s2}{msat}\PY{l+s+s2}{\PYZdq{}}\PY{p}{)} \PY{k}{as} \PY{n}{s}\PY{p}{:}
        
        \PY{c+c1}{\PYZsh{} Criar todos os estados que poderão vir a ser necessários.}
        \PY{n}{X} \PY{o}{=} \PY{p}{[}\PY{n}{genState}\PY{p}{(}\PY{n+nb}{vars}\PY{p}{,}\PY{l+s+s1}{\PYZsq{}}\PY{l+s+s1}{X}\PY{l+s+s1}{\PYZsq{}}\PY{p}{,}\PY{n}{i}\PY{p}{,}\PY{n}{tamanho}\PY{p}{)} \PY{k}{for} \PY{n}{i} \PY{o+ow}{in} \PY{n+nb}{range}\PY{p}{(}\PY{n}{N}\PY{o}{+}\PY{l+m+mi}{1}\PY{p}{)}\PY{p}{]} \PY{c+c1}{\PYZsh{} Com a função genState, criar todos os estados que possam ser necessário, TODOS. \PYZsh{} X SFOTS original}
        \PY{n}{Y} \PY{o}{=} \PY{p}{[}\PY{n}{genState}\PY{p}{(}\PY{n+nb}{vars}\PY{p}{,}\PY{l+s+s1}{\PYZsq{}}\PY{l+s+s1}{Y}\PY{l+s+s1}{\PYZsq{}}\PY{p}{,}\PY{n}{i}\PY{p}{,}\PY{n}{tamanho}\PY{p}{)} \PY{k}{for} \PY{n}{i} \PY{o+ow}{in} \PY{n+nb}{range}\PY{p}{(}\PY{n}{M}\PY{o}{+}\PY{l+m+mi}{1}\PY{p}{)}\PY{p}{]} \PY{c+c1}{\PYZsh{} Y SFOTS invertido}

        \PY{c+c1}{\PYZsh{} Estabelecer a ordem pela qual os pares (n,m) vão surgir. Por exemplo:}
        \PY{n}{order} \PY{o}{=} \PY{n+nb}{sorted}\PY{p}{(}\PY{p}{[}\PY{p}{(}\PY{n}{a}\PY{p}{,}\PY{n}{b}\PY{p}{)} \PY{k}{for} \PY{n}{a} \PY{o+ow}{in} \PY{n+nb}{range}\PY{p}{(}\PY{l+m+mi}{1}\PY{p}{,}\PY{n}{N}\PY{o}{+}\PY{l+m+mi}{1}\PY{p}{)} \PY{k}{for} \PY{n}{b} \PY{o+ow}{in} \PY{n+nb}{range}\PY{p}{(}\PY{l+m+mi}{1}\PY{p}{,}\PY{n}{M}\PY{o}{+}\PY{l+m+mi}{1}\PY{p}{)}\PY{p}{]}\PY{p}{,}\PY{n}{key}\PY{o}{=}\PY{k}{lambda} \PY{n}{tup}\PY{p}{:}\PY{n}{tup}\PY{p}{[}\PY{l+m+mi}{0}\PY{p}{]}\PY{o}{+}\PY{n}{tup}\PY{p}{[}\PY{l+m+mi}{1}\PY{p}{]}\PY{p}{)} \PY{c+c1}{\PYZsh{} Estabelecer ordem que criamos o n e o m \PYZsh{} ideia da stora: usar o sorted,}
                                                                                                         \PY{c+c1}{\PYZsh{} gerar todos os pares possíveis }
                                                                                                         \PY{c+c1}{\PYZsh{} e ter como critério de ordenação as soma dos elementos dos pares}
        
        \PY{k}{for} \PY{p}{(}\PY{n}{n}\PY{p}{,}\PY{n}{m}\PY{p}{)} \PY{o+ow}{in} \PY{n}{order}\PY{p}{:} \PY{c+c1}{\PYZsh{} Seguir o algoritmo}
            \PY{c+c1}{\PYZsh{} completar}
            \PY{n}{I} \PY{o}{=} \PY{n}{init}\PY{p}{(}\PY{n}{X}\PY{p}{[}\PY{l+m+mi}{0}\PY{p}{]}\PY{p}{,}\PY{n}{a}\PY{p}{,}\PY{n}{b}\PY{p}{,}\PY{n}{tamanho}\PY{p}{)} \PY{c+c1}{\PYZsh{} o X é uma lista de estados}
            \PY{n}{Tn} \PY{o}{=} \PY{n}{And}\PY{p}{(}\PY{p}{[}\PY{n}{trans}\PY{p}{(}\PY{n}{X}\PY{p}{[}\PY{n}{i}\PY{p}{]}\PY{p}{,}\PY{n}{X}\PY{p}{[}\PY{n}{i}\PY{o}{+}\PY{l+m+mi}{1}\PY{p}{]}\PY{p}{,}\PY{n}{tamanho}\PY{p}{)} \PY{k}{for} \PY{n}{i} \PY{o+ow}{in} \PY{n+nb}{range}\PY{p}{(}\PY{n}{n}\PY{p}{)}\PY{p}{]}\PY{p}{)}
            \PY{n}{Rn} \PY{o}{=} \PY{n}{And}\PY{p}{(}\PY{n}{I}\PY{p}{,}\PY{n}{Tn}\PY{p}{)} \PY{c+c1}{\PYZsh{} estados acessíveis em n transições}
            \PY{c+c1}{\PYZsh{}print(X[0])}
            \PY{n}{E} \PY{o}{=} \PY{n}{error}\PY{p}{(}\PY{n}{Y}\PY{p}{[}\PY{l+m+mi}{0}\PY{p}{]}\PY{p}{,}\PY{n}{tamanho}\PY{p}{)}
            \PY{n}{Bm} \PY{o}{=} \PY{n}{And}\PY{p}{(}\PY{p}{[}\PY{n}{invert}\PY{p}{(}\PY{n}{trans}\PY{p}{,}\PY{n}{tamanho}\PY{p}{)}\PY{p}{(}\PY{n}{Y}\PY{p}{[}\PY{n}{i}\PY{p}{]}\PY{p}{,}\PY{n}{Y}\PY{p}{[}\PY{n}{i}\PY{o}{+}\PY{l+m+mi}{1}\PY{p}{]}\PY{p}{)} \PY{k}{for} \PY{n}{i} \PY{o+ow}{in} \PY{n+nb}{range}\PY{p}{(}\PY{n}{m}\PY{p}{)}\PY{p}{]}\PY{p}{)}
            \PY{n}{Um} \PY{o}{=} \PY{n}{And}\PY{p}{(}\PY{n}{E}\PY{p}{,}\PY{n}{Bm}\PY{p}{)} \PY{c+c1}{\PYZsh{} estados inseguros em m transições}
            
            \PY{n}{Vnm} \PY{o}{=} \PY{n}{And}\PY{p}{(}\PY{n}{Rn}\PY{p}{,}\PY{n}{same}\PY{p}{(}\PY{n}{X}\PY{p}{[}\PY{n}{n}\PY{p}{]}\PY{p}{,}\PY{n}{Y}\PY{p}{[}\PY{n}{m}\PY{p}{]}\PY{p}{)}\PY{p}{,}\PY{n}{Um}\PY{p}{)} \PY{c+c1}{\PYZsh{} temos de testar se dois estados estão iguais e, portanto, usamos a função same dada acima}
            
            \PY{c+c1}{\PYZsh{}pprint(Vnm.serialize())}
            
            \PY{k}{if} \PY{n}{s}\PY{o}{.}\PY{n}{solve}\PY{p}{(}\PY{p}{[}\PY{n}{Vnm}\PY{p}{]}\PY{p}{)}\PY{p}{:}
                \PY{n+nb}{print}\PY{p}{(}\PY{l+s+s2}{\PYZdq{}}\PY{l+s+s2}{unsafe}\PY{l+s+s2}{\PYZdq{}}\PY{p}{)}
                \PY{k}{return} 
           
            \PY{c+c1}{\PYZsh{} Se for insatisfazível, temos de criar o interpolante para n fórmulas}
            \PY{n}{C} \PY{o}{=} \PY{n}{binary\PYZus{}interpolant}\PY{p}{(}\PY{n}{And}\PY{p}{(}\PY{n}{Rn}\PY{p}{,}\PY{n}{same}\PY{p}{(}\PY{n}{X}\PY{p}{[}\PY{n}{n}\PY{p}{]}\PY{p}{,}\PY{n}{Y}\PY{p}{[}\PY{n}{m}\PY{p}{]}\PY{p}{)}\PY{p}{)}\PY{p}{,} \PY{n}{Um}\PY{p}{)}
            \PY{k}{if} \PY{n}{C} \PY{o+ow}{is} \PY{k+kc}{None}\PY{p}{:}
                \PY{n+nb}{print}\PY{p}{(}\PY{l+s+s2}{\PYZdq{}}\PY{l+s+s2}{Interpolante None}\PY{l+s+s2}{\PYZdq{}}\PY{p}{)}
                \PY{k}{continue}
            
            \PY{n}{C0} \PY{o}{=} \PY{n}{rename}\PY{p}{(}\PY{n}{C}\PY{p}{,}\PY{n}{X}\PY{p}{[}\PY{l+m+mi}{0}\PY{p}{]}\PY{p}{)} \PY{c+c1}{\PYZsh{} Rename do C com o estado envolvido, neste caso o X[0] }
            \PY{n}{C1} \PY{o}{=} \PY{n}{rename}\PY{p}{(}\PY{n}{C}\PY{p}{,}\PY{n}{X}\PY{p}{[}\PY{l+m+mi}{1}\PY{p}{]}\PY{p}{)}
            \PY{c+c1}{\PYZsh{} Trabalhamos com X[0] e X[1] porque T pode ser escrito como T = (X0,X1)}
            
            \PY{n}{T} \PY{o}{=} \PY{n}{trans}\PY{p}{(}\PY{n}{X}\PY{p}{[}\PY{l+m+mi}{0}\PY{p}{]}\PY{p}{,}\PY{n}{X}\PY{p}{[}\PY{l+m+mi}{1}\PY{p}{]}\PY{p}{,}\PY{n}{tamanho}\PY{p}{)}
            
            \PY{k}{if} \PY{o+ow}{not} \PY{n}{s}\PY{o}{.}\PY{n}{solve}\PY{p}{(}\PY{p}{[}\PY{n}{C0}\PY{p}{,}\PY{n}{T}\PY{p}{,}\PY{n}{Not}\PY{p}{(}\PY{n}{C1}\PY{p}{)}\PY{p}{]}\PY{p}{)}\PY{p}{:}
                \PY{n+nb}{print}\PY{p}{(}\PY{n}{n}\PY{p}{,}\PY{n}{m}\PY{p}{)}
                \PY{n+nb}{print}\PY{p}{(}\PY{l+s+s2}{\PYZdq{}}\PY{l+s+s2}{Safe}\PY{l+s+s2}{\PYZdq{}}\PY{p}{)}
                \PY{k}{return}
            \PY{k}{else}\PY{p}{:}
                    \PY{c+c1}{\PYZsh{}\PYZsh{}\PYZsh{}\PYZsh{} gerar o S (Parte que descreve o Majorante S)}
                
                \PY{c+c1}{\PYZsh{}Passo 1:}
                \PY{n}{S} \PY{o}{=} \PY{n}{rename}\PY{p}{(}\PY{n}{C}\PY{p}{,}\PY{n}{X}\PY{p}{[}\PY{n}{n}\PY{p}{]}\PY{p}{)}
                \PY{k}{while} \PY{k+kc}{True}\PY{p}{:}
                    \PY{c+c1}{\PYZsh{}Passo 2:}
                    \PY{n}{A} \PY{o}{=} \PY{n}{And}\PY{p}{(}\PY{n}{S}\PY{p}{,}\PY{n}{trans}\PY{p}{(}\PY{n}{X}\PY{p}{[}\PY{n}{n}\PY{p}{]}\PY{p}{,}\PY{n}{Y}\PY{p}{[}\PY{n}{m}\PY{p}{]}\PY{p}{,}\PY{n}{tamanho}\PY{p}{)}\PY{p}{)}
                    \PY{k}{if} \PY{n}{s}\PY{o}{.}\PY{n}{solve}\PY{p}{(}\PY{p}{[}\PY{n}{A}\PY{p}{,}\PY{n}{Um}\PY{p}{]}\PY{p}{)}\PY{p}{:}
                        \PY{c+c1}{\PYZsh{}N = int(input(\PYZdq{}new N\PYZdq{}))}
                        \PY{c+c1}{\PYZsh{}M = int(input(\PYZdq{}new M\PYZdq{}))}
                        \PY{n+nb}{print}\PY{p}{(}\PY{l+s+s2}{\PYZdq{}}\PY{l+s+s2}{Não foi possível encontrar o majorante.}\PY{l+s+s2}{\PYZdq{}}\PY{p}{)}
                        \PY{k}{break}
                    \PY{k}{else}\PY{p}{:}
                        \PY{n}{CNew} \PY{o}{=} \PY{n}{binary\PYZus{}interpolant}\PY{p}{(}\PY{n}{A}\PY{p}{,}\PY{n}{Um}\PY{p}{)} \PY{c+c1}{\PYZsh{} as duas formulas têm de ser inconsistentes para que faça sentido para usar binary\PYZus{}interpolant}
                        \PY{n}{Cn} \PY{o}{=} \PY{n}{rename}\PY{p}{(}\PY{n}{CNew}\PY{p}{,}\PY{n}{X}\PY{p}{[}\PY{n}{n}\PY{p}{]}\PY{p}{)}
                        
                        \PY{k}{if} \PY{n}{s}\PY{o}{.}\PY{n}{solve}\PY{p}{(}\PY{p}{[}\PY{n}{Cn}\PY{p}{,}\PY{n}{Not}\PY{p}{(}\PY{n}{S}\PY{p}{)}\PY{p}{]}\PY{p}{)}\PY{p}{:}
                            \PY{n}{S} \PY{o}{=} \PY{n}{Or}\PY{p}{(}\PY{n}{S}\PY{p}{,}\PY{n}{Cn}\PY{p}{)}
                        \PY{k}{else}\PY{p}{:}
                            \PY{n+nb}{print}\PY{p}{(}\PY{n}{n}\PY{p}{,}\PY{n}{m}\PY{p}{)}
                            \PY{n+nb}{print}\PY{p}{(}\PY{l+s+s2}{\PYZdq{}}\PY{l+s+s2}{Safe}\PY{l+s+s2}{\PYZdq{}}\PY{p}{)}
                            \PY{k}{return}
            \PY{c+c1}{\PYZsh{}new\PYZus{}n = int(input(\PYZdq{}new n POGCHAMP\PYZdq{}))}
            \PY{c+c1}{\PYZsh{}new\PYZus{}m = int(input(\PYZdq{}new m POGCHAMP\PYZdq{}))}
            \PY{c+c1}{\PYZsh{}order.append((new\PYZus{}n,new\PYZus{}m))}
                
        \PY{n+nb}{print}\PY{p}{(}\PY{l+s+s2}{\PYZdq{}}\PY{l+s+s2}{unknown}\PY{l+s+s2}{\PYZdq{}}\PY{p}{)}
\end{Verbatim}
\end{tcolorbox}

    Versão com input.

    \begin{tcolorbox}[breakable, size=fbox, boxrule=1pt, pad at break*=1mm,colback=cellbackground, colframe=cellborder]
\prompt{In}{incolor}{11}{\boxspacing}
\begin{Verbatim}[commandchars=\\\{\}]
\PY{k}{def} \PY{n+nf}{model\PYZus{}checking\PYZus{}input}\PY{p}{(}\PY{n+nb}{vars}\PY{p}{,}\PY{n}{init}\PY{p}{,}\PY{n}{trans}\PY{p}{,}\PY{n}{error}\PY{p}{,}\PY{n}{N}\PY{p}{,}\PY{n}{M}\PY{p}{,}\PY{n}{a}\PY{p}{,}\PY{n}{b}\PY{p}{,}\PY{n}{tamanho}\PY{p}{)}\PY{p}{:}
    \PY{k}{with} \PY{n}{Solver}\PY{p}{(}\PY{n}{name}\PY{o}{=}\PY{l+s+s2}{\PYZdq{}}\PY{l+s+s2}{msat}\PY{l+s+s2}{\PYZdq{}}\PY{p}{)} \PY{k}{as} \PY{n}{s}\PY{p}{:}
        
        \PY{c+c1}{\PYZsh{} Criar todos os estados que poderão vir a ser necessários.}
        \PY{n}{X} \PY{o}{=} \PY{p}{[}\PY{n}{genState}\PY{p}{(}\PY{n+nb}{vars}\PY{p}{,}\PY{l+s+s1}{\PYZsq{}}\PY{l+s+s1}{X}\PY{l+s+s1}{\PYZsq{}}\PY{p}{,}\PY{n}{i}\PY{p}{,}\PY{n}{tamanho}\PY{p}{)} \PY{k}{for} \PY{n}{i} \PY{o+ow}{in} \PY{n+nb}{range}\PY{p}{(}\PY{n}{N}\PY{o}{+}\PY{l+m+mi}{1}\PY{p}{)}\PY{p}{]} \PY{c+c1}{\PYZsh{} Com a função genState, criar todos os estados que possam ser necessário, TODOS. \PYZsh{} X SFOTS original}
        \PY{n}{Y} \PY{o}{=} \PY{p}{[}\PY{n}{genState}\PY{p}{(}\PY{n+nb}{vars}\PY{p}{,}\PY{l+s+s1}{\PYZsq{}}\PY{l+s+s1}{Y}\PY{l+s+s1}{\PYZsq{}}\PY{p}{,}\PY{n}{i}\PY{p}{,}\PY{n}{tamanho}\PY{p}{)} \PY{k}{for} \PY{n}{i} \PY{o+ow}{in} \PY{n+nb}{range}\PY{p}{(}\PY{n}{M}\PY{o}{+}\PY{l+m+mi}{1}\PY{p}{)}\PY{p}{]} \PY{c+c1}{\PYZsh{} Y SFOTS invertido}

        \PY{c+c1}{\PYZsh{} Estabelecer a ordem pela qual os pares (n,m) vão surgir. Por exemplo:}
        \PY{c+c1}{\PYZsh{}order = sorted([(a,b) for a in range(1,N+1) for b in range(1,M+1)],key=lambda tup:tup[0]+tup[1]) \PYZsh{} Estabelecer ordem que criamos o n e o m \PYZsh{} ideia da stora: usar o sorted,}
                                                                                                         \PY{c+c1}{\PYZsh{} gerar todos os pares possíveis }
                                                                                                         \PY{c+c1}{\PYZsh{} e ter como critério de ordenação as soma dos elementos dos pares}
        \PY{p}{(}\PY{n}{n}\PY{p}{,}\PY{n}{m}\PY{p}{)} \PY{o}{=} \PY{l+m+mi}{1}\PY{p}{,}\PY{l+m+mi}{1}
        
        \PY{k}{while}\PY{p}{(}\PY{n}{n} \PY{o}{\PYZlt{}}\PY{o}{=} \PY{n}{N} \PY{o+ow}{and} \PY{n}{m} \PY{o}{\PYZlt{}}\PY{o}{=} \PY{n}{M}\PY{p}{)}\PY{p}{:}
            \PY{c+c1}{\PYZsh{} completar}
            \PY{n}{I} \PY{o}{=} \PY{n}{init}\PY{p}{(}\PY{n}{X}\PY{p}{[}\PY{l+m+mi}{0}\PY{p}{]}\PY{p}{,}\PY{n}{a}\PY{p}{,}\PY{n}{b}\PY{p}{,}\PY{n}{tamanho}\PY{p}{)} \PY{c+c1}{\PYZsh{} o X é uma lista de estados}
            \PY{n}{Tn} \PY{o}{=} \PY{n}{And}\PY{p}{(}\PY{p}{[}\PY{n}{trans}\PY{p}{(}\PY{n}{X}\PY{p}{[}\PY{n}{i}\PY{p}{]}\PY{p}{,}\PY{n}{X}\PY{p}{[}\PY{n}{i}\PY{o}{+}\PY{l+m+mi}{1}\PY{p}{]}\PY{p}{,}\PY{n}{tamanho}\PY{p}{)} \PY{k}{for} \PY{n}{i} \PY{o+ow}{in} \PY{n+nb}{range}\PY{p}{(}\PY{n}{n}\PY{p}{)}\PY{p}{]}\PY{p}{)}
            \PY{n}{Rn} \PY{o}{=} \PY{n}{And}\PY{p}{(}\PY{n}{I}\PY{p}{,}\PY{n}{Tn}\PY{p}{)} \PY{c+c1}{\PYZsh{} estados acessíveis em n transições}
            \PY{c+c1}{\PYZsh{}print(X[0])}
            \PY{n}{E} \PY{o}{=} \PY{n}{error}\PY{p}{(}\PY{n}{Y}\PY{p}{[}\PY{l+m+mi}{0}\PY{p}{]}\PY{p}{,}\PY{n}{tamanho}\PY{p}{)}
            \PY{n}{Bm} \PY{o}{=} \PY{n}{And}\PY{p}{(}\PY{p}{[}\PY{n}{invert}\PY{p}{(}\PY{n}{trans}\PY{p}{,}\PY{n}{tamanho}\PY{p}{)}\PY{p}{(}\PY{n}{Y}\PY{p}{[}\PY{n}{i}\PY{p}{]}\PY{p}{,}\PY{n}{Y}\PY{p}{[}\PY{n}{i}\PY{o}{+}\PY{l+m+mi}{1}\PY{p}{]}\PY{p}{)} \PY{k}{for} \PY{n}{i} \PY{o+ow}{in} \PY{n+nb}{range}\PY{p}{(}\PY{n}{m}\PY{p}{)}\PY{p}{]}\PY{p}{)}
            \PY{n}{Um} \PY{o}{=} \PY{n}{And}\PY{p}{(}\PY{n}{E}\PY{p}{,}\PY{n}{Bm}\PY{p}{)} \PY{c+c1}{\PYZsh{} estados inseguros em m transições}
            
            \PY{n}{Vnm} \PY{o}{=} \PY{n}{And}\PY{p}{(}\PY{n}{Rn}\PY{p}{,}\PY{n}{same}\PY{p}{(}\PY{n}{X}\PY{p}{[}\PY{n}{n}\PY{p}{]}\PY{p}{,}\PY{n}{Y}\PY{p}{[}\PY{n}{m}\PY{p}{]}\PY{p}{)}\PY{p}{,}\PY{n}{Um}\PY{p}{)} \PY{c+c1}{\PYZsh{} temos de testar se dois estados estão iguais e, portanto, usamos a função same dada acima}
            
            \PY{c+c1}{\PYZsh{}pprint(Vnm.serialize())}
            
            \PY{k}{if} \PY{n}{s}\PY{o}{.}\PY{n}{solve}\PY{p}{(}\PY{p}{[}\PY{n}{Vnm}\PY{p}{]}\PY{p}{)}\PY{p}{:}
                \PY{n+nb}{print}\PY{p}{(}\PY{l+s+s2}{\PYZdq{}}\PY{l+s+s2}{unsafe}\PY{l+s+s2}{\PYZdq{}}\PY{p}{)}
                \PY{k}{return} 
           
            \PY{c+c1}{\PYZsh{} Se for insatisfazível, temos de criar o interpolante para n fórmulas}
            \PY{n}{C} \PY{o}{=} \PY{n}{binary\PYZus{}interpolant}\PY{p}{(}\PY{n}{And}\PY{p}{(}\PY{n}{Rn}\PY{p}{,}\PY{n}{same}\PY{p}{(}\PY{n}{X}\PY{p}{[}\PY{n}{n}\PY{p}{]}\PY{p}{,}\PY{n}{Y}\PY{p}{[}\PY{n}{m}\PY{p}{]}\PY{p}{)}\PY{p}{)}\PY{p}{,} \PY{n}{Um}\PY{p}{)}
            \PY{k}{if} \PY{n}{C} \PY{o+ow}{is} \PY{k+kc}{None}\PY{p}{:}
                \PY{n+nb}{print}\PY{p}{(}\PY{l+s+s2}{\PYZdq{}}\PY{l+s+s2}{Interpolante None}\PY{l+s+s2}{\PYZdq{}}\PY{p}{)}
                \PY{k}{continue}
            
            \PY{n}{C0} \PY{o}{=} \PY{n}{rename}\PY{p}{(}\PY{n}{C}\PY{p}{,}\PY{n}{X}\PY{p}{[}\PY{l+m+mi}{0}\PY{p}{]}\PY{p}{)} \PY{c+c1}{\PYZsh{} Rename do C com o estado envolvido, neste caso o X[0] }
            \PY{n}{C1} \PY{o}{=} \PY{n}{rename}\PY{p}{(}\PY{n}{C}\PY{p}{,}\PY{n}{X}\PY{p}{[}\PY{l+m+mi}{1}\PY{p}{]}\PY{p}{)}
            \PY{c+c1}{\PYZsh{} Trabalhamos com X[0] e X[1] porque T pode ser escrito como T = (X0,X1)}
            
            \PY{n}{T} \PY{o}{=} \PY{n}{trans}\PY{p}{(}\PY{n}{X}\PY{p}{[}\PY{l+m+mi}{0}\PY{p}{]}\PY{p}{,}\PY{n}{X}\PY{p}{[}\PY{l+m+mi}{1}\PY{p}{]}\PY{p}{,}\PY{n}{tamanho}\PY{p}{)}
            
            \PY{k}{if} \PY{o+ow}{not} \PY{n}{s}\PY{o}{.}\PY{n}{solve}\PY{p}{(}\PY{p}{[}\PY{n}{C0}\PY{p}{,}\PY{n}{T}\PY{p}{,}\PY{n}{Not}\PY{p}{(}\PY{n}{C1}\PY{p}{)}\PY{p}{]}\PY{p}{)}\PY{p}{:}
                \PY{n+nb}{print}\PY{p}{(}\PY{l+s+s2}{\PYZdq{}}\PY{l+s+s2}{Safe}\PY{l+s+s2}{\PYZdq{}}\PY{p}{)}
                \PY{k}{return}
            \PY{k}{else}\PY{p}{:}
                    \PY{c+c1}{\PYZsh{}\PYZsh{}\PYZsh{}\PYZsh{} gerar o S (Parte que descreve o Majorante S)}
                
                \PY{c+c1}{\PYZsh{}Passo 1:}
                \PY{n}{S} \PY{o}{=} \PY{n}{rename}\PY{p}{(}\PY{n}{C}\PY{p}{,}\PY{n}{X}\PY{p}{[}\PY{n}{n}\PY{p}{]}\PY{p}{)}
                \PY{k}{while} \PY{k+kc}{True}\PY{p}{:}
                    \PY{c+c1}{\PYZsh{}Passo 2:}
                    \PY{n}{A} \PY{o}{=} \PY{n}{And}\PY{p}{(}\PY{n}{S}\PY{p}{,}\PY{n}{trans}\PY{p}{(}\PY{n}{X}\PY{p}{[}\PY{n}{n}\PY{p}{]}\PY{p}{,}\PY{n}{Y}\PY{p}{[}\PY{n}{m}\PY{p}{]}\PY{p}{,}\PY{n}{tamanho}\PY{p}{)}\PY{p}{)}
                    \PY{k}{if} \PY{n}{s}\PY{o}{.}\PY{n}{solve}\PY{p}{(}\PY{p}{[}\PY{n}{A}\PY{p}{,}\PY{n}{Um}\PY{p}{]}\PY{p}{)}\PY{p}{:}
                        \PY{c+c1}{\PYZsh{}N = int(input(\PYZdq{}new N\PYZdq{}))}
                        \PY{c+c1}{\PYZsh{}M = int(input(\PYZdq{}new M\PYZdq{}))}
                        \PY{n+nb}{print}\PY{p}{(}\PY{l+s+s2}{\PYZdq{}}\PY{l+s+s2}{Não foi possível encontrar o majorante.}\PY{l+s+s2}{\PYZdq{}}\PY{p}{)}

                        \PY{c+c1}{\PYZsh{}n = int(input(\PYZdq{}new n\PYZdq{}))}
                        \PY{c+c1}{\PYZsh{}m = int(input(\PYZdq{}new m\PYZdq{}))}
                        
                        \PY{k}{break}
                    \PY{k}{else}\PY{p}{:}
                        \PY{n}{CNew} \PY{o}{=} \PY{n}{binary\PYZus{}interpolant}\PY{p}{(}\PY{n}{A}\PY{p}{,}\PY{n}{Um}\PY{p}{)} \PY{c+c1}{\PYZsh{} as duas formulas têm de ser inconsistentes para que faça sentido para usar binary\PYZus{}interpolant}
                        \PY{n}{Cn} \PY{o}{=} \PY{n}{rename}\PY{p}{(}\PY{n}{CNew}\PY{p}{,}\PY{n}{X}\PY{p}{[}\PY{n}{n}\PY{p}{]}\PY{p}{)}
                        
                        \PY{k}{if} \PY{n}{s}\PY{o}{.}\PY{n}{solve}\PY{p}{(}\PY{p}{[}\PY{n}{Cn}\PY{p}{,}\PY{n}{Not}\PY{p}{(}\PY{n}{S}\PY{p}{)}\PY{p}{]}\PY{p}{)}\PY{p}{:}
                            \PY{n}{S} \PY{o}{=} \PY{n}{Or}\PY{p}{(}\PY{n}{S}\PY{p}{,}\PY{n}{Cn}\PY{p}{)}
                        \PY{k}{else}\PY{p}{:}
                            \PY{n+nb}{print}\PY{p}{(}\PY{l+s+s2}{\PYZdq{}}\PY{l+s+s2}{Safe}\PY{l+s+s2}{\PYZdq{}}\PY{p}{)}
                            \PY{k}{return}

            \PY{n}{var} \PY{o}{=} \PY{n+nb}{input}\PY{p}{(}\PY{l+s+s2}{\PYZdq{}}\PY{l+s+s2}{Qual é a variável que pretende incrementar.}\PY{l+s+s2}{\PYZdq{}}\PY{p}{)}\PY{o}{.}\PY{n}{lower}\PY{p}{(}\PY{p}{)}
            \PY{n}{new\PYZus{}val} \PY{o}{=} \PY{n+nb}{int}\PY{p}{(}\PY{n+nb}{input}\PY{p}{(}\PY{l+s+s2}{\PYZdq{}}\PY{l+s+s2}{Quantidade}\PY{l+s+s2}{\PYZdq{}}\PY{p}{)}\PY{p}{)}
            \PY{p}{(}\PY{n}{n}\PY{p}{,}\PY{n}{m}\PY{p}{)} \PY{o}{=} \PY{p}{(}\PY{n}{n}\PY{o}{+}\PY{n}{new\PYZus{}val} \PY{k}{if} \PY{n}{var} \PY{o}{==} \PY{l+s+s1}{\PYZsq{}}\PY{l+s+s1}{n}\PY{l+s+s1}{\PYZsq{}} \PY{k}{else} \PY{n}{n}\PY{p}{,} \PY{n}{m}\PY{o}{+}\PY{n}{new\PYZus{}val} \PY{k}{if} \PY{n}{var} \PY{o}{==} \PY{l+s+s1}{\PYZsq{}}\PY{l+s+s1}{m}\PY{l+s+s1}{\PYZsq{}} \PY{k}{else} \PY{n}{m}\PY{p}{)}
            \PY{c+c1}{\PYZsh{}new\PYZus{}m = int(input(\PYZdq{}new m POGCHAMP\PYZdq{}))}
            \PY{c+c1}{\PYZsh{}order.append((new\PYZus{}n,new\PYZus{}m))}
                
        \PY{n+nb}{print}\PY{p}{(}\PY{l+s+s2}{\PYZdq{}}\PY{l+s+s2}{unknown}\PY{l+s+s2}{\PYZdq{}}\PY{p}{)}
\end{Verbatim}
\end{tcolorbox}

    \hypertarget{exemplos}{%
\section{Exemplos}\label{exemplos}}

Nesta secção apresentamos alguns exemplos.

    \hypertarget{exemplo-1}{%
\subsubsection{Exemplo 1}\label{exemplo-1}}

Nestes exemplos mostramos varias execuções da função com \(x\) e \(y\)
fixos à partida.

    \begin{tcolorbox}[breakable, size=fbox, boxrule=1pt, pad at break*=1mm,colback=cellbackground, colframe=cellborder]
\prompt{In}{incolor}{10}{\boxspacing}
\begin{Verbatim}[commandchars=\\\{\}]
\PY{n}{N} \PY{o}{=} \PY{l+m+mi}{100}
\PY{n}{M} \PY{o}{=} \PY{l+m+mi}{100}
\PY{n}{a} \PY{o}{=} \PY{l+m+mi}{10}
\PY{n}{b} \PY{o}{=} \PY{l+m+mi}{10}
\PY{n}{tamanho} \PY{o}{=} \PY{l+m+mi}{16}
\end{Verbatim}
\end{tcolorbox}

    \begin{tcolorbox}[breakable, size=fbox, boxrule=1pt, pad at break*=1mm,colback=cellbackground, colframe=cellborder]
\prompt{In}{incolor}{11}{\boxspacing}
\begin{Verbatim}[commandchars=\\\{\}]
\PY{n}{model\PYZus{}checking}\PY{p}{(}\PY{p}{[}\PY{l+s+s1}{\PYZsq{}}\PY{l+s+s1}{pc}\PY{l+s+s1}{\PYZsq{}}\PY{p}{,}\PY{l+s+s1}{\PYZsq{}}\PY{l+s+s1}{x}\PY{l+s+s1}{\PYZsq{}}\PY{p}{,}\PY{l+s+s1}{\PYZsq{}}\PY{l+s+s1}{y}\PY{l+s+s1}{\PYZsq{}}\PY{p}{,}\PY{l+s+s1}{\PYZsq{}}\PY{l+s+s1}{z}\PY{l+s+s1}{\PYZsq{}}\PY{p}{]}\PY{p}{,} \PY{n}{init\PYZus{}fixed}\PY{p}{,} \PY{n}{trans}\PY{p}{,} \PY{n}{error}\PY{p}{,} \PY{n}{N}\PY{p}{,} \PY{n}{M}\PY{p}{,} \PY{n}{a}\PY{p}{,} \PY{n}{b}\PY{p}{,} \PY{n}{tamanho}\PY{p}{)}    
\end{Verbatim}
\end{tcolorbox}

    \begin{Verbatim}[commandchars=\\\{\}]
Não foi possível encontrar o majorante.
1 2
Safe
    \end{Verbatim}

    \begin{tcolorbox}[breakable, size=fbox, boxrule=1pt, pad at break*=1mm,colback=cellbackground, colframe=cellborder]
\prompt{In}{incolor}{12}{\boxspacing}
\begin{Verbatim}[commandchars=\\\{\}]
\PY{n}{model\PYZus{}checking\PYZus{}input}\PY{p}{(}\PY{p}{[}\PY{l+s+s1}{\PYZsq{}}\PY{l+s+s1}{pc}\PY{l+s+s1}{\PYZsq{}}\PY{p}{,}\PY{l+s+s1}{\PYZsq{}}\PY{l+s+s1}{x}\PY{l+s+s1}{\PYZsq{}}\PY{p}{,}\PY{l+s+s1}{\PYZsq{}}\PY{l+s+s1}{y}\PY{l+s+s1}{\PYZsq{}}\PY{p}{,}\PY{l+s+s1}{\PYZsq{}}\PY{l+s+s1}{z}\PY{l+s+s1}{\PYZsq{}}\PY{p}{]}\PY{p}{,} \PY{n}{init\PYZus{}fixed}\PY{p}{,} \PY{n}{trans}\PY{p}{,} \PY{n}{error}\PY{p}{,} \PY{n}{N}\PY{p}{,} \PY{n}{M}\PY{p}{,} \PY{n}{a}\PY{p}{,} \PY{n}{b}\PY{p}{,} \PY{n}{tamanho}\PY{p}{)}    
\end{Verbatim}
\end{tcolorbox}

    \begin{Verbatim}[commandchars=\\\{\}]
Não foi possível encontrar o majorante.
Qual é a variável que pretende incrementar. m
Quantidade 2
Safe
    \end{Verbatim}

    \begin{tcolorbox}[breakable, size=fbox, boxrule=1pt, pad at break*=1mm,colback=cellbackground, colframe=cellborder]
\prompt{In}{incolor}{13}{\boxspacing}
\begin{Verbatim}[commandchars=\\\{\}]
\PY{n}{model\PYZus{}checking}\PY{p}{(}\PY{p}{[}\PY{l+s+s1}{\PYZsq{}}\PY{l+s+s1}{pc}\PY{l+s+s1}{\PYZsq{}}\PY{p}{,}\PY{l+s+s1}{\PYZsq{}}\PY{l+s+s1}{x}\PY{l+s+s1}{\PYZsq{}}\PY{p}{,}\PY{l+s+s1}{\PYZsq{}}\PY{l+s+s1}{y}\PY{l+s+s1}{\PYZsq{}}\PY{p}{,}\PY{l+s+s1}{\PYZsq{}}\PY{l+s+s1}{z}\PY{l+s+s1}{\PYZsq{}}\PY{p}{]}\PY{p}{,} \PY{n}{init\PYZus{}fixed}\PY{p}{,} \PY{n}{trans\PYZus{}more\PYZus{}states}\PY{p}{,} \PY{n}{error}\PY{p}{,} \PY{n}{N}\PY{p}{,} \PY{n}{M}\PY{p}{,} \PY{n}{a}\PY{p}{,} \PY{n}{b}\PY{p}{,} \PY{n}{tamanho}\PY{p}{)}    
\end{Verbatim}
\end{tcolorbox}

    \begin{Verbatim}[commandchars=\\\{\}]
Não foi possível encontrar o majorante.
Não foi possível encontrar o majorante.
Não foi possível encontrar o majorante.
Não foi possível encontrar o majorante.
Não foi possível encontrar o majorante.
Não foi possível encontrar o majorante.
Não foi possível encontrar o majorante.
Não foi possível encontrar o majorante.
Não foi possível encontrar o majorante.
Não foi possível encontrar o majorante.
Não foi possível encontrar o majorante.
Não foi possível encontrar o majorante.
Não foi possível encontrar o majorante.
Não foi possível encontrar o majorante.
Não foi possível encontrar o majorante.
1 6
Safe
    \end{Verbatim}

    \begin{tcolorbox}[breakable, size=fbox, boxrule=1pt, pad at break*=1mm,colback=cellbackground, colframe=cellborder]
\prompt{In}{incolor}{14}{\boxspacing}
\begin{Verbatim}[commandchars=\\\{\}]
\PY{n}{model\PYZus{}checking\PYZus{}input}\PY{p}{(}\PY{p}{[}\PY{l+s+s1}{\PYZsq{}}\PY{l+s+s1}{pc}\PY{l+s+s1}{\PYZsq{}}\PY{p}{,}\PY{l+s+s1}{\PYZsq{}}\PY{l+s+s1}{x}\PY{l+s+s1}{\PYZsq{}}\PY{p}{,}\PY{l+s+s1}{\PYZsq{}}\PY{l+s+s1}{y}\PY{l+s+s1}{\PYZsq{}}\PY{p}{,}\PY{l+s+s1}{\PYZsq{}}\PY{l+s+s1}{z}\PY{l+s+s1}{\PYZsq{}}\PY{p}{]}\PY{p}{,} \PY{n}{init\PYZus{}fixed}\PY{p}{,} \PY{n}{trans\PYZus{}more\PYZus{}states}\PY{p}{,} \PY{n}{error}\PY{p}{,} \PY{n}{N}\PY{p}{,} \PY{n}{M}\PY{p}{,} \PY{n}{a}\PY{p}{,} \PY{n}{b}\PY{p}{,} \PY{n}{tamanho}\PY{p}{)}    
\end{Verbatim}
\end{tcolorbox}

    \begin{Verbatim}[commandchars=\\\{\}]
Não foi possível encontrar o majorante.
Qual é a variável que pretende incrementar. m
Quantidade 6
Safe
    \end{Verbatim}

    \hypertarget{exemplo-2}{%
\subsubsection{Exemplo 2}\label{exemplo-2}}

Nestes exemplos mostramos várias execuções da função com \(x\) e \(y\)
arbitrários. Nota: Nestas condições o resultado será sempre ``unsafe''
pois para qualquer tamanho é sempre possível encontrar ``x'' e ``y'' de
modo a que a sua multiplicação resulte em \(overflow\).

    \begin{tcolorbox}[breakable, size=fbox, boxrule=1pt, pad at break*=1mm,colback=cellbackground, colframe=cellborder]
\prompt{In}{incolor}{15}{\boxspacing}
\begin{Verbatim}[commandchars=\\\{\}]
\PY{n}{model\PYZus{}checking}\PY{p}{(}\PY{p}{[}\PY{l+s+s1}{\PYZsq{}}\PY{l+s+s1}{pc}\PY{l+s+s1}{\PYZsq{}}\PY{p}{,}\PY{l+s+s1}{\PYZsq{}}\PY{l+s+s1}{x}\PY{l+s+s1}{\PYZsq{}}\PY{p}{,}\PY{l+s+s1}{\PYZsq{}}\PY{l+s+s1}{y}\PY{l+s+s1}{\PYZsq{}}\PY{p}{,}\PY{l+s+s1}{\PYZsq{}}\PY{l+s+s1}{z}\PY{l+s+s1}{\PYZsq{}}\PY{p}{]}\PY{p}{,} \PY{n}{init\PYZus{}unbounded}\PY{p}{,} \PY{n}{trans}\PY{p}{,} \PY{n}{error}\PY{p}{,} \PY{n}{N}\PY{p}{,} \PY{n}{M}\PY{p}{,} \PY{n}{a}\PY{p}{,} \PY{n}{b}\PY{p}{,} \PY{n}{tamanho}\PY{p}{)}    
\end{Verbatim}
\end{tcolorbox}

    \begin{Verbatim}[commandchars=\\\{\}]
unsafe
    \end{Verbatim}

    \begin{tcolorbox}[breakable, size=fbox, boxrule=1pt, pad at break*=1mm,colback=cellbackground, colframe=cellborder]
\prompt{In}{incolor}{16}{\boxspacing}
\begin{Verbatim}[commandchars=\\\{\}]
\PY{n}{model\PYZus{}checking\PYZus{}input}\PY{p}{(}\PY{p}{[}\PY{l+s+s1}{\PYZsq{}}\PY{l+s+s1}{pc}\PY{l+s+s1}{\PYZsq{}}\PY{p}{,}\PY{l+s+s1}{\PYZsq{}}\PY{l+s+s1}{x}\PY{l+s+s1}{\PYZsq{}}\PY{p}{,}\PY{l+s+s1}{\PYZsq{}}\PY{l+s+s1}{y}\PY{l+s+s1}{\PYZsq{}}\PY{p}{,}\PY{l+s+s1}{\PYZsq{}}\PY{l+s+s1}{z}\PY{l+s+s1}{\PYZsq{}}\PY{p}{]}\PY{p}{,} \PY{n}{init\PYZus{}unbounded}\PY{p}{,} \PY{n}{trans}\PY{p}{,} \PY{n}{error}\PY{p}{,} \PY{n}{N}\PY{p}{,} \PY{n}{M}\PY{p}{,} \PY{n}{a}\PY{p}{,} \PY{n}{b}\PY{p}{,} \PY{n}{tamanho}\PY{p}{)}    
\end{Verbatim}
\end{tcolorbox}

    \begin{Verbatim}[commandchars=\\\{\}]
unsafe
    \end{Verbatim}

    \begin{tcolorbox}[breakable, size=fbox, boxrule=1pt, pad at break*=1mm,colback=cellbackground, colframe=cellborder]
\prompt{In}{incolor}{17}{\boxspacing}
\begin{Verbatim}[commandchars=\\\{\}]
\PY{n}{model\PYZus{}checking}\PY{p}{(}\PY{p}{[}\PY{l+s+s1}{\PYZsq{}}\PY{l+s+s1}{pc}\PY{l+s+s1}{\PYZsq{}}\PY{p}{,}\PY{l+s+s1}{\PYZsq{}}\PY{l+s+s1}{x}\PY{l+s+s1}{\PYZsq{}}\PY{p}{,}\PY{l+s+s1}{\PYZsq{}}\PY{l+s+s1}{y}\PY{l+s+s1}{\PYZsq{}}\PY{p}{,}\PY{l+s+s1}{\PYZsq{}}\PY{l+s+s1}{z}\PY{l+s+s1}{\PYZsq{}}\PY{p}{]}\PY{p}{,} \PY{n}{init\PYZus{}unbounded}\PY{p}{,} \PY{n}{trans\PYZus{}more\PYZus{}states}\PY{p}{,} \PY{n}{error}\PY{p}{,} \PY{n}{N}\PY{p}{,} \PY{n}{M}\PY{p}{,} \PY{n}{a}\PY{p}{,} \PY{n}{b}\PY{p}{,} \PY{n}{tamanho}\PY{p}{)}    
\end{Verbatim}
\end{tcolorbox}

    \begin{Verbatim}[commandchars=\\\{\}]
unsafe
    \end{Verbatim}

    \begin{tcolorbox}[breakable, size=fbox, boxrule=1pt, pad at break*=1mm,colback=cellbackground, colframe=cellborder]
\prompt{In}{incolor}{18}{\boxspacing}
\begin{Verbatim}[commandchars=\\\{\}]
\PY{n}{model\PYZus{}checking\PYZus{}input}\PY{p}{(}\PY{p}{[}\PY{l+s+s1}{\PYZsq{}}\PY{l+s+s1}{pc}\PY{l+s+s1}{\PYZsq{}}\PY{p}{,}\PY{l+s+s1}{\PYZsq{}}\PY{l+s+s1}{x}\PY{l+s+s1}{\PYZsq{}}\PY{p}{,}\PY{l+s+s1}{\PYZsq{}}\PY{l+s+s1}{y}\PY{l+s+s1}{\PYZsq{}}\PY{p}{,}\PY{l+s+s1}{\PYZsq{}}\PY{l+s+s1}{z}\PY{l+s+s1}{\PYZsq{}}\PY{p}{]}\PY{p}{,} \PY{n}{init\PYZus{}unbounded}\PY{p}{,} \PY{n}{trans\PYZus{}more\PYZus{}states}\PY{p}{,} \PY{n}{error}\PY{p}{,} \PY{n}{N}\PY{p}{,} \PY{n}{M}\PY{p}{,} \PY{n}{a}\PY{p}{,} \PY{n}{b}\PY{p}{,} \PY{n}{tamanho}\PY{p}{)}    
\end{Verbatim}
\end{tcolorbox}

    \begin{Verbatim}[commandchars=\\\{\}]
unsafe
    \end{Verbatim}

    \hypertarget{exemplo-3}{%
\subsubsection{Exemplo 3}\label{exemplo-3}}

Nestes exemplos mostramos várias execuções da função com \(x\) e \(y\)
iniciais limitados superiormente. Nota: Este modo de execução é o que,
na nossa experiência, requer mais tempo de execução.

    \begin{tcolorbox}[breakable, size=fbox, boxrule=1pt, pad at break*=1mm,colback=cellbackground, colframe=cellborder]
\prompt{In}{incolor}{19}{\boxspacing}
\begin{Verbatim}[commandchars=\\\{\}]
\PY{n}{model\PYZus{}checking}\PY{p}{(}\PY{p}{[}\PY{l+s+s1}{\PYZsq{}}\PY{l+s+s1}{pc}\PY{l+s+s1}{\PYZsq{}}\PY{p}{,}\PY{l+s+s1}{\PYZsq{}}\PY{l+s+s1}{x}\PY{l+s+s1}{\PYZsq{}}\PY{p}{,}\PY{l+s+s1}{\PYZsq{}}\PY{l+s+s1}{y}\PY{l+s+s1}{\PYZsq{}}\PY{p}{,}\PY{l+s+s1}{\PYZsq{}}\PY{l+s+s1}{z}\PY{l+s+s1}{\PYZsq{}}\PY{p}{]}\PY{p}{,} \PY{n}{init\PYZus{}bounded}\PY{p}{,} \PY{n}{trans}\PY{p}{,} \PY{n}{error}\PY{p}{,} \PY{n}{N}\PY{p}{,} \PY{n}{M}\PY{p}{,} \PY{n}{a}\PY{p}{,} \PY{n}{b}\PY{p}{,} \PY{n}{tamanho}\PY{p}{)}    
\end{Verbatim}
\end{tcolorbox}

    \begin{Verbatim}[commandchars=\\\{\}]
Não foi possível encontrar o majorante.
Não foi possível encontrar o majorante.
Não foi possível encontrar o majorante.
Não foi possível encontrar o majorante.
Não foi possível encontrar o majorante.
Não foi possível encontrar o majorante.
Não foi possível encontrar o majorante.
Não foi possível encontrar o majorante.
Não foi possível encontrar o majorante.
Não foi possível encontrar o majorante.
Não foi possível encontrar o majorante.
Não foi possível encontrar o majorante.
3 3
Safe
    \end{Verbatim}

    \begin{tcolorbox}[breakable, size=fbox, boxrule=1pt, pad at break*=1mm,colback=cellbackground, colframe=cellborder]
\prompt{In}{incolor}{20}{\boxspacing}
\begin{Verbatim}[commandchars=\\\{\}]
\PY{n}{model\PYZus{}checking\PYZus{}input}\PY{p}{(}\PY{p}{[}\PY{l+s+s1}{\PYZsq{}}\PY{l+s+s1}{pc}\PY{l+s+s1}{\PYZsq{}}\PY{p}{,}\PY{l+s+s1}{\PYZsq{}}\PY{l+s+s1}{x}\PY{l+s+s1}{\PYZsq{}}\PY{p}{,}\PY{l+s+s1}{\PYZsq{}}\PY{l+s+s1}{y}\PY{l+s+s1}{\PYZsq{}}\PY{p}{,}\PY{l+s+s1}{\PYZsq{}}\PY{l+s+s1}{z}\PY{l+s+s1}{\PYZsq{}}\PY{p}{]}\PY{p}{,} \PY{n}{init\PYZus{}bounded}\PY{p}{,} \PY{n}{trans}\PY{p}{,} \PY{n}{error}\PY{p}{,} \PY{n}{N}\PY{p}{,} \PY{n}{M}\PY{p}{,} \PY{n}{a}\PY{p}{,} \PY{n}{b}\PY{p}{,} \PY{n}{tamanho}\PY{p}{)}    
\end{Verbatim}
\end{tcolorbox}

    \begin{Verbatim}[commandchars=\\\{\}]
Não foi possível encontrar o majorante.
Qual é a variável que pretende incrementar. n
Quantidade 2
Não foi possível encontrar o majorante.
Qual é a variável que pretende incrementar. n
Quantidade 3
Safe
    \end{Verbatim}

    \begin{tcolorbox}[breakable, size=fbox, boxrule=1pt, pad at break*=1mm,colback=cellbackground, colframe=cellborder]
\prompt{In}{incolor}{21}{\boxspacing}
\begin{Verbatim}[commandchars=\\\{\}]
\PY{n}{model\PYZus{}checking}\PY{p}{(}\PY{p}{[}\PY{l+s+s1}{\PYZsq{}}\PY{l+s+s1}{pc}\PY{l+s+s1}{\PYZsq{}}\PY{p}{,}\PY{l+s+s1}{\PYZsq{}}\PY{l+s+s1}{x}\PY{l+s+s1}{\PYZsq{}}\PY{p}{,}\PY{l+s+s1}{\PYZsq{}}\PY{l+s+s1}{y}\PY{l+s+s1}{\PYZsq{}}\PY{p}{,}\PY{l+s+s1}{\PYZsq{}}\PY{l+s+s1}{z}\PY{l+s+s1}{\PYZsq{}}\PY{p}{]}\PY{p}{,} \PY{n}{init\PYZus{}bounded}\PY{p}{,} \PY{n}{trans\PYZus{}more\PYZus{}states}\PY{p}{,} \PY{n}{error}\PY{p}{,} \PY{n}{N}\PY{p}{,} \PY{n}{M}\PY{p}{,} \PY{n}{a}\PY{p}{,} \PY{n}{b}\PY{p}{,} \PY{n}{tamanho}\PY{p}{)}    
\end{Verbatim}
\end{tcolorbox}

    \begin{Verbatim}[commandchars=\\\{\}]
Não foi possível encontrar o majorante.
Não foi possível encontrar o majorante.
Não foi possível encontrar o majorante.
Não foi possível encontrar o majorante.
Não foi possível encontrar o majorante.
Não foi possível encontrar o majorante.
Não foi possível encontrar o majorante.
Não foi possível encontrar o majorante.
Não foi possível encontrar o majorante.
Não foi possível encontrar o majorante.
Não foi possível encontrar o majorante.
Não foi possível encontrar o majorante.
Não foi possível encontrar o majorante.
Não foi possível encontrar o majorante.
Não foi possível encontrar o majorante.
Não foi possível encontrar o majorante.
Não foi possível encontrar o majorante.
Não foi possível encontrar o majorante.
Não foi possível encontrar o majorante.
Não foi possível encontrar o majorante.
Não foi possível encontrar o majorante.
Não foi possível encontrar o majorante.
Não foi possível encontrar o majorante.
Não foi possível encontrar o majorante.
Não foi possível encontrar o majorante.
Não foi possível encontrar o majorante.
Não foi possível encontrar o majorante.
Não foi possível encontrar o majorante.
Não foi possível encontrar o majorante.
Não foi possível encontrar o majorante.
Não foi possível encontrar o majorante.
Não foi possível encontrar o majorante.
Não foi possível encontrar o majorante.
Não foi possível encontrar o majorante.
Não foi possível encontrar o majorante.
Não foi possível encontrar o majorante.
Não foi possível encontrar o majorante.
Não foi possível encontrar o majorante.
Não foi possível encontrar o majorante.
Não foi possível encontrar o majorante.
Não foi possível encontrar o majorante.
Não foi possível encontrar o majorante.
Não foi possível encontrar o majorante.
Não foi possível encontrar o majorante.
Não foi possível encontrar o majorante.
1 10
Safe
    \end{Verbatim}

    \begin{tcolorbox}[breakable, size=fbox, boxrule=1pt, pad at break*=1mm,colback=cellbackground, colframe=cellborder]
\prompt{In}{incolor}{22}{\boxspacing}
\begin{Verbatim}[commandchars=\\\{\}]
\PY{n}{model\PYZus{}checking\PYZus{}input}\PY{p}{(}\PY{p}{[}\PY{l+s+s1}{\PYZsq{}}\PY{l+s+s1}{pc}\PY{l+s+s1}{\PYZsq{}}\PY{p}{,}\PY{l+s+s1}{\PYZsq{}}\PY{l+s+s1}{x}\PY{l+s+s1}{\PYZsq{}}\PY{p}{,}\PY{l+s+s1}{\PYZsq{}}\PY{l+s+s1}{y}\PY{l+s+s1}{\PYZsq{}}\PY{p}{,}\PY{l+s+s1}{\PYZsq{}}\PY{l+s+s1}{z}\PY{l+s+s1}{\PYZsq{}}\PY{p}{]}\PY{p}{,} \PY{n}{init\PYZus{}bounded}\PY{p}{,} \PY{n}{trans\PYZus{}more\PYZus{}states}\PY{p}{,} \PY{n}{error}\PY{p}{,} \PY{n}{N}\PY{p}{,} \PY{n}{M}\PY{p}{,} \PY{n}{a}\PY{p}{,} \PY{n}{b}\PY{p}{,} \PY{n}{tamanho}\PY{p}{)}    
\end{Verbatim}
\end{tcolorbox}

    \begin{Verbatim}[commandchars=\\\{\}]
Não foi possível encontrar o majorante.
Qual é a variável que pretende incrementar. m
Quantidade 9
Safe
    \end{Verbatim}

    \hypertarget{exemplo-4}{%
\subsubsection{Exemplo 4}\label{exemplo-4}}

Aqui mostramos mais exemplos da execução da função
\texttt{model\_checking}

    \begin{tcolorbox}[breakable, size=fbox, boxrule=1pt, pad at break*=1mm,colback=cellbackground, colframe=cellborder]
\prompt{In}{incolor}{10}{\boxspacing}
\begin{Verbatim}[commandchars=\\\{\}]
\PY{n}{N} \PY{o}{=} \PY{l+m+mi}{100}
\PY{n}{M} \PY{o}{=} \PY{l+m+mi}{100}
\PY{n}{a} \PY{o}{=} \PY{l+m+mi}{10}
\PY{n}{b} \PY{o}{=} \PY{l+m+mi}{10}
\PY{n}{tamanho} \PY{o}{=} \PY{l+m+mi}{4}
\end{Verbatim}
\end{tcolorbox}

    \begin{tcolorbox}[breakable, size=fbox, boxrule=1pt, pad at break*=1mm,colback=cellbackground, colframe=cellborder]
\prompt{In}{incolor}{11}{\boxspacing}
\begin{Verbatim}[commandchars=\\\{\}]
\PY{n}{model\PYZus{}checking}\PY{p}{(}\PY{p}{[}\PY{l+s+s1}{\PYZsq{}}\PY{l+s+s1}{pc}\PY{l+s+s1}{\PYZsq{}}\PY{p}{,}\PY{l+s+s1}{\PYZsq{}}\PY{l+s+s1}{x}\PY{l+s+s1}{\PYZsq{}}\PY{p}{,}\PY{l+s+s1}{\PYZsq{}}\PY{l+s+s1}{y}\PY{l+s+s1}{\PYZsq{}}\PY{p}{,}\PY{l+s+s1}{\PYZsq{}}\PY{l+s+s1}{z}\PY{l+s+s1}{\PYZsq{}}\PY{p}{]}\PY{p}{,} \PY{n}{init\PYZus{}fixed}\PY{p}{,} \PY{n}{trans}\PY{p}{,} \PY{n}{error}\PY{p}{,} \PY{n}{N}\PY{p}{,} \PY{n}{M}\PY{p}{,} \PY{n}{a}\PY{p}{,} \PY{n}{b}\PY{p}{,} \PY{n}{tamanho}\PY{p}{)}    
\end{Verbatim}
\end{tcolorbox}

    \begin{Verbatim}[commandchars=\\\{\}]
unsafe
    \end{Verbatim}

    \begin{tcolorbox}[breakable, size=fbox, boxrule=1pt, pad at break*=1mm,colback=cellbackground, colframe=cellborder]
\prompt{In}{incolor}{20}{\boxspacing}
\begin{Verbatim}[commandchars=\\\{\}]
\PY{n}{N} \PY{o}{=} \PY{l+m+mi}{100}
\PY{n}{M} \PY{o}{=} \PY{l+m+mi}{100}
\PY{n}{a} \PY{o}{=} \PY{l+m+mi}{2}
\PY{n}{b} \PY{o}{=} \PY{l+m+mi}{15}
\PY{n}{tamanho} \PY{o}{=} \PY{l+m+mi}{5}
\end{Verbatim}
\end{tcolorbox}

    \begin{tcolorbox}[breakable, size=fbox, boxrule=1pt, pad at break*=1mm,colback=cellbackground, colframe=cellborder]
\prompt{In}{incolor}{21}{\boxspacing}
\begin{Verbatim}[commandchars=\\\{\}]
\PY{n}{model\PYZus{}checking}\PY{p}{(}\PY{p}{[}\PY{l+s+s1}{\PYZsq{}}\PY{l+s+s1}{pc}\PY{l+s+s1}{\PYZsq{}}\PY{p}{,}\PY{l+s+s1}{\PYZsq{}}\PY{l+s+s1}{x}\PY{l+s+s1}{\PYZsq{}}\PY{p}{,}\PY{l+s+s1}{\PYZsq{}}\PY{l+s+s1}{y}\PY{l+s+s1}{\PYZsq{}}\PY{p}{,}\PY{l+s+s1}{\PYZsq{}}\PY{l+s+s1}{z}\PY{l+s+s1}{\PYZsq{}}\PY{p}{]}\PY{p}{,} \PY{n}{init\PYZus{}bounded}\PY{p}{,} \PY{n}{trans}\PY{p}{,} \PY{n}{error}\PY{p}{,} \PY{n}{N}\PY{p}{,} \PY{n}{M}\PY{p}{,} \PY{n}{a}\PY{p}{,} \PY{n}{b}\PY{p}{,} \PY{n}{tamanho}\PY{p}{)}    
\end{Verbatim}
\end{tcolorbox}

    \begin{Verbatim}[commandchars=\\\{\}]
Não foi possível encontrar o majorante.
Não foi possível encontrar o majorante.
Não foi possível encontrar o majorante.
Não foi possível encontrar o majorante.
Não foi possível encontrar o majorante.
Não foi possível encontrar o majorante.
Não foi possível encontrar o majorante.
Não foi possível encontrar o majorante.
Não foi possível encontrar o majorante.
Não foi possível encontrar o majorante.
Não foi possível encontrar o majorante.
Não foi possível encontrar o majorante.
Não foi possível encontrar o majorante.
Não foi possível encontrar o majorante.
Não foi possível encontrar o majorante.
1 6
Safe
    \end{Verbatim}


    % Add a bibliography block to the postdoc
    
    
    
\end{document}
