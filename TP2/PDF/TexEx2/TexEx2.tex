\documentclass[11pt]{article}

    \usepackage[breakable]{tcolorbox}
    \usepackage{parskip} % Stop auto-indenting (to mimic markdown behaviour)
    

    % Basic figure setup, for now with no caption control since it's done
    % automatically by Pandoc (which extracts ![](path) syntax from Markdown).
    \usepackage{graphicx}
    % Maintain compatibility with old templates. Remove in nbconvert 6.0
    \let\Oldincludegraphics\includegraphics
    % Ensure that by default, figures have no caption (until we provide a
    % proper Figure object with a Caption API and a way to capture that
    % in the conversion process - todo).
    \usepackage{caption}
    \DeclareCaptionFormat{nocaption}{}
    \captionsetup{format=nocaption,aboveskip=0pt,belowskip=0pt}

    \usepackage{float}
    \floatplacement{figure}{H} % forces figures to be placed at the correct location
    \usepackage{xcolor} % Allow colors to be defined
    \usepackage{enumerate} % Needed for markdown enumerations to work
    \usepackage{geometry} % Used to adjust the document margins
    \usepackage{amsmath} % Equations
    \usepackage{amssymb} % Equations
    \usepackage{textcomp} % defines textquotesingle
    % Hack from http://tex.stackexchange.com/a/47451/13684:
    \AtBeginDocument{%
        \def\PYZsq{\textquotesingle}% Upright quotes in Pygmentized code
    }
    \usepackage{upquote} % Upright quotes for verbatim code
    \usepackage{eurosym} % defines \euro

    \usepackage{iftex}
    \ifPDFTeX
        \usepackage[T1]{fontenc}
        \IfFileExists{alphabeta.sty}{
              \usepackage{alphabeta}
          }{
              \usepackage[mathletters]{ucs}
              \usepackage[utf8x]{inputenc}
          }
    \else
        \usepackage{fontspec}
        \usepackage{unicode-math}
    \fi

    \usepackage{fancyvrb} % verbatim replacement that allows latex
    \usepackage{grffile} % extends the file name processing of package graphics 
                         % to support a larger range
    \makeatletter % fix for old versions of grffile with XeLaTeX
    \@ifpackagelater{grffile}{2019/11/01}
    {
      % Do nothing on new versions
    }
    {
      \def\Gread@@xetex#1{%
        \IfFileExists{"\Gin@base".bb}%
        {\Gread@eps{\Gin@base.bb}}%
        {\Gread@@xetex@aux#1}%
      }
    }
    \makeatother
    \usepackage[Export]{adjustbox} % Used to constrain images to a maximum size
    \adjustboxset{max size={0.9\linewidth}{0.9\paperheight}}

    % The hyperref package gives us a pdf with properly built
    % internal navigation ('pdf bookmarks' for the table of contents,
    % internal cross-reference links, web links for URLs, etc.)
    \usepackage{hyperref}
    % The default LaTeX title has an obnoxious amount of whitespace. By default,
    % titling removes some of it. It also provides customization options.
    \usepackage{titling}
    \usepackage{longtable} % longtable support required by pandoc >1.10
    \usepackage{booktabs}  % table support for pandoc > 1.12.2
    \usepackage{array}     % table support for pandoc >= 2.11.3
    \usepackage{calc}      % table minipage width calculation for pandoc >= 2.11.1
    \usepackage[inline]{enumitem} % IRkernel/repr support (it uses the enumerate* environment)
    \usepackage[normalem]{ulem} % ulem is needed to support strikethroughs (\sout)
                                % normalem makes italics be italics, not underlines
    \usepackage{mathrsfs}
    

    
    % Colors for the hyperref package
    \definecolor{urlcolor}{rgb}{0,.145,.698}
    \definecolor{linkcolor}{rgb}{.71,0.21,0.01}
    \definecolor{citecolor}{rgb}{.12,.54,.11}

    % ANSI colors
    \definecolor{ansi-black}{HTML}{3E424D}
    \definecolor{ansi-black-intense}{HTML}{282C36}
    \definecolor{ansi-red}{HTML}{E75C58}
    \definecolor{ansi-red-intense}{HTML}{B22B31}
    \definecolor{ansi-green}{HTML}{00A250}
    \definecolor{ansi-green-intense}{HTML}{007427}
    \definecolor{ansi-yellow}{HTML}{DDB62B}
    \definecolor{ansi-yellow-intense}{HTML}{B27D12}
    \definecolor{ansi-blue}{HTML}{208FFB}
    \definecolor{ansi-blue-intense}{HTML}{0065CA}
    \definecolor{ansi-magenta}{HTML}{D160C4}
    \definecolor{ansi-magenta-intense}{HTML}{A03196}
    \definecolor{ansi-cyan}{HTML}{60C6C8}
    \definecolor{ansi-cyan-intense}{HTML}{258F8F}
    \definecolor{ansi-white}{HTML}{C5C1B4}
    \definecolor{ansi-white-intense}{HTML}{A1A6B2}
    \definecolor{ansi-default-inverse-fg}{HTML}{FFFFFF}
    \definecolor{ansi-default-inverse-bg}{HTML}{000000}

    % common color for the border for error outputs.
    \definecolor{outerrorbackground}{HTML}{FFDFDF}

    % commands and environments needed by pandoc snippets
    % extracted from the output of `pandoc -s`
    \providecommand{\tightlist}{%
      \setlength{\itemsep}{0pt}\setlength{\parskip}{0pt}}
    \DefineVerbatimEnvironment{Highlighting}{Verbatim}{commandchars=\\\{\}}
    % Add ',fontsize=\small' for more characters per line
    \newenvironment{Shaded}{}{}
    \newcommand{\KeywordTok}[1]{\textcolor[rgb]{0.00,0.44,0.13}{\textbf{{#1}}}}
    \newcommand{\DataTypeTok}[1]{\textcolor[rgb]{0.56,0.13,0.00}{{#1}}}
    \newcommand{\DecValTok}[1]{\textcolor[rgb]{0.25,0.63,0.44}{{#1}}}
    \newcommand{\BaseNTok}[1]{\textcolor[rgb]{0.25,0.63,0.44}{{#1}}}
    \newcommand{\FloatTok}[1]{\textcolor[rgb]{0.25,0.63,0.44}{{#1}}}
    \newcommand{\CharTok}[1]{\textcolor[rgb]{0.25,0.44,0.63}{{#1}}}
    \newcommand{\StringTok}[1]{\textcolor[rgb]{0.25,0.44,0.63}{{#1}}}
    \newcommand{\CommentTok}[1]{\textcolor[rgb]{0.38,0.63,0.69}{\textit{{#1}}}}
    \newcommand{\OtherTok}[1]{\textcolor[rgb]{0.00,0.44,0.13}{{#1}}}
    \newcommand{\AlertTok}[1]{\textcolor[rgb]{1.00,0.00,0.00}{\textbf{{#1}}}}
    \newcommand{\FunctionTok}[1]{\textcolor[rgb]{0.02,0.16,0.49}{{#1}}}
    \newcommand{\RegionMarkerTok}[1]{{#1}}
    \newcommand{\ErrorTok}[1]{\textcolor[rgb]{1.00,0.00,0.00}{\textbf{{#1}}}}
    \newcommand{\NormalTok}[1]{{#1}}
    
    % Additional commands for more recent versions of Pandoc
    \newcommand{\ConstantTok}[1]{\textcolor[rgb]{0.53,0.00,0.00}{{#1}}}
    \newcommand{\SpecialCharTok}[1]{\textcolor[rgb]{0.25,0.44,0.63}{{#1}}}
    \newcommand{\VerbatimStringTok}[1]{\textcolor[rgb]{0.25,0.44,0.63}{{#1}}}
    \newcommand{\SpecialStringTok}[1]{\textcolor[rgb]{0.73,0.40,0.53}{{#1}}}
    \newcommand{\ImportTok}[1]{{#1}}
    \newcommand{\DocumentationTok}[1]{\textcolor[rgb]{0.73,0.13,0.13}{\textit{{#1}}}}
    \newcommand{\AnnotationTok}[1]{\textcolor[rgb]{0.38,0.63,0.69}{\textbf{\textit{{#1}}}}}
    \newcommand{\CommentVarTok}[1]{\textcolor[rgb]{0.38,0.63,0.69}{\textbf{\textit{{#1}}}}}
    \newcommand{\VariableTok}[1]{\textcolor[rgb]{0.10,0.09,0.49}{{#1}}}
    \newcommand{\ControlFlowTok}[1]{\textcolor[rgb]{0.00,0.44,0.13}{\textbf{{#1}}}}
    \newcommand{\OperatorTok}[1]{\textcolor[rgb]{0.40,0.40,0.40}{{#1}}}
    \newcommand{\BuiltInTok}[1]{{#1}}
    \newcommand{\ExtensionTok}[1]{{#1}}
    \newcommand{\PreprocessorTok}[1]{\textcolor[rgb]{0.74,0.48,0.00}{{#1}}}
    \newcommand{\AttributeTok}[1]{\textcolor[rgb]{0.49,0.56,0.16}{{#1}}}
    \newcommand{\InformationTok}[1]{\textcolor[rgb]{0.38,0.63,0.69}{\textbf{\textit{{#1}}}}}
    \newcommand{\WarningTok}[1]{\textcolor[rgb]{0.38,0.63,0.69}{\textbf{\textit{{#1}}}}}
    
    
    % Define a nice break command that doesn't care if a line doesn't already
    % exist.
    \def\br{\hspace*{\fill} \\* }
    % Math Jax compatibility definitions
    \def\gt{>}
    \def\lt{<}
    \let\Oldtex\TeX
    \let\Oldlatex\LaTeX
    \renewcommand{\TeX}{\textrm{\Oldtex}}
    \renewcommand{\LaTeX}{\textrm{\Oldlatex}}
    % Document parameters
    % Document title
    \title{TP2-Exercicio2}
    
    
    
    
    
% Pygments definitions
\makeatletter
\def\PY@reset{\let\PY@it=\relax \let\PY@bf=\relax%
    \let\PY@ul=\relax \let\PY@tc=\relax%
    \let\PY@bc=\relax \let\PY@ff=\relax}
\def\PY@tok#1{\csname PY@tok@#1\endcsname}
\def\PY@toks#1+{\ifx\relax#1\empty\else%
    \PY@tok{#1}\expandafter\PY@toks\fi}
\def\PY@do#1{\PY@bc{\PY@tc{\PY@ul{%
    \PY@it{\PY@bf{\PY@ff{#1}}}}}}}
\def\PY#1#2{\PY@reset\PY@toks#1+\relax+\PY@do{#2}}

\@namedef{PY@tok@w}{\def\PY@tc##1{\textcolor[rgb]{0.73,0.73,0.73}{##1}}}
\@namedef{PY@tok@c}{\let\PY@it=\textit\def\PY@tc##1{\textcolor[rgb]{0.24,0.48,0.48}{##1}}}
\@namedef{PY@tok@cp}{\def\PY@tc##1{\textcolor[rgb]{0.61,0.40,0.00}{##1}}}
\@namedef{PY@tok@k}{\let\PY@bf=\textbf\def\PY@tc##1{\textcolor[rgb]{0.00,0.50,0.00}{##1}}}
\@namedef{PY@tok@kp}{\def\PY@tc##1{\textcolor[rgb]{0.00,0.50,0.00}{##1}}}
\@namedef{PY@tok@kt}{\def\PY@tc##1{\textcolor[rgb]{0.69,0.00,0.25}{##1}}}
\@namedef{PY@tok@o}{\def\PY@tc##1{\textcolor[rgb]{0.40,0.40,0.40}{##1}}}
\@namedef{PY@tok@ow}{\let\PY@bf=\textbf\def\PY@tc##1{\textcolor[rgb]{0.67,0.13,1.00}{##1}}}
\@namedef{PY@tok@nb}{\def\PY@tc##1{\textcolor[rgb]{0.00,0.50,0.00}{##1}}}
\@namedef{PY@tok@nf}{\def\PY@tc##1{\textcolor[rgb]{0.00,0.00,1.00}{##1}}}
\@namedef{PY@tok@nc}{\let\PY@bf=\textbf\def\PY@tc##1{\textcolor[rgb]{0.00,0.00,1.00}{##1}}}
\@namedef{PY@tok@nn}{\let\PY@bf=\textbf\def\PY@tc##1{\textcolor[rgb]{0.00,0.00,1.00}{##1}}}
\@namedef{PY@tok@ne}{\let\PY@bf=\textbf\def\PY@tc##1{\textcolor[rgb]{0.80,0.25,0.22}{##1}}}
\@namedef{PY@tok@nv}{\def\PY@tc##1{\textcolor[rgb]{0.10,0.09,0.49}{##1}}}
\@namedef{PY@tok@no}{\def\PY@tc##1{\textcolor[rgb]{0.53,0.00,0.00}{##1}}}
\@namedef{PY@tok@nl}{\def\PY@tc##1{\textcolor[rgb]{0.46,0.46,0.00}{##1}}}
\@namedef{PY@tok@ni}{\let\PY@bf=\textbf\def\PY@tc##1{\textcolor[rgb]{0.44,0.44,0.44}{##1}}}
\@namedef{PY@tok@na}{\def\PY@tc##1{\textcolor[rgb]{0.41,0.47,0.13}{##1}}}
\@namedef{PY@tok@nt}{\let\PY@bf=\textbf\def\PY@tc##1{\textcolor[rgb]{0.00,0.50,0.00}{##1}}}
\@namedef{PY@tok@nd}{\def\PY@tc##1{\textcolor[rgb]{0.67,0.13,1.00}{##1}}}
\@namedef{PY@tok@s}{\def\PY@tc##1{\textcolor[rgb]{0.73,0.13,0.13}{##1}}}
\@namedef{PY@tok@sd}{\let\PY@it=\textit\def\PY@tc##1{\textcolor[rgb]{0.73,0.13,0.13}{##1}}}
\@namedef{PY@tok@si}{\let\PY@bf=\textbf\def\PY@tc##1{\textcolor[rgb]{0.64,0.35,0.47}{##1}}}
\@namedef{PY@tok@se}{\let\PY@bf=\textbf\def\PY@tc##1{\textcolor[rgb]{0.67,0.36,0.12}{##1}}}
\@namedef{PY@tok@sr}{\def\PY@tc##1{\textcolor[rgb]{0.64,0.35,0.47}{##1}}}
\@namedef{PY@tok@ss}{\def\PY@tc##1{\textcolor[rgb]{0.10,0.09,0.49}{##1}}}
\@namedef{PY@tok@sx}{\def\PY@tc##1{\textcolor[rgb]{0.00,0.50,0.00}{##1}}}
\@namedef{PY@tok@m}{\def\PY@tc##1{\textcolor[rgb]{0.40,0.40,0.40}{##1}}}
\@namedef{PY@tok@gh}{\let\PY@bf=\textbf\def\PY@tc##1{\textcolor[rgb]{0.00,0.00,0.50}{##1}}}
\@namedef{PY@tok@gu}{\let\PY@bf=\textbf\def\PY@tc##1{\textcolor[rgb]{0.50,0.00,0.50}{##1}}}
\@namedef{PY@tok@gd}{\def\PY@tc##1{\textcolor[rgb]{0.63,0.00,0.00}{##1}}}
\@namedef{PY@tok@gi}{\def\PY@tc##1{\textcolor[rgb]{0.00,0.52,0.00}{##1}}}
\@namedef{PY@tok@gr}{\def\PY@tc##1{\textcolor[rgb]{0.89,0.00,0.00}{##1}}}
\@namedef{PY@tok@ge}{\let\PY@it=\textit}
\@namedef{PY@tok@gs}{\let\PY@bf=\textbf}
\@namedef{PY@tok@gp}{\let\PY@bf=\textbf\def\PY@tc##1{\textcolor[rgb]{0.00,0.00,0.50}{##1}}}
\@namedef{PY@tok@go}{\def\PY@tc##1{\textcolor[rgb]{0.44,0.44,0.44}{##1}}}
\@namedef{PY@tok@gt}{\def\PY@tc##1{\textcolor[rgb]{0.00,0.27,0.87}{##1}}}
\@namedef{PY@tok@err}{\def\PY@bc##1{{\setlength{\fboxsep}{\string -\fboxrule}\fcolorbox[rgb]{1.00,0.00,0.00}{1,1,1}{\strut ##1}}}}
\@namedef{PY@tok@kc}{\let\PY@bf=\textbf\def\PY@tc##1{\textcolor[rgb]{0.00,0.50,0.00}{##1}}}
\@namedef{PY@tok@kd}{\let\PY@bf=\textbf\def\PY@tc##1{\textcolor[rgb]{0.00,0.50,0.00}{##1}}}
\@namedef{PY@tok@kn}{\let\PY@bf=\textbf\def\PY@tc##1{\textcolor[rgb]{0.00,0.50,0.00}{##1}}}
\@namedef{PY@tok@kr}{\let\PY@bf=\textbf\def\PY@tc##1{\textcolor[rgb]{0.00,0.50,0.00}{##1}}}
\@namedef{PY@tok@bp}{\def\PY@tc##1{\textcolor[rgb]{0.00,0.50,0.00}{##1}}}
\@namedef{PY@tok@fm}{\def\PY@tc##1{\textcolor[rgb]{0.00,0.00,1.00}{##1}}}
\@namedef{PY@tok@vc}{\def\PY@tc##1{\textcolor[rgb]{0.10,0.09,0.49}{##1}}}
\@namedef{PY@tok@vg}{\def\PY@tc##1{\textcolor[rgb]{0.10,0.09,0.49}{##1}}}
\@namedef{PY@tok@vi}{\def\PY@tc##1{\textcolor[rgb]{0.10,0.09,0.49}{##1}}}
\@namedef{PY@tok@vm}{\def\PY@tc##1{\textcolor[rgb]{0.10,0.09,0.49}{##1}}}
\@namedef{PY@tok@sa}{\def\PY@tc##1{\textcolor[rgb]{0.73,0.13,0.13}{##1}}}
\@namedef{PY@tok@sb}{\def\PY@tc##1{\textcolor[rgb]{0.73,0.13,0.13}{##1}}}
\@namedef{PY@tok@sc}{\def\PY@tc##1{\textcolor[rgb]{0.73,0.13,0.13}{##1}}}
\@namedef{PY@tok@dl}{\def\PY@tc##1{\textcolor[rgb]{0.73,0.13,0.13}{##1}}}
\@namedef{PY@tok@s2}{\def\PY@tc##1{\textcolor[rgb]{0.73,0.13,0.13}{##1}}}
\@namedef{PY@tok@sh}{\def\PY@tc##1{\textcolor[rgb]{0.73,0.13,0.13}{##1}}}
\@namedef{PY@tok@s1}{\def\PY@tc##1{\textcolor[rgb]{0.73,0.13,0.13}{##1}}}
\@namedef{PY@tok@mb}{\def\PY@tc##1{\textcolor[rgb]{0.40,0.40,0.40}{##1}}}
\@namedef{PY@tok@mf}{\def\PY@tc##1{\textcolor[rgb]{0.40,0.40,0.40}{##1}}}
\@namedef{PY@tok@mh}{\def\PY@tc##1{\textcolor[rgb]{0.40,0.40,0.40}{##1}}}
\@namedef{PY@tok@mi}{\def\PY@tc##1{\textcolor[rgb]{0.40,0.40,0.40}{##1}}}
\@namedef{PY@tok@il}{\def\PY@tc##1{\textcolor[rgb]{0.40,0.40,0.40}{##1}}}
\@namedef{PY@tok@mo}{\def\PY@tc##1{\textcolor[rgb]{0.40,0.40,0.40}{##1}}}
\@namedef{PY@tok@ch}{\let\PY@it=\textit\def\PY@tc##1{\textcolor[rgb]{0.24,0.48,0.48}{##1}}}
\@namedef{PY@tok@cm}{\let\PY@it=\textit\def\PY@tc##1{\textcolor[rgb]{0.24,0.48,0.48}{##1}}}
\@namedef{PY@tok@cpf}{\let\PY@it=\textit\def\PY@tc##1{\textcolor[rgb]{0.24,0.48,0.48}{##1}}}
\@namedef{PY@tok@c1}{\let\PY@it=\textit\def\PY@tc##1{\textcolor[rgb]{0.24,0.48,0.48}{##1}}}
\@namedef{PY@tok@cs}{\let\PY@it=\textit\def\PY@tc##1{\textcolor[rgb]{0.24,0.48,0.48}{##1}}}

\def\PYZbs{\char`\\}
\def\PYZus{\char`\_}
\def\PYZob{\char`\{}
\def\PYZcb{\char`\}}
\def\PYZca{\char`\^}
\def\PYZam{\char`\&}
\def\PYZlt{\char`\<}
\def\PYZgt{\char`\>}
\def\PYZsh{\char`\#}
\def\PYZpc{\char`\%}
\def\PYZdl{\char`\$}
\def\PYZhy{\char`\-}
\def\PYZsq{\char`\'}
\def\PYZdq{\char`\"}
\def\PYZti{\char`\~}
% for compatibility with earlier versions
\def\PYZat{@}
\def\PYZlb{[}
\def\PYZrb{]}
\makeatother


    % For linebreaks inside Verbatim environment from package fancyvrb. 
    \makeatletter
        \newbox\Wrappedcontinuationbox 
        \newbox\Wrappedvisiblespacebox 
        \newcommand*\Wrappedvisiblespace {\textcolor{red}{\textvisiblespace}} 
        \newcommand*\Wrappedcontinuationsymbol {\textcolor{red}{\llap{\tiny$\m@th\hookrightarrow$}}} 
        \newcommand*\Wrappedcontinuationindent {3ex } 
        \newcommand*\Wrappedafterbreak {\kern\Wrappedcontinuationindent\copy\Wrappedcontinuationbox} 
        % Take advantage of the already applied Pygments mark-up to insert 
        % potential linebreaks for TeX processing. 
        %        {, <, #, %, $, ' and ": go to next line. 
        %        _, }, ^, &, >, - and ~: stay at end of broken line. 
        % Use of \textquotesingle for straight quote. 
        \newcommand*\Wrappedbreaksatspecials {% 
            \def\PYGZus{\discretionary{\char`\_}{\Wrappedafterbreak}{\char`\_}}% 
            \def\PYGZob{\discretionary{}{\Wrappedafterbreak\char`\{}{\char`\{}}% 
            \def\PYGZcb{\discretionary{\char`\}}{\Wrappedafterbreak}{\char`\}}}% 
            \def\PYGZca{\discretionary{\char`\^}{\Wrappedafterbreak}{\char`\^}}% 
            \def\PYGZam{\discretionary{\char`\&}{\Wrappedafterbreak}{\char`\&}}% 
            \def\PYGZlt{\discretionary{}{\Wrappedafterbreak\char`\<}{\char`\<}}% 
            \def\PYGZgt{\discretionary{\char`\>}{\Wrappedafterbreak}{\char`\>}}% 
            \def\PYGZsh{\discretionary{}{\Wrappedafterbreak\char`\#}{\char`\#}}% 
            \def\PYGZpc{\discretionary{}{\Wrappedafterbreak\char`\%}{\char`\%}}% 
            \def\PYGZdl{\discretionary{}{\Wrappedafterbreak\char`\$}{\char`\$}}% 
            \def\PYGZhy{\discretionary{\char`\-}{\Wrappedafterbreak}{\char`\-}}% 
            \def\PYGZsq{\discretionary{}{\Wrappedafterbreak\textquotesingle}{\textquotesingle}}% 
            \def\PYGZdq{\discretionary{}{\Wrappedafterbreak\char`\"}{\char`\"}}% 
            \def\PYGZti{\discretionary{\char`\~}{\Wrappedafterbreak}{\char`\~}}% 
        } 
        % Some characters . , ; ? ! / are not pygmentized. 
        % This macro makes them "active" and they will insert potential linebreaks 
        \newcommand*\Wrappedbreaksatpunct {% 
            \lccode`\~`\.\lowercase{\def~}{\discretionary{\hbox{\char`\.}}{\Wrappedafterbreak}{\hbox{\char`\.}}}% 
            \lccode`\~`\,\lowercase{\def~}{\discretionary{\hbox{\char`\,}}{\Wrappedafterbreak}{\hbox{\char`\,}}}% 
            \lccode`\~`\;\lowercase{\def~}{\discretionary{\hbox{\char`\;}}{\Wrappedafterbreak}{\hbox{\char`\;}}}% 
            \lccode`\~`\:\lowercase{\def~}{\discretionary{\hbox{\char`\:}}{\Wrappedafterbreak}{\hbox{\char`\:}}}% 
            \lccode`\~`\?\lowercase{\def~}{\discretionary{\hbox{\char`\?}}{\Wrappedafterbreak}{\hbox{\char`\?}}}% 
            \lccode`\~`\!\lowercase{\def~}{\discretionary{\hbox{\char`\!}}{\Wrappedafterbreak}{\hbox{\char`\!}}}% 
            \lccode`\~`\/\lowercase{\def~}{\discretionary{\hbox{\char`\/}}{\Wrappedafterbreak}{\hbox{\char`\/}}}% 
            \catcode`\.\active
            \catcode`\,\active 
            \catcode`\;\active
            \catcode`\:\active
            \catcode`\?\active
            \catcode`\!\active
            \catcode`\/\active 
            \lccode`\~`\~ 	
        }
    \makeatother

    \let\OriginalVerbatim=\Verbatim
    \makeatletter
    \renewcommand{\Verbatim}[1][1]{%
        %\parskip\z@skip
        \sbox\Wrappedcontinuationbox {\Wrappedcontinuationsymbol}%
        \sbox\Wrappedvisiblespacebox {\FV@SetupFont\Wrappedvisiblespace}%
        \def\FancyVerbFormatLine ##1{\hsize\linewidth
            \vtop{\raggedright\hyphenpenalty\z@\exhyphenpenalty\z@
                \doublehyphendemerits\z@\finalhyphendemerits\z@
                \strut ##1\strut}%
        }%
        % If the linebreak is at a space, the latter will be displayed as visible
        % space at end of first line, and a continuation symbol starts next line.
        % Stretch/shrink are however usually zero for typewriter font.
        \def\FV@Space {%
            \nobreak\hskip\z@ plus\fontdimen3\font minus\fontdimen4\font
            \discretionary{\copy\Wrappedvisiblespacebox}{\Wrappedafterbreak}
            {\kern\fontdimen2\font}%
        }%
        
        % Allow breaks at special characters using \PYG... macros.
        \Wrappedbreaksatspecials
        % Breaks at punctuation characters . , ; ? ! and / need catcode=\active 	
        \OriginalVerbatim[#1,codes*=\Wrappedbreaksatpunct]%
    }
    \makeatother

    % Exact colors from NB
    \definecolor{incolor}{HTML}{303F9F}
    \definecolor{outcolor}{HTML}{D84315}
    \definecolor{cellborder}{HTML}{CFCFCF}
    \definecolor{cellbackground}{HTML}{F7F7F7}
    
    % prompt
    \makeatletter
    \newcommand{\boxspacing}{\kern\kvtcb@left@rule\kern\kvtcb@boxsep}
    \makeatother
    \newcommand{\prompt}[4]{
        {\ttfamily\llap{{\color{#2}[#3]:\hspace{3pt}#4}}\vspace{-\baselineskip}}
    }
    

    
    % Prevent overflowing lines due to hard-to-break entities
    \sloppy 
    % Setup hyperref package
    \hypersetup{
      breaklinks=true,  % so long urls are correctly broken across lines
      colorlinks=true,
      urlcolor=urlcolor,
      linkcolor=linkcolor,
      citecolor=citecolor,
      }
    % Slightly bigger margins than the latex defaults
    
    \geometry{verbose,tmargin=1in,bmargin=1in,lmargin=1in,rmargin=1in}
    
    

\begin{document}
    
    \maketitle
    
    

    
    \hypertarget{tp2}{%
\section{TP2}\label{tp2}}

\hypertarget{grupo-15}{%
\subsection{Grupo 15}\label{grupo-15}}

Carlos Eduardo Da Silva Machado A96936

Gonçalo Manuel Maia de Sousa A97485

    \hypertarget{problema-2}{%
\subsection{Problema 2}\label{problema-2}}

    \hypertarget{descriuxe7uxe3o-do-problema}{%
\subsubsection{Descrição do
Problema}\label{descriuxe7uxe3o-do-problema}}

    \begin{enumerate}
\def\labelenumi{\arabic{enumi}.}
\setcounter{enumi}{1}
\tightlist
\item
  O Conway's Game of Life é um exemplo bastante conhecido de um autómato
  celular . Neste problema vamos modificar as regras do autómato da
  seguinte forma

  \begin{enumerate}
  \def\labelenumii{\arabic{enumii}.}
  \tightlist
  \item
    O espaço de estados é finito definido por uma grelha de células
    booleanas (morta=0/viva=1) de dimensão \(\,N\times N\,\) (com
    \(N>3\)) identificadas por índices \(\,(i,j)\in \{1..N\}\). Estas
    \(\;N^2\;\) células são aqui referidas como ``normais''.
  \item
    No estado inicial todas as células normais estão mortas excepto um
    quadrado \(\,3\times 3\,\), designado por ``centro'', aleatoriamente
    posicionado formado apenas por células vivas.
  \item
    Adicionalmente existem \(\,2\,N+1\,\) ``células da borda'' que
    correspondem a um dos índices, \(i\) ou \(j\), ser zero. As células
    da borda têm valores constantes que, no estado inicial, são gerados
    aleatoriamente com uma probabilidade \(\,\rho\,\) de estarem vivas.
  \item
    As células normais o autómato modificam o estado de acordo com a
    regra ``B3/S23'': i.e.~a célula nasce (passa de \(0\) a \(1\)) se
    tem exatamente 3 vizinhos vivos e sobrevive (mantém-se viva) se o
    número de vizinhos vivos é 2 ou 3, caso contrário morre ou continua
    morta.
  \end{enumerate}
\end{enumerate}

    \hypertarget{abordagem-do-problema}{%
\subsubsection{Abordagem do Problema}\label{abordagem-do-problema}}

    Para resolver este problema construimos uma máquina de estados finita
que representa esse autómato, de modo que cada estado é representado
unicamente por uma configuração específica de células vivas e mortas.

São parametros deste problema: 1. \(\rho\) a probabilidade de cada uma
das células da borda estar viva. 2. \(N\) o tamaho do quadro. 3. \(k\) o
número de estados inicialmente reservados para o autómato (caso seja
necessário, este número é duplicado). 4. \(bound\) o número de estados
até ao primeiro repetido. 5. \(centro\) a posição do centro
aleatóriamente posicionado.

Para além do autómato é necessário verificar de as seguintes
propriedades de aplicam: 1. Todos os estados acessíveis contém pelo
menos uma célula viva (apenas são verificadas as células normais). 2.
Toda a célula normal está viva pelo menos uma vez em algum estado
acessível.

    \hypertarget{cuxf3digo-python}{%
\subsection{Código Python}\label{cuxf3digo-python}}

    Esta secção de codigo serve para importar as bibliotecas nesseçárias
para a realização do trabalho.

    \begin{tcolorbox}[breakable, size=fbox, boxrule=1pt, pad at break*=1mm,colback=cellbackground, colframe=cellborder]
\prompt{In}{incolor}{3}{\boxspacing}
\begin{Verbatim}[commandchars=\\\{\}]
\PY{k+kn}{import} \PY{n+nn}{matplotlib}\PY{n+nn}{.}\PY{n+nn}{pyplot} \PY{k}{as} \PY{n+nn}{plt}
\PY{k+kn}{from} \PY{n+nn}{pysmt}\PY{n+nn}{.}\PY{n+nn}{shortcuts} \PY{k+kn}{import} \PY{o}{*}
\PY{k+kn}{from} \PY{n+nn}{pysmt}\PY{n+nn}{.}\PY{n+nn}{typing} \PY{k+kn}{import} \PY{n}{INT}
\PY{k+kn}{import} \PY{n+nn}{random} \PY{k}{as} \PY{n+nn}{rn}
\PY{k+kn}{from} \PY{n+nn}{pprint} \PY{k+kn}{import} \PY{n}{pprint}
\end{Verbatim}
\end{tcolorbox}

    Aqui são apresentadas funções auxiliares, uma delas que gera
aleatóriamente a borda, outro que gera aleatoriamente o centro e uma que
calcula a vizinhança de um ponto.

    \begin{tcolorbox}[breakable, size=fbox, boxrule=1pt, pad at break*=1mm,colback=cellbackground, colframe=cellborder]
\prompt{In}{incolor}{4}{\boxspacing}
\begin{Verbatim}[commandchars=\\\{\}]
\PY{k}{def} \PY{n+nf}{gera\PYZus{}rand}\PY{p}{(}\PY{n}{p}\PY{p}{,}\PY{n}{size}\PY{p}{)}\PY{p}{:}
    \PY{k}{return} \PY{n}{rn}\PY{o}{.}\PY{n}{choices}\PY{p}{(}\PY{p}{[}\PY{l+m+mi}{0}\PY{p}{,}\PY{l+m+mi}{1}\PY{p}{]}\PY{p}{,}\PY{p}{[}\PY{l+m+mi}{1}\PY{o}{\PYZhy{}}\PY{n}{p}\PY{p}{,}\PY{n}{p}\PY{p}{]}\PY{p}{,}\PY{n}{k}\PY{o}{=}\PY{n}{size}\PY{p}{)}

\PY{k}{def} \PY{n+nf}{gera\PYZus{}border}\PY{p}{(}\PY{n}{p}\PY{p}{,}\PY{n}{size}\PY{p}{)}\PY{p}{:}
    \PY{k}{return} \PY{p}{(}\PY{n}{gera\PYZus{}rand}\PY{p}{(}\PY{n}{p}\PY{p}{,}\PY{n}{size}\PY{p}{)}\PY{p}{,}\PY{n}{gera\PYZus{}rand}\PY{p}{(}\PY{n}{p}\PY{p}{,}\PY{n}{size}\PY{o}{\PYZhy{}}\PY{l+m+mi}{1}\PY{p}{)}\PY{p}{)}

\PY{k}{def} \PY{n+nf}{gera\PYZus{}centro}\PY{p}{(}\PY{n}{N}\PY{p}{)}\PY{p}{:}
    \PY{k}{return} \PY{p}{(}\PY{n}{rn}\PY{o}{.}\PY{n}{randint}\PY{p}{(}\PY{l+m+mi}{3}\PY{p}{,}\PY{n}{N}\PY{o}{\PYZhy{}}\PY{l+m+mi}{1}\PY{p}{)}\PY{p}{,}\PY{n}{rn}\PY{o}{.}\PY{n}{randint}\PY{p}{(}\PY{l+m+mi}{3}\PY{p}{,}\PY{n}{N}\PY{o}{\PYZhy{}}\PY{l+m+mi}{1}\PY{p}{)}\PY{p}{)}

\PY{k}{def} \PY{n+nf}{PontosProximos}\PY{p}{(}\PY{n}{pos}\PY{p}{,}\PY{n}{N}\PY{p}{)}\PY{p}{:}
    \PY{n}{x}\PY{p}{,}\PY{n}{y} \PY{o}{=} \PY{n}{pos}
    \PY{n}{p} \PY{o}{=} \PY{p}{[}\PY{p}{(}\PY{n}{x}\PY{o}{+}\PY{l+m+mi}{1}\PY{p}{,}\PY{n}{y}\PY{p}{)}\PY{p}{,}\PY{p}{(}\PY{n}{x}\PY{o}{\PYZhy{}}\PY{l+m+mi}{1}\PY{p}{,}\PY{n}{y}\PY{p}{)}\PY{p}{,}\PY{p}{(}\PY{n}{x}\PY{p}{,}\PY{n}{y}\PY{o}{+}\PY{l+m+mi}{1}\PY{p}{)}\PY{p}{,}\PY{p}{(}\PY{n}{x}\PY{p}{,}\PY{n}{y}\PY{o}{\PYZhy{}}\PY{l+m+mi}{1}\PY{p}{)}\PY{p}{,}\PY{p}{(}\PY{n}{x}\PY{o}{\PYZhy{}}\PY{l+m+mi}{1}\PY{p}{,}\PY{n}{y}\PY{o}{\PYZhy{}}\PY{l+m+mi}{1}\PY{p}{)}\PY{p}{,}\PY{p}{(}\PY{n}{x}\PY{o}{+}\PY{l+m+mi}{1}\PY{p}{,}\PY{n}{y}\PY{o}{+}\PY{l+m+mi}{1}\PY{p}{)}\PY{p}{,}\PY{p}{(}\PY{n}{x}\PY{o}{\PYZhy{}}\PY{l+m+mi}{1}\PY{p}{,}\PY{n}{y}\PY{o}{+}\PY{l+m+mi}{1}\PY{p}{)}\PY{p}{,}\PY{p}{(}\PY{n}{x}\PY{o}{+}\PY{l+m+mi}{1}\PY{p}{,} \PY{n}{y}\PY{o}{\PYZhy{}}\PY{l+m+mi}{1}\PY{p}{)}\PY{p}{]}
    \PY{k}{return} \PY{n+nb}{list}\PY{p}{(}\PY{n+nb}{filter}\PY{p}{(}\PY{k}{lambda} \PY{n}{x}\PY{p}{:} \PY{n}{x}\PY{p}{[}\PY{l+m+mi}{0}\PY{p}{]} \PY{o}{\PYZlt{}} \PY{n}{N}\PY{o}{+}\PY{l+m+mi}{1} \PY{o+ow}{and} \PY{n}{x}\PY{p}{[}\PY{l+m+mi}{1}\PY{p}{]} \PY{o}{\PYZlt{}} \PY{n}{N}\PY{o}{+}\PY{l+m+mi}{1}\PY{p}{,}  \PY{n}{p}\PY{p}{)}\PY{p}{)}
\end{Verbatim}
\end{tcolorbox}

    \hypertarget{construuxe7uxe3o-da-muxe1quina-de-estados-finita}{%
\section{Construção da máquina de estados
finita}\label{construuxe7uxe3o-da-muxe1quina-de-estados-finita}}

    Função de declaração.

    \begin{tcolorbox}[breakable, size=fbox, boxrule=1pt, pad at break*=1mm,colback=cellbackground, colframe=cellborder]
\prompt{In}{incolor}{5}{\boxspacing}
\begin{Verbatim}[commandchars=\\\{\}]
\PY{k}{def} \PY{n+nf}{declare}\PY{p}{(}\PY{n}{c}\PY{p}{,}\PY{n}{N}\PY{p}{)}\PY{p}{:}
    \PY{n}{state} \PY{o}{=} \PY{n+nb}{dict}\PY{p}{(}\PY{p}{)}
    \PY{k}{for} \PY{n}{i} \PY{o+ow}{in} \PY{n+nb}{range}\PY{p}{(}\PY{l+m+mi}{1}\PY{p}{,}\PY{n}{N}\PY{o}{+}\PY{l+m+mi}{1}\PY{p}{)}\PY{p}{:}
        \PY{k}{for} \PY{n}{j} \PY{o+ow}{in} \PY{n+nb}{range}\PY{p}{(}\PY{l+m+mi}{1}\PY{p}{,}\PY{n}{N}\PY{o}{+}\PY{l+m+mi}{1}\PY{p}{)}\PY{p}{:}
            \PY{n}{state}\PY{p}{[}\PY{l+s+sa}{f}\PY{l+s+s1}{\PYZsq{}}\PY{l+s+s1}{(}\PY{l+s+si}{\PYZob{}}\PY{n}{i}\PY{l+s+si}{\PYZcb{}}\PY{l+s+s1}{,}\PY{l+s+si}{\PYZob{}}\PY{n}{j}\PY{l+s+si}{\PYZcb{}}\PY{l+s+s1}{)}\PY{l+s+s1}{\PYZsq{}}\PY{p}{]} \PY{o}{=} \PY{n}{Symbol}\PY{p}{(}\PY{l+s+sa}{f}\PY{l+s+s1}{\PYZsq{}}\PY{l+s+s1}{(}\PY{l+s+si}{\PYZob{}}\PY{n}{i}\PY{l+s+si}{\PYZcb{}}\PY{l+s+s1}{,}\PY{l+s+si}{\PYZob{}}\PY{n}{j}\PY{l+s+si}{\PYZcb{}}\PY{l+s+s1}{)}\PY{l+s+s1}{\PYZsq{}}\PY{o}{+}\PY{n+nb}{str}\PY{p}{(}\PY{n}{c}\PY{p}{)}\PY{p}{,}\PY{n}{INT}\PY{p}{)}
    \PY{k}{return} \PY{n}{state}
\end{Verbatim}
\end{tcolorbox}

    Função inicial. Nesta função, garantimos que a borda e o centro são
iguais aos que foram gerados aleatoriamente e garantimos também que os
pontos do centro estão vivos, enquanto que os restantes pontos normais
estão mortos.

    \begin{tcolorbox}[breakable, size=fbox, boxrule=1pt, pad at break*=1mm,colback=cellbackground, colframe=cellborder]
\prompt{In}{incolor}{6}{\boxspacing}
\begin{Verbatim}[commandchars=\\\{\}]
\PY{k}{def} \PY{n+nf}{init}\PY{p}{(}\PY{n}{state}\PY{p}{,}\PY{n}{N}\PY{p}{,}\PY{n}{centro}\PY{p}{,}\PY{n}{border}\PY{p}{)}\PY{p}{:}
    
    \PY{n}{l} \PY{o}{=} \PY{n+nb}{list}\PY{p}{(}\PY{p}{)}
    
    \PY{k}{for} \PY{n}{i}\PY{p}{,}\PY{n}{x} \PY{o+ow}{in} \PY{n+nb}{enumerate}\PY{p}{(}\PY{n}{border}\PY{p}{[}\PY{l+m+mi}{0}\PY{p}{]}\PY{p}{,}\PY{n}{start}\PY{o}{=}\PY{l+m+mi}{1}\PY{p}{)}\PY{p}{:}
        \PY{n}{l}\PY{o}{.}\PY{n}{append}\PY{p}{(}\PY{n}{Equals}\PY{p}{(}\PY{n}{state}\PY{p}{[}\PY{l+s+sa}{f}\PY{l+s+s1}{\PYZsq{}}\PY{l+s+s1}{(}\PY{l+s+si}{\PYZob{}}\PY{n}{i}\PY{l+s+si}{\PYZcb{}}\PY{l+s+s1}{,1)}\PY{l+s+s1}{\PYZsq{}}\PY{p}{]}\PY{p}{,} \PY{n}{Int}\PY{p}{(}\PY{n}{x}\PY{p}{)}\PY{p}{)}\PY{p}{)}
    \PY{k}{for} \PY{n}{j}\PY{p}{,}\PY{n}{x} \PY{o+ow}{in} \PY{n+nb}{enumerate}\PY{p}{(}\PY{n}{border}\PY{p}{[}\PY{l+m+mi}{1}\PY{p}{]}\PY{p}{,}\PY{n}{start}\PY{o}{=}\PY{l+m+mi}{2}\PY{p}{)}\PY{p}{:}
        \PY{n}{l}\PY{o}{.}\PY{n}{append}\PY{p}{(}\PY{n}{Equals}\PY{p}{(}\PY{n}{state}\PY{p}{[}\PY{l+s+sa}{f}\PY{l+s+s1}{\PYZsq{}}\PY{l+s+s1}{(1,}\PY{l+s+si}{\PYZob{}}\PY{n}{j}\PY{l+s+si}{\PYZcb{}}\PY{l+s+s1}{)}\PY{l+s+s1}{\PYZsq{}}\PY{p}{]}\PY{p}{,} \PY{n}{Int}\PY{p}{(}\PY{n}{x}\PY{p}{)}\PY{p}{)}\PY{p}{)}
        
    \PY{n}{ic}\PY{p}{,}\PY{n}{jc} \PY{o}{=} \PY{n}{centro}
    
    \PY{k}{for} \PY{n}{i} \PY{o+ow}{in} \PY{n+nb}{range}\PY{p}{(}\PY{l+m+mi}{2}\PY{p}{,}\PY{n}{N}\PY{o}{+}\PY{l+m+mi}{1}\PY{p}{)}\PY{p}{:}
        \PY{k}{for} \PY{n}{j} \PY{o+ow}{in} \PY{n+nb}{range}\PY{p}{(}\PY{l+m+mi}{2}\PY{p}{,}\PY{n}{N}\PY{o}{+}\PY{l+m+mi}{1}\PY{p}{)}\PY{p}{:}
            \PY{k}{if} \PY{n}{i} \PY{o+ow}{in} \PY{p}{[}\PY{n}{ic}\PY{o}{\PYZhy{}}\PY{l+m+mi}{1}\PY{p}{,}\PY{n}{ic}\PY{p}{,} \PY{n}{ic}\PY{o}{+}\PY{l+m+mi}{1}\PY{p}{]} \PY{o+ow}{and} \PY{n}{j} \PY{o+ow}{in} \PY{p}{[}\PY{n}{jc}\PY{o}{\PYZhy{}}\PY{l+m+mi}{1}\PY{p}{,} \PY{n}{jc}\PY{p}{,} \PY{n}{jc}\PY{o}{+}\PY{l+m+mi}{1}\PY{p}{]}\PY{p}{:}
                \PY{n}{l}\PY{o}{.}\PY{n}{append}\PY{p}{(}\PY{n}{Equals}\PY{p}{(}\PY{n}{state}\PY{p}{[}\PY{l+s+sa}{f}\PY{l+s+s1}{\PYZsq{}}\PY{l+s+s1}{(}\PY{l+s+si}{\PYZob{}}\PY{n}{i}\PY{l+s+si}{\PYZcb{}}\PY{l+s+s1}{,}\PY{l+s+si}{\PYZob{}}\PY{n}{j}\PY{l+s+si}{\PYZcb{}}\PY{l+s+s1}{)}\PY{l+s+s1}{\PYZsq{}}\PY{p}{]}\PY{p}{,} \PY{n}{Int}\PY{p}{(}\PY{l+m+mi}{1}\PY{p}{)}\PY{p}{)}\PY{p}{)}
            \PY{k}{else}\PY{p}{:}
                \PY{n}{l}\PY{o}{.}\PY{n}{append}\PY{p}{(}\PY{n}{Equals}\PY{p}{(}\PY{n}{state}\PY{p}{[}\PY{l+s+sa}{f}\PY{l+s+s1}{\PYZsq{}}\PY{l+s+s1}{(}\PY{l+s+si}{\PYZob{}}\PY{n}{i}\PY{l+s+si}{\PYZcb{}}\PY{l+s+s1}{,}\PY{l+s+si}{\PYZob{}}\PY{n}{j}\PY{l+s+si}{\PYZcb{}}\PY{l+s+s1}{)}\PY{l+s+s1}{\PYZsq{}}\PY{p}{]}\PY{p}{,} \PY{n}{Int}\PY{p}{(}\PY{l+m+mi}{0}\PY{p}{)}\PY{p}{)}\PY{p}{)}
            
    \PY{c+c1}{\PYZsh{}l.append(Equals(state[\PYZsq{}pc\PYZsq{}], Int(0)))}
    \PY{k}{return} \PY{n}{And}\PY{p}{(}\PY{n}{l}\PY{p}{)}
\end{Verbatim}
\end{tcolorbox}

    Função de transição. Garantimos que a borda se mantem constante e
realiza uma de quatro ações para as células normais: 1. Se a célula
estiver morta e possuir exatamente 3 vizinhos vivos, a mesma passa a
viva no estado seguinte. 2. Se a célula estiver viva e possuir
exatamente 2 ou 3 vizinhos vivos, mantém-se viva no estado seguinte. 3.
Se a célula estiver viva e não possuir exatamente 2 ou 3 vizinhos vivos,
então morre no próximo estado. 4. Se a célula estiver morta e não
possuir exatamente 3 vizinhos vivos, então mantém-se morta no próximo
estado.

\[\mathsf{trans}(x_{i,j},x'_{i,j} : i,j \in N, )\;\equiv\;\] \[
 \left\{\begin{array}{lr}
 \forall_{i \in \{1...N\}} \quad (x'_{i,1} = x_{i,1}) & \land \\ 
 \forall_{j \in \{1...N\}} \quad (x'_{1,j} = x_{1,j}) & \end{array}\right.\\
 \land\\
 \left\{\begin{array}{lr}
 \forall_{i \in \{2...N\}} \cdot \forall{j \in \{2...N\}} \quad (x_{i,j} = 0) \land (neighbors(x_{i,j}) = 3) \land (x'_{i,j} = 1) & \lor \\
 \forall_{i \in \{2...N\}} \cdot \forall{j \in \{2...N\}} \quad (x_{i,j} = 1) \land ((neighbors(x_{i,j}) = 2) \lor neighbors(x_{i,j} = 3)) \land (x'_{i,j} = 1) & \lor\\ 
 \forall_{i \in \{2...N\}} \cdot \forall{j \in \{2...N\}} \quad (x_{i,j} = 1) \land ((neighbors(x_{i,j}) \lt 2) \lor neighbors(x_{i,j} \gt 3)) \land (x'_{i,j} = 0) & \lor\\ 
 \forall_{i \in \{2...N\}} \cdot \forall{j \in \{2...N\}} \quad (x_{i,j} = 0) \land ((neighbors(x_{i,j}) \le 2) \lor neighbors(x_{i,j} \gt 3)) \land (x'_{i,j} = 0) & \end{array}\right.
 \]

    \begin{tcolorbox}[breakable, size=fbox, boxrule=1pt, pad at break*=1mm,colback=cellbackground, colframe=cellborder]
\prompt{In}{incolor}{7}{\boxspacing}
\begin{Verbatim}[commandchars=\\\{\}]
\PY{k}{def} \PY{n+nf}{trans}\PY{p}{(}\PY{n}{curr}\PY{p}{,}\PY{n}{prox}\PY{p}{,}\PY{n}{N}\PY{p}{)}\PY{p}{:}
    \PY{n}{l} \PY{o}{=} \PY{p}{[}\PY{p}{]}
    \PY{k}{for} \PY{n}{i} \PY{o+ow}{in} \PY{n+nb}{range}\PY{p}{(}\PY{l+m+mi}{1}\PY{p}{,}\PY{n}{N}\PY{o}{+}\PY{l+m+mi}{1}\PY{p}{)}\PY{p}{:}
        \PY{n}{l}\PY{o}{.}\PY{n}{append}\PY{p}{(}\PY{n}{Equals}\PY{p}{(}\PY{n}{prox}\PY{p}{[}\PY{l+s+sa}{f}\PY{l+s+s1}{\PYZsq{}}\PY{l+s+s1}{(}\PY{l+s+si}{\PYZob{}}\PY{n}{i}\PY{l+s+si}{\PYZcb{}}\PY{l+s+s1}{,1)}\PY{l+s+s1}{\PYZsq{}}\PY{p}{]}\PY{p}{,} \PY{n}{curr}\PY{p}{[}\PY{l+s+sa}{f}\PY{l+s+s1}{\PYZsq{}}\PY{l+s+s1}{(}\PY{l+s+si}{\PYZob{}}\PY{n}{i}\PY{l+s+si}{\PYZcb{}}\PY{l+s+s1}{,1)}\PY{l+s+s1}{\PYZsq{}}\PY{p}{]}\PY{p}{)}\PY{p}{)}
    \PY{k}{for} \PY{n}{j} \PY{o+ow}{in} \PY{n+nb}{range}\PY{p}{(}\PY{l+m+mi}{2}\PY{p}{,}\PY{n}{N}\PY{o}{+}\PY{l+m+mi}{1}\PY{p}{)}\PY{p}{:}
        \PY{n}{l}\PY{o}{.}\PY{n}{append}\PY{p}{(}\PY{n}{Equals}\PY{p}{(}\PY{n}{prox}\PY{p}{[}\PY{l+s+sa}{f}\PY{l+s+s1}{\PYZsq{}}\PY{l+s+s1}{(1,}\PY{l+s+si}{\PYZob{}}\PY{n}{j}\PY{l+s+si}{\PYZcb{}}\PY{l+s+s1}{)}\PY{l+s+s1}{\PYZsq{}}\PY{p}{]}\PY{p}{,} \PY{n}{curr}\PY{p}{[}\PY{l+s+sa}{f}\PY{l+s+s1}{\PYZsq{}}\PY{l+s+s1}{(1,}\PY{l+s+si}{\PYZob{}}\PY{n}{j}\PY{l+s+si}{\PYZcb{}}\PY{l+s+s1}{)}\PY{l+s+s1}{\PYZsq{}}\PY{p}{]}\PY{p}{)}\PY{p}{)}
    \PY{k}{for} \PY{n}{i} \PY{o+ow}{in} \PY{n+nb}{range}\PY{p}{(}\PY{l+m+mi}{2}\PY{p}{,}\PY{n}{N}\PY{o}{+}\PY{l+m+mi}{1}\PY{p}{)}\PY{p}{:}
        \PY{k}{for} \PY{n}{j} \PY{o+ow}{in} \PY{n+nb}{range}\PY{p}{(}\PY{l+m+mi}{2}\PY{p}{,}\PY{n}{N}\PY{o}{+}\PY{l+m+mi}{1}\PY{p}{)}\PY{p}{:}
            \PY{n}{valor} \PY{o}{=} \PY{n}{Plus}\PY{p}{(}\PY{p}{[}\PY{n}{curr}\PY{p}{[}\PY{l+s+sa}{f}\PY{l+s+s1}{\PYZsq{}}\PY{l+s+s1}{(}\PY{l+s+si}{\PYZob{}}\PY{n}{x}\PY{l+s+si}{\PYZcb{}}\PY{l+s+s1}{,}\PY{l+s+si}{\PYZob{}}\PY{n}{y}\PY{l+s+si}{\PYZcb{}}\PY{l+s+s1}{)}\PY{l+s+s1}{\PYZsq{}}\PY{p}{]} \PY{k}{for} \PY{n}{x}\PY{p}{,}\PY{n}{y} \PY{o+ow}{in} \PY{n}{PontosProximos}\PY{p}{(}\PY{p}{(}\PY{n}{i}\PY{p}{,}\PY{n}{j}\PY{p}{)}\PY{p}{,}\PY{n}{N}\PY{p}{)}\PY{p}{]}\PY{p}{)}
            \PY{n}{tBorn} \PY{o}{=} \PY{n}{And}\PY{p}{(}\PY{n}{Equals}\PY{p}{(}\PY{n}{curr}\PY{p}{[}\PY{l+s+sa}{f}\PY{l+s+s1}{\PYZsq{}}\PY{l+s+s1}{(}\PY{l+s+si}{\PYZob{}}\PY{n}{i}\PY{l+s+si}{\PYZcb{}}\PY{l+s+s1}{,}\PY{l+s+si}{\PYZob{}}\PY{n}{j}\PY{l+s+si}{\PYZcb{}}\PY{l+s+s1}{)}\PY{l+s+s1}{\PYZsq{}}\PY{p}{]}\PY{p}{,} \PY{n}{Int}\PY{p}{(}\PY{l+m+mi}{0}\PY{p}{)}\PY{p}{)}\PY{p}{,} \PY{n}{Equals}\PY{p}{(}\PY{n}{valor}\PY{p}{,} \PY{n}{Int}\PY{p}{(}\PY{l+m+mi}{3}\PY{p}{)}\PY{p}{)}\PY{p}{,} \PY{n}{Equals}\PY{p}{(}\PY{n}{prox}\PY{p}{[}\PY{l+s+sa}{f}\PY{l+s+s1}{\PYZsq{}}\PY{l+s+s1}{(}\PY{l+s+si}{\PYZob{}}\PY{n}{i}\PY{l+s+si}{\PYZcb{}}\PY{l+s+s1}{,}\PY{l+s+si}{\PYZob{}}\PY{n}{j}\PY{l+s+si}{\PYZcb{}}\PY{l+s+s1}{)}\PY{l+s+s1}{\PYZsq{}}\PY{p}{]}\PY{p}{,} \PY{n}{Int}\PY{p}{(}\PY{l+m+mi}{1}\PY{p}{)}\PY{p}{)}\PY{p}{)}
            \PY{n}{tStayAlive} \PY{o}{=} \PY{n}{And}\PY{p}{(}\PY{n}{Equals}\PY{p}{(}\PY{n}{curr}\PY{p}{[}\PY{l+s+sa}{f}\PY{l+s+s1}{\PYZsq{}}\PY{l+s+s1}{(}\PY{l+s+si}{\PYZob{}}\PY{n}{i}\PY{l+s+si}{\PYZcb{}}\PY{l+s+s1}{,}\PY{l+s+si}{\PYZob{}}\PY{n}{j}\PY{l+s+si}{\PYZcb{}}\PY{l+s+s1}{)}\PY{l+s+s1}{\PYZsq{}}\PY{p}{]}\PY{p}{,} \PY{n}{Int}\PY{p}{(}\PY{l+m+mi}{1}\PY{p}{)}\PY{p}{)}\PY{p}{,} \PY{n}{Or}\PY{p}{(}\PY{n}{Equals}\PY{p}{(}\PY{n}{valor}\PY{p}{,} \PY{n}{Int}\PY{p}{(}\PY{l+m+mi}{2}\PY{p}{)}\PY{p}{)}\PY{p}{,}\PY{n}{Equals}\PY{p}{(}\PY{n}{valor}\PY{p}{,} \PY{n}{Int}\PY{p}{(}\PY{l+m+mi}{3}\PY{p}{)}\PY{p}{)}\PY{p}{)}\PY{p}{,} \PY{n}{Equals}\PY{p}{(}\PY{n}{prox}\PY{p}{[}\PY{l+s+sa}{f}\PY{l+s+s1}{\PYZsq{}}\PY{l+s+s1}{(}\PY{l+s+si}{\PYZob{}}\PY{n}{i}\PY{l+s+si}{\PYZcb{}}\PY{l+s+s1}{,}\PY{l+s+si}{\PYZob{}}\PY{n}{j}\PY{l+s+si}{\PYZcb{}}\PY{l+s+s1}{)}\PY{l+s+s1}{\PYZsq{}}\PY{p}{]}\PY{p}{,} \PY{n}{Int}\PY{p}{(}\PY{l+m+mi}{1}\PY{p}{)}\PY{p}{)}\PY{p}{)}
            \PY{n}{tDie} \PY{o}{=} \PY{n}{And}\PY{p}{(}\PY{n}{Equals}\PY{p}{(}\PY{n}{curr}\PY{p}{[}\PY{l+s+sa}{f}\PY{l+s+s1}{\PYZsq{}}\PY{l+s+s1}{(}\PY{l+s+si}{\PYZob{}}\PY{n}{i}\PY{l+s+si}{\PYZcb{}}\PY{l+s+s1}{,}\PY{l+s+si}{\PYZob{}}\PY{n}{j}\PY{l+s+si}{\PYZcb{}}\PY{l+s+s1}{)}\PY{l+s+s1}{\PYZsq{}}\PY{p}{]}\PY{p}{,} \PY{n}{Int}\PY{p}{(}\PY{l+m+mi}{1}\PY{p}{)}\PY{p}{)}\PY{p}{,} \PY{n}{Or}\PY{p}{(}\PY{n}{valor} \PY{o}{\PYZlt{}} \PY{n}{Int}\PY{p}{(}\PY{l+m+mi}{2}\PY{p}{)}\PY{p}{,}\PY{n}{valor} \PY{o}{\PYZgt{}} \PY{n}{Int}\PY{p}{(}\PY{l+m+mi}{3}\PY{p}{)}\PY{p}{)}\PY{p}{,} \PY{n}{Equals}\PY{p}{(}\PY{n}{prox}\PY{p}{[}\PY{l+s+sa}{f}\PY{l+s+s1}{\PYZsq{}}\PY{l+s+s1}{(}\PY{l+s+si}{\PYZob{}}\PY{n}{i}\PY{l+s+si}{\PYZcb{}}\PY{l+s+s1}{,}\PY{l+s+si}{\PYZob{}}\PY{n}{j}\PY{l+s+si}{\PYZcb{}}\PY{l+s+s1}{)}\PY{l+s+s1}{\PYZsq{}}\PY{p}{]}\PY{p}{,} \PY{n}{Int}\PY{p}{(}\PY{l+m+mi}{0}\PY{p}{)}\PY{p}{)}\PY{p}{)}
            \PY{n}{tStillDead} \PY{o}{=} \PY{n}{And}\PY{p}{(}\PY{n}{Equals}\PY{p}{(}\PY{n}{curr}\PY{p}{[}\PY{l+s+sa}{f}\PY{l+s+s1}{\PYZsq{}}\PY{l+s+s1}{(}\PY{l+s+si}{\PYZob{}}\PY{n}{i}\PY{l+s+si}{\PYZcb{}}\PY{l+s+s1}{,}\PY{l+s+si}{\PYZob{}}\PY{n}{j}\PY{l+s+si}{\PYZcb{}}\PY{l+s+s1}{)}\PY{l+s+s1}{\PYZsq{}}\PY{p}{]}\PY{p}{,} \PY{n}{Int}\PY{p}{(}\PY{l+m+mi}{0}\PY{p}{)}\PY{p}{)}\PY{p}{,} \PY{n}{Or}\PY{p}{(}\PY{n}{valor} \PY{o}{\PYZlt{}}\PY{o}{=} \PY{n}{Int}\PY{p}{(}\PY{l+m+mi}{2}\PY{p}{)}\PY{p}{,}\PY{n}{valor} \PY{o}{\PYZgt{}} \PY{n}{Int}\PY{p}{(}\PY{l+m+mi}{3}\PY{p}{)}\PY{p}{)}\PY{p}{,} \PY{n}{Equals}\PY{p}{(}\PY{n}{prox}\PY{p}{[}\PY{l+s+sa}{f}\PY{l+s+s1}{\PYZsq{}}\PY{l+s+s1}{(}\PY{l+s+si}{\PYZob{}}\PY{n}{i}\PY{l+s+si}{\PYZcb{}}\PY{l+s+s1}{,}\PY{l+s+si}{\PYZob{}}\PY{n}{j}\PY{l+s+si}{\PYZcb{}}\PY{l+s+s1}{)}\PY{l+s+s1}{\PYZsq{}}\PY{p}{]}\PY{p}{,}\PY{n}{Int}\PY{p}{(}\PY{l+m+mi}{0}\PY{p}{)}\PY{p}{)}\PY{p}{)}
            
            \PY{n}{l}\PY{o}{.}\PY{n}{append}\PY{p}{(}\PY{n}{Or}\PY{p}{(}\PY{n}{tBorn}\PY{p}{,} \PY{n}{tStayAlive}\PY{p}{,} \PY{n}{tDie}\PY{p}{,}\PY{n}{tStillDead}\PY{p}{)}\PY{p}{)}
    \PY{k}{return} \PY{n}{And}\PY{p}{(}\PY{n}{l}\PY{p}{)}
\end{Verbatim}
\end{tcolorbox}

    Função que usa \(\textit{SMT solver}\) para gerar um possível traço de
execução do programa, e mostra o estado do autómato na forma de um
quadrado \(N \times N\) tal que as células mortas são apresentadas com
cor escura e as vivas com cor clara.

    \begin{tcolorbox}[breakable, size=fbox, boxrule=1pt, pad at break*=1mm,colback=cellbackground, colframe=cellborder]
\prompt{In}{incolor}{8}{\boxspacing}
\begin{Verbatim}[commandchars=\\\{\}]
    
\PY{k}{def} \PY{n+nf}{resolve}\PY{p}{(}\PY{n}{N}\PY{p}{,}\PY{n}{k}\PY{p}{,}\PY{n}{centro}\PY{p}{,}\PY{n}{border}\PY{p}{)}\PY{p}{:}    
    \PY{k}{with} \PY{n}{Solver}\PY{p}{(}\PY{n}{name}\PY{o}{=}\PY{l+s+s2}{\PYZdq{}}\PY{l+s+s2}{z3}\PY{l+s+s2}{\PYZdq{}}\PY{p}{)} \PY{k}{as} \PY{n}{s}\PY{p}{:}
        
        \PY{n}{flag} \PY{o}{=} \PY{k+kc}{True}
        
        \PY{k}{while} \PY{n}{flag}\PY{p}{:}
            
            \PY{n}{bound} \PY{o}{=} \PY{l+m+mi}{0}
            
            \PY{n}{mat\PYZus{}list} \PY{o}{=} \PY{n+nb}{list}\PY{p}{(}\PY{p}{)}
            
            \PY{n}{trace} \PY{o}{=} \PY{p}{[}\PY{n}{declare}\PY{p}{(}\PY{n}{i}\PY{p}{,}\PY{n}{N}\PY{p}{)} \PY{k}{for} \PY{n}{i} \PY{o+ow}{in} \PY{n+nb}{range}\PY{p}{(}\PY{n}{k}\PY{p}{)}\PY{p}{]}
            
            \PY{n}{s}\PY{o}{.}\PY{n}{add\PYZus{}assertion}\PY{p}{(}\PY{n}{init}\PY{p}{(}\PY{n}{trace}\PY{p}{[}\PY{l+m+mi}{0}\PY{p}{]}\PY{p}{,}\PY{n}{N}\PY{p}{,}\PY{n}{centro}\PY{p}{,}\PY{n}{border}\PY{p}{)}\PY{p}{)}

            \PY{k}{for} \PY{n}{i} \PY{o+ow}{in} \PY{n+nb}{range}\PY{p}{(}\PY{n}{k}\PY{o}{\PYZhy{}}\PY{l+m+mi}{1}\PY{p}{)}\PY{p}{:}
                \PY{n}{s}\PY{o}{.}\PY{n}{add\PYZus{}assertion}\PY{p}{(}\PY{n}{trans}\PY{p}{(}\PY{n}{trace}\PY{p}{[}\PY{n}{i}\PY{p}{]}\PY{p}{,} \PY{n}{trace}\PY{p}{[}\PY{n}{i}\PY{o}{+}\PY{l+m+mi}{1}\PY{p}{]}\PY{p}{,}\PY{n}{N}\PY{p}{)}\PY{p}{)}
            
            \PY{n}{mat\PYZus{}set} \PY{o}{=} \PY{n+nb}{set}\PY{p}{(}\PY{p}{)}
            
            \PY{k}{if} \PY{n}{s}\PY{o}{.}\PY{n}{solve}\PY{p}{(}\PY{p}{)}\PY{p}{:}
                \PY{k}{for} \PY{n}{i} \PY{o+ow}{in} \PY{n+nb}{range}\PY{p}{(}\PY{n}{k}\PY{p}{)}\PY{p}{:}
                    \PY{n}{matrix} \PY{o}{=} \PY{p}{[}\PY{p}{]}
                    \PY{k}{for} \PY{n}{x} \PY{o+ow}{in} \PY{n+nb}{range}\PY{p}{(}\PY{l+m+mi}{1}\PY{p}{,}\PY{n}{N}\PY{o}{+}\PY{l+m+mi}{1}\PY{p}{)}\PY{p}{:}
                        \PY{c+c1}{\PYZsh{}print([s.get\PYZus{}value(trace[i][f\PYZsq{}(\PYZob{}k\PYZcb{},\PYZob{}x\PYZcb{})\PYZsq{}]) for k in range(1,N+1)])}
                        \PY{n}{matrix}\PY{o}{.}\PY{n}{append}\PY{p}{(}\PY{p}{[}\PY{p}{(}\PY{n}{s}\PY{o}{.}\PY{n}{get\PYZus{}value}\PY{p}{(}\PY{n}{trace}\PY{p}{[}\PY{n}{i}\PY{p}{]}\PY{p}{[}\PY{l+s+sa}{f}\PY{l+s+s1}{\PYZsq{}}\PY{l+s+s1}{(}\PY{l+s+si}{\PYZob{}}\PY{n}{k}\PY{l+s+si}{\PYZcb{}}\PY{l+s+s1}{,}\PY{l+s+si}{\PYZob{}}\PY{n}{x}\PY{l+s+si}{\PYZcb{}}\PY{l+s+s1}{)}\PY{l+s+s1}{\PYZsq{}}\PY{p}{]}\PY{p}{)}\PY{o}{.}\PY{n}{constant\PYZus{}value}\PY{p}{(}\PY{p}{)}\PY{p}{)} \PY{k}{for} \PY{n}{k} \PY{o+ow}{in} \PY{n+nb}{range}\PY{p}{(}\PY{l+m+mi}{1}\PY{p}{,}\PY{n}{N}\PY{o}{+}\PY{l+m+mi}{1}\PY{p}{)}\PY{p}{]}\PY{p}{)}
                    \PY{c+c1}{\PYZsh{}pprint(matrix)}
                    \PY{n}{matrix\PYZus{}str} \PY{o}{=} \PY{n+nb}{str}\PY{p}{(}\PY{n}{matrix}\PY{p}{)}
                    \PY{k}{if} \PY{n}{matrix\PYZus{}str} \PY{o+ow}{in} \PY{n}{mat\PYZus{}set}\PY{p}{:}
                       \PY{n}{flag} \PY{o}{=} \PY{k+kc}{False}
                       \PY{k}{break}
                    \PY{n}{mat\PYZus{}set}\PY{o}{.}\PY{n}{add}\PY{p}{(}\PY{n}{matrix\PYZus{}str}\PY{p}{)}
                    \PY{n}{mat\PYZus{}list}\PY{o}{.}\PY{n}{append}\PY{p}{(}\PY{n}{matrix}\PY{p}{)}
                    \PY{c+c1}{\PYZsh{}pprint(matrix)}
                    \PY{n}{bound} \PY{o}{+}\PY{o}{=} \PY{l+m+mi}{1}
            \PY{k}{else}\PY{p}{:}
                \PY{n+nb}{print}\PY{p}{(}\PY{l+s+s1}{\PYZsq{}}\PY{l+s+s1}{Não foi possivel resolver}\PY{l+s+s1}{\PYZsq{}}\PY{p}{)}
            \PY{c+c1}{\PYZsh{}print(trace)   }
            
            \PY{n}{k} \PY{o}{=} \PY{l+m+mi}{2}\PY{o}{*}\PY{n}{k}
            \PY{c+c1}{\PYZsh{}print(\PYZsq{}cicle\PYZsq{})}
            
            \PY{n}{s}\PY{o}{.}\PY{n}{reset\PYZus{}assertions}\PY{p}{(}\PY{p}{)}
              
        \PY{k}{for} \PY{n}{i}\PY{p}{,}\PY{n}{l} \PY{o+ow}{in} \PY{n+nb}{enumerate}\PY{p}{(}\PY{n}{mat\PYZus{}list}\PY{p}{)}\PY{p}{:}
            \PY{n}{plt}\PY{o}{.}\PY{n}{imshow}\PY{p}{(}\PY{n}{l}\PY{p}{)}
            \PY{n}{plt}\PY{o}{.}\PY{n}{axis}\PY{p}{(}\PY{l+s+s1}{\PYZsq{}}\PY{l+s+s1}{off}\PY{l+s+s1}{\PYZsq{}}\PY{p}{)}
            \PY{n+nb}{print}\PY{p}{(}\PY{n}{i}\PY{p}{)}
            \PY{n}{plt}\PY{o}{.}\PY{n}{show}\PY{p}{(}\PY{p}{)}
        \PY{k}{return} \PY{n}{bound}
\end{Verbatim}
\end{tcolorbox}

    Função para testar invariante para um traço limitado.

    \begin{tcolorbox}[breakable, size=fbox, boxrule=1pt, pad at break*=1mm,colback=cellbackground, colframe=cellborder]
\prompt{In}{incolor}{9}{\boxspacing}
\begin{Verbatim}[commandchars=\\\{\}]
\PY{k}{def} \PY{n+nf}{bmc\PYZus{}always}\PY{p}{(}\PY{n}{declare}\PY{p}{,}\PY{n}{init}\PY{p}{,}\PY{n}{trans}\PY{p}{,}\PY{n}{inv}\PY{p}{,}\PY{n}{K}\PY{p}{,}\PY{n}{N}\PY{p}{,}\PY{n}{centro}\PY{p}{,}\PY{n}{borda}\PY{p}{)}\PY{p}{:}
    \PY{k}{for} \PY{n}{k} \PY{o+ow}{in} \PY{n+nb}{range}\PY{p}{(}\PY{l+m+mi}{1}\PY{p}{,}\PY{n}{K}\PY{o}{+}\PY{l+m+mi}{1}\PY{p}{)}\PY{p}{:}
        \PY{k}{with} \PY{n}{Solver}\PY{p}{(}\PY{n}{name}\PY{o}{=}\PY{l+s+s2}{\PYZdq{}}\PY{l+s+s2}{z3}\PY{l+s+s2}{\PYZdq{}}\PY{p}{)} \PY{k}{as} \PY{n}{s}\PY{p}{:}
            \PY{n}{trace} \PY{o}{=} \PY{p}{[}\PY{n}{declare}\PY{p}{(}\PY{n}{i}\PY{p}{,}\PY{n}{N}\PY{p}{)} \PY{k}{for} \PY{n}{i} \PY{o+ow}{in} \PY{n+nb}{range}\PY{p}{(}\PY{n}{k}\PY{p}{)}\PY{p}{]}
            \PY{n}{s}\PY{o}{.}\PY{n}{add\PYZus{}assertion}\PY{p}{(}\PY{n}{init}\PY{p}{(}\PY{n}{trace}\PY{p}{[}\PY{l+m+mi}{0}\PY{p}{]}\PY{p}{,}\PY{n}{N}\PY{p}{,}\PY{n}{centro}\PY{p}{,}\PY{n}{borda}\PY{p}{)}\PY{p}{)}
            \PY{k}{for} \PY{n}{i} \PY{o+ow}{in} \PY{n+nb}{range}\PY{p}{(}\PY{n}{k}\PY{o}{\PYZhy{}}\PY{l+m+mi}{1}\PY{p}{)}\PY{p}{:}
                \PY{n}{s}\PY{o}{.}\PY{n}{add\PYZus{}assertion}\PY{p}{(}\PY{n}{trans}\PY{p}{(}\PY{n}{trace}\PY{p}{[}\PY{n}{i}\PY{p}{]}\PY{p}{,} \PY{n}{trace}\PY{p}{[}\PY{n}{i}\PY{o}{+}\PY{l+m+mi}{1}\PY{p}{]}\PY{p}{,}\PY{n}{N}\PY{p}{)}\PY{p}{)}
            \PY{n}{s}\PY{o}{.}\PY{n}{add\PYZus{}assertion}\PY{p}{(}\PY{n}{Not}\PY{p}{(}\PY{n}{inv}\PY{p}{(}\PY{n}{trace}\PY{p}{[}\PY{n}{k}\PY{o}{\PYZhy{}}\PY{l+m+mi}{1}\PY{p}{]}\PY{p}{,}\PY{n}{N}\PY{p}{)}\PY{p}{)}\PY{p}{)}
            \PY{k}{if} \PY{n}{s}\PY{o}{.}\PY{n}{solve}\PY{p}{(}\PY{p}{)}\PY{p}{:}
                \PY{n+nb}{print}\PY{p}{(}\PY{l+s+s2}{\PYZdq{}}\PY{l+s+s2}{A propriedade é inválida}\PY{l+s+s2}{\PYZdq{}}\PY{p}{)}
                \PY{k}{return}
    \PY{n+nb}{print}\PY{p}{(}\PY{l+s+s2}{\PYZdq{}}\PY{l+s+s2}{A propriedade é válida}\PY{l+s+s2}{\PYZdq{}}\PY{p}{)}
   
\end{Verbatim}
\end{tcolorbox}

    Função para testar a inevitabilidade de alguma propriedade.

    \begin{tcolorbox}[breakable, size=fbox, boxrule=1pt, pad at break*=1mm,colback=cellbackground, colframe=cellborder]
\prompt{In}{incolor}{10}{\boxspacing}
\begin{Verbatim}[commandchars=\\\{\}]
\PY{k}{def} \PY{n+nf}{bmc\PYZus{}eventually}\PY{p}{(}\PY{n}{declare}\PY{p}{,}\PY{n}{init}\PY{p}{,}\PY{n}{trans}\PY{p}{,}\PY{n}{prop}\PY{p}{,}\PY{n}{bound}\PY{p}{,}\PY{n}{N}\PY{p}{,}\PY{n}{centro}\PY{p}{,}\PY{n}{borda}\PY{p}{)}\PY{p}{:}
    \PY{k}{for} \PY{n}{k} \PY{o+ow}{in} \PY{n+nb}{range}\PY{p}{(}\PY{l+m+mi}{1}\PY{p}{,}\PY{n}{bound}\PY{o}{+}\PY{l+m+mi}{1}\PY{p}{)}\PY{p}{:}
         \PY{k}{with} \PY{n}{Solver}\PY{p}{(}\PY{n}{name}\PY{o}{=}\PY{l+s+s2}{\PYZdq{}}\PY{l+s+s2}{z3}\PY{l+s+s2}{\PYZdq{}}\PY{p}{)} \PY{k}{as} \PY{n}{s}\PY{p}{:}
            \PY{c+c1}{\PYZsh{} completar}
            \PY{n}{trace} \PY{o}{=} \PY{p}{[}\PY{n}{declare}\PY{p}{(}\PY{n}{i}\PY{p}{,}\PY{n}{N}\PY{p}{)} \PY{k}{for} \PY{n}{i} \PY{o+ow}{in} \PY{n+nb}{range}\PY{p}{(}\PY{n}{k}\PY{p}{)}\PY{p}{]}
    
            \PY{n}{s}\PY{o}{.}\PY{n}{add\PYZus{}assertion}\PY{p}{(}\PY{n}{init}\PY{p}{(}\PY{n}{trace}\PY{p}{[}\PY{l+m+mi}{0}\PY{p}{]}\PY{p}{,}\PY{n}{N}\PY{p}{,}\PY{n}{centro}\PY{p}{,}\PY{n}{borda}\PY{p}{)}\PY{p}{)}
            \PY{k}{for} \PY{n}{i} \PY{o+ow}{in} \PY{n+nb}{range}\PY{p}{(}\PY{n}{k}\PY{o}{\PYZhy{}}\PY{l+m+mi}{1}\PY{p}{)}\PY{p}{:}
                \PY{n}{s}\PY{o}{.}\PY{n}{add\PYZus{}assertion}\PY{p}{(}\PY{n}{trans}\PY{p}{(}\PY{n}{trace}\PY{p}{[}\PY{n}{i}\PY{p}{]}\PY{p}{,} \PY{n}{trace}\PY{p}{[}\PY{n}{i}\PY{o}{+}\PY{l+m+mi}{1}\PY{p}{]}\PY{p}{,}\PY{n}{N}\PY{p}{)}\PY{p}{)}
            
            \PY{k}{for} \PY{n}{i} \PY{o+ow}{in} \PY{n+nb}{range}\PY{p}{(}\PY{n}{k}\PY{p}{)}\PY{p}{:}
                \PY{n}{s}\PY{o}{.}\PY{n}{add\PYZus{}assertion}\PY{p}{(}\PY{n}{Not}\PY{p}{(}\PY{n}{prop}\PY{p}{(}\PY{n}{trace}\PY{p}{,}\PY{n}{k}\PY{p}{,}\PY{n}{N}\PY{p}{)}\PY{p}{)}\PY{p}{)}
            
            \PY{n}{s}\PY{o}{.}\PY{n}{add\PYZus{}assertion}\PY{p}{(}\PY{n}{Or}\PY{p}{(}\PY{n}{trans}\PY{p}{(}\PY{n}{trace}\PY{p}{[}\PY{n}{k}\PY{o}{\PYZhy{}}\PY{l+m+mi}{1}\PY{p}{]}\PY{p}{,}\PY{n}{trace}\PY{p}{[}\PY{n}{i}\PY{p}{]}\PY{p}{,}\PY{n}{N}\PY{p}{)} \PY{k}{for} \PY{n}{i} \PY{o+ow}{in} \PY{n+nb}{range}\PY{p}{(}\PY{n}{k}\PY{p}{)}\PY{p}{)}\PY{p}{)}
            
            \PY{k}{if} \PY{n}{s}\PY{o}{.}\PY{n}{solve}\PY{p}{(}\PY{p}{)}\PY{p}{:}
                \PY{n+nb}{print}\PY{p}{(}\PY{l+s+s2}{\PYZdq{}}\PY{l+s+s2}{Propriedade é inválida}\PY{l+s+s2}{\PYZdq{}}\PY{p}{)}
                \PY{k}{return}
            
    \PY{n+nb}{print}\PY{p}{(}\PY{l+s+s2}{\PYZdq{}}\PY{l+s+s2}{A propriedade é válida}\PY{l+s+s2}{\PYZdq{}}\PY{p}{)}
\end{Verbatim}
\end{tcolorbox}

    Propriedade A: Todos os estados acessíveis contém pelo menos uma célula
viva.

    \begin{tcolorbox}[breakable, size=fbox, boxrule=1pt, pad at break*=1mm,colback=cellbackground, colframe=cellborder]
\prompt{In}{incolor}{11}{\boxspacing}
\begin{Verbatim}[commandchars=\\\{\}]
\PY{k}{def} \PY{n+nf}{propA}\PY{p}{(}\PY{n}{state}\PY{p}{,}\PY{n}{N}\PY{p}{)}\PY{p}{:}
    \PY{k}{return} \PY{n}{Plus}\PY{p}{(}\PY{p}{[}\PY{n}{state}\PY{p}{[}\PY{l+s+sa}{f}\PY{l+s+s1}{\PYZsq{}}\PY{l+s+s1}{(}\PY{l+s+si}{\PYZob{}}\PY{n}{i}\PY{l+s+si}{\PYZcb{}}\PY{l+s+s1}{,}\PY{l+s+si}{\PYZob{}}\PY{n}{j}\PY{l+s+si}{\PYZcb{}}\PY{l+s+s1}{)}\PY{l+s+s1}{\PYZsq{}}\PY{p}{]} \PY{k}{for} \PY{n}{i} \PY{o+ow}{in} \PY{n+nb}{range}\PY{p}{(}\PY{l+m+mi}{1}\PY{p}{,}\PY{n}{N}\PY{o}{+}\PY{l+m+mi}{1}\PY{p}{)} \PY{k}{for} \PY{n}{j} \PY{o+ow}{in} \PY{n+nb}{range}\PY{p}{(}\PY{l+m+mi}{1}\PY{p}{,}\PY{n}{N}\PY{o}{+}\PY{l+m+mi}{1}\PY{p}{)}\PY{p}{]}\PY{p}{)} \PY{o}{\PYZgt{}}\PY{o}{=} \PY{l+m+mi}{1}
\end{Verbatim}
\end{tcolorbox}

    Propriedade A2: Todos os estados acessíveis contém pelo menos uma célula
normal viva.

    \begin{tcolorbox}[breakable, size=fbox, boxrule=1pt, pad at break*=1mm,colback=cellbackground, colframe=cellborder]
\prompt{In}{incolor}{12}{\boxspacing}
\begin{Verbatim}[commandchars=\\\{\}]
\PY{k}{def} \PY{n+nf}{propA2}\PY{p}{(}\PY{n}{state}\PY{p}{,}\PY{n}{N}\PY{p}{)}\PY{p}{:}
    \PY{k}{return} \PY{n}{Plus}\PY{p}{(}\PY{p}{[}\PY{n}{state}\PY{p}{[}\PY{l+s+sa}{f}\PY{l+s+s1}{\PYZsq{}}\PY{l+s+s1}{(}\PY{l+s+si}{\PYZob{}}\PY{n}{i}\PY{l+s+si}{\PYZcb{}}\PY{l+s+s1}{,}\PY{l+s+si}{\PYZob{}}\PY{n}{j}\PY{l+s+si}{\PYZcb{}}\PY{l+s+s1}{)}\PY{l+s+s1}{\PYZsq{}}\PY{p}{]} \PY{k}{for} \PY{n}{i} \PY{o+ow}{in} \PY{n+nb}{range}\PY{p}{(}\PY{l+m+mi}{2}\PY{p}{,}\PY{n}{N}\PY{o}{+}\PY{l+m+mi}{1}\PY{p}{)} \PY{k}{for} \PY{n}{j} \PY{o+ow}{in} \PY{n+nb}{range}\PY{p}{(}\PY{l+m+mi}{2}\PY{p}{,}\PY{n}{N}\PY{o}{+}\PY{l+m+mi}{1}\PY{p}{)}\PY{p}{]}\PY{p}{)} \PY{o}{\PYZgt{}}\PY{o}{=} \PY{l+m+mi}{1}
\end{Verbatim}
\end{tcolorbox}

    Propriedade B: Toda a célula normal está viva pelo menos uma vez em
algum estado acessível.

    \begin{tcolorbox}[breakable, size=fbox, boxrule=1pt, pad at break*=1mm,colback=cellbackground, colframe=cellborder]
\prompt{In}{incolor}{13}{\boxspacing}
\begin{Verbatim}[commandchars=\\\{\}]
\PY{k}{def} \PY{n+nf}{propB}\PY{p}{(}\PY{n}{trace}\PY{p}{,}\PY{n}{k}\PY{p}{,}\PY{n}{N}\PY{p}{)}\PY{p}{:}
    \PY{n}{l} \PY{o}{=} \PY{p}{[}\PY{p}{]}
    \PY{k}{for} \PY{n}{i} \PY{o+ow}{in} \PY{n+nb}{range}\PY{p}{(}\PY{l+m+mi}{2}\PY{p}{,}\PY{n}{N}\PY{o}{+}\PY{l+m+mi}{1}\PY{p}{)}\PY{p}{:}
        \PY{k}{for} \PY{n}{j} \PY{o+ow}{in} \PY{n+nb}{range}\PY{p}{(}\PY{l+m+mi}{2}\PY{p}{,}\PY{n}{N}\PY{o}{+}\PY{l+m+mi}{1}\PY{p}{)}\PY{p}{:}
            \PY{n}{l}\PY{o}{.}\PY{n}{append}\PY{p}{(}\PY{n}{Plus}\PY{p}{(}\PY{p}{[}\PY{n}{trace}\PY{p}{[}\PY{n}{x}\PY{p}{]}\PY{p}{[}\PY{l+s+sa}{f}\PY{l+s+s1}{\PYZsq{}}\PY{l+s+s1}{(}\PY{l+s+si}{\PYZob{}}\PY{n}{i}\PY{l+s+si}{\PYZcb{}}\PY{l+s+s1}{,}\PY{l+s+si}{\PYZob{}}\PY{n}{j}\PY{l+s+si}{\PYZcb{}}\PY{l+s+s1}{)}\PY{l+s+s1}{\PYZsq{}}\PY{p}{]} \PY{k}{for} \PY{n}{x} \PY{o+ow}{in} \PY{n+nb}{range}\PY{p}{(}\PY{n}{k}\PY{p}{)}\PY{p}{]}\PY{p}{)} \PY{o}{\PYZgt{}}\PY{o}{=} \PY{l+m+mi}{1}\PY{p}{)}
    \PY{k}{return} \PY{n}{And}\PY{p}{(}\PY{n}{l}\PY{p}{)}
\end{Verbatim}
\end{tcolorbox}

    \hypertarget{exemplos-e-testes-de-aplicauxe7uxe3o.}{%
\subsubsection{Exemplos e testes de
aplicação.}\label{exemplos-e-testes-de-aplicauxe7uxe3o.}}

    \hypertarget{exemplo-1}{%
\paragraph{Exemplo 1}\label{exemplo-1}}

    Neste exemplo procuramos mostrar que é encontrado um loop mesmo que
\(k\) inicial não seja suficientemente grande.

    \begin{tcolorbox}[breakable, size=fbox, boxrule=1pt, pad at break*=1mm,colback=cellbackground, colframe=cellborder]
\prompt{In}{incolor}{12}{\boxspacing}
\begin{Verbatim}[commandchars=\\\{\}]
\PY{n}{N} \PY{o}{=} \PY{l+m+mi}{4}
\PY{n}{p} \PY{o}{=} \PY{l+m+mf}{0.75}
\PY{n}{k} \PY{o}{=} \PY{l+m+mi}{2}
\PY{n}{centro} \PY{o}{=} \PY{n}{gera\PYZus{}centro}\PY{p}{(}\PY{n}{N}\PY{p}{)}
\PY{n}{border} \PY{o}{=} \PY{n}{gera\PYZus{}border}\PY{p}{(}\PY{n}{p}\PY{p}{,}\PY{n}{N}\PY{p}{)}
\PY{n+nb}{print}\PY{p}{(}\PY{n}{centro}\PY{p}{)}
\PY{n+nb}{print}\PY{p}{(}\PY{n}{border}\PY{p}{)}
\end{Verbatim}
\end{tcolorbox}

    \begin{Verbatim}[commandchars=\\\{\}]
(3, 3)
([1, 0, 1, 1], [1, 0, 0])
    \end{Verbatim}

    \begin{tcolorbox}[breakable, size=fbox, boxrule=1pt, pad at break*=1mm,colback=cellbackground, colframe=cellborder]
\prompt{In}{incolor}{13}{\boxspacing}
\begin{Verbatim}[commandchars=\\\{\}]
\PY{n+nb}{print}\PY{p}{(}\PY{n}{bound} \PY{o}{:=} \PY{n}{resolve}\PY{p}{(}\PY{n}{N}\PY{p}{,}\PY{n}{k}\PY{p}{,}\PY{n}{centro}\PY{p}{,}\PY{n}{border}\PY{p}{)}\PY{p}{)}
\end{Verbatim}
\end{tcolorbox}

    \begin{Verbatim}[commandchars=\\\{\}]
0
    \end{Verbatim}

    \begin{center}
    \adjustimage{max size={0.9\linewidth}{0.9\paperheight}}{output_34_1.png}
    \end{center}
    { \hspace*{\fill} \\}
    
    \begin{Verbatim}[commandchars=\\\{\}]
1
    \end{Verbatim}

    \begin{center}
    \adjustimage{max size={0.9\linewidth}{0.9\paperheight}}{output_34_3.png}
    \end{center}
    { \hspace*{\fill} \\}
    
    \begin{Verbatim}[commandchars=\\\{\}]
2
    \end{Verbatim}

    \begin{center}
    \adjustimage{max size={0.9\linewidth}{0.9\paperheight}}{output_34_5.png}
    \end{center}
    { \hspace*{\fill} \\}
    
    \begin{Verbatim}[commandchars=\\\{\}]
3
    \end{Verbatim}

    \begin{center}
    \adjustimage{max size={0.9\linewidth}{0.9\paperheight}}{output_34_7.png}
    \end{center}
    { \hspace*{\fill} \\}
    
    \begin{Verbatim}[commandchars=\\\{\}]
4
    \end{Verbatim}

    \begin{center}
    \adjustimage{max size={0.9\linewidth}{0.9\paperheight}}{output_34_9.png}
    \end{center}
    { \hspace*{\fill} \\}
    
    \begin{Verbatim}[commandchars=\\\{\}]
5
    \end{Verbatim}

    \begin{center}
    \adjustimage{max size={0.9\linewidth}{0.9\paperheight}}{output_34_11.png}
    \end{center}
    { \hspace*{\fill} \\}
    
    \begin{Verbatim}[commandchars=\\\{\}]
6
    \end{Verbatim}

    \begin{tcolorbox}[breakable, size=fbox, boxrule=1pt, pad at break*=1mm,colback=cellbackground, colframe=cellborder]
\prompt{In}{incolor}{14}{\boxspacing}
\begin{Verbatim}[commandchars=\\\{\}]
\PY{n+nb}{print}\PY{p}{(}\PY{l+s+s2}{\PYZdq{}}\PY{l+s+s2}{Propriedade A:}\PY{l+s+s2}{\PYZdq{}}\PY{p}{)}
\PY{n}{bmc\PYZus{}always}\PY{p}{(}\PY{n}{declare}\PY{p}{,}\PY{n}{init}\PY{p}{,}\PY{n}{trans}\PY{p}{,}\PY{n}{propA}\PY{p}{,}\PY{n}{bound}\PY{p}{,}\PY{n}{N}\PY{p}{,}\PY{n}{centro}\PY{p}{,}\PY{n}{border}\PY{p}{)}
\PY{n+nb}{print}\PY{p}{(}\PY{l+s+s2}{\PYZdq{}}\PY{l+s+s2}{Propriedade A, mas para células normais apenas:}\PY{l+s+s2}{\PYZdq{}}\PY{p}{)}
\PY{n}{bmc\PYZus{}always}\PY{p}{(}\PY{n}{declare}\PY{p}{,}\PY{n}{init}\PY{p}{,}\PY{n}{trans}\PY{p}{,}\PY{n}{propA2}\PY{p}{,}\PY{n}{bound}\PY{p}{,}\PY{n}{N}\PY{p}{,}\PY{n}{centro}\PY{p}{,}\PY{n}{border}\PY{p}{)}
\PY{n+nb}{print}\PY{p}{(}\PY{l+s+s2}{\PYZdq{}}\PY{l+s+s2}{Propriedade B:}\PY{l+s+s2}{\PYZdq{}}\PY{p}{)}
\PY{n}{bmc\PYZus{}eventually}\PY{p}{(}\PY{n}{declare}\PY{p}{,}\PY{n}{init}\PY{p}{,}\PY{n}{trans}\PY{p}{,}\PY{n}{propB}\PY{p}{,}\PY{n}{bound}\PY{p}{,}\PY{n}{N}\PY{p}{,}\PY{n}{centro}\PY{p}{,}\PY{n}{border}\PY{p}{)}
\end{Verbatim}
\end{tcolorbox}

    \begin{Verbatim}[commandchars=\\\{\}]
Propriedade A:
A propriedade é válida
Propriedade A, mas para células normais apenas:
A propriedade é válida
Propriedade B:
A propriedade é válida
    \end{Verbatim}

    \hypertarget{exemplo-2}{%
\paragraph{Exemplo 2}\label{exemplo-2}}

    Este é um exemplo em a proposição A é verdadeira mas a B é falsa.

    \begin{tcolorbox}[breakable, size=fbox, boxrule=1pt, pad at break*=1mm,colback=cellbackground, colframe=cellborder]
\prompt{In}{incolor}{15}{\boxspacing}
\begin{Verbatim}[commandchars=\\\{\}]
\PY{n}{N} \PY{o}{=} \PY{l+m+mi}{5}
\PY{n}{p} \PY{o}{=} \PY{l+m+mi}{1}
\PY{n}{k} \PY{o}{=} \PY{l+m+mi}{2}
\PY{n}{centro} \PY{o}{=} \PY{n}{gera\PYZus{}centro}\PY{p}{(}\PY{n}{N}\PY{p}{)}
\PY{n}{border} \PY{o}{=} \PY{n}{gera\PYZus{}border}\PY{p}{(}\PY{n}{p}\PY{p}{,}\PY{n}{N}\PY{p}{)}
\PY{n+nb}{print}\PY{p}{(}\PY{n}{centro}\PY{p}{)}
\PY{n+nb}{print}\PY{p}{(}\PY{n}{border}\PY{p}{)}
\end{Verbatim}
\end{tcolorbox}

    \begin{Verbatim}[commandchars=\\\{\}]
(3, 4)
([1, 1, 1, 1, 1], [1, 1, 1, 1])
    \end{Verbatim}

    \begin{tcolorbox}[breakable, size=fbox, boxrule=1pt, pad at break*=1mm,colback=cellbackground, colframe=cellborder]
\prompt{In}{incolor}{16}{\boxspacing}
\begin{Verbatim}[commandchars=\\\{\}]
\PY{n+nb}{print}\PY{p}{(}\PY{n}{bound} \PY{o}{:=} \PY{n}{resolve}\PY{p}{(}\PY{n}{N}\PY{p}{,}\PY{n}{k}\PY{p}{,}\PY{n}{centro}\PY{p}{,}\PY{n}{border}\PY{p}{)}\PY{p}{)}
\end{Verbatim}
\end{tcolorbox}

    \begin{Verbatim}[commandchars=\\\{\}]
0
    \end{Verbatim}

    \begin{center}
    \adjustimage{max size={0.9\linewidth}{0.9\paperheight}}{output_39_1.png}
    \end{center}
    { \hspace*{\fill} \\}
    
    \begin{Verbatim}[commandchars=\\\{\}]
1
    \end{Verbatim}

    \begin{center}
    \adjustimage{max size={0.9\linewidth}{0.9\paperheight}}{output_39_3.png}
    \end{center}
    { \hspace*{\fill} \\}
    
    \begin{Verbatim}[commandchars=\\\{\}]
2
    \end{Verbatim}

    \begin{center}
    \adjustimage{max size={0.9\linewidth}{0.9\paperheight}}{output_39_5.png}
    \end{center}
    { \hspace*{\fill} \\}
    
    \begin{Verbatim}[commandchars=\\\{\}]
3
    \end{Verbatim}

    \begin{center}
    \adjustimage{max size={0.9\linewidth}{0.9\paperheight}}{output_39_7.png}
    \end{center}
    { \hspace*{\fill} \\}
    
    \begin{Verbatim}[commandchars=\\\{\}]
4
    \end{Verbatim}

    \begin{center}
    \adjustimage{max size={0.9\linewidth}{0.9\paperheight}}{output_39_9.png}
    \end{center}
    { \hspace*{\fill} \\}
    
    \begin{Verbatim}[commandchars=\\\{\}]
5
    \end{Verbatim}

    \begin{center}
    \adjustimage{max size={0.9\linewidth}{0.9\paperheight}}{output_39_11.png}
    \end{center}
    { \hspace*{\fill} \\}
    
    \begin{Verbatim}[commandchars=\\\{\}]
6
    \end{Verbatim}

    \begin{center}
    \adjustimage{max size={0.9\linewidth}{0.9\paperheight}}{output_39_13.png}
    \end{center}
    { \hspace*{\fill} \\}
    
    \begin{Verbatim}[commandchars=\\\{\}]
7
    \end{Verbatim}

    \begin{center}
    \adjustimage{max size={0.9\linewidth}{0.9\paperheight}}{output_39_15.png}
    \end{center}
    { \hspace*{\fill} \\}
    
    \begin{Verbatim}[commandchars=\\\{\}]
8
    \end{Verbatim}

    \begin{tcolorbox}[breakable, size=fbox, boxrule=1pt, pad at break*=1mm,colback=cellbackground, colframe=cellborder]
\prompt{In}{incolor}{17}{\boxspacing}
\begin{Verbatim}[commandchars=\\\{\}]
\PY{n+nb}{print}\PY{p}{(}\PY{l+s+s2}{\PYZdq{}}\PY{l+s+s2}{Propriedade A:}\PY{l+s+s2}{\PYZdq{}}\PY{p}{)}
\PY{n}{bmc\PYZus{}always}\PY{p}{(}\PY{n}{declare}\PY{p}{,}\PY{n}{init}\PY{p}{,}\PY{n}{trans}\PY{p}{,}\PY{n}{propA}\PY{p}{,}\PY{n}{bound}\PY{p}{,}\PY{n}{N}\PY{p}{,}\PY{n}{centro}\PY{p}{,}\PY{n}{border}\PY{p}{)}
\PY{n+nb}{print}\PY{p}{(}\PY{l+s+s2}{\PYZdq{}}\PY{l+s+s2}{Propriedade A, mas para células normais apenas:}\PY{l+s+s2}{\PYZdq{}}\PY{p}{)}
\PY{n}{bmc\PYZus{}always}\PY{p}{(}\PY{n}{declare}\PY{p}{,}\PY{n}{init}\PY{p}{,}\PY{n}{trans}\PY{p}{,}\PY{n}{propA2}\PY{p}{,}\PY{n}{bound}\PY{p}{,}\PY{n}{N}\PY{p}{,}\PY{n}{centro}\PY{p}{,}\PY{n}{border}\PY{p}{)}
\PY{n+nb}{print}\PY{p}{(}\PY{l+s+s2}{\PYZdq{}}\PY{l+s+s2}{Propriedade B:}\PY{l+s+s2}{\PYZdq{}}\PY{p}{)}
\PY{n}{bmc\PYZus{}eventually}\PY{p}{(}\PY{n}{declare}\PY{p}{,}\PY{n}{init}\PY{p}{,}\PY{n}{trans}\PY{p}{,}\PY{n}{propB}\PY{p}{,}\PY{n}{bound}\PY{p}{,}\PY{n}{N}\PY{p}{,}\PY{n}{centro}\PY{p}{,}\PY{n}{border}\PY{p}{)}
\end{Verbatim}
\end{tcolorbox}

    \begin{Verbatim}[commandchars=\\\{\}]
Propriedade A:
A propriedade é válida
Propriedade A, mas para células normais apenas:
A propriedade é válida
Propriedade B:
Propriedade é inválida
    \end{Verbatim}

    \hypertarget{exemplo-3}{%
\paragraph{Exemplo 3}\label{exemplo-3}}

    Este é um exemplo com um quadro maior.

    \begin{tcolorbox}[breakable, size=fbox, boxrule=1pt, pad at break*=1mm,colback=cellbackground, colframe=cellborder]
\prompt{In}{incolor}{21}{\boxspacing}
\begin{Verbatim}[commandchars=\\\{\}]
\PY{n}{N} \PY{o}{=} \PY{l+m+mi}{15}
\PY{n}{p} \PY{o}{=} \PY{l+m+mf}{0.5}
\PY{n}{k} \PY{o}{=} \PY{l+m+mi}{50}
\PY{n}{centro} \PY{o}{=} \PY{n}{gera\PYZus{}centro}\PY{p}{(}\PY{n}{N}\PY{p}{)}
\PY{n}{border} \PY{o}{=} \PY{n}{gera\PYZus{}border}\PY{p}{(}\PY{n}{p}\PY{p}{,}\PY{n}{N}\PY{p}{)}
\PY{n+nb}{print}\PY{p}{(}\PY{n}{centro}\PY{p}{)}
\PY{n+nb}{print}\PY{p}{(}\PY{n}{border}\PY{p}{)}
\end{Verbatim}
\end{tcolorbox}

    \begin{Verbatim}[commandchars=\\\{\}]
(13, 12)
([0, 1, 0, 0, 1, 1, 0, 0, 0, 0, 0, 1, 0, 0, 0], [0, 1, 0, 0, 1, 1, 1, 0, 0, 1,
1, 0, 1, 0])
    \end{Verbatim}

    \begin{tcolorbox}[breakable, size=fbox, boxrule=1pt, pad at break*=1mm,colback=cellbackground, colframe=cellborder]
\prompt{In}{incolor}{22}{\boxspacing}
\begin{Verbatim}[commandchars=\\\{\}]
\PY{n+nb}{print}\PY{p}{(}\PY{n}{bound} \PY{o}{:=} \PY{n}{resolve}\PY{p}{(}\PY{n}{N}\PY{p}{,}\PY{n}{k}\PY{p}{,}\PY{n}{centro}\PY{p}{,}\PY{n}{border}\PY{p}{)}\PY{p}{)}
\end{Verbatim}
\end{tcolorbox}

    \begin{Verbatim}[commandchars=\\\{\}]
0
    \end{Verbatim}

    \begin{center}
    \adjustimage{max size={0.9\linewidth}{0.9\paperheight}}{output_44_1.png}
    \end{center}
    { \hspace*{\fill} \\}
    
    \begin{Verbatim}[commandchars=\\\{\}]
1
    \end{Verbatim}

    \begin{center}
    \adjustimage{max size={0.9\linewidth}{0.9\paperheight}}{output_44_3.png}
    \end{center}
    { \hspace*{\fill} \\}
    
    \begin{Verbatim}[commandchars=\\\{\}]
2
    \end{Verbatim}

    \begin{center}
    \adjustimage{max size={0.9\linewidth}{0.9\paperheight}}{output_44_5.png}
    \end{center}
    { \hspace*{\fill} \\}
    
    \begin{Verbatim}[commandchars=\\\{\}]
3
    \end{Verbatim}

    \begin{center}
    \adjustimage{max size={0.9\linewidth}{0.9\paperheight}}{output_44_7.png}
    \end{center}
    { \hspace*{\fill} \\}
    
    \begin{Verbatim}[commandchars=\\\{\}]
4
    \end{Verbatim}

    \begin{center}
    \adjustimage{max size={0.9\linewidth}{0.9\paperheight}}{output_44_9.png}
    \end{center}
    { \hspace*{\fill} \\}
    
    \begin{Verbatim}[commandchars=\\\{\}]
5
    \end{Verbatim}

    \begin{center}
    \adjustimage{max size={0.9\linewidth}{0.9\paperheight}}{output_44_11.png}
    \end{center}
    { \hspace*{\fill} \\}
    
    \begin{Verbatim}[commandchars=\\\{\}]
6
    \end{Verbatim}

    \begin{center}
    \adjustimage{max size={0.9\linewidth}{0.9\paperheight}}{output_44_13.png}
    \end{center}
    { \hspace*{\fill} \\}
    
    \begin{Verbatim}[commandchars=\\\{\}]
7
    \end{Verbatim}

    \begin{center}
    \adjustimage{max size={0.9\linewidth}{0.9\paperheight}}{output_44_15.png}
    \end{center}
    { \hspace*{\fill} \\}
    
    \begin{Verbatim}[commandchars=\\\{\}]
8
    \end{Verbatim}

    \begin{center}
    \adjustimage{max size={0.9\linewidth}{0.9\paperheight}}{output_44_17.png}
    \end{center}
    { \hspace*{\fill} \\}
    
    \begin{Verbatim}[commandchars=\\\{\}]
9
    \end{Verbatim}

    \begin{center}
    \adjustimage{max size={0.9\linewidth}{0.9\paperheight}}{output_44_19.png}
    \end{center}
    { \hspace*{\fill} \\}
    
    \begin{Verbatim}[commandchars=\\\{\}]
10
    \end{Verbatim}

    \begin{center}
    \adjustimage{max size={0.9\linewidth}{0.9\paperheight}}{output_44_21.png}
    \end{center}
    { \hspace*{\fill} \\}
    
    \begin{Verbatim}[commandchars=\\\{\}]
11
    \end{Verbatim}

    \begin{center}
    \adjustimage{max size={0.9\linewidth}{0.9\paperheight}}{output_44_23.png}
    \end{center}
    { \hspace*{\fill} \\}
    
    \begin{Verbatim}[commandchars=\\\{\}]
12
    \end{Verbatim}

    \begin{center}
    \adjustimage{max size={0.9\linewidth}{0.9\paperheight}}{output_44_25.png}
    \end{center}
    { \hspace*{\fill} \\}
    
    \begin{Verbatim}[commandchars=\\\{\}]
13
    \end{Verbatim}

    \begin{center}
    \adjustimage{max size={0.9\linewidth}{0.9\paperheight}}{output_44_27.png}
    \end{center}
    { \hspace*{\fill} \\}
    
    \begin{Verbatim}[commandchars=\\\{\}]
14
    \end{Verbatim}

    \begin{center}
    \adjustimage{max size={0.9\linewidth}{0.9\paperheight}}{output_44_29.png}
    \end{center}
    { \hspace*{\fill} \\}
    
    \begin{Verbatim}[commandchars=\\\{\}]
15
    \end{Verbatim}

    \begin{center}
    \adjustimage{max size={0.9\linewidth}{0.9\paperheight}}{output_44_31.png}
    \end{center}
    { \hspace*{\fill} \\}
    
    \begin{Verbatim}[commandchars=\\\{\}]
16
    \end{Verbatim}

    \begin{center}
    \adjustimage{max size={0.9\linewidth}{0.9\paperheight}}{output_44_33.png}
    \end{center}
    { \hspace*{\fill} \\}
    
    \begin{Verbatim}[commandchars=\\\{\}]
17
    \end{Verbatim}

    \begin{center}
    \adjustimage{max size={0.9\linewidth}{0.9\paperheight}}{output_44_35.png}
    \end{center}
    { \hspace*{\fill} \\}
    
    \begin{Verbatim}[commandchars=\\\{\}]
18
    \end{Verbatim}

    \begin{center}
    \adjustimage{max size={0.9\linewidth}{0.9\paperheight}}{output_44_37.png}
    \end{center}
    { \hspace*{\fill} \\}
    
    \begin{Verbatim}[commandchars=\\\{\}]
19
    \end{Verbatim}

    \begin{center}
    \adjustimage{max size={0.9\linewidth}{0.9\paperheight}}{output_44_39.png}
    \end{center}
    { \hspace*{\fill} \\}
    
    \begin{Verbatim}[commandchars=\\\{\}]
20
    \end{Verbatim}

    \begin{center}
    \adjustimage{max size={0.9\linewidth}{0.9\paperheight}}{output_44_41.png}
    \end{center}
    { \hspace*{\fill} \\}
    
    \begin{Verbatim}[commandchars=\\\{\}]
21
    \end{Verbatim}

    \begin{center}
    \adjustimage{max size={0.9\linewidth}{0.9\paperheight}}{output_44_43.png}
    \end{center}
    { \hspace*{\fill} \\}
    
    \begin{Verbatim}[commandchars=\\\{\}]
22
    \end{Verbatim}

    \begin{center}
    \adjustimage{max size={0.9\linewidth}{0.9\paperheight}}{output_44_45.png}
    \end{center}
    { \hspace*{\fill} \\}
    
    \begin{Verbatim}[commandchars=\\\{\}]
23
    \end{Verbatim}

    \begin{center}
    \adjustimage{max size={0.9\linewidth}{0.9\paperheight}}{output_44_47.png}
    \end{center}
    { \hspace*{\fill} \\}
    
    \begin{Verbatim}[commandchars=\\\{\}]
24
    \end{Verbatim}

    \begin{center}
    \adjustimage{max size={0.9\linewidth}{0.9\paperheight}}{output_44_49.png}
    \end{center}
    { \hspace*{\fill} \\}
    
    \begin{Verbatim}[commandchars=\\\{\}]
25
    \end{Verbatim}

    \begin{center}
    \adjustimage{max size={0.9\linewidth}{0.9\paperheight}}{output_44_51.png}
    \end{center}
    { \hspace*{\fill} \\}
    
    \begin{Verbatim}[commandchars=\\\{\}]
26
    \end{Verbatim}

    \begin{tcolorbox}[breakable, size=fbox, boxrule=1pt, pad at break*=1mm,colback=cellbackground, colframe=cellborder]
\prompt{In}{incolor}{23}{\boxspacing}
\begin{Verbatim}[commandchars=\\\{\}]
\PY{n+nb}{print}\PY{p}{(}\PY{l+s+s2}{\PYZdq{}}\PY{l+s+s2}{Propriedade A:}\PY{l+s+s2}{\PYZdq{}}\PY{p}{)}
\PY{n}{bmc\PYZus{}always}\PY{p}{(}\PY{n}{declare}\PY{p}{,}\PY{n}{init}\PY{p}{,}\PY{n}{trans}\PY{p}{,}\PY{n}{propA}\PY{p}{,}\PY{n}{bound}\PY{p}{,}\PY{n}{N}\PY{p}{,}\PY{n}{centro}\PY{p}{,}\PY{n}{border}\PY{p}{)}
\PY{n+nb}{print}\PY{p}{(}\PY{l+s+s2}{\PYZdq{}}\PY{l+s+s2}{Propriedade A, mas para células normais apenas:}\PY{l+s+s2}{\PYZdq{}}\PY{p}{)}
\PY{n}{bmc\PYZus{}always}\PY{p}{(}\PY{n}{declare}\PY{p}{,}\PY{n}{init}\PY{p}{,}\PY{n}{trans}\PY{p}{,}\PY{n}{propA2}\PY{p}{,}\PY{n}{bound}\PY{p}{,}\PY{n}{N}\PY{p}{,}\PY{n}{centro}\PY{p}{,}\PY{n}{border}\PY{p}{)}
\PY{n+nb}{print}\PY{p}{(}\PY{l+s+s2}{\PYZdq{}}\PY{l+s+s2}{Propriedade B:}\PY{l+s+s2}{\PYZdq{}}\PY{p}{)}
\PY{n}{bmc\PYZus{}eventually}\PY{p}{(}\PY{n}{declare}\PY{p}{,}\PY{n}{init}\PY{p}{,}\PY{n}{trans}\PY{p}{,}\PY{n}{propB}\PY{p}{,}\PY{n}{bound}\PY{p}{,}\PY{n}{N}\PY{p}{,}\PY{n}{centro}\PY{p}{,}\PY{n}{border}\PY{p}{)}
\end{Verbatim}
\end{tcolorbox}

    \begin{Verbatim}[commandchars=\\\{\}]
Propriedade A:
A propriedade é válida
Propriedade A, mas para células normais apenas:
A propriedade é válida
Propriedade B:
Propriedade é inválida
    \end{Verbatim}

    \hypertarget{exemplo-4}{%
\paragraph{Exemplo 4}\label{exemplo-4}}

    Este é um exemplo em que a propriedade A e B são falsas.

    \begin{tcolorbox}[breakable, size=fbox, boxrule=1pt, pad at break*=1mm,colback=cellbackground, colframe=cellborder]
\prompt{In}{incolor}{24}{\boxspacing}
\begin{Verbatim}[commandchars=\\\{\}]
\PY{n}{N} \PY{o}{=} \PY{l+m+mi}{7}
\PY{n}{p} \PY{o}{=} \PY{l+m+mi}{0}
\PY{n}{k} \PY{o}{=} \PY{l+m+mi}{20}
\PY{n}{centro} \PY{o}{=} \PY{n}{gera\PYZus{}centro}\PY{p}{(}\PY{n}{N}\PY{p}{)}
\PY{n}{border} \PY{o}{=} \PY{n}{gera\PYZus{}border}\PY{p}{(}\PY{n}{p}\PY{p}{,}\PY{n}{N}\PY{p}{)}
\PY{n+nb}{print}\PY{p}{(}\PY{n}{centro}\PY{p}{)}
\PY{n+nb}{print}\PY{p}{(}\PY{n}{border}\PY{p}{)}
\end{Verbatim}
\end{tcolorbox}

    \begin{Verbatim}[commandchars=\\\{\}]
(4, 4)
([0, 0, 0, 0, 0, 0, 0], [0, 0, 0, 0, 0, 0])
    \end{Verbatim}

    \begin{tcolorbox}[breakable, size=fbox, boxrule=1pt, pad at break*=1mm,colback=cellbackground, colframe=cellborder]
\prompt{In}{incolor}{25}{\boxspacing}
\begin{Verbatim}[commandchars=\\\{\}]
\PY{n+nb}{print}\PY{p}{(}\PY{n}{bound} \PY{o}{:=} \PY{n}{resolve}\PY{p}{(}\PY{n}{N}\PY{p}{,}\PY{n}{k}\PY{p}{,}\PY{n}{centro}\PY{p}{,}\PY{n}{border}\PY{p}{)}\PY{p}{)}
\end{Verbatim}
\end{tcolorbox}

    \begin{Verbatim}[commandchars=\\\{\}]
0
    \end{Verbatim}

    \begin{center}
    \adjustimage{max size={0.9\linewidth}{0.9\paperheight}}{output_49_1.png}
    \end{center}
    { \hspace*{\fill} \\}
    
    \begin{Verbatim}[commandchars=\\\{\}]
1
    \end{Verbatim}

    \begin{center}
    \adjustimage{max size={0.9\linewidth}{0.9\paperheight}}{output_49_3.png}
    \end{center}
    { \hspace*{\fill} \\}
    
    \begin{Verbatim}[commandchars=\\\{\}]
2
    \end{Verbatim}

    \begin{center}
    \adjustimage{max size={0.9\linewidth}{0.9\paperheight}}{output_49_5.png}
    \end{center}
    { \hspace*{\fill} \\}
    
    \begin{Verbatim}[commandchars=\\\{\}]
3
    \end{Verbatim}

    \begin{center}
    \adjustimage{max size={0.9\linewidth}{0.9\paperheight}}{output_49_7.png}
    \end{center}
    { \hspace*{\fill} \\}
    
    \begin{Verbatim}[commandchars=\\\{\}]
4
    \end{Verbatim}

    \begin{center}
    \adjustimage{max size={0.9\linewidth}{0.9\paperheight}}{output_49_9.png}
    \end{center}
    { \hspace*{\fill} \\}
    
    \begin{Verbatim}[commandchars=\\\{\}]
5
    \end{Verbatim}

    \begin{center}
    \adjustimage{max size={0.9\linewidth}{0.9\paperheight}}{output_49_11.png}
    \end{center}
    { \hspace*{\fill} \\}
    
    \begin{Verbatim}[commandchars=\\\{\}]
6
    \end{Verbatim}

    \begin{center}
    \adjustimage{max size={0.9\linewidth}{0.9\paperheight}}{output_49_13.png}
    \end{center}
    { \hspace*{\fill} \\}
    
    \begin{Verbatim}[commandchars=\\\{\}]
7
    \end{Verbatim}

    \begin{center}
    \adjustimage{max size={0.9\linewidth}{0.9\paperheight}}{output_49_15.png}
    \end{center}
    { \hspace*{\fill} \\}
    
    \begin{Verbatim}[commandchars=\\\{\}]
8
    \end{Verbatim}

    \begin{center}
    \adjustimage{max size={0.9\linewidth}{0.9\paperheight}}{output_49_17.png}
    \end{center}
    { \hspace*{\fill} \\}
    
    \begin{Verbatim}[commandchars=\\\{\}]
9
    \end{Verbatim}

    \begin{center}
    \adjustimage{max size={0.9\linewidth}{0.9\paperheight}}{output_49_19.png}
    \end{center}
    { \hspace*{\fill} \\}
    
    \begin{Verbatim}[commandchars=\\\{\}]
10
    \end{Verbatim}

    \begin{center}
    \adjustimage{max size={0.9\linewidth}{0.9\paperheight}}{output_49_21.png}
    \end{center}
    { \hspace*{\fill} \\}
    
    \begin{Verbatim}[commandchars=\\\{\}]
11
    \end{Verbatim}

    \begin{center}
    \adjustimage{max size={0.9\linewidth}{0.9\paperheight}}{output_49_23.png}
    \end{center}
    { \hspace*{\fill} \\}
    
    \begin{Verbatim}[commandchars=\\\{\}]
12
    \end{Verbatim}

    \begin{center}
    \adjustimage{max size={0.9\linewidth}{0.9\paperheight}}{output_49_25.png}
    \end{center}
    { \hspace*{\fill} \\}
    
    \begin{Verbatim}[commandchars=\\\{\}]
13
    \end{Verbatim}

    \begin{center}
    \adjustimage{max size={0.9\linewidth}{0.9\paperheight}}{output_49_27.png}
    \end{center}
    { \hspace*{\fill} \\}
    
    \begin{Verbatim}[commandchars=\\\{\}]
14
    \end{Verbatim}

    \begin{center}
    \adjustimage{max size={0.9\linewidth}{0.9\paperheight}}{output_49_29.png}
    \end{center}
    { \hspace*{\fill} \\}
    
    \begin{Verbatim}[commandchars=\\\{\}]
15
    \end{Verbatim}

    \begin{tcolorbox}[breakable, size=fbox, boxrule=1pt, pad at break*=1mm,colback=cellbackground, colframe=cellborder]
\prompt{In}{incolor}{26}{\boxspacing}
\begin{Verbatim}[commandchars=\\\{\}]
\PY{n+nb}{print}\PY{p}{(}\PY{l+s+s2}{\PYZdq{}}\PY{l+s+s2}{Propriedade A:}\PY{l+s+s2}{\PYZdq{}}\PY{p}{)}
\PY{n}{bmc\PYZus{}always}\PY{p}{(}\PY{n}{declare}\PY{p}{,}\PY{n}{init}\PY{p}{,}\PY{n}{trans}\PY{p}{,}\PY{n}{propA}\PY{p}{,}\PY{n}{bound}\PY{p}{,}\PY{n}{N}\PY{p}{,}\PY{n}{centro}\PY{p}{,}\PY{n}{border}\PY{p}{)}
\PY{n+nb}{print}\PY{p}{(}\PY{l+s+s2}{\PYZdq{}}\PY{l+s+s2}{Propriedade A, mas para células normais apenas:}\PY{l+s+s2}{\PYZdq{}}\PY{p}{)}
\PY{n}{bmc\PYZus{}always}\PY{p}{(}\PY{n}{declare}\PY{p}{,}\PY{n}{init}\PY{p}{,}\PY{n}{trans}\PY{p}{,}\PY{n}{propA2}\PY{p}{,}\PY{n}{bound}\PY{p}{,}\PY{n}{N}\PY{p}{,}\PY{n}{centro}\PY{p}{,}\PY{n}{border}\PY{p}{)}
\PY{n+nb}{print}\PY{p}{(}\PY{l+s+s2}{\PYZdq{}}\PY{l+s+s2}{Propriedade B:}\PY{l+s+s2}{\PYZdq{}}\PY{p}{)}
\PY{n}{bmc\PYZus{}eventually}\PY{p}{(}\PY{n}{declare}\PY{p}{,}\PY{n}{init}\PY{p}{,}\PY{n}{trans}\PY{p}{,}\PY{n}{propB}\PY{p}{,}\PY{n}{bound}\PY{p}{,}\PY{n}{N}\PY{p}{,}\PY{n}{centro}\PY{p}{,}\PY{n}{border}\PY{p}{)}
\end{Verbatim}
\end{tcolorbox}

    \begin{Verbatim}[commandchars=\\\{\}]
Propriedade A:
A propriedade é inválida
Propriedade A, mas para células normais apenas:
A propriedade é inválida
Propriedade B:
Propriedade é inválida
    \end{Verbatim}

    \hypertarget{exemplo-5}{%
\subparagraph{Exemplo 5}\label{exemplo-5}}

    Este é outro exemplo de uma execução com um quadro relativamente grande.

    \begin{tcolorbox}[breakable, size=fbox, boxrule=1pt, pad at break*=1mm,colback=cellbackground, colframe=cellborder]
\prompt{In}{incolor}{14}{\boxspacing}
\begin{Verbatim}[commandchars=\\\{\}]
\PY{n}{N} \PY{o}{=} \PY{l+m+mi}{20}
\PY{n}{p} \PY{o}{=} \PY{l+m+mf}{0.40}
\PY{n}{k} \PY{o}{=} \PY{l+m+mi}{100}
\PY{n}{centro} \PY{o}{=} \PY{n}{gera\PYZus{}centro}\PY{p}{(}\PY{n}{N}\PY{p}{)}
\PY{n}{border} \PY{o}{=} \PY{n}{gera\PYZus{}border}\PY{p}{(}\PY{n}{p}\PY{p}{,}\PY{n}{N}\PY{p}{)}
\PY{n+nb}{print}\PY{p}{(}\PY{n}{centro}\PY{p}{)}
\PY{n+nb}{print}\PY{p}{(}\PY{n}{border}\PY{p}{)}
\end{Verbatim}
\end{tcolorbox}

    \begin{Verbatim}[commandchars=\\\{\}]
(14, 4)
([1, 0, 0, 1, 1, 1, 0, 0, 0, 0, 0, 1, 1, 1, 1, 0, 1, 0, 0, 0], [0, 1, 0, 0, 0,
0, 0, 1, 0, 0, 1, 1, 1, 1, 0, 1, 0, 0, 1])
    \end{Verbatim}

    \begin{tcolorbox}[breakable, size=fbox, boxrule=1pt, pad at break*=1mm,colback=cellbackground, colframe=cellborder]
\prompt{In}{incolor}{15}{\boxspacing}
\begin{Verbatim}[commandchars=\\\{\}]
\PY{n+nb}{print}\PY{p}{(}\PY{n}{bound} \PY{o}{:=} \PY{n}{resolve}\PY{p}{(}\PY{n}{N}\PY{p}{,}\PY{n}{k}\PY{p}{,}\PY{n}{centro}\PY{p}{,}\PY{n}{border}\PY{p}{)}\PY{p}{)}
\end{Verbatim}
\end{tcolorbox}

    \begin{Verbatim}[commandchars=\\\{\}]
0
    \end{Verbatim}

    \begin{center}
    \adjustimage{max size={0.9\linewidth}{0.9\paperheight}}{output_54_1.png}
    \end{center}
    { \hspace*{\fill} \\}
    
    \begin{Verbatim}[commandchars=\\\{\}]
1
    \end{Verbatim}

    \begin{center}
    \adjustimage{max size={0.9\linewidth}{0.9\paperheight}}{output_54_3.png}
    \end{center}
    { \hspace*{\fill} \\}
    
    \begin{Verbatim}[commandchars=\\\{\}]
2
    \end{Verbatim}

    \begin{center}
    \adjustimage{max size={0.9\linewidth}{0.9\paperheight}}{output_54_5.png}
    \end{center}
    { \hspace*{\fill} \\}
    
    \begin{Verbatim}[commandchars=\\\{\}]
3
    \end{Verbatim}

    \begin{center}
    \adjustimage{max size={0.9\linewidth}{0.9\paperheight}}{output_54_7.png}
    \end{center}
    { \hspace*{\fill} \\}
    
    \begin{Verbatim}[commandchars=\\\{\}]
4
    \end{Verbatim}

    \begin{center}
    \adjustimage{max size={0.9\linewidth}{0.9\paperheight}}{output_54_9.png}
    \end{center}
    { \hspace*{\fill} \\}
    
    \begin{Verbatim}[commandchars=\\\{\}]
5
    \end{Verbatim}

    \begin{center}
    \adjustimage{max size={0.9\linewidth}{0.9\paperheight}}{output_54_11.png}
    \end{center}
    { \hspace*{\fill} \\}
    
    \begin{Verbatim}[commandchars=\\\{\}]
6
    \end{Verbatim}

    \begin{center}
    \adjustimage{max size={0.9\linewidth}{0.9\paperheight}}{output_54_13.png}
    \end{center}
    { \hspace*{\fill} \\}
    
    \begin{Verbatim}[commandchars=\\\{\}]
7
    \end{Verbatim}

    \begin{center}
    \adjustimage{max size={0.9\linewidth}{0.9\paperheight}}{output_54_15.png}
    \end{center}
    { \hspace*{\fill} \\}
    
    \begin{Verbatim}[commandchars=\\\{\}]
8
    \end{Verbatim}

    \begin{center}
    \adjustimage{max size={0.9\linewidth}{0.9\paperheight}}{output_54_17.png}
    \end{center}
    { \hspace*{\fill} \\}
    
    \begin{Verbatim}[commandchars=\\\{\}]
9
    \end{Verbatim}

    \begin{center}
    \adjustimage{max size={0.9\linewidth}{0.9\paperheight}}{output_54_19.png}
    \end{center}
    { \hspace*{\fill} \\}
    
    \begin{Verbatim}[commandchars=\\\{\}]
10
    \end{Verbatim}

    \begin{center}
    \adjustimage{max size={0.9\linewidth}{0.9\paperheight}}{output_54_21.png}
    \end{center}
    { \hspace*{\fill} \\}
    
    \begin{Verbatim}[commandchars=\\\{\}]
11
    \end{Verbatim}

    \begin{center}
    \adjustimage{max size={0.9\linewidth}{0.9\paperheight}}{output_54_23.png}
    \end{center}
    { \hspace*{\fill} \\}
    
    \begin{Verbatim}[commandchars=\\\{\}]
12
    \end{Verbatim}

    \begin{center}
    \adjustimage{max size={0.9\linewidth}{0.9\paperheight}}{output_54_25.png}
    \end{center}
    { \hspace*{\fill} \\}
    
    \begin{Verbatim}[commandchars=\\\{\}]
13
    \end{Verbatim}

    \begin{center}
    \adjustimage{max size={0.9\linewidth}{0.9\paperheight}}{output_54_27.png}
    \end{center}
    { \hspace*{\fill} \\}
    
    \begin{Verbatim}[commandchars=\\\{\}]
14
    \end{Verbatim}

    \begin{center}
    \adjustimage{max size={0.9\linewidth}{0.9\paperheight}}{output_54_29.png}
    \end{center}
    { \hspace*{\fill} \\}
    
    \begin{Verbatim}[commandchars=\\\{\}]
15
    \end{Verbatim}

    \begin{center}
    \adjustimage{max size={0.9\linewidth}{0.9\paperheight}}{output_54_31.png}
    \end{center}
    { \hspace*{\fill} \\}
    
    \begin{Verbatim}[commandchars=\\\{\}]
16
    \end{Verbatim}

    \begin{center}
    \adjustimage{max size={0.9\linewidth}{0.9\paperheight}}{output_54_33.png}
    \end{center}
    { \hspace*{\fill} \\}
    
    \begin{Verbatim}[commandchars=\\\{\}]
17
    \end{Verbatim}

    \begin{center}
    \adjustimage{max size={0.9\linewidth}{0.9\paperheight}}{output_54_35.png}
    \end{center}
    { \hspace*{\fill} \\}
    
    \begin{Verbatim}[commandchars=\\\{\}]
18
    \end{Verbatim}

    \begin{center}
    \adjustimage{max size={0.9\linewidth}{0.9\paperheight}}{output_54_37.png}
    \end{center}
    { \hspace*{\fill} \\}
    
    \begin{Verbatim}[commandchars=\\\{\}]
19
    \end{Verbatim}

    \begin{center}
    \adjustimage{max size={0.9\linewidth}{0.9\paperheight}}{output_54_39.png}
    \end{center}
    { \hspace*{\fill} \\}
    
    \begin{Verbatim}[commandchars=\\\{\}]
20
    \end{Verbatim}

    \begin{center}
    \adjustimage{max size={0.9\linewidth}{0.9\paperheight}}{output_54_41.png}
    \end{center}
    { \hspace*{\fill} \\}
    
    \begin{Verbatim}[commandchars=\\\{\}]
21
    \end{Verbatim}

    \begin{center}
    \adjustimage{max size={0.9\linewidth}{0.9\paperheight}}{output_54_43.png}
    \end{center}
    { \hspace*{\fill} \\}
    
    \begin{Verbatim}[commandchars=\\\{\}]
22
    \end{Verbatim}

    \begin{center}
    \adjustimage{max size={0.9\linewidth}{0.9\paperheight}}{output_54_45.png}
    \end{center}
    { \hspace*{\fill} \\}
    
    \begin{Verbatim}[commandchars=\\\{\}]
23
    \end{Verbatim}

    \begin{center}
    \adjustimage{max size={0.9\linewidth}{0.9\paperheight}}{output_54_47.png}
    \end{center}
    { \hspace*{\fill} \\}
    
    \begin{Verbatim}[commandchars=\\\{\}]
24
    \end{Verbatim}

    \begin{center}
    \adjustimage{max size={0.9\linewidth}{0.9\paperheight}}{output_54_49.png}
    \end{center}
    { \hspace*{\fill} \\}
    
    \begin{Verbatim}[commandchars=\\\{\}]
25
    \end{Verbatim}

    \begin{center}
    \adjustimage{max size={0.9\linewidth}{0.9\paperheight}}{output_54_51.png}
    \end{center}
    { \hspace*{\fill} \\}
    
    \begin{Verbatim}[commandchars=\\\{\}]
26
    \end{Verbatim}

    \begin{center}
    \adjustimage{max size={0.9\linewidth}{0.9\paperheight}}{output_54_53.png}
    \end{center}
    { \hspace*{\fill} \\}
    
    \begin{Verbatim}[commandchars=\\\{\}]
27
    \end{Verbatim}

    \begin{center}
    \adjustimage{max size={0.9\linewidth}{0.9\paperheight}}{output_54_55.png}
    \end{center}
    { \hspace*{\fill} \\}
    
    \begin{Verbatim}[commandchars=\\\{\}]
28
    \end{Verbatim}

    \begin{center}
    \adjustimage{max size={0.9\linewidth}{0.9\paperheight}}{output_54_57.png}
    \end{center}
    { \hspace*{\fill} \\}
    
    \begin{Verbatim}[commandchars=\\\{\}]
29
    \end{Verbatim}

    \begin{center}
    \adjustimage{max size={0.9\linewidth}{0.9\paperheight}}{output_54_59.png}
    \end{center}
    { \hspace*{\fill} \\}
    
    \begin{Verbatim}[commandchars=\\\{\}]
30
    \end{Verbatim}

    \begin{center}
    \adjustimage{max size={0.9\linewidth}{0.9\paperheight}}{output_54_61.png}
    \end{center}
    { \hspace*{\fill} \\}
    
    \begin{Verbatim}[commandchars=\\\{\}]
31
    \end{Verbatim}

    \begin{center}
    \adjustimage{max size={0.9\linewidth}{0.9\paperheight}}{output_54_63.png}
    \end{center}
    { \hspace*{\fill} \\}
    
    \begin{Verbatim}[commandchars=\\\{\}]
32
    \end{Verbatim}

    \begin{center}
    \adjustimage{max size={0.9\linewidth}{0.9\paperheight}}{output_54_65.png}
    \end{center}
    { \hspace*{\fill} \\}
    
    \begin{Verbatim}[commandchars=\\\{\}]
33
    \end{Verbatim}

    \begin{center}
    \adjustimage{max size={0.9\linewidth}{0.9\paperheight}}{output_54_67.png}
    \end{center}
    { \hspace*{\fill} \\}
    
    \begin{Verbatim}[commandchars=\\\{\}]
34
    \end{Verbatim}

    \begin{center}
    \adjustimage{max size={0.9\linewidth}{0.9\paperheight}}{output_54_69.png}
    \end{center}
    { \hspace*{\fill} \\}
    
    \begin{Verbatim}[commandchars=\\\{\}]
35
    \end{Verbatim}

    \begin{center}
    \adjustimage{max size={0.9\linewidth}{0.9\paperheight}}{output_54_71.png}
    \end{center}
    { \hspace*{\fill} \\}
    
    \begin{Verbatim}[commandchars=\\\{\}]
36
    \end{Verbatim}

    \begin{center}
    \adjustimage{max size={0.9\linewidth}{0.9\paperheight}}{output_54_73.png}
    \end{center}
    { \hspace*{\fill} \\}
    
    \begin{Verbatim}[commandchars=\\\{\}]
37
    \end{Verbatim}

    \begin{center}
    \adjustimage{max size={0.9\linewidth}{0.9\paperheight}}{output_54_75.png}
    \end{center}
    { \hspace*{\fill} \\}
    
    \begin{Verbatim}[commandchars=\\\{\}]
38
    \end{Verbatim}

    \begin{center}
    \adjustimage{max size={0.9\linewidth}{0.9\paperheight}}{output_54_77.png}
    \end{center}
    { \hspace*{\fill} \\}
    
    \begin{Verbatim}[commandchars=\\\{\}]
39
    \end{Verbatim}

    \begin{center}
    \adjustimage{max size={0.9\linewidth}{0.9\paperheight}}{output_54_79.png}
    \end{center}
    { \hspace*{\fill} \\}
    
    \begin{Verbatim}[commandchars=\\\{\}]
40
    \end{Verbatim}

    \begin{center}
    \adjustimage{max size={0.9\linewidth}{0.9\paperheight}}{output_54_81.png}
    \end{center}
    { \hspace*{\fill} \\}
    
    \begin{Verbatim}[commandchars=\\\{\}]
41
    \end{Verbatim}

    \begin{center}
    \adjustimage{max size={0.9\linewidth}{0.9\paperheight}}{output_54_83.png}
    \end{center}
    { \hspace*{\fill} \\}
    
    \begin{Verbatim}[commandchars=\\\{\}]
42
    \end{Verbatim}

    \begin{center}
    \adjustimage{max size={0.9\linewidth}{0.9\paperheight}}{output_54_85.png}
    \end{center}
    { \hspace*{\fill} \\}
    
    \begin{Verbatim}[commandchars=\\\{\}]
43
    \end{Verbatim}

    \begin{center}
    \adjustimage{max size={0.9\linewidth}{0.9\paperheight}}{output_54_87.png}
    \end{center}
    { \hspace*{\fill} \\}
    
    \begin{Verbatim}[commandchars=\\\{\}]
44
    \end{Verbatim}

    \begin{center}
    \adjustimage{max size={0.9\linewidth}{0.9\paperheight}}{output_54_89.png}
    \end{center}
    { \hspace*{\fill} \\}
    
    \begin{Verbatim}[commandchars=\\\{\}]
45
    \end{Verbatim}

    \begin{center}
    \adjustimage{max size={0.9\linewidth}{0.9\paperheight}}{output_54_91.png}
    \end{center}
    { \hspace*{\fill} \\}
    
    \begin{Verbatim}[commandchars=\\\{\}]
46
    \end{Verbatim}

    \begin{center}
    \adjustimage{max size={0.9\linewidth}{0.9\paperheight}}{output_54_93.png}
    \end{center}
    { \hspace*{\fill} \\}
    
    \begin{Verbatim}[commandchars=\\\{\}]
47
    \end{Verbatim}

    \begin{center}
    \adjustimage{max size={0.9\linewidth}{0.9\paperheight}}{output_54_95.png}
    \end{center}
    { \hspace*{\fill} \\}
    
    \begin{Verbatim}[commandchars=\\\{\}]
48
    \end{Verbatim}

    \begin{center}
    \adjustimage{max size={0.9\linewidth}{0.9\paperheight}}{output_54_97.png}
    \end{center}
    { \hspace*{\fill} \\}
    
    \begin{Verbatim}[commandchars=\\\{\}]
49
    \end{Verbatim}

    \begin{center}
    \adjustimage{max size={0.9\linewidth}{0.9\paperheight}}{output_54_99.png}
    \end{center}
    { \hspace*{\fill} \\}
    
    \begin{Verbatim}[commandchars=\\\{\}]
50
    \end{Verbatim}

    \begin{center}
    \adjustimage{max size={0.9\linewidth}{0.9\paperheight}}{output_54_101.png}
    \end{center}
    { \hspace*{\fill} \\}
    
    \begin{Verbatim}[commandchars=\\\{\}]
51
    \end{Verbatim}

    \begin{center}
    \adjustimage{max size={0.9\linewidth}{0.9\paperheight}}{output_54_103.png}
    \end{center}
    { \hspace*{\fill} \\}
    
    \begin{Verbatim}[commandchars=\\\{\}]
52
    \end{Verbatim}

    \begin{center}
    \adjustimage{max size={0.9\linewidth}{0.9\paperheight}}{output_54_105.png}
    \end{center}
    { \hspace*{\fill} \\}
    
    \begin{Verbatim}[commandchars=\\\{\}]
53
    \end{Verbatim}

    \begin{center}
    \adjustimage{max size={0.9\linewidth}{0.9\paperheight}}{output_54_107.png}
    \end{center}
    { \hspace*{\fill} \\}
    
    \begin{Verbatim}[commandchars=\\\{\}]
54
    \end{Verbatim}

    \begin{center}
    \adjustimage{max size={0.9\linewidth}{0.9\paperheight}}{output_54_109.png}
    \end{center}
    { \hspace*{\fill} \\}
    
    \begin{Verbatim}[commandchars=\\\{\}]
55
    \end{Verbatim}

    \begin{center}
    \adjustimage{max size={0.9\linewidth}{0.9\paperheight}}{output_54_111.png}
    \end{center}
    { \hspace*{\fill} \\}
    
    \begin{Verbatim}[commandchars=\\\{\}]
56
    \end{Verbatim}

    \begin{center}
    \adjustimage{max size={0.9\linewidth}{0.9\paperheight}}{output_54_113.png}
    \end{center}
    { \hspace*{\fill} \\}
    
    \begin{Verbatim}[commandchars=\\\{\}]
57
    \end{Verbatim}

    \begin{center}
    \adjustimage{max size={0.9\linewidth}{0.9\paperheight}}{output_54_115.png}
    \end{center}
    { \hspace*{\fill} \\}
    
    \begin{Verbatim}[commandchars=\\\{\}]
58
    \end{Verbatim}

    \begin{center}
    \adjustimage{max size={0.9\linewidth}{0.9\paperheight}}{output_54_117.png}
    \end{center}
    { \hspace*{\fill} \\}
    
    \begin{Verbatim}[commandchars=\\\{\}]
59
    \end{Verbatim}

    \begin{center}
    \adjustimage{max size={0.9\linewidth}{0.9\paperheight}}{output_54_119.png}
    \end{center}
    { \hspace*{\fill} \\}
    
    \begin{Verbatim}[commandchars=\\\{\}]
60
    \end{Verbatim}

    \begin{center}
    \adjustimage{max size={0.9\linewidth}{0.9\paperheight}}{output_54_121.png}
    \end{center}
    { \hspace*{\fill} \\}
    
    \begin{Verbatim}[commandchars=\\\{\}]
61
    \end{Verbatim}

    \begin{center}
    \adjustimage{max size={0.9\linewidth}{0.9\paperheight}}{output_54_123.png}
    \end{center}
    { \hspace*{\fill} \\}
    
    \begin{Verbatim}[commandchars=\\\{\}]
62
    \end{Verbatim}

    \begin{center}
    \adjustimage{max size={0.9\linewidth}{0.9\paperheight}}{output_54_125.png}
    \end{center}
    { \hspace*{\fill} \\}
    
    \begin{Verbatim}[commandchars=\\\{\}]
63
    \end{Verbatim}

    \begin{center}
    \adjustimage{max size={0.9\linewidth}{0.9\paperheight}}{output_54_127.png}
    \end{center}
    { \hspace*{\fill} \\}
    
    \begin{Verbatim}[commandchars=\\\{\}]
64
    \end{Verbatim}

    \begin{center}
    \adjustimage{max size={0.9\linewidth}{0.9\paperheight}}{output_54_129.png}
    \end{center}
    { \hspace*{\fill} \\}
    
    \begin{Verbatim}[commandchars=\\\{\}]
65
    \end{Verbatim}

    \begin{tcolorbox}[breakable, size=fbox, boxrule=1pt, pad at break*=1mm,colback=cellbackground, colframe=cellborder]
\prompt{In}{incolor}{16}{\boxspacing}
\begin{Verbatim}[commandchars=\\\{\}]
\PY{n+nb}{print}\PY{p}{(}\PY{l+s+s2}{\PYZdq{}}\PY{l+s+s2}{Propriedade A:}\PY{l+s+s2}{\PYZdq{}}\PY{p}{)}
\PY{n}{bmc\PYZus{}always}\PY{p}{(}\PY{n}{declare}\PY{p}{,}\PY{n}{init}\PY{p}{,}\PY{n}{trans}\PY{p}{,}\PY{n}{propA}\PY{p}{,}\PY{n}{bound}\PY{p}{,}\PY{n}{N}\PY{p}{,}\PY{n}{centro}\PY{p}{,}\PY{n}{border}\PY{p}{)}
\PY{n+nb}{print}\PY{p}{(}\PY{l+s+s2}{\PYZdq{}}\PY{l+s+s2}{Propriedade A, mas para células normais apenas:}\PY{l+s+s2}{\PYZdq{}}\PY{p}{)}
\PY{n}{bmc\PYZus{}always}\PY{p}{(}\PY{n}{declare}\PY{p}{,}\PY{n}{init}\PY{p}{,}\PY{n}{trans}\PY{p}{,}\PY{n}{propA2}\PY{p}{,}\PY{n}{bound}\PY{p}{,}\PY{n}{N}\PY{p}{,}\PY{n}{centro}\PY{p}{,}\PY{n}{border}\PY{p}{)}
\PY{n+nb}{print}\PY{p}{(}\PY{l+s+s2}{\PYZdq{}}\PY{l+s+s2}{Propriedade B:}\PY{l+s+s2}{\PYZdq{}}\PY{p}{)}
\PY{n}{bmc\PYZus{}eventually}\PY{p}{(}\PY{n}{declare}\PY{p}{,}\PY{n}{init}\PY{p}{,}\PY{n}{trans}\PY{p}{,}\PY{n}{propB}\PY{p}{,}\PY{n}{bound}\PY{p}{,}\PY{n}{N}\PY{p}{,}\PY{n}{centro}\PY{p}{,}\PY{n}{border}\PY{p}{)}
\end{Verbatim}
\end{tcolorbox}

    \begin{Verbatim}[commandchars=\\\{\}]
Propriedade A:
A propriedade é válida
Propriedade A, mas para células normais apenas:
A propriedade é válida
Propriedade B:
Propriedade é inválida
    \end{Verbatim}


    % Add a bibliography block to the postdoc
    
    
    
\end{document}
