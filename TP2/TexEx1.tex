\documentclass[11pt]{article}

    \usepackage[breakable]{tcolorbox}
    \usepackage{parskip} % Stop auto-indenting (to mimic markdown behaviour)
    

    % Basic figure setup, for now with no caption control since it's done
    % automatically by Pandoc (which extracts ![](path) syntax from Markdown).
    \usepackage{graphicx}
    % Maintain compatibility with old templates. Remove in nbconvert 6.0
    \let\Oldincludegraphics\includegraphics
    % Ensure that by default, figures have no caption (until we provide a
    % proper Figure object with a Caption API and a way to capture that
    % in the conversion process - todo).
    \usepackage{caption}
    \DeclareCaptionFormat{nocaption}{}
    \captionsetup{format=nocaption,aboveskip=0pt,belowskip=0pt}

    \usepackage{float}
    \floatplacement{figure}{H} % forces figures to be placed at the correct location
    \usepackage{xcolor} % Allow colors to be defined
    \usepackage{enumerate} % Needed for markdown enumerations to work
    \usepackage{geometry} % Used to adjust the document margins
    \usepackage{amsmath} % Equations
    \usepackage{amssymb} % Equations
    \usepackage{textcomp} % defines textquotesingle
    % Hack from http://tex.stackexchange.com/a/47451/13684:
    \AtBeginDocument{%
        \def\PYZsq{\textquotesingle}% Upright quotes in Pygmentized code
    }
    \usepackage{upquote} % Upright quotes for verbatim code
    \usepackage{eurosym} % defines \euro

    \usepackage{iftex}
    \ifPDFTeX
        \usepackage[T1]{fontenc}
        \IfFileExists{alphabeta.sty}{
              \usepackage{alphabeta}
          }{
              \usepackage[mathletters]{ucs}
              \usepackage[utf8x]{inputenc}
          }
    \else
        \usepackage{fontspec}
        \usepackage{unicode-math}
    \fi

    \usepackage{fancyvrb} % verbatim replacement that allows latex
    \usepackage{grffile} % extends the file name processing of package graphics 
                         % to support a larger range
    \makeatletter % fix for old versions of grffile with XeLaTeX
    \@ifpackagelater{grffile}{2019/11/01}
    {
      % Do nothing on new versions
    }
    {
      \def\Gread@@xetex#1{%
        \IfFileExists{"\Gin@base".bb}%
        {\Gread@eps{\Gin@base.bb}}%
        {\Gread@@xetex@aux#1}%
      }
    }
    \makeatother
    \usepackage[Export]{adjustbox} % Used to constrain images to a maximum size
    \adjustboxset{max size={0.9\linewidth}{0.9\paperheight}}

    % The hyperref package gives us a pdf with properly built
    % internal navigation ('pdf bookmarks' for the table of contents,
    % internal cross-reference links, web links for URLs, etc.)
    \usepackage{hyperref}
    % The default LaTeX title has an obnoxious amount of whitespace. By default,
    % titling removes some of it. It also provides customization options.
    \usepackage{titling}
    \usepackage{longtable} % longtable support required by pandoc >1.10
    \usepackage{booktabs}  % table support for pandoc > 1.12.2
    \usepackage{array}     % table support for pandoc >= 2.11.3
    \usepackage{calc}      % table minipage width calculation for pandoc >= 2.11.1
    \usepackage[inline]{enumitem} % IRkernel/repr support (it uses the enumerate* environment)
    \usepackage[normalem]{ulem} % ulem is needed to support strikethroughs (\sout)
                                % normalem makes italics be italics, not underlines
    \usepackage{mathrsfs}
    

    
    % Colors for the hyperref package
    \definecolor{urlcolor}{rgb}{0,.145,.698}
    \definecolor{linkcolor}{rgb}{.71,0.21,0.01}
    \definecolor{citecolor}{rgb}{.12,.54,.11}

    % ANSI colors
    \definecolor{ansi-black}{HTML}{3E424D}
    \definecolor{ansi-black-intense}{HTML}{282C36}
    \definecolor{ansi-red}{HTML}{E75C58}
    \definecolor{ansi-red-intense}{HTML}{B22B31}
    \definecolor{ansi-green}{HTML}{00A250}
    \definecolor{ansi-green-intense}{HTML}{007427}
    \definecolor{ansi-yellow}{HTML}{DDB62B}
    \definecolor{ansi-yellow-intense}{HTML}{B27D12}
    \definecolor{ansi-blue}{HTML}{208FFB}
    \definecolor{ansi-blue-intense}{HTML}{0065CA}
    \definecolor{ansi-magenta}{HTML}{D160C4}
    \definecolor{ansi-magenta-intense}{HTML}{A03196}
    \definecolor{ansi-cyan}{HTML}{60C6C8}
    \definecolor{ansi-cyan-intense}{HTML}{258F8F}
    \definecolor{ansi-white}{HTML}{C5C1B4}
    \definecolor{ansi-white-intense}{HTML}{A1A6B2}
    \definecolor{ansi-default-inverse-fg}{HTML}{FFFFFF}
    \definecolor{ansi-default-inverse-bg}{HTML}{000000}

    % common color for the border for error outputs.
    \definecolor{outerrorbackground}{HTML}{FFDFDF}

    % commands and environments needed by pandoc snippets
    % extracted from the output of `pandoc -s`
    \providecommand{\tightlist}{%
      \setlength{\itemsep}{0pt}\setlength{\parskip}{0pt}}
    \DefineVerbatimEnvironment{Highlighting}{Verbatim}{commandchars=\\\{\}}
    % Add ',fontsize=\small' for more characters per line
    \newenvironment{Shaded}{}{}
    \newcommand{\KeywordTok}[1]{\textcolor[rgb]{0.00,0.44,0.13}{\textbf{{#1}}}}
    \newcommand{\DataTypeTok}[1]{\textcolor[rgb]{0.56,0.13,0.00}{{#1}}}
    \newcommand{\DecValTok}[1]{\textcolor[rgb]{0.25,0.63,0.44}{{#1}}}
    \newcommand{\BaseNTok}[1]{\textcolor[rgb]{0.25,0.63,0.44}{{#1}}}
    \newcommand{\FloatTok}[1]{\textcolor[rgb]{0.25,0.63,0.44}{{#1}}}
    \newcommand{\CharTok}[1]{\textcolor[rgb]{0.25,0.44,0.63}{{#1}}}
    \newcommand{\StringTok}[1]{\textcolor[rgb]{0.25,0.44,0.63}{{#1}}}
    \newcommand{\CommentTok}[1]{\textcolor[rgb]{0.38,0.63,0.69}{\textit{{#1}}}}
    \newcommand{\OtherTok}[1]{\textcolor[rgb]{0.00,0.44,0.13}{{#1}}}
    \newcommand{\AlertTok}[1]{\textcolor[rgb]{1.00,0.00,0.00}{\textbf{{#1}}}}
    \newcommand{\FunctionTok}[1]{\textcolor[rgb]{0.02,0.16,0.49}{{#1}}}
    \newcommand{\RegionMarkerTok}[1]{{#1}}
    \newcommand{\ErrorTok}[1]{\textcolor[rgb]{1.00,0.00,0.00}{\textbf{{#1}}}}
    \newcommand{\NormalTok}[1]{{#1}}
    
    % Additional commands for more recent versions of Pandoc
    \newcommand{\ConstantTok}[1]{\textcolor[rgb]{0.53,0.00,0.00}{{#1}}}
    \newcommand{\SpecialCharTok}[1]{\textcolor[rgb]{0.25,0.44,0.63}{{#1}}}
    \newcommand{\VerbatimStringTok}[1]{\textcolor[rgb]{0.25,0.44,0.63}{{#1}}}
    \newcommand{\SpecialStringTok}[1]{\textcolor[rgb]{0.73,0.40,0.53}{{#1}}}
    \newcommand{\ImportTok}[1]{{#1}}
    \newcommand{\DocumentationTok}[1]{\textcolor[rgb]{0.73,0.13,0.13}{\textit{{#1}}}}
    \newcommand{\AnnotationTok}[1]{\textcolor[rgb]{0.38,0.63,0.69}{\textbf{\textit{{#1}}}}}
    \newcommand{\CommentVarTok}[1]{\textcolor[rgb]{0.38,0.63,0.69}{\textbf{\textit{{#1}}}}}
    \newcommand{\VariableTok}[1]{\textcolor[rgb]{0.10,0.09,0.49}{{#1}}}
    \newcommand{\ControlFlowTok}[1]{\textcolor[rgb]{0.00,0.44,0.13}{\textbf{{#1}}}}
    \newcommand{\OperatorTok}[1]{\textcolor[rgb]{0.40,0.40,0.40}{{#1}}}
    \newcommand{\BuiltInTok}[1]{{#1}}
    \newcommand{\ExtensionTok}[1]{{#1}}
    \newcommand{\PreprocessorTok}[1]{\textcolor[rgb]{0.74,0.48,0.00}{{#1}}}
    \newcommand{\AttributeTok}[1]{\textcolor[rgb]{0.49,0.56,0.16}{{#1}}}
    \newcommand{\InformationTok}[1]{\textcolor[rgb]{0.38,0.63,0.69}{\textbf{\textit{{#1}}}}}
    \newcommand{\WarningTok}[1]{\textcolor[rgb]{0.38,0.63,0.69}{\textbf{\textit{{#1}}}}}
    
    
    % Define a nice break command that doesn't care if a line doesn't already
    % exist.
    \def\br{\hspace*{\fill} \\* }
    % Math Jax compatibility definitions
    \def\gt{>}
    \def\lt{<}
    \let\Oldtex\TeX
    \let\Oldlatex\LaTeX
    \renewcommand{\TeX}{\textrm{\Oldtex}}
    \renewcommand{\LaTeX}{\textrm{\Oldlatex}}
    % Document parameters
    % Document title
    \title{TP2-Exercicio1}
    
    
    
    
    
% Pygments definitions
\makeatletter
\def\PY@reset{\let\PY@it=\relax \let\PY@bf=\relax%
    \let\PY@ul=\relax \let\PY@tc=\relax%
    \let\PY@bc=\relax \let\PY@ff=\relax}
\def\PY@tok#1{\csname PY@tok@#1\endcsname}
\def\PY@toks#1+{\ifx\relax#1\empty\else%
    \PY@tok{#1}\expandafter\PY@toks\fi}
\def\PY@do#1{\PY@bc{\PY@tc{\PY@ul{%
    \PY@it{\PY@bf{\PY@ff{#1}}}}}}}
\def\PY#1#2{\PY@reset\PY@toks#1+\relax+\PY@do{#2}}

\@namedef{PY@tok@w}{\def\PY@tc##1{\textcolor[rgb]{0.73,0.73,0.73}{##1}}}
\@namedef{PY@tok@c}{\let\PY@it=\textit\def\PY@tc##1{\textcolor[rgb]{0.24,0.48,0.48}{##1}}}
\@namedef{PY@tok@cp}{\def\PY@tc##1{\textcolor[rgb]{0.61,0.40,0.00}{##1}}}
\@namedef{PY@tok@k}{\let\PY@bf=\textbf\def\PY@tc##1{\textcolor[rgb]{0.00,0.50,0.00}{##1}}}
\@namedef{PY@tok@kp}{\def\PY@tc##1{\textcolor[rgb]{0.00,0.50,0.00}{##1}}}
\@namedef{PY@tok@kt}{\def\PY@tc##1{\textcolor[rgb]{0.69,0.00,0.25}{##1}}}
\@namedef{PY@tok@o}{\def\PY@tc##1{\textcolor[rgb]{0.40,0.40,0.40}{##1}}}
\@namedef{PY@tok@ow}{\let\PY@bf=\textbf\def\PY@tc##1{\textcolor[rgb]{0.67,0.13,1.00}{##1}}}
\@namedef{PY@tok@nb}{\def\PY@tc##1{\textcolor[rgb]{0.00,0.50,0.00}{##1}}}
\@namedef{PY@tok@nf}{\def\PY@tc##1{\textcolor[rgb]{0.00,0.00,1.00}{##1}}}
\@namedef{PY@tok@nc}{\let\PY@bf=\textbf\def\PY@tc##1{\textcolor[rgb]{0.00,0.00,1.00}{##1}}}
\@namedef{PY@tok@nn}{\let\PY@bf=\textbf\def\PY@tc##1{\textcolor[rgb]{0.00,0.00,1.00}{##1}}}
\@namedef{PY@tok@ne}{\let\PY@bf=\textbf\def\PY@tc##1{\textcolor[rgb]{0.80,0.25,0.22}{##1}}}
\@namedef{PY@tok@nv}{\def\PY@tc##1{\textcolor[rgb]{0.10,0.09,0.49}{##1}}}
\@namedef{PY@tok@no}{\def\PY@tc##1{\textcolor[rgb]{0.53,0.00,0.00}{##1}}}
\@namedef{PY@tok@nl}{\def\PY@tc##1{\textcolor[rgb]{0.46,0.46,0.00}{##1}}}
\@namedef{PY@tok@ni}{\let\PY@bf=\textbf\def\PY@tc##1{\textcolor[rgb]{0.44,0.44,0.44}{##1}}}
\@namedef{PY@tok@na}{\def\PY@tc##1{\textcolor[rgb]{0.41,0.47,0.13}{##1}}}
\@namedef{PY@tok@nt}{\let\PY@bf=\textbf\def\PY@tc##1{\textcolor[rgb]{0.00,0.50,0.00}{##1}}}
\@namedef{PY@tok@nd}{\def\PY@tc##1{\textcolor[rgb]{0.67,0.13,1.00}{##1}}}
\@namedef{PY@tok@s}{\def\PY@tc##1{\textcolor[rgb]{0.73,0.13,0.13}{##1}}}
\@namedef{PY@tok@sd}{\let\PY@it=\textit\def\PY@tc##1{\textcolor[rgb]{0.73,0.13,0.13}{##1}}}
\@namedef{PY@tok@si}{\let\PY@bf=\textbf\def\PY@tc##1{\textcolor[rgb]{0.64,0.35,0.47}{##1}}}
\@namedef{PY@tok@se}{\let\PY@bf=\textbf\def\PY@tc##1{\textcolor[rgb]{0.67,0.36,0.12}{##1}}}
\@namedef{PY@tok@sr}{\def\PY@tc##1{\textcolor[rgb]{0.64,0.35,0.47}{##1}}}
\@namedef{PY@tok@ss}{\def\PY@tc##1{\textcolor[rgb]{0.10,0.09,0.49}{##1}}}
\@namedef{PY@tok@sx}{\def\PY@tc##1{\textcolor[rgb]{0.00,0.50,0.00}{##1}}}
\@namedef{PY@tok@m}{\def\PY@tc##1{\textcolor[rgb]{0.40,0.40,0.40}{##1}}}
\@namedef{PY@tok@gh}{\let\PY@bf=\textbf\def\PY@tc##1{\textcolor[rgb]{0.00,0.00,0.50}{##1}}}
\@namedef{PY@tok@gu}{\let\PY@bf=\textbf\def\PY@tc##1{\textcolor[rgb]{0.50,0.00,0.50}{##1}}}
\@namedef{PY@tok@gd}{\def\PY@tc##1{\textcolor[rgb]{0.63,0.00,0.00}{##1}}}
\@namedef{PY@tok@gi}{\def\PY@tc##1{\textcolor[rgb]{0.00,0.52,0.00}{##1}}}
\@namedef{PY@tok@gr}{\def\PY@tc##1{\textcolor[rgb]{0.89,0.00,0.00}{##1}}}
\@namedef{PY@tok@ge}{\let\PY@it=\textit}
\@namedef{PY@tok@gs}{\let\PY@bf=\textbf}
\@namedef{PY@tok@gp}{\let\PY@bf=\textbf\def\PY@tc##1{\textcolor[rgb]{0.00,0.00,0.50}{##1}}}
\@namedef{PY@tok@go}{\def\PY@tc##1{\textcolor[rgb]{0.44,0.44,0.44}{##1}}}
\@namedef{PY@tok@gt}{\def\PY@tc##1{\textcolor[rgb]{0.00,0.27,0.87}{##1}}}
\@namedef{PY@tok@err}{\def\PY@bc##1{{\setlength{\fboxsep}{\string -\fboxrule}\fcolorbox[rgb]{1.00,0.00,0.00}{1,1,1}{\strut ##1}}}}
\@namedef{PY@tok@kc}{\let\PY@bf=\textbf\def\PY@tc##1{\textcolor[rgb]{0.00,0.50,0.00}{##1}}}
\@namedef{PY@tok@kd}{\let\PY@bf=\textbf\def\PY@tc##1{\textcolor[rgb]{0.00,0.50,0.00}{##1}}}
\@namedef{PY@tok@kn}{\let\PY@bf=\textbf\def\PY@tc##1{\textcolor[rgb]{0.00,0.50,0.00}{##1}}}
\@namedef{PY@tok@kr}{\let\PY@bf=\textbf\def\PY@tc##1{\textcolor[rgb]{0.00,0.50,0.00}{##1}}}
\@namedef{PY@tok@bp}{\def\PY@tc##1{\textcolor[rgb]{0.00,0.50,0.00}{##1}}}
\@namedef{PY@tok@fm}{\def\PY@tc##1{\textcolor[rgb]{0.00,0.00,1.00}{##1}}}
\@namedef{PY@tok@vc}{\def\PY@tc##1{\textcolor[rgb]{0.10,0.09,0.49}{##1}}}
\@namedef{PY@tok@vg}{\def\PY@tc##1{\textcolor[rgb]{0.10,0.09,0.49}{##1}}}
\@namedef{PY@tok@vi}{\def\PY@tc##1{\textcolor[rgb]{0.10,0.09,0.49}{##1}}}
\@namedef{PY@tok@vm}{\def\PY@tc##1{\textcolor[rgb]{0.10,0.09,0.49}{##1}}}
\@namedef{PY@tok@sa}{\def\PY@tc##1{\textcolor[rgb]{0.73,0.13,0.13}{##1}}}
\@namedef{PY@tok@sb}{\def\PY@tc##1{\textcolor[rgb]{0.73,0.13,0.13}{##1}}}
\@namedef{PY@tok@sc}{\def\PY@tc##1{\textcolor[rgb]{0.73,0.13,0.13}{##1}}}
\@namedef{PY@tok@dl}{\def\PY@tc##1{\textcolor[rgb]{0.73,0.13,0.13}{##1}}}
\@namedef{PY@tok@s2}{\def\PY@tc##1{\textcolor[rgb]{0.73,0.13,0.13}{##1}}}
\@namedef{PY@tok@sh}{\def\PY@tc##1{\textcolor[rgb]{0.73,0.13,0.13}{##1}}}
\@namedef{PY@tok@s1}{\def\PY@tc##1{\textcolor[rgb]{0.73,0.13,0.13}{##1}}}
\@namedef{PY@tok@mb}{\def\PY@tc##1{\textcolor[rgb]{0.40,0.40,0.40}{##1}}}
\@namedef{PY@tok@mf}{\def\PY@tc##1{\textcolor[rgb]{0.40,0.40,0.40}{##1}}}
\@namedef{PY@tok@mh}{\def\PY@tc##1{\textcolor[rgb]{0.40,0.40,0.40}{##1}}}
\@namedef{PY@tok@mi}{\def\PY@tc##1{\textcolor[rgb]{0.40,0.40,0.40}{##1}}}
\@namedef{PY@tok@il}{\def\PY@tc##1{\textcolor[rgb]{0.40,0.40,0.40}{##1}}}
\@namedef{PY@tok@mo}{\def\PY@tc##1{\textcolor[rgb]{0.40,0.40,0.40}{##1}}}
\@namedef{PY@tok@ch}{\let\PY@it=\textit\def\PY@tc##1{\textcolor[rgb]{0.24,0.48,0.48}{##1}}}
\@namedef{PY@tok@cm}{\let\PY@it=\textit\def\PY@tc##1{\textcolor[rgb]{0.24,0.48,0.48}{##1}}}
\@namedef{PY@tok@cpf}{\let\PY@it=\textit\def\PY@tc##1{\textcolor[rgb]{0.24,0.48,0.48}{##1}}}
\@namedef{PY@tok@c1}{\let\PY@it=\textit\def\PY@tc##1{\textcolor[rgb]{0.24,0.48,0.48}{##1}}}
\@namedef{PY@tok@cs}{\let\PY@it=\textit\def\PY@tc##1{\textcolor[rgb]{0.24,0.48,0.48}{##1}}}

\def\PYZbs{\char`\\}
\def\PYZus{\char`\_}
\def\PYZob{\char`\{}
\def\PYZcb{\char`\}}
\def\PYZca{\char`\^}
\def\PYZam{\char`\&}
\def\PYZlt{\char`\<}
\def\PYZgt{\char`\>}
\def\PYZsh{\char`\#}
\def\PYZpc{\char`\%}
\def\PYZdl{\char`\$}
\def\PYZhy{\char`\-}
\def\PYZsq{\char`\'}
\def\PYZdq{\char`\"}
\def\PYZti{\char`\~}
% for compatibility with earlier versions
\def\PYZat{@}
\def\PYZlb{[}
\def\PYZrb{]}
\makeatother


    % For linebreaks inside Verbatim environment from package fancyvrb. 
    \makeatletter
        \newbox\Wrappedcontinuationbox 
        \newbox\Wrappedvisiblespacebox 
        \newcommand*\Wrappedvisiblespace {\textcolor{red}{\textvisiblespace}} 
        \newcommand*\Wrappedcontinuationsymbol {\textcolor{red}{\llap{\tiny$\m@th\hookrightarrow$}}} 
        \newcommand*\Wrappedcontinuationindent {3ex } 
        \newcommand*\Wrappedafterbreak {\kern\Wrappedcontinuationindent\copy\Wrappedcontinuationbox} 
        % Take advantage of the already applied Pygments mark-up to insert 
        % potential linebreaks for TeX processing. 
        %        {, <, #, %, $, ' and ": go to next line. 
        %        _, }, ^, &, >, - and ~: stay at end of broken line. 
        % Use of \textquotesingle for straight quote. 
        \newcommand*\Wrappedbreaksatspecials {% 
            \def\PYGZus{\discretionary{\char`\_}{\Wrappedafterbreak}{\char`\_}}% 
            \def\PYGZob{\discretionary{}{\Wrappedafterbreak\char`\{}{\char`\{}}% 
            \def\PYGZcb{\discretionary{\char`\}}{\Wrappedafterbreak}{\char`\}}}% 
            \def\PYGZca{\discretionary{\char`\^}{\Wrappedafterbreak}{\char`\^}}% 
            \def\PYGZam{\discretionary{\char`\&}{\Wrappedafterbreak}{\char`\&}}% 
            \def\PYGZlt{\discretionary{}{\Wrappedafterbreak\char`\<}{\char`\<}}% 
            \def\PYGZgt{\discretionary{\char`\>}{\Wrappedafterbreak}{\char`\>}}% 
            \def\PYGZsh{\discretionary{}{\Wrappedafterbreak\char`\#}{\char`\#}}% 
            \def\PYGZpc{\discretionary{}{\Wrappedafterbreak\char`\%}{\char`\%}}% 
            \def\PYGZdl{\discretionary{}{\Wrappedafterbreak\char`\$}{\char`\$}}% 
            \def\PYGZhy{\discretionary{\char`\-}{\Wrappedafterbreak}{\char`\-}}% 
            \def\PYGZsq{\discretionary{}{\Wrappedafterbreak\textquotesingle}{\textquotesingle}}% 
            \def\PYGZdq{\discretionary{}{\Wrappedafterbreak\char`\"}{\char`\"}}% 
            \def\PYGZti{\discretionary{\char`\~}{\Wrappedafterbreak}{\char`\~}}% 
        } 
        % Some characters . , ; ? ! / are not pygmentized. 
        % This macro makes them "active" and they will insert potential linebreaks 
        \newcommand*\Wrappedbreaksatpunct {% 
            \lccode`\~`\.\lowercase{\def~}{\discretionary{\hbox{\char`\.}}{\Wrappedafterbreak}{\hbox{\char`\.}}}% 
            \lccode`\~`\,\lowercase{\def~}{\discretionary{\hbox{\char`\,}}{\Wrappedafterbreak}{\hbox{\char`\,}}}% 
            \lccode`\~`\;\lowercase{\def~}{\discretionary{\hbox{\char`\;}}{\Wrappedafterbreak}{\hbox{\char`\;}}}% 
            \lccode`\~`\:\lowercase{\def~}{\discretionary{\hbox{\char`\:}}{\Wrappedafterbreak}{\hbox{\char`\:}}}% 
            \lccode`\~`\?\lowercase{\def~}{\discretionary{\hbox{\char`\?}}{\Wrappedafterbreak}{\hbox{\char`\?}}}% 
            \lccode`\~`\!\lowercase{\def~}{\discretionary{\hbox{\char`\!}}{\Wrappedafterbreak}{\hbox{\char`\!}}}% 
            \lccode`\~`\/\lowercase{\def~}{\discretionary{\hbox{\char`\/}}{\Wrappedafterbreak}{\hbox{\char`\/}}}% 
            \catcode`\.\active
            \catcode`\,\active 
            \catcode`\;\active
            \catcode`\:\active
            \catcode`\?\active
            \catcode`\!\active
            \catcode`\/\active 
            \lccode`\~`\~ 	
        }
    \makeatother

    \let\OriginalVerbatim=\Verbatim
    \makeatletter
    \renewcommand{\Verbatim}[1][1]{%
        %\parskip\z@skip
        \sbox\Wrappedcontinuationbox {\Wrappedcontinuationsymbol}%
        \sbox\Wrappedvisiblespacebox {\FV@SetupFont\Wrappedvisiblespace}%
        \def\FancyVerbFormatLine ##1{\hsize\linewidth
            \vtop{\raggedright\hyphenpenalty\z@\exhyphenpenalty\z@
                \doublehyphendemerits\z@\finalhyphendemerits\z@
                \strut ##1\strut}%
        }%
        % If the linebreak is at a space, the latter will be displayed as visible
        % space at end of first line, and a continuation symbol starts next line.
        % Stretch/shrink are however usually zero for typewriter font.
        \def\FV@Space {%
            \nobreak\hskip\z@ plus\fontdimen3\font minus\fontdimen4\font
            \discretionary{\copy\Wrappedvisiblespacebox}{\Wrappedafterbreak}
            {\kern\fontdimen2\font}%
        }%
        
        % Allow breaks at special characters using \PYG... macros.
        \Wrappedbreaksatspecials
        % Breaks at punctuation characters . , ; ? ! and / need catcode=\active 	
        \OriginalVerbatim[#1,codes*=\Wrappedbreaksatpunct]%
    }
    \makeatother

    % Exact colors from NB
    \definecolor{incolor}{HTML}{303F9F}
    \definecolor{outcolor}{HTML}{D84315}
    \definecolor{cellborder}{HTML}{CFCFCF}
    \definecolor{cellbackground}{HTML}{F7F7F7}
    
    % prompt
    \makeatletter
    \newcommand{\boxspacing}{\kern\kvtcb@left@rule\kern\kvtcb@boxsep}
    \makeatother
    \newcommand{\prompt}[4]{
        {\ttfamily\llap{{\color{#2}[#3]:\hspace{3pt}#4}}\vspace{-\baselineskip}}
    }
    

    
    % Prevent overflowing lines due to hard-to-break entities
    \sloppy 
    % Setup hyperref package
    \hypersetup{
      breaklinks=true,  % so long urls are correctly broken across lines
      colorlinks=true,
      urlcolor=urlcolor,
      linkcolor=linkcolor,
      citecolor=citecolor,
      }
    % Slightly bigger margins than the latex defaults
    
    \geometry{verbose,tmargin=1in,bmargin=1in,lmargin=1in,rmargin=1in}
    
    

\begin{document}
    
    \maketitle
    
    

    
    \hypertarget{tp2}{%
\section{TP2}\label{tp2}}

\hypertarget{grupo-15}{%
\subsection{Grupo 15}\label{grupo-15}}

Carlos Eduardo Da Silva Machado A96936

Gonçalo Manuel Maia de Sousa A97485

    \hypertarget{problema-1}{%
\subsection{Problema 1}\label{problema-1}}

    \hypertarget{descriuxe7uxe3o-do-problema}{%
\subsubsection{Descrição do
Problema}\label{descriuxe7uxe3o-do-problema}}

    É nos dado um Control Flow Automaton (CFA) que descreve um programa
imperativo cujo objetivo é implementar a multiplicação de dois inteiros
a,b, fornecidos como ``input'' e um n, também fornecido como ``input'',
de precisão limitada. Para alem disso, temos de ter em conta os
seguintes aspetos: - Existe a possibilidade de alguma das operações do
programa produzir um erro de ``overflow''; - Os nós do grafo representam
ações que actuam sobre os ``inputs'' do nó e produzem um ``output'' com
as operações indicadas; - Os ramos do grafo representam ligações que
transferem o ``output'' de um nodo para o ``input'' do nodo seguinte.
Esta transferência é condicionada pela satisfação da condição associada
ao ramo.

    \hypertarget{abordagem-do-problema}{%
\subsubsection{Abordagem do Problema}\label{abordagem-do-problema}}

    Para resolver este problema, vamos construir um First Order Transition
System (FOTS) usando \(BitVector\)'s de tamanho \(n\) de forma a
descrever o comportamento do autómato acima mencionado.

São parâmetros do problema \(a\), \(b\), \(n\), e \(k\) tais que: 1.
\(a\) é o valor inicial de \(x\) 2. \(b\) é o valor inicial de \(y\) 3.
\(k\) é o número máximo de estados num traço do problema, toma o valor
de \(n+1\), este valor é o resultado do seguinte calculo: \[
2.\log(2^{\frac{n}{2}-1})-1 \approx 2.\log(2^{\frac{n}{2}})-1 = 2\frac{n}{2}-1 = n-1;\\
\text{Este é o número de operações para o pior caso, com } y = 2^\frac{n}{2}-1, \text{pois são realizados }\\ \log(2^{\frac{n}{2}-1}) \text{ shifts e }\log(2^{\frac{n}{2}-1})-1\text{ subtrações no } y \\
\text{Para utilizar este valor resta apenas adicionar }2\text{, pois para além de }n-1\text{ estados é necessário incluir o estado inicial e o estado final.} 
\]\\
4. \(n\) é o número de \(bit\)'s máximo das variáveis

O autómato consiste na seguinte estrutura: 1. Um estado final
(\(pc=1\)). 2. Um estado de erro (\(pc=2\)) que marca o estado de
\(overflow\) 3. Um estado de operações (\(pc=0\)) no o qual todas as
operações sobre as variaveis serão realizadas

De modo a tratar de casos de \(overflow\) as variáveis \(x\), \(y\) e
\(z\) são declaradas como \(BitVector\)'s de tamanho \(n+1\). Assim se o
primeiro bit de uma delas for \(1\) podemos transitar para o estado de
\(overflow\)

Além disso, por motivos de optimização no caso da variavel \(b\) ser
maior do que \(a\), são trocadas para que o número de transições seja
minimizado.

Para além do FOTS, também vamos verificar se \(P≡(x∗y+z=a∗b)\) é um
invariante do comportamento que estamos a estudar.

    \hypertarget{cuxf3digo-python}{%
\subsection{Código Python}\label{cuxf3digo-python}}

    \hypertarget{algoritmo-buxe1sico}{%
\paragraph{Algoritmo básico}\label{algoritmo-buxe1sico}}

    variaveis -\textgreater{} x,y,z,pc

0: while(y!=0):

\begin{verbatim}
if even(y) then x,y,z = 2*x,y/2,z

if odd(y)  then x,y,z = x,y-1,z+x
\end{verbatim}

1: stop

    Vamos Utilizar a biblioteca do \(\textit{Pysmt}\) e a biblioteca
\(\textit{random}\) para resolver este exercício.

    \begin{tcolorbox}[breakable, size=fbox, boxrule=1pt, pad at break*=1mm,colback=cellbackground, colframe=cellborder]
\prompt{In}{incolor}{1}{\boxspacing}
\begin{Verbatim}[commandchars=\\\{\}]
\PY{k+kn}{from} \PY{n+nn}{pysmt}\PY{n+nn}{.}\PY{n+nn}{shortcuts} \PY{k+kn}{import} \PY{o}{*}
\PY{k+kn}{from} \PY{n+nn}{pysmt}\PY{n+nn}{.}\PY{n+nn}{typing} \PY{k+kn}{import} \PY{n}{INT}
\PY{k+kn}{import} \PY{n+nn}{random} \PY{k}{as} \PY{n+nn}{rn}
\end{Verbatim}
\end{tcolorbox}

    Construção do FOTS:

    Função de declaração:

    \begin{tcolorbox}[breakable, size=fbox, boxrule=1pt, pad at break*=1mm,colback=cellbackground, colframe=cellborder]
\prompt{In}{incolor}{2}{\boxspacing}
\begin{Verbatim}[commandchars=\\\{\}]
\PY{k}{def} \PY{n+nf}{declare}\PY{p}{(}\PY{n}{i}\PY{p}{,}\PY{n}{n}\PY{p}{)}\PY{p}{:}
    \PY{n}{state} \PY{o}{=} \PY{p}{\PYZob{}}\PY{p}{\PYZcb{}}
    \PY{n}{state}\PY{p}{[}\PY{l+s+s1}{\PYZsq{}}\PY{l+s+s1}{pc}\PY{l+s+s1}{\PYZsq{}}\PY{p}{]} \PY{o}{=} \PY{n}{Symbol}\PY{p}{(}\PY{l+s+s1}{\PYZsq{}}\PY{l+s+s1}{pc}\PY{l+s+s1}{\PYZsq{}}\PY{o}{+}\PY{n+nb}{str}\PY{p}{(}\PY{n}{i}\PY{p}{)}\PY{p}{,}\PY{n}{INT}\PY{p}{)}
    \PY{n}{state}\PY{p}{[}\PY{l+s+s1}{\PYZsq{}}\PY{l+s+s1}{x}\PY{l+s+s1}{\PYZsq{}}\PY{p}{]} \PY{o}{=} \PY{n}{Symbol}\PY{p}{(}\PY{l+s+s1}{\PYZsq{}}\PY{l+s+s1}{x}\PY{l+s+s1}{\PYZsq{}}\PY{o}{+}\PY{n+nb}{str}\PY{p}{(}\PY{n}{i}\PY{p}{)}\PY{p}{,}\PY{n}{types}\PY{o}{.}\PY{n}{BVType}\PY{p}{(}\PY{n}{n}\PY{o}{+}\PY{l+m+mi}{1}\PY{p}{)}\PY{p}{)}
    \PY{n}{state}\PY{p}{[}\PY{l+s+s1}{\PYZsq{}}\PY{l+s+s1}{y}\PY{l+s+s1}{\PYZsq{}}\PY{p}{]} \PY{o}{=} \PY{n}{Symbol}\PY{p}{(}\PY{l+s+s1}{\PYZsq{}}\PY{l+s+s1}{y}\PY{l+s+s1}{\PYZsq{}}\PY{o}{+}\PY{n+nb}{str}\PY{p}{(}\PY{n}{i}\PY{p}{)}\PY{p}{,}\PY{n}{types}\PY{o}{.}\PY{n}{BVType}\PY{p}{(}\PY{n}{n}\PY{o}{+}\PY{l+m+mi}{1}\PY{p}{)}\PY{p}{)}
    \PY{n}{state}\PY{p}{[}\PY{l+s+s1}{\PYZsq{}}\PY{l+s+s1}{z}\PY{l+s+s1}{\PYZsq{}}\PY{p}{]} \PY{o}{=} \PY{n}{Symbol}\PY{p}{(}\PY{l+s+s1}{\PYZsq{}}\PY{l+s+s1}{z}\PY{l+s+s1}{\PYZsq{}}\PY{o}{+}\PY{n+nb}{str}\PY{p}{(}\PY{n}{i}\PY{p}{)}\PY{p}{,}\PY{n}{types}\PY{o}{.}\PY{n}{BVType}\PY{p}{(}\PY{n}{n}\PY{o}{+}\PY{l+m+mi}{1}\PY{p}{)}\PY{p}{)}
    \PY{k}{return} \PY{n}{state}
\end{Verbatim}
\end{tcolorbox}

    Função de inicialização:

    \begin{tcolorbox}[breakable, size=fbox, boxrule=1pt, pad at break*=1mm,colback=cellbackground, colframe=cellborder]
\prompt{In}{incolor}{3}{\boxspacing}
\begin{Verbatim}[commandchars=\\\{\}]
\PY{k}{def} \PY{n+nf}{init}\PY{p}{(}\PY{n}{state}\PY{p}{,}\PY{n}{a}\PY{p}{,}\PY{n}{b}\PY{p}{,}\PY{n}{n}\PY{p}{)}\PY{p}{:}
    \PY{k}{if} \PY{n}{b} \PY{o}{\PYZgt{}} \PY{n}{a}\PY{p}{:}
        \PY{n}{a}\PY{p}{,}\PY{n}{b} \PY{o}{=} \PY{n}{b}\PY{p}{,}\PY{n}{a}
        
    \PY{n}{tPc} \PY{o}{=} \PY{n}{Equals}\PY{p}{(}\PY{n}{state}\PY{p}{[}\PY{l+s+s1}{\PYZsq{}}\PY{l+s+s1}{pc}\PY{l+s+s1}{\PYZsq{}}\PY{p}{]}\PY{p}{,}\PY{n}{Int}\PY{p}{(}\PY{l+m+mi}{0}\PY{p}{)}\PY{p}{)} \PY{c+c1}{\PYZsh{} Program counter a zero}
    \PY{n}{tZ} \PY{o}{=} \PY{n}{Equals}\PY{p}{(}\PY{n}{state}\PY{p}{[}\PY{l+s+s1}{\PYZsq{}}\PY{l+s+s1}{z}\PY{l+s+s1}{\PYZsq{}}\PY{p}{]}\PY{p}{,}\PY{n}{BVZero}\PY{p}{(}\PY{n}{n}\PY{o}{+}\PY{l+m+mi}{1}\PY{p}{)}\PY{p}{)} \PY{c+c1}{\PYZsh{} Z a zero}
    \PY{n}{tX} \PY{o}{=} \PY{n}{Equals}\PY{p}{(}\PY{n}{state}\PY{p}{[}\PY{l+s+s1}{\PYZsq{}}\PY{l+s+s1}{x}\PY{l+s+s1}{\PYZsq{}}\PY{p}{]}\PY{p}{,} \PY{n}{BV}\PY{p}{(}\PY{n}{a}\PY{p}{,}\PY{n}{n}\PY{o}{+}\PY{l+m+mi}{1}\PY{p}{)}\PY{p}{)} \PY{c+c1}{\PYZsh{} x inicilizado com valor de a}
    \PY{n}{tY} \PY{o}{=} \PY{n}{Equals}\PY{p}{(}\PY{n}{state}\PY{p}{[}\PY{l+s+s1}{\PYZsq{}}\PY{l+s+s1}{y}\PY{l+s+s1}{\PYZsq{}}\PY{p}{]}\PY{p}{,} \PY{n}{BV}\PY{p}{(}\PY{n}{b}\PY{p}{,}\PY{n}{n}\PY{o}{+}\PY{l+m+mi}{1}\PY{p}{)}\PY{p}{)} \PY{c+c1}{\PYZsh{} y inicilizado com valor de b}
    \PY{k}{return} \PY{n}{And}\PY{p}{(}\PY{n}{tPc}\PY{p}{,}\PY{n}{tX}\PY{p}{,}\PY{n}{tY}\PY{p}{,}\PY{n}{tZ}\PY{p}{)}
\end{Verbatim}
\end{tcolorbox}

    Função de Transição:

\[\mathsf{trans}(x,y,z,pc,x',y',z',pc')\;\equiv\;\] \[
 \left\{\begin{array}{lr}
 (pc=0)\land even(y) \land (y > 0) \land (x'=2x)\land (y'= \frac{y}{2}) \land (z'=c) \land (pc'=0) & \lor \\ 
 (pc=0) \land odd(y) \land (x'=x) \land (y'=y- 1) \land (z'=x+z) \land (pc'=0)  & \lor \\
 (pc=0)\land (y = 0) \land overflow(z) \land (x'=x)\land (y'=y) \land (z'=c) \land (pc'=1) & \lor \\
 (pc=1)\land(x'=x)\land (y'=y)\land (z'=z)\land (pc'=1)& \lor \\
 (pc=0)\land overflow(y) \land overflow(x) \land overflow(z) \land (x'=x)\land (y'=y) \land (z'=c) \land (pc'=2) & \lor \\ 
 (pc=2) \land (x'=x) \land (y'=y) \land (z'=z) \land (pc'= 2) & \end{array}\right.
 \]

    \begin{tcolorbox}[breakable, size=fbox, boxrule=1pt, pad at break*=1mm,colback=cellbackground, colframe=cellborder]
\prompt{In}{incolor}{4}{\boxspacing}
\begin{Verbatim}[commandchars=\\\{\}]
\PY{k}{def} \PY{n+nf}{BVFirst}\PY{p}{(}\PY{n}{n}\PY{p}{)}\PY{p}{:}
    \PY{k}{return} \PY{n}{BV}\PY{p}{(}\PY{l+m+mi}{2}\PY{o}{*}\PY{o}{*}\PY{p}{(}\PY{n}{n}\PY{o}{\PYZhy{}}\PY{l+m+mi}{1}\PY{p}{)}\PY{p}{,}\PY{n}{n}\PY{p}{)}

\PY{k}{def} \PY{n+nf}{tEven}\PY{p}{(}\PY{n}{curr}\PY{p}{,}\PY{n}{prox}\PY{p}{,}\PY{n}{n}\PY{p}{)}\PY{p}{:}
    \PY{n}{tPcZero} \PY{o}{=} \PY{n}{Equals}\PY{p}{(}\PY{n}{curr}\PY{p}{[}\PY{l+s+s1}{\PYZsq{}}\PY{l+s+s1}{pc}\PY{l+s+s1}{\PYZsq{}}\PY{p}{]}\PY{p}{,}\PY{n}{Int}\PY{p}{(}\PY{l+m+mi}{0}\PY{p}{)}\PY{p}{)}
    \PY{n}{tYLast} \PY{o}{=} \PY{n}{Equals}\PY{p}{(}\PY{n}{BVAnd}\PY{p}{(}\PY{n}{curr}\PY{p}{[}\PY{l+s+s1}{\PYZsq{}}\PY{l+s+s1}{y}\PY{l+s+s1}{\PYZsq{}}\PY{p}{]}\PY{p}{,}\PY{n}{BVOne}\PY{p}{(}\PY{n}{n}\PY{o}{+}\PY{l+m+mi}{1}\PY{p}{)}\PY{p}{)}\PY{p}{,}\PY{n}{BVZero}\PY{p}{(}\PY{n}{n}\PY{o}{+}\PY{l+m+mi}{1}\PY{p}{)}\PY{p}{)}\PY{c+c1}{\PYZsh{}ultimo bit = 0}
    \PY{n}{tYGt} \PY{o}{=} \PY{n}{BVUGT}\PY{p}{(}\PY{n}{curr}\PY{p}{[}\PY{l+s+s1}{\PYZsq{}}\PY{l+s+s1}{y}\PY{l+s+s1}{\PYZsq{}}\PY{p}{]}\PY{p}{,}\PY{n}{BVZero}\PY{p}{(}\PY{n}{n}\PY{o}{+}\PY{l+m+mi}{1}\PY{p}{)}\PY{p}{)}\PY{c+c1}{\PYZsh{}y \PYZgt{} 0}
    \PY{n}{tX} \PY{o}{=} \PY{n}{Equals}\PY{p}{(}\PY{n}{prox}\PY{p}{[}\PY{l+s+s1}{\PYZsq{}}\PY{l+s+s1}{x}\PY{l+s+s1}{\PYZsq{}}\PY{p}{]}\PY{p}{,} \PY{n}{BVLShl}\PY{p}{(}\PY{n}{curr}\PY{p}{[}\PY{l+s+s1}{\PYZsq{}}\PY{l+s+s1}{x}\PY{l+s+s1}{\PYZsq{}}\PY{p}{]}\PY{p}{,}\PY{n}{BVOne}\PY{p}{(}\PY{n}{n}\PY{o}{+}\PY{l+m+mi}{1}\PY{p}{)}\PY{p}{)}\PY{p}{)}\PY{c+c1}{\PYZsh{}2*x}
    \PY{n}{tY} \PY{o}{=} \PY{n}{Equals}\PY{p}{(}\PY{n}{prox}\PY{p}{[}\PY{l+s+s1}{\PYZsq{}}\PY{l+s+s1}{y}\PY{l+s+s1}{\PYZsq{}}\PY{p}{]}\PY{p}{,} \PY{n}{BVLShr}\PY{p}{(}\PY{n}{curr}\PY{p}{[}\PY{l+s+s1}{\PYZsq{}}\PY{l+s+s1}{y}\PY{l+s+s1}{\PYZsq{}}\PY{p}{]}\PY{p}{,}\PY{n}{BVOne}\PY{p}{(}\PY{n}{n}\PY{o}{+}\PY{l+m+mi}{1}\PY{p}{)}\PY{p}{)}\PY{p}{)}\PY{c+c1}{\PYZsh{}y/2}
    \PY{n}{tZ} \PY{o}{=} \PY{n}{Equals}\PY{p}{(}\PY{n}{prox}\PY{p}{[}\PY{l+s+s1}{\PYZsq{}}\PY{l+s+s1}{z}\PY{l+s+s1}{\PYZsq{}}\PY{p}{]}\PY{p}{,}\PY{n}{curr}\PY{p}{[}\PY{l+s+s1}{\PYZsq{}}\PY{l+s+s1}{z}\PY{l+s+s1}{\PYZsq{}}\PY{p}{]}\PY{p}{)}\PY{c+c1}{\PYZsh{}z}
    \PY{n}{tPc} \PY{o}{=} \PY{n}{Equals}\PY{p}{(}\PY{n}{prox}\PY{p}{[}\PY{l+s+s1}{\PYZsq{}}\PY{l+s+s1}{pc}\PY{l+s+s1}{\PYZsq{}}\PY{p}{]}\PY{p}{,}\PY{n}{Int}\PY{p}{(}\PY{l+m+mi}{0}\PY{p}{)}\PY{p}{)}
    \PY{k}{return} \PY{n}{And}\PY{p}{(}\PY{n}{tPcZero}\PY{p}{,}\PY{n}{tYLast}\PY{p}{,}\PY{n}{tYGt}\PY{p}{,}\PY{n}{tX}\PY{p}{,}\PY{n}{tY}\PY{p}{,}\PY{n}{tZ}\PY{p}{,}\PY{n}{tPc}\PY{p}{)}

\PY{k}{def} \PY{n+nf}{tOdd}\PY{p}{(}\PY{n}{curr}\PY{p}{,}\PY{n}{prox}\PY{p}{,}\PY{n}{n}\PY{p}{)}\PY{p}{:}
    \PY{n}{tPcZero} \PY{o}{=} \PY{n}{Equals}\PY{p}{(}\PY{n}{curr}\PY{p}{[}\PY{l+s+s1}{\PYZsq{}}\PY{l+s+s1}{pc}\PY{l+s+s1}{\PYZsq{}}\PY{p}{]}\PY{p}{,}\PY{n}{Int}\PY{p}{(}\PY{l+m+mi}{0}\PY{p}{)}\PY{p}{)}
    \PY{n}{tYLast} \PY{o}{=} \PY{n}{Equals}\PY{p}{(}\PY{n}{BVAnd}\PY{p}{(}\PY{n}{curr}\PY{p}{[}\PY{l+s+s1}{\PYZsq{}}\PY{l+s+s1}{y}\PY{l+s+s1}{\PYZsq{}}\PY{p}{]}\PY{p}{,}\PY{n}{BVOne}\PY{p}{(}\PY{n}{n}\PY{o}{+}\PY{l+m+mi}{1}\PY{p}{)}\PY{p}{)}\PY{p}{,}\PY{n}{BVOne}\PY{p}{(}\PY{n}{n}\PY{o}{+}\PY{l+m+mi}{1}\PY{p}{)}\PY{p}{)}
    \PY{n}{tX} \PY{o}{=} \PY{n}{Equals}\PY{p}{(}\PY{n}{prox}\PY{p}{[}\PY{l+s+s1}{\PYZsq{}}\PY{l+s+s1}{x}\PY{l+s+s1}{\PYZsq{}}\PY{p}{]}\PY{p}{,} \PY{n}{curr}\PY{p}{[}\PY{l+s+s1}{\PYZsq{}}\PY{l+s+s1}{x}\PY{l+s+s1}{\PYZsq{}}\PY{p}{]}\PY{p}{)}
    \PY{n}{tY} \PY{o}{=} \PY{n}{Equals}\PY{p}{(}\PY{n}{prox}\PY{p}{[}\PY{l+s+s1}{\PYZsq{}}\PY{l+s+s1}{y}\PY{l+s+s1}{\PYZsq{}}\PY{p}{]}\PY{p}{,}\PY{n}{BVSub}\PY{p}{(}\PY{n}{curr}\PY{p}{[}\PY{l+s+s1}{\PYZsq{}}\PY{l+s+s1}{y}\PY{l+s+s1}{\PYZsq{}}\PY{p}{]}\PY{p}{,}\PY{n}{BVOne}\PY{p}{(}\PY{n}{n}\PY{o}{+}\PY{l+m+mi}{1}\PY{p}{)}\PY{p}{)}\PY{p}{)}
    \PY{n}{tZ} \PY{o}{=} \PY{n}{Equals}\PY{p}{(}\PY{n}{prox}\PY{p}{[}\PY{l+s+s1}{\PYZsq{}}\PY{l+s+s1}{z}\PY{l+s+s1}{\PYZsq{}}\PY{p}{]}\PY{p}{,}\PY{n}{BVAdd}\PY{p}{(}\PY{n}{curr}\PY{p}{[}\PY{l+s+s1}{\PYZsq{}}\PY{l+s+s1}{x}\PY{l+s+s1}{\PYZsq{}}\PY{p}{]}\PY{p}{,}\PY{n}{curr}\PY{p}{[}\PY{l+s+s1}{\PYZsq{}}\PY{l+s+s1}{z}\PY{l+s+s1}{\PYZsq{}}\PY{p}{]}\PY{p}{)}\PY{p}{)} 
    \PY{n}{tPc} \PY{o}{=} \PY{n}{Equals}\PY{p}{(}\PY{n}{prox}\PY{p}{[}\PY{l+s+s1}{\PYZsq{}}\PY{l+s+s1}{pc}\PY{l+s+s1}{\PYZsq{}}\PY{p}{]}\PY{p}{,}\PY{n}{Int}\PY{p}{(}\PY{l+m+mi}{0}\PY{p}{)}\PY{p}{)}   
    \PY{k}{return} \PY{n}{And}\PY{p}{(}\PY{n}{tPcZero}\PY{p}{,}\PY{n}{tYLast}\PY{p}{,}\PY{n}{tX}\PY{p}{,}\PY{n}{tY}\PY{p}{,}\PY{n}{tZ}\PY{p}{,}\PY{n}{tPc}\PY{p}{)}

\PY{k}{def} \PY{n+nf}{tStop}\PY{p}{(}\PY{n}{curr}\PY{p}{,}\PY{n}{prox}\PY{p}{,}\PY{n}{n}\PY{p}{)}\PY{p}{:}
    \PY{n}{tPcZero} \PY{o}{=} \PY{n}{Equals}\PY{p}{(}\PY{n}{curr}\PY{p}{[}\PY{l+s+s1}{\PYZsq{}}\PY{l+s+s1}{pc}\PY{l+s+s1}{\PYZsq{}}\PY{p}{]}\PY{p}{,}\PY{n}{Int}\PY{p}{(}\PY{l+m+mi}{0}\PY{p}{)}\PY{p}{)}
    \PY{n}{tYZero} \PY{o}{=} \PY{n}{Equals}\PY{p}{(}\PY{n}{curr}\PY{p}{[}\PY{l+s+s1}{\PYZsq{}}\PY{l+s+s1}{y}\PY{l+s+s1}{\PYZsq{}}\PY{p}{]}\PY{p}{,}\PY{n}{BVZero}\PY{p}{(}\PY{n}{n}\PY{o}{+}\PY{l+m+mi}{1}\PY{p}{)}\PY{p}{)}\PY{c+c1}{\PYZsh{}y=0}
    \PY{n}{tZFirst} \PY{o}{=} \PY{n}{Equals}\PY{p}{(}\PY{n}{BVAnd}\PY{p}{(}\PY{n}{curr}\PY{p}{[}\PY{l+s+s1}{\PYZsq{}}\PY{l+s+s1}{z}\PY{l+s+s1}{\PYZsq{}}\PY{p}{]}\PY{p}{,}\PY{n}{BVFirst}\PY{p}{(}\PY{n}{n}\PY{o}{+}\PY{l+m+mi}{1}\PY{p}{)}\PY{p}{)}\PY{p}{,}\PY{n}{BVZero}\PY{p}{(}\PY{n}{n}\PY{o}{+}\PY{l+m+mi}{1}\PY{p}{)}\PY{p}{)}\PY{c+c1}{\PYZsh{}primriro bit de z = 0}
    \PY{n}{tX} \PY{o}{=} \PY{n}{Equals}\PY{p}{(}\PY{n}{prox}\PY{p}{[}\PY{l+s+s1}{\PYZsq{}}\PY{l+s+s1}{x}\PY{l+s+s1}{\PYZsq{}}\PY{p}{]}\PY{p}{,}\PY{n}{curr}\PY{p}{[}\PY{l+s+s1}{\PYZsq{}}\PY{l+s+s1}{x}\PY{l+s+s1}{\PYZsq{}}\PY{p}{]}\PY{p}{)}
    \PY{n}{tY} \PY{o}{=} \PY{n}{Equals}\PY{p}{(}\PY{n}{prox}\PY{p}{[}\PY{l+s+s1}{\PYZsq{}}\PY{l+s+s1}{y}\PY{l+s+s1}{\PYZsq{}}\PY{p}{]}\PY{p}{,}\PY{n}{curr}\PY{p}{[}\PY{l+s+s1}{\PYZsq{}}\PY{l+s+s1}{y}\PY{l+s+s1}{\PYZsq{}}\PY{p}{]}\PY{p}{)}
    \PY{n}{tZ} \PY{o}{=} \PY{n}{Equals}\PY{p}{(}\PY{n}{prox}\PY{p}{[}\PY{l+s+s1}{\PYZsq{}}\PY{l+s+s1}{z}\PY{l+s+s1}{\PYZsq{}}\PY{p}{]}\PY{p}{,}\PY{n}{curr}\PY{p}{[}\PY{l+s+s1}{\PYZsq{}}\PY{l+s+s1}{z}\PY{l+s+s1}{\PYZsq{}}\PY{p}{]}\PY{p}{)}
    \PY{n}{tPc} \PY{o}{=} \PY{n}{Equals}\PY{p}{(}\PY{n}{prox}\PY{p}{[}\PY{l+s+s1}{\PYZsq{}}\PY{l+s+s1}{pc}\PY{l+s+s1}{\PYZsq{}}\PY{p}{]}\PY{p}{,}\PY{n}{Int}\PY{p}{(}\PY{l+m+mi}{1}\PY{p}{)}\PY{p}{)}
    \PY{k}{return} \PY{n}{And}\PY{p}{(}\PY{n}{tYZero}\PY{p}{,}\PY{n}{tZFirst}\PY{p}{,}\PY{n}{tPcZero}\PY{p}{,}\PY{n}{tX}\PY{p}{,}\PY{n}{tY}\PY{p}{,}\PY{n}{tZ}\PY{p}{,}\PY{n}{tPc}\PY{p}{)}

\PY{k}{def} \PY{n+nf}{tEnd}\PY{p}{(}\PY{n}{curr}\PY{p}{,}\PY{n}{prox}\PY{p}{)}\PY{p}{:}
    \PY{n}{tPcOne} \PY{o}{=} \PY{n}{Equals}\PY{p}{(}\PY{n}{curr}\PY{p}{[}\PY{l+s+s1}{\PYZsq{}}\PY{l+s+s1}{pc}\PY{l+s+s1}{\PYZsq{}}\PY{p}{]}\PY{p}{,}\PY{n}{Int}\PY{p}{(}\PY{l+m+mi}{1}\PY{p}{)}\PY{p}{)}
    \PY{n}{tX} \PY{o}{=} \PY{n}{Equals}\PY{p}{(}\PY{n}{prox}\PY{p}{[}\PY{l+s+s1}{\PYZsq{}}\PY{l+s+s1}{x}\PY{l+s+s1}{\PYZsq{}}\PY{p}{]}\PY{p}{,}\PY{n}{curr}\PY{p}{[}\PY{l+s+s1}{\PYZsq{}}\PY{l+s+s1}{x}\PY{l+s+s1}{\PYZsq{}}\PY{p}{]}\PY{p}{)}
    \PY{n}{tY} \PY{o}{=} \PY{n}{Equals}\PY{p}{(}\PY{n}{prox}\PY{p}{[}\PY{l+s+s1}{\PYZsq{}}\PY{l+s+s1}{y}\PY{l+s+s1}{\PYZsq{}}\PY{p}{]}\PY{p}{,}\PY{n}{curr}\PY{p}{[}\PY{l+s+s1}{\PYZsq{}}\PY{l+s+s1}{y}\PY{l+s+s1}{\PYZsq{}}\PY{p}{]}\PY{p}{)}
    \PY{n}{tZ} \PY{o}{=} \PY{n}{Equals}\PY{p}{(}\PY{n}{prox}\PY{p}{[}\PY{l+s+s1}{\PYZsq{}}\PY{l+s+s1}{z}\PY{l+s+s1}{\PYZsq{}}\PY{p}{]}\PY{p}{,}\PY{n}{curr}\PY{p}{[}\PY{l+s+s1}{\PYZsq{}}\PY{l+s+s1}{z}\PY{l+s+s1}{\PYZsq{}}\PY{p}{]}\PY{p}{)}
    \PY{n}{tPc} \PY{o}{=} \PY{n}{Equals}\PY{p}{(}\PY{n}{prox}\PY{p}{[}\PY{l+s+s1}{\PYZsq{}}\PY{l+s+s1}{pc}\PY{l+s+s1}{\PYZsq{}}\PY{p}{]}\PY{p}{,}\PY{n}{Int}\PY{p}{(}\PY{l+m+mi}{1}\PY{p}{)}\PY{p}{)}
    \PY{k}{return} \PY{n}{And}\PY{p}{(}\PY{n}{tPcOne}\PY{p}{,}\PY{n}{tX}\PY{p}{,}\PY{n}{tY}\PY{p}{,}\PY{n}{tZ}\PY{p}{,}\PY{n}{tPc}\PY{p}{)}

\PY{k}{def} \PY{n+nf}{tError}\PY{p}{(}\PY{n}{curr}\PY{p}{,}\PY{n}{prox}\PY{p}{,}\PY{n}{n}\PY{p}{)}\PY{p}{:}
    \PY{n}{tPcZero} \PY{o}{=} \PY{n}{Equals}\PY{p}{(}\PY{n}{curr}\PY{p}{[}\PY{l+s+s1}{\PYZsq{}}\PY{l+s+s1}{pc}\PY{l+s+s1}{\PYZsq{}}\PY{p}{]}\PY{p}{,}\PY{n}{Int}\PY{p}{(}\PY{l+m+mi}{0}\PY{p}{)}\PY{p}{)}
    \PY{n}{tYFirst} \PY{o}{=} \PY{n}{Equals}\PY{p}{(}\PY{n}{BVAnd}\PY{p}{(}\PY{n}{curr}\PY{p}{[}\PY{l+s+s1}{\PYZsq{}}\PY{l+s+s1}{y}\PY{l+s+s1}{\PYZsq{}}\PY{p}{]}\PY{p}{,}\PY{n}{BVFirst}\PY{p}{(}\PY{n}{n}\PY{o}{+}\PY{l+m+mi}{1}\PY{p}{)}\PY{p}{)}\PY{p}{,}\PY{n}{BVFirst}\PY{p}{(}\PY{n}{n}\PY{o}{+}\PY{l+m+mi}{1}\PY{p}{)}\PY{p}{)}
    \PY{n}{tXFirst} \PY{o}{=} \PY{n}{Equals}\PY{p}{(}\PY{n}{BVAnd}\PY{p}{(}\PY{n}{curr}\PY{p}{[}\PY{l+s+s1}{\PYZsq{}}\PY{l+s+s1}{x}\PY{l+s+s1}{\PYZsq{}}\PY{p}{]}\PY{p}{,}\PY{n}{BVFirst}\PY{p}{(}\PY{n}{n}\PY{o}{+}\PY{l+m+mi}{1}\PY{p}{)}\PY{p}{)}\PY{p}{,}\PY{n}{BVFirst}\PY{p}{(}\PY{n}{n}\PY{o}{+}\PY{l+m+mi}{1}\PY{p}{)}\PY{p}{)}
    \PY{n}{tZFirst} \PY{o}{=} \PY{n}{Equals}\PY{p}{(}\PY{n}{BVAnd}\PY{p}{(}\PY{n}{curr}\PY{p}{[}\PY{l+s+s1}{\PYZsq{}}\PY{l+s+s1}{z}\PY{l+s+s1}{\PYZsq{}}\PY{p}{]}\PY{p}{,}\PY{n}{BVFirst}\PY{p}{(}\PY{n}{n}\PY{o}{+}\PY{l+m+mi}{1}\PY{p}{)}\PY{p}{)}\PY{p}{,}\PY{n}{BVFirst}\PY{p}{(}\PY{n}{n}\PY{o}{+}\PY{l+m+mi}{1}\PY{p}{)}\PY{p}{)}
    \PY{n}{tX} \PY{o}{=} \PY{n}{Equals}\PY{p}{(}\PY{n}{prox}\PY{p}{[}\PY{l+s+s1}{\PYZsq{}}\PY{l+s+s1}{x}\PY{l+s+s1}{\PYZsq{}}\PY{p}{]}\PY{p}{,} \PY{n}{curr}\PY{p}{[}\PY{l+s+s1}{\PYZsq{}}\PY{l+s+s1}{x}\PY{l+s+s1}{\PYZsq{}}\PY{p}{]}\PY{p}{)}
    \PY{n}{tY} \PY{o}{=} \PY{n}{Equals}\PY{p}{(}\PY{n}{prox}\PY{p}{[}\PY{l+s+s1}{\PYZsq{}}\PY{l+s+s1}{y}\PY{l+s+s1}{\PYZsq{}}\PY{p}{]}\PY{p}{,}\PY{n}{curr}\PY{p}{[}\PY{l+s+s1}{\PYZsq{}}\PY{l+s+s1}{y}\PY{l+s+s1}{\PYZsq{}}\PY{p}{]}\PY{p}{)}
    \PY{n}{tZ} \PY{o}{=} \PY{n}{Equals}\PY{p}{(}\PY{n}{prox}\PY{p}{[}\PY{l+s+s1}{\PYZsq{}}\PY{l+s+s1}{z}\PY{l+s+s1}{\PYZsq{}}\PY{p}{]}\PY{p}{,}\PY{n}{curr}\PY{p}{[}\PY{l+s+s1}{\PYZsq{}}\PY{l+s+s1}{z}\PY{l+s+s1}{\PYZsq{}}\PY{p}{]}\PY{p}{)}
    \PY{n}{tPc} \PY{o}{=} \PY{n}{Equals}\PY{p}{(}\PY{n}{prox}\PY{p}{[}\PY{l+s+s1}{\PYZsq{}}\PY{l+s+s1}{pc}\PY{l+s+s1}{\PYZsq{}}\PY{p}{]}\PY{p}{,}\PY{n}{Int}\PY{p}{(}\PY{l+m+mi}{2}\PY{p}{)}\PY{p}{)}
    \PY{k}{return} \PY{n}{And}\PY{p}{(}\PY{n}{tPcZero}\PY{p}{,}\PY{n}{Or}\PY{p}{(}\PY{n}{tYFirst}\PY{p}{,}\PY{n}{tXFirst}\PY{p}{,}\PY{n}{tZFirst}\PY{p}{)}\PY{p}{,}\PY{n}{tX}\PY{p}{,}\PY{n}{tY}\PY{p}{,}\PY{n}{tZ}\PY{p}{,}\PY{n}{tPc}\PY{p}{)}

\PY{k}{def} \PY{n+nf}{tEndError}\PY{p}{(}\PY{n}{curr}\PY{p}{,}\PY{n}{prox}\PY{p}{)}\PY{p}{:}
    \PY{n}{tPcTwo} \PY{o}{=} \PY{n}{Equals}\PY{p}{(}\PY{n}{curr}\PY{p}{[}\PY{l+s+s1}{\PYZsq{}}\PY{l+s+s1}{pc}\PY{l+s+s1}{\PYZsq{}}\PY{p}{]}\PY{p}{,}\PY{n}{Int}\PY{p}{(}\PY{l+m+mi}{2}\PY{p}{)}\PY{p}{)}
    \PY{n}{tX} \PY{o}{=} \PY{n}{Equals}\PY{p}{(}\PY{n}{prox}\PY{p}{[}\PY{l+s+s1}{\PYZsq{}}\PY{l+s+s1}{x}\PY{l+s+s1}{\PYZsq{}}\PY{p}{]}\PY{p}{,} \PY{n}{curr}\PY{p}{[}\PY{l+s+s1}{\PYZsq{}}\PY{l+s+s1}{x}\PY{l+s+s1}{\PYZsq{}}\PY{p}{]}\PY{p}{)}
    \PY{n}{tY} \PY{o}{=} \PY{n}{Equals}\PY{p}{(}\PY{n}{prox}\PY{p}{[}\PY{l+s+s1}{\PYZsq{}}\PY{l+s+s1}{y}\PY{l+s+s1}{\PYZsq{}}\PY{p}{]}\PY{p}{,}\PY{n}{curr}\PY{p}{[}\PY{l+s+s1}{\PYZsq{}}\PY{l+s+s1}{y}\PY{l+s+s1}{\PYZsq{}}\PY{p}{]}\PY{p}{)}
    \PY{n}{tZ} \PY{o}{=} \PY{n}{Equals}\PY{p}{(}\PY{n}{prox}\PY{p}{[}\PY{l+s+s1}{\PYZsq{}}\PY{l+s+s1}{z}\PY{l+s+s1}{\PYZsq{}}\PY{p}{]}\PY{p}{,}\PY{n}{curr}\PY{p}{[}\PY{l+s+s1}{\PYZsq{}}\PY{l+s+s1}{z}\PY{l+s+s1}{\PYZsq{}}\PY{p}{]}\PY{p}{)}
    \PY{n}{tPc} \PY{o}{=} \PY{n}{Equals}\PY{p}{(}\PY{n}{prox}\PY{p}{[}\PY{l+s+s1}{\PYZsq{}}\PY{l+s+s1}{pc}\PY{l+s+s1}{\PYZsq{}}\PY{p}{]}\PY{p}{,}\PY{n}{Int}\PY{p}{(}\PY{l+m+mi}{2}\PY{p}{)}\PY{p}{)}
    \PY{k}{return} \PY{n}{And}\PY{p}{(}\PY{n}{tPcTwo}\PY{p}{,}\PY{n}{tX}\PY{p}{,}\PY{n}{tY}\PY{p}{,}\PY{n}{tZ}\PY{p}{,}\PY{n}{tPc}\PY{p}{)}

\PY{k}{def} \PY{n+nf}{trans}\PY{p}{(}\PY{n}{curr}\PY{p}{,}\PY{n}{prox}\PY{p}{,}\PY{n}{n}\PY{p}{)}\PY{p}{:}
    \PY{n}{tToStop} \PY{o}{=} \PY{n}{tStop}\PY{p}{(}\PY{n}{curr}\PY{p}{,}\PY{n}{prox}\PY{p}{,}\PY{n}{n}\PY{p}{)}
    \PY{n}{tToEven}   \PY{o}{=} \PY{n}{tEven}\PY{p}{(}\PY{n}{curr}\PY{p}{,}\PY{n}{prox}\PY{p}{,}\PY{n}{n}\PY{p}{)}
    \PY{n}{tToError}  \PY{o}{=} \PY{n}{tError}\PY{p}{(}\PY{n}{curr}\PY{p}{,}\PY{n}{prox}\PY{p}{,}\PY{n}{n}\PY{p}{)}
    \PY{n}{tToEndError} \PY{o}{=} \PY{n}{tEndError}\PY{p}{(}\PY{n}{curr}\PY{p}{,}\PY{n}{prox}\PY{p}{)}
    \PY{n}{tToOdd}    \PY{o}{=} \PY{n}{tOdd}\PY{p}{(}\PY{n}{curr}\PY{p}{,}\PY{n}{prox}\PY{p}{,}\PY{n}{n}\PY{p}{)}
    \PY{n}{tToEnd}    \PY{o}{=} \PY{n}{tEnd}\PY{p}{(}\PY{n}{curr}\PY{p}{,}\PY{n}{prox}\PY{p}{)}
    \PY{k}{return} \PY{n}{Or}\PY{p}{(}\PY{n}{tToStop}\PY{p}{,}\PY{n}{tToEven}\PY{p}{,}\PY{n}{tToError}\PY{p}{,}\PY{n}{tToEndError}\PY{p}{,}\PY{n}{tToOdd}\PY{p}{,}\PY{n}{tToEnd}\PY{p}{)}
\end{Verbatim}
\end{tcolorbox}

    Função que usa \(\textit{SMT solver}\) para gerar um possível traço de
execução do programa, imprimindo, para cada estado, as variáveis x,y,z e
o program counter e função que auxiliar na conversão das variáveis para
inteiro.

    \begin{tcolorbox}[breakable, size=fbox, boxrule=1pt, pad at break*=1mm,colback=cellbackground, colframe=cellborder]
\prompt{In}{incolor}{5}{\boxspacing}
\begin{Verbatim}[commandchars=\\\{\}]
\PY{k}{def} \PY{n+nf}{toInt}\PY{p}{(}\PY{n}{s}\PY{p}{)}\PY{p}{:}
    \PY{k}{return} \PY{n+nb}{sum}\PY{p}{(}\PY{p}{[}\PY{n+nb}{int}\PY{p}{(}\PY{n}{x}\PY{p}{)}\PY{o}{*}\PY{l+m+mi}{2}\PY{o}{*}\PY{o}{*}\PY{p}{(}\PY{n+nb}{len}\PY{p}{(}\PY{n}{s}\PY{p}{)}\PY{o}{\PYZhy{}}\PY{n}{i}\PY{o}{\PYZhy{}}\PY{l+m+mi}{1}\PY{p}{)} \PY{k}{for} \PY{n}{i}\PY{p}{,}\PY{n}{x} \PY{o+ow}{in} \PY{p}{(}\PY{n+nb}{enumerate}\PY{p}{(}\PY{n}{s}\PY{p}{)}\PY{p}{)}\PY{p}{]}\PY{p}{)}
\end{Verbatim}
\end{tcolorbox}

    \begin{tcolorbox}[breakable, size=fbox, boxrule=1pt, pad at break*=1mm,colback=cellbackground, colframe=cellborder]
\prompt{In}{incolor}{6}{\boxspacing}
\begin{Verbatim}[commandchars=\\\{\}]
\PY{k}{def} \PY{n+nf}{resolve}\PY{p}{(}\PY{n}{a}\PY{p}{,}\PY{n}{b}\PY{p}{,}\PY{n}{n}\PY{p}{,}\PY{n}{k}\PY{p}{)}\PY{p}{:}
    \PY{k}{with} \PY{n}{Solver}\PY{p}{(}\PY{n}{name}\PY{o}{=}\PY{l+s+s2}{\PYZdq{}}\PY{l+s+s2}{msat}\PY{l+s+s2}{\PYZdq{}}\PY{p}{)} \PY{k}{as} \PY{n}{s}\PY{p}{:}
            \PY{c+c1}{\PYZsh{} cria k copias do estado}
            \PY{n}{trace} \PY{o}{=} \PY{p}{[}\PY{n}{declare}\PY{p}{(}\PY{n}{i}\PY{p}{,}\PY{n}{n}\PY{p}{)} \PY{k}{for} \PY{n}{i} \PY{o+ow}{in} \PY{n+nb}{range}\PY{p}{(}\PY{n}{k}\PY{p}{)}\PY{p}{]}
            \PY{c+c1}{\PYZsh{}print(trace)}
            \PY{c+c1}{\PYZsh{} criar o traço}
            \PY{n}{s}\PY{o}{.}\PY{n}{add\PYZus{}assertion}\PY{p}{(}\PY{n}{init}\PY{p}{(}\PY{n}{trace}\PY{p}{[}\PY{l+m+mi}{0}\PY{p}{]}\PY{p}{,}\PY{n}{a}\PY{p}{,}\PY{n}{b}\PY{p}{,}\PY{n}{n}\PY{p}{)}\PY{p}{)}
            \PY{c+c1}{\PYZsh{}print(init(trace[0]))}
            \PY{k}{for} \PY{n}{i} \PY{o+ow}{in} \PY{n+nb}{range}\PY{p}{(}\PY{n}{k}\PY{o}{\PYZhy{}}\PY{l+m+mi}{1}\PY{p}{)}\PY{p}{:}
                \PY{n}{s}\PY{o}{.}\PY{n}{add\PYZus{}assertion}\PY{p}{(}\PY{n}{trans}\PY{p}{(}\PY{n}{trace}\PY{p}{[}\PY{n}{i}\PY{p}{]}\PY{p}{,} \PY{n}{trace}\PY{p}{[}\PY{n}{i}\PY{o}{+}\PY{l+m+mi}{1}\PY{p}{]}\PY{p}{,}\PY{n}{n}\PY{p}{)}\PY{p}{)}

            \PY{k}{if} \PY{n}{s}\PY{o}{.}\PY{n}{solve}\PY{p}{(}\PY{p}{)}\PY{p}{:}
                \PY{k}{for} \PY{n}{i} \PY{o+ow}{in} \PY{n+nb}{range}\PY{p}{(}\PY{n}{k}\PY{p}{)}\PY{p}{:}
                    \PY{n+nb}{print}\PY{p}{(}\PY{n}{i}\PY{p}{)}
                    \PY{n+nb}{print}\PY{p}{(}\PY{l+s+s2}{\PYZdq{}}\PY{l+s+s2}{pc=}\PY{l+s+s2}{\PYZdq{}}\PY{p}{,} \PY{n}{pc} \PY{o}{:=} \PY{n}{s}\PY{o}{.}\PY{n}{get\PYZus{}value}\PY{p}{(}\PY{n}{trace}\PY{p}{[}\PY{n}{i}\PY{p}{]}\PY{p}{[}\PY{l+s+s1}{\PYZsq{}}\PY{l+s+s1}{pc}\PY{l+s+s1}{\PYZsq{}}\PY{p}{]}\PY{p}{)}\PY{o}{.}\PY{n}{constant\PYZus{}value}\PY{p}{(}\PY{p}{)}\PY{p}{)}
                    \PY{n+nb}{print}\PY{p}{(}\PY{l+s+s2}{\PYZdq{}}\PY{l+s+s2}{x=}\PY{l+s+s2}{\PYZdq{}}\PY{p}{,} \PY{n}{toInt}\PY{p}{(}\PY{n}{s}\PY{o}{.}\PY{n}{get\PYZus{}value}\PY{p}{(}\PY{n}{trace}\PY{p}{[}\PY{n}{i}\PY{p}{]}\PY{p}{[}\PY{l+s+s1}{\PYZsq{}}\PY{l+s+s1}{x}\PY{l+s+s1}{\PYZsq{}}\PY{p}{]}\PY{p}{)}\PY{o}{.}\PY{n}{bv\PYZus{}str}\PY{p}{(}\PY{p}{)}\PY{p}{)}\PY{p}{)}
                    \PY{n+nb}{print}\PY{p}{(}\PY{l+s+s2}{\PYZdq{}}\PY{l+s+s2}{y=}\PY{l+s+s2}{\PYZdq{}}\PY{p}{,} \PY{n}{toInt}\PY{p}{(}\PY{n}{s}\PY{o}{.}\PY{n}{get\PYZus{}value}\PY{p}{(}\PY{n}{trace}\PY{p}{[}\PY{n}{i}\PY{p}{]}\PY{p}{[}\PY{l+s+s1}{\PYZsq{}}\PY{l+s+s1}{y}\PY{l+s+s1}{\PYZsq{}}\PY{p}{]}\PY{p}{)}\PY{o}{.}\PY{n}{bv\PYZus{}str}\PY{p}{(}\PY{p}{)}\PY{p}{)}\PY{p}{)}
                    \PY{n+nb}{print}\PY{p}{(}\PY{l+s+s2}{\PYZdq{}}\PY{l+s+s2}{z=}\PY{l+s+s2}{\PYZdq{}}\PY{p}{,} \PY{n}{toInt}\PY{p}{(}\PY{n}{s}\PY{o}{.}\PY{n}{get\PYZus{}value}\PY{p}{(}\PY{n}{trace}\PY{p}{[}\PY{n}{i}\PY{p}{]}\PY{p}{[}\PY{l+s+s1}{\PYZsq{}}\PY{l+s+s1}{z}\PY{l+s+s1}{\PYZsq{}}\PY{p}{]}\PY{p}{)}\PY{o}{.}\PY{n}{bv\PYZus{}str}\PY{p}{(}\PY{p}{)}\PY{p}{)}\PY{p}{)}
                    \PY{n+nb}{print}\PY{p}{(}\PY{p}{)}
                    \PY{k}{if} \PY{n}{pc} \PY{o+ow}{in} \PY{p}{(}\PY{l+m+mi}{1}\PY{p}{,}\PY{l+m+mi}{2}\PY{p}{)}\PY{p}{:}
                        \PY{k}{break}
            \PY{k}{else}\PY{p}{:}
                \PY{n+nb}{print}\PY{p}{(}\PY{l+s+s1}{\PYZsq{}}\PY{l+s+s1}{Não foi possível resolver}\PY{l+s+s1}{\PYZsq{}}\PY{p}{)}
\end{Verbatim}
\end{tcolorbox}

    O invariante \(P≡(x∗y+z=a∗b)\) como função \textbf{invariant(state,a,b)}
e a função de ordem superior
\textbf{bmc\_always(declare,init,trans,inv,K,a,b,n)} que testa se o
invariante é verificado para traços de tamanho maximo \(k\).

    \begin{tcolorbox}[breakable, size=fbox, boxrule=1pt, pad at break*=1mm,colback=cellbackground, colframe=cellborder]
\prompt{In}{incolor}{7}{\boxspacing}
\begin{Verbatim}[commandchars=\\\{\}]
\PY{k}{def} \PY{n+nf}{invariant}\PY{p}{(}\PY{n}{state}\PY{p}{,}\PY{n}{a}\PY{p}{,}\PY{n}{b}\PY{p}{)}\PY{p}{:}
    \PY{k}{return} \PY{n}{Equals}\PY{p}{(}\PY{n}{BVAdd}\PY{p}{(}\PY{n}{BVMul}\PY{p}{(}\PY{n}{state}\PY{p}{[}\PY{l+s+s1}{\PYZsq{}}\PY{l+s+s1}{x}\PY{l+s+s1}{\PYZsq{}}\PY{p}{]}\PY{p}{,} \PY{n}{state}\PY{p}{[}\PY{l+s+s1}{\PYZsq{}}\PY{l+s+s1}{y}\PY{l+s+s1}{\PYZsq{}}\PY{p}{]}\PY{p}{)}\PY{p}{,} \PY{n}{state}\PY{p}{[}\PY{l+s+s1}{\PYZsq{}}\PY{l+s+s1}{z}\PY{l+s+s1}{\PYZsq{}}\PY{p}{]}\PY{p}{)}\PY{p}{,} \PY{n}{BVMul}\PY{p}{(}\PY{n}{BV}\PY{p}{(}\PY{n}{a}\PY{p}{,} \PY{n}{n}\PY{o}{+}\PY{l+m+mi}{1}\PY{p}{)}\PY{p}{,} \PY{n}{BV}\PY{p}{(}\PY{n}{b}\PY{p}{,} \PY{n}{n}\PY{o}{+}\PY{l+m+mi}{1}\PY{p}{)}\PY{p}{)}\PY{p}{)}


\PY{k}{def} \PY{n+nf}{bmc\PYZus{}always}\PY{p}{(}\PY{n}{declare}\PY{p}{,}\PY{n}{init}\PY{p}{,}\PY{n}{trans}\PY{p}{,}\PY{n}{inv}\PY{p}{,}\PY{n}{K}\PY{p}{,}\PY{n}{n}\PY{p}{,}\PY{n}{a}\PY{p}{,}\PY{n}{b}\PY{p}{)}\PY{p}{:}
    \PY{k}{for} \PY{n}{k} \PY{o+ow}{in} \PY{n+nb}{range}\PY{p}{(}\PY{l+m+mi}{1}\PY{p}{,}\PY{n}{K}\PY{o}{+}\PY{l+m+mi}{1}\PY{p}{)}\PY{p}{:}
        \PY{k}{with} \PY{n}{Solver}\PY{p}{(}\PY{n}{name}\PY{o}{=}\PY{l+s+s2}{\PYZdq{}}\PY{l+s+s2}{z3}\PY{l+s+s2}{\PYZdq{}}\PY{p}{)} \PY{k}{as} \PY{n}{s}\PY{p}{:}

            \PY{n}{trace} \PY{o}{=} \PY{p}{[}\PY{n}{declare}\PY{p}{(}\PY{n}{i}\PY{p}{,}\PY{n}{n}\PY{p}{)} \PY{k}{for} \PY{n}{i} \PY{o+ow}{in} \PY{n+nb}{range}\PY{p}{(}\PY{n}{k}\PY{p}{)}\PY{p}{]}
    
            \PY{n}{s}\PY{o}{.}\PY{n}{add\PYZus{}assertion}\PY{p}{(}\PY{n}{init}\PY{p}{(}\PY{n}{trace}\PY{p}{[}\PY{l+m+mi}{0}\PY{p}{]}\PY{p}{,}\PY{n}{a}\PY{p}{,}\PY{n}{b}\PY{p}{,}\PY{n}{n}\PY{p}{)}\PY{p}{)}
            \PY{k}{for} \PY{n}{i} \PY{o+ow}{in} \PY{n+nb}{range}\PY{p}{(}\PY{n}{k}\PY{o}{\PYZhy{}}\PY{l+m+mi}{1}\PY{p}{)}\PY{p}{:}
                \PY{n}{s}\PY{o}{.}\PY{n}{add\PYZus{}assertion}\PY{p}{(}\PY{n}{trans}\PY{p}{(}\PY{n}{trace}\PY{p}{[}\PY{n}{i}\PY{p}{]}\PY{p}{,} \PY{n}{trace}\PY{p}{[}\PY{n}{i}\PY{o}{+}\PY{l+m+mi}{1}\PY{p}{]}\PY{p}{,}\PY{n}{n}\PY{p}{)}\PY{p}{)}
            
            \PY{n}{s}\PY{o}{.}\PY{n}{add\PYZus{}assertion}\PY{p}{(}\PY{n}{Not}\PY{p}{(}\PY{n}{inv}\PY{p}{(}\PY{n}{trace}\PY{p}{[}\PY{n}{k}\PY{o}{\PYZhy{}}\PY{l+m+mi}{1}\PY{p}{]}\PY{p}{,}\PY{n}{a}\PY{p}{,}\PY{n}{b}\PY{p}{)}\PY{p}{)}\PY{p}{)}
            \PY{k}{if} \PY{n}{s}\PY{o}{.}\PY{n}{solve}\PY{p}{(}\PY{p}{)}\PY{p}{:}
                \PY{k}{for} \PY{n}{i} \PY{o+ow}{in} \PY{n+nb}{range}\PY{p}{(}\PY{n}{k}\PY{p}{)}\PY{p}{:}
                    \PY{k}{for} \PY{n}{v} \PY{o+ow}{in} \PY{n}{trace}\PY{p}{[}\PY{l+m+mi}{0}\PY{p}{]}\PY{p}{:}
                        \PY{n+nb}{print}\PY{p}{(}\PY{n}{v}\PY{p}{,}\PY{l+s+s1}{\PYZsq{}}\PY{l+s+s1}{=}\PY{l+s+s1}{\PYZsq{}}\PY{p}{,}\PY{n}{s}\PY{o}{.}\PY{n}{get\PYZus{}value}\PY{p}{(}\PY{n}{trace}\PY{p}{[}\PY{l+m+mi}{0}\PY{p}{]}\PY{p}{[}\PY{n}{v}\PY{p}{]}\PY{p}{)}\PY{p}{)}
                \PY{k}{return}
            
    \PY{n+nb}{print}\PY{p}{(}\PY{l+s+s2}{\PYZdq{}}\PY{l+s+s2}{A propriedade é válida para traços de tamanho até }\PY{l+s+s2}{\PYZdq{}} \PY{o}{+} \PY{n+nb}{str}\PY{p}{(}\PY{n}{k}\PY{p}{)}\PY{p}{)}
\end{Verbatim}
\end{tcolorbox}

    \hypertarget{exemplos-e-testes-de-aplicauxe7uxe3o}{%
\subsubsection{Exemplos e Testes de
Aplicação}\label{exemplos-e-testes-de-aplicauxe7uxe3o}}

    \hypertarget{exemplo-1}{%
\paragraph{Exemplo 1}\label{exemplo-1}}

    \begin{tcolorbox}[breakable, size=fbox, boxrule=1pt, pad at break*=1mm,colback=cellbackground, colframe=cellborder]
\prompt{In}{incolor}{8}{\boxspacing}
\begin{Verbatim}[commandchars=\\\{\}]
\PY{n}{n} \PY{o}{=} \PY{l+m+mi}{7}
\PY{n}{a} \PY{o}{=} \PY{l+m+mi}{4}
\PY{n}{b} \PY{o}{=} \PY{l+m+mi}{3}
\PY{n}{k} \PY{o}{=} \PY{n}{n}\PY{o}{+}\PY{l+m+mi}{1}

\PY{n+nb}{print}\PY{p}{(}\PY{l+s+s1}{\PYZsq{}}\PY{l+s+s1}{x = }\PY{l+s+s1}{\PYZsq{}}\PY{p}{,} \PY{n}{BV}\PY{p}{(}\PY{n}{a}\PY{p}{,}\PY{n}{n}\PY{p}{)}\PY{o}{.}\PY{n}{bv\PYZus{}str}\PY{p}{(}\PY{p}{)}\PY{p}{)}
\PY{n+nb}{print}\PY{p}{(}\PY{l+s+s1}{\PYZsq{}}\PY{l+s+s1}{y = }\PY{l+s+s1}{\PYZsq{}}\PY{p}{,} \PY{n}{BV}\PY{p}{(}\PY{n}{b}\PY{p}{,}\PY{n}{n}\PY{p}{)}\PY{o}{.}\PY{n}{bv\PYZus{}str}\PY{p}{(}\PY{p}{)}\PY{p}{)}
\end{Verbatim}
\end{tcolorbox}

    \begin{Verbatim}[commandchars=\\\{\}]
x =  0000100
y =  0000011
    \end{Verbatim}

    \hypertarget{resoluuxe7uxe3o-do-exemplo-1}{%
\paragraph{Resolução do Exemplo 1}\label{resoluuxe7uxe3o-do-exemplo-1}}

    Este exemplo é apenas uma mostra de uma multiplicação básica.

    \begin{tcolorbox}[breakable, size=fbox, boxrule=1pt, pad at break*=1mm,colback=cellbackground, colframe=cellborder]
\prompt{In}{incolor}{9}{\boxspacing}
\begin{Verbatim}[commandchars=\\\{\}]
\PY{n}{resolve}\PY{p}{(}\PY{n}{a}\PY{p}{,}\PY{n}{b}\PY{p}{,}\PY{n}{n}\PY{p}{,}\PY{n}{k}\PY{p}{)}
\end{Verbatim}
\end{tcolorbox}

    \begin{Verbatim}[commandchars=\\\{\}]
0
pc= 0
x= 4
y= 3
z= 0

1
pc= 0
x= 4
y= 2
z= 4

2
pc= 0
x= 8
y= 1
z= 4

3
pc= 0
x= 8
y= 0
z= 12

4
pc= 1
x= 8
y= 0
z= 12

    \end{Verbatim}

    \begin{tcolorbox}[breakable, size=fbox, boxrule=1pt, pad at break*=1mm,colback=cellbackground, colframe=cellborder]
\prompt{In}{incolor}{10}{\boxspacing}
\begin{Verbatim}[commandchars=\\\{\}]
\PY{n}{bmc\PYZus{}always}\PY{p}{(}\PY{n}{declare}\PY{p}{,}\PY{n}{init}\PY{p}{,}\PY{n}{trans}\PY{p}{,}\PY{n}{invariant}\PY{p}{,}\PY{n}{k}\PY{p}{,}\PY{n}{n}\PY{p}{,}\PY{n}{a}\PY{p}{,}\PY{n}{b}\PY{p}{)}
\end{Verbatim}
\end{tcolorbox}

    \begin{Verbatim}[commandchars=\\\{\}]
A propriedade é válida para traços de tamanho até 8
    \end{Verbatim}

    \hypertarget{exemplo-2}{%
\paragraph{Exemplo 2}\label{exemplo-2}}

    Neste exemplo procuramos apresentar um dos piores casos em termos de
transições de estado.

    \begin{tcolorbox}[breakable, size=fbox, boxrule=1pt, pad at break*=1mm,colback=cellbackground, colframe=cellborder]
\prompt{In}{incolor}{8}{\boxspacing}
\begin{Verbatim}[commandchars=\\\{\}]
\PY{n}{n} \PY{o}{=} \PY{l+m+mi}{6}
\PY{n}{a} \PY{o}{=} \PY{l+m+mi}{7}
\PY{n}{b} \PY{o}{=} \PY{l+m+mi}{7}
\PY{n}{k} \PY{o}{=} \PY{n}{n}\PY{o}{+}\PY{l+m+mi}{1}
\PY{n}{k\PYZus{}inv} \PY{o}{=} \PY{l+m+mi}{1}

\PY{n+nb}{print}\PY{p}{(}\PY{l+s+s1}{\PYZsq{}}\PY{l+s+s1}{x = }\PY{l+s+s1}{\PYZsq{}}\PY{p}{,} \PY{n}{BV}\PY{p}{(}\PY{n}{a}\PY{p}{,}\PY{n}{n}\PY{p}{)}\PY{o}{.}\PY{n}{bv\PYZus{}str}\PY{p}{(}\PY{p}{)}\PY{p}{)}
\PY{n+nb}{print}\PY{p}{(}\PY{l+s+s1}{\PYZsq{}}\PY{l+s+s1}{y = }\PY{l+s+s1}{\PYZsq{}}\PY{p}{,} \PY{n}{BV}\PY{p}{(}\PY{n}{b}\PY{p}{,}\PY{n}{n}\PY{p}{)}\PY{o}{.}\PY{n}{bv\PYZus{}str}\PY{p}{(}\PY{p}{)}\PY{p}{)}
\end{Verbatim}
\end{tcolorbox}

    \begin{Verbatim}[commandchars=\\\{\}]
x =  000111
y =  000111
    \end{Verbatim}

    \hypertarget{resoluuxe7uxe3o-do-exemplo-2}{%
\paragraph{Resolução do Exemplo 2}\label{resoluuxe7uxe3o-do-exemplo-2}}

    \begin{tcolorbox}[breakable, size=fbox, boxrule=1pt, pad at break*=1mm,colback=cellbackground, colframe=cellborder]
\prompt{In}{incolor}{9}{\boxspacing}
\begin{Verbatim}[commandchars=\\\{\}]
\PY{n}{resolve}\PY{p}{(}\PY{n}{a}\PY{p}{,}\PY{n}{b}\PY{p}{,}\PY{n}{n}\PY{p}{,}\PY{n}{k}\PY{p}{)}
\end{Verbatim}
\end{tcolorbox}

    \begin{Verbatim}[commandchars=\\\{\}]
0
pc= 0
x= 7
y= 7
z= 0

1
pc= 0
x= 7
y= 6
z= 7

2
pc= 0
x= 14
y= 3
z= 7

3
pc= 0
x= 14
y= 2
z= 21

4
pc= 0
x= 28
y= 1
z= 21

5
pc= 0
x= 28
y= 0
z= 49

6
pc= 1
x= 28
y= 0
z= 49

    \end{Verbatim}

    \begin{tcolorbox}[breakable, size=fbox, boxrule=1pt, pad at break*=1mm,colback=cellbackground, colframe=cellborder]
\prompt{In}{incolor}{10}{\boxspacing}
\begin{Verbatim}[commandchars=\\\{\}]
\PY{n}{bmc\PYZus{}always}\PY{p}{(}\PY{n}{declare}\PY{p}{,}\PY{n}{init}\PY{p}{,}\PY{n}{trans}\PY{p}{,}\PY{n}{invariant}\PY{p}{,}\PY{n}{k}\PY{p}{,}\PY{n}{n}\PY{p}{,}\PY{n}{a}\PY{p}{,}\PY{n}{b}\PY{p}{)}
\end{Verbatim}
\end{tcolorbox}

    \begin{Verbatim}[commandchars=\\\{\}]
A propriedade é válida para traços de tamanho até 7
    \end{Verbatim}

    \hypertarget{exemplo-3}{%
\paragraph{Exemplo 3}\label{exemplo-3}}

    Neste exemplo procuramos mostrar a optimização feita de modo a que sejam
efetuadas o menor número de transições possiveis.

    \begin{tcolorbox}[breakable, size=fbox, boxrule=1pt, pad at break*=1mm,colback=cellbackground, colframe=cellborder]
\prompt{In}{incolor}{8}{\boxspacing}
\begin{Verbatim}[commandchars=\\\{\}]
\PY{n}{n} \PY{o}{=} \PY{l+m+mi}{15}
\PY{n}{a} \PY{o}{=} \PY{l+m+mi}{1}
\PY{n}{b} \PY{o}{=} \PY{l+m+mi}{32767}
\PY{n}{k} \PY{o}{=} \PY{n}{n}\PY{o}{+}\PY{l+m+mi}{1}

\PY{n+nb}{print}\PY{p}{(}\PY{l+s+s1}{\PYZsq{}}\PY{l+s+s1}{x = }\PY{l+s+s1}{\PYZsq{}}\PY{p}{,} \PY{n}{BV}\PY{p}{(}\PY{n}{a}\PY{p}{,}\PY{n}{n}\PY{p}{)}\PY{o}{.}\PY{n}{bv\PYZus{}str}\PY{p}{(}\PY{p}{)}\PY{p}{)}
\PY{n+nb}{print}\PY{p}{(}\PY{l+s+s1}{\PYZsq{}}\PY{l+s+s1}{y = }\PY{l+s+s1}{\PYZsq{}}\PY{p}{,} \PY{n}{BV}\PY{p}{(}\PY{n}{b}\PY{p}{,}\PY{n}{n}\PY{p}{)}\PY{o}{.}\PY{n}{bv\PYZus{}str}\PY{p}{(}\PY{p}{)}\PY{p}{)}
\end{Verbatim}
\end{tcolorbox}

    \begin{Verbatim}[commandchars=\\\{\}]
x =  000000000000001
y =  111111111111111
    \end{Verbatim}

    \hypertarget{resoluuxe7uxe3o-do-exemplo-3}{%
\paragraph{Resolução do exemplo 3}\label{resoluuxe7uxe3o-do-exemplo-3}}

    \begin{tcolorbox}[breakable, size=fbox, boxrule=1pt, pad at break*=1mm,colback=cellbackground, colframe=cellborder]
\prompt{In}{incolor}{9}{\boxspacing}
\begin{Verbatim}[commandchars=\\\{\}]
\PY{n}{resolve}\PY{p}{(}\PY{n}{a}\PY{p}{,}\PY{n}{b}\PY{p}{,}\PY{n}{n}\PY{p}{,}\PY{n}{k}\PY{p}{)}
\end{Verbatim}
\end{tcolorbox}

    \begin{Verbatim}[commandchars=\\\{\}]
0
pc= 0
x= 32767
y= 1
z= 0

1
pc= 0
x= 32767
y= 0
z= 32767

2
pc= 1
x= 32767
y= 0
z= 32767

    \end{Verbatim}

    \begin{tcolorbox}[breakable, size=fbox, boxrule=1pt, pad at break*=1mm,colback=cellbackground, colframe=cellborder]
\prompt{In}{incolor}{10}{\boxspacing}
\begin{Verbatim}[commandchars=\\\{\}]
\PY{n}{bmc\PYZus{}always}\PY{p}{(}\PY{n}{declare}\PY{p}{,}\PY{n}{init}\PY{p}{,}\PY{n}{trans}\PY{p}{,}\PY{n}{invariant}\PY{p}{,}\PY{n}{k}\PY{p}{,}\PY{n}{n}\PY{p}{,}\PY{n}{a}\PY{p}{,}\PY{n}{b}\PY{p}{)}
\end{Verbatim}
\end{tcolorbox}

    \begin{Verbatim}[commandchars=\\\{\}]
A propriedade é válida para traços de tamanho até 16
    \end{Verbatim}

    \hypertarget{exemplo-4}{%
\paragraph{Exemplo 4}\label{exemplo-4}}

    Neste exemplo procuramos mostrar um caso de \(overflow\).

    \begin{tcolorbox}[breakable, size=fbox, boxrule=1pt, pad at break*=1mm,colback=cellbackground, colframe=cellborder]
\prompt{In}{incolor}{7}{\boxspacing}
\begin{Verbatim}[commandchars=\\\{\}]
\PY{n}{n} \PY{o}{=} \PY{l+m+mi}{32}
\PY{n}{a} \PY{o}{=} \PY{l+m+mi}{65535}
\PY{n}{b} \PY{o}{=} \PY{l+m+mi}{131069}
\PY{n}{k} \PY{o}{=} \PY{n}{n}\PY{o}{+}\PY{l+m+mi}{1}

\PY{n+nb}{print}\PY{p}{(}\PY{l+s+s1}{\PYZsq{}}\PY{l+s+s1}{x = }\PY{l+s+s1}{\PYZsq{}}\PY{p}{,} \PY{n}{BV}\PY{p}{(}\PY{n}{a}\PY{p}{,}\PY{n}{n}\PY{p}{)}\PY{o}{.}\PY{n}{bv\PYZus{}str}\PY{p}{(}\PY{p}{)}\PY{p}{)}
\PY{n+nb}{print}\PY{p}{(}\PY{l+s+s1}{\PYZsq{}}\PY{l+s+s1}{y = }\PY{l+s+s1}{\PYZsq{}}\PY{p}{,} \PY{n}{BV}\PY{p}{(}\PY{n}{b}\PY{p}{,}\PY{n}{n}\PY{p}{)}\PY{o}{.}\PY{n}{bv\PYZus{}str}\PY{p}{(}\PY{p}{)}\PY{p}{)}
\end{Verbatim}
\end{tcolorbox}

    \begin{Verbatim}[commandchars=\\\{\}]
x =  00000000000000001111111111111111
y =  00000000000000011111111111111101
    \end{Verbatim}

    \hypertarget{resoluuxe7uxe3o-do-exemplo-4}{%
\paragraph{Resolução do exemplo 4}\label{resoluuxe7uxe3o-do-exemplo-4}}

    \begin{tcolorbox}[breakable, size=fbox, boxrule=1pt, pad at break*=1mm,colback=cellbackground, colframe=cellborder]
\prompt{In}{incolor}{8}{\boxspacing}
\begin{Verbatim}[commandchars=\\\{\}]
\PY{n}{resolve}\PY{p}{(}\PY{n}{a}\PY{p}{,}\PY{n}{b}\PY{p}{,}\PY{n}{n}\PY{p}{,}\PY{n}{k}\PY{p}{)}
\end{Verbatim}
\end{tcolorbox}

    \begin{Verbatim}[commandchars=\\\{\}]
0
pc= 0
x= 131069
y= 65535
z= 0

1
pc= 0
x= 131069
y= 65534
z= 131069

2
pc= 0
x= 262138
y= 32767
z= 131069

3
pc= 0
x= 262138
y= 32766
z= 393207

4
pc= 0
x= 524276
y= 16383
z= 393207

5
pc= 0
x= 524276
y= 16382
z= 917483

6
pc= 0
x= 1048552
y= 8191
z= 917483

7
pc= 0
x= 1048552
y= 8190
z= 1966035

8
pc= 0
x= 2097104
y= 4095
z= 1966035

9
pc= 0
x= 2097104
y= 4094
z= 4063139

10
pc= 0
x= 4194208
y= 2047
z= 4063139

11
pc= 0
x= 4194208
y= 2046
z= 8257347

12
pc= 0
x= 8388416
y= 1023
z= 8257347

13
pc= 0
x= 8388416
y= 1022
z= 16645763

14
pc= 0
x= 16776832
y= 511
z= 16645763

15
pc= 0
x= 16776832
y= 510
z= 33422595

16
pc= 0
x= 33553664
y= 255
z= 33422595

17
pc= 0
x= 33553664
y= 254
z= 66976259

18
pc= 0
x= 67107328
y= 127
z= 66976259

19
pc= 0
x= 67107328
y= 126
z= 134083587

20
pc= 0
x= 134214656
y= 63
z= 134083587

21
pc= 0
x= 134214656
y= 62
z= 268298243

22
pc= 0
x= 268429312
y= 31
z= 268298243

23
pc= 0
x= 268429312
y= 30
z= 536727555

24
pc= 0
x= 536858624
y= 15
z= 536727555

25
pc= 0
x= 536858624
y= 14
z= 1073586179

26
pc= 0
x= 1073717248
y= 7
z= 1073586179

27
pc= 0
x= 1073717248
y= 6
z= 2147303427

28
pc= 0
x= 2147434496
y= 3
z= 2147303427

29
pc= 0
x= 2147434496
y= 2
z= 4294737923

30
pc= 0
x= 4294868992
y= 1
z= 4294737923

31
pc= 0
x= 4294868992
y= 0
z= 8589606915

32
pc= 2
x= 4294868992
y= 0
z= 8589606915

    \end{Verbatim}

    \begin{tcolorbox}[breakable, size=fbox, boxrule=1pt, pad at break*=1mm,colback=cellbackground, colframe=cellborder]
\prompt{In}{incolor}{11}{\boxspacing}
\begin{Verbatim}[commandchars=\\\{\}]
\PY{n}{bmc\PYZus{}always}\PY{p}{(}\PY{n}{declare}\PY{p}{,}\PY{n}{init}\PY{p}{,}\PY{n}{trans}\PY{p}{,}\PY{n}{invariant}\PY{p}{,}\PY{n}{k}\PY{p}{,}\PY{n}{n}\PY{p}{,}\PY{n}{a}\PY{p}{,}\PY{n}{b}\PY{p}{)}
\end{Verbatim}
\end{tcolorbox}

    \begin{Verbatim}[commandchars=\\\{\}]
A propriedade é válida para traços de tamanho até 33
    \end{Verbatim}

    \hypertarget{exemplo-5}{%
\paragraph{Exemplo 5}\label{exemplo-5}}

    Este exemplo serve para ser possivel efetuar testes repetidos com
variaveis aleatórias.

    \begin{tcolorbox}[breakable, size=fbox, boxrule=1pt, pad at break*=1mm,colback=cellbackground, colframe=cellborder]
\prompt{In}{incolor}{19}{\boxspacing}
\begin{Verbatim}[commandchars=\\\{\}]
\PY{n}{n} \PY{o}{=} \PY{l+m+mi}{32}
\PY{n}{a} \PY{o}{=} \PY{n}{rn}\PY{o}{.}\PY{n}{randrange}\PY{p}{(}\PY{l+m+mi}{1}\PY{p}{,} \PY{l+m+mi}{2}\PY{o}{*}\PY{o}{*}\PY{p}{(}\PY{n}{n}\PY{p}{)}\PY{p}{)}
\PY{n}{b} \PY{o}{=} \PY{n}{rn}\PY{o}{.}\PY{n}{randrange}\PY{p}{(}\PY{l+m+mi}{1}\PY{p}{,} \PY{l+m+mi}{2}\PY{o}{*}\PY{o}{*}\PY{p}{(}\PY{n}{n}\PY{p}{)}\PY{p}{)}
\PY{n}{k} \PY{o}{=} \PY{n}{n}\PY{o}{+}\PY{l+m+mi}{1}

\PY{n+nb}{print}\PY{p}{(}\PY{l+s+s1}{\PYZsq{}}\PY{l+s+s1}{x = }\PY{l+s+s1}{\PYZsq{}}\PY{p}{,} \PY{n}{BV}\PY{p}{(}\PY{n}{a}\PY{p}{,}\PY{n}{n}\PY{p}{)}\PY{o}{.}\PY{n}{bv\PYZus{}str}\PY{p}{(}\PY{p}{)}\PY{p}{)}
\PY{n+nb}{print}\PY{p}{(}\PY{l+s+s1}{\PYZsq{}}\PY{l+s+s1}{y = }\PY{l+s+s1}{\PYZsq{}}\PY{p}{,} \PY{n}{BV}\PY{p}{(}\PY{n}{b}\PY{p}{,}\PY{n}{n}\PY{p}{)}\PY{o}{.}\PY{n}{bv\PYZus{}str}\PY{p}{(}\PY{p}{)}\PY{p}{)}
\end{Verbatim}
\end{tcolorbox}

    \begin{Verbatim}[commandchars=\\\{\}]
x =  10011100101110110100100011100000
y =  01101011111101100010110011100010
    \end{Verbatim}

    \hypertarget{resoluuxe7uxe3o-do-exemplo-5}{%
\paragraph{Resolução do exemplo 5}\label{resoluuxe7uxe3o-do-exemplo-5}}

    \begin{tcolorbox}[breakable, size=fbox, boxrule=1pt, pad at break*=1mm,colback=cellbackground, colframe=cellborder]
\prompt{In}{incolor}{20}{\boxspacing}
\begin{Verbatim}[commandchars=\\\{\}]
\PY{n}{resolve}\PY{p}{(}\PY{n}{a}\PY{p}{,}\PY{n}{b}\PY{p}{,}\PY{n}{n}\PY{p}{,}\PY{n}{k}\PY{p}{)}
\end{Verbatim}
\end{tcolorbox}

    \begin{Verbatim}[commandchars=\\\{\}]
0
pc= 0
x= 2629519584
y= 1811295458
z= 0

1
pc= 0
x= 5259039168
y= 905647729
z= 0

2
pc= 0
x= 5259039168
y= 905647728
z= 5259039168

3
pc= 0
x= 1928143744
y= 452823864
z= 5259039168

4
pc= 0
x= 3856287488
y= 226411932
z= 5259039168

5
pc= 0
x= 7712574976
y= 113205966
z= 5259039168

6
pc= 0
x= 6835215360
y= 56602983
z= 5259039168

7
pc= 2
x= 6835215360
y= 56602983
z= 5259039168

    \end{Verbatim}

    \begin{tcolorbox}[breakable, size=fbox, boxrule=1pt, pad at break*=1mm,colback=cellbackground, colframe=cellborder]
\prompt{In}{incolor}{21}{\boxspacing}
\begin{Verbatim}[commandchars=\\\{\}]
\PY{n}{bmc\PYZus{}always}\PY{p}{(}\PY{n}{declare}\PY{p}{,}\PY{n}{init}\PY{p}{,}\PY{n}{trans}\PY{p}{,}\PY{n}{invariant}\PY{p}{,}\PY{n}{k}\PY{p}{,}\PY{n}{n}\PY{p}{,}\PY{n}{a}\PY{p}{,}\PY{n}{b}\PY{p}{)}
\end{Verbatim}
\end{tcolorbox}

    \begin{Verbatim}[commandchars=\\\{\}]
A propriedade é válida para traços de tamanho até 33
    \end{Verbatim}


    % Add a bibliography block to the postdoc
    
    
    
\end{document}
